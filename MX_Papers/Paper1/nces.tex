Net Condition/Event Systems (NCES) is a finite state formalism that preserves the graphical notation and the non-interleaving semantics of Petri nets \cite{Petri62}, and extends them with a clear and concise notion of signal inputs and outputs. The formalism was introduced in \cite{RaHA95} \VV{in 1995} and has been used in dozens of applications, especially in embedded industrial automation systems. 

%\fgx{8}{module}{Graphical notation of a NCES module.}{CEN}

\begin {figure}
    \centering
    \includegraphics [width = .5 \textwidth] {images/nces_module.jpg}
    \caption {Graphical notation of an NCES module.}
    \label {fig:nces}
\end {figure}

Given a place/transition net $N=(P,T,F,m_0)$, the
Net Condition/Event System (NCES) is defined as a tuple
$\NN=(N,\theta_N,\Psi_N, Gr)$, where $\theta_N$ is an internal
structure of signal arcs, $\Psi_N$ is an input/output structure,
and $Gr \subseteq T$ is a set of so-called "obliged" transitions that fire as soon as it is enabled.
Fig. \ref{fig:nces} shows an example of an NCES module. 
The structure $\Psi_N$ consists of
condition and event inputs and outputs ($ci,ei,eo,co$). The
structure $\theta_N$ is formed from two types of {\it signal}
arcs. Condition arcs lead from places and condition inputs to
transitions and condition outputs. They provide additional
enableness conditions of the recipient transitions. Event arcs
from transitions and event inputs to transitions and event
outputs provide one-sided synchronization of the recipient
transitions: the firing of the source transition forces the firing of the recipient if the latter is enabled by the marking and conditions.

The NCES modules can be interconnected by the condition and event arcs, forming thus distributed and hierarchical models \VV{as illustrated in Fig.\ref{fig:composition}}.
NCES having no inputs can be analyzed without any additional
information about its external environment.

\begin {figure}
    \centering
    \includegraphics [width = .5 \textwidth] {images/composition.jpg}
    \caption {Composition of NCES modules.}
    \label {fig:composition}
\end {figure}

The semantics of NCES cover both asynchronous and synchronous behaviour (required to model plants and controllers respectively). NCES are supported by a family of model-developing and model-checking tools, such as a graphic editor, SESA and ViVe (\cite{vive}).

The state of an NCES module is completely determined by the
current marking $m: P \righarrow \N_0$ of places and values of
inputs. A state transition is determined by the subset $\tau
\subseteq T$ of simultaneously fired transitions, called {\it
step}. The transitions having no incoming event arcs are called
{\it spontaneous}, otherwise {\it forced}. The step fully determines the
values of the event outputs of the module. In the original NCES version the step is formed by
choosing some\footnote{This means the step in NCES is non-deterministic.} of the enabled spontaneous transitions, and all the enabled transitions forced by the transitions already included in the step. 

A state of NCES is fully described by the marking of all its places (in the timed version also by clocks). A transition step specifies a state transition. 
When used for system analysis, a set of all reachable states (complete or partial) 
of NCES model is generated and then analyzed.

For describing the execution model of function blocks we use a deterministic dialect 
of NCES and the modeling approach that guarantee certain properties of the models as follows: 
\begin{enumerate}
\item In the chosen dialect a step is formed from all enabled spontaneous transitions and all forced transitions;
\item The models are designed so that there is no conflicts (i.e. deficient marking in some places) leading to non-deterministic choice of some of the enabled transitions;
\item The models also guarantee bounded marking in all places.
\end{enumerate}
