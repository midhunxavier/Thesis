IEC 61499 is an executable event-based language for control software that allows visual and textual implementation of individual software components, so-called Function Blocks (FBs). In heterogeneous environments, relevant features are the transfer of software and libraries between tools (portability) and the configuration of multiple runtime environments from a single IDE (configurability). Due to ambiguities and deviations from the standard, the FB behaviour slightly varies on each available platform. Such variations are not only observed in IEC~61499-based software and limit portability in practice. This paper therefore investigates an approach for software testing which involves generating portable test code from a specification model. For cross-platform validation, the test code is executed in each relevant run-time environment. We have evaluated our approach in a test-driven development process for a drilling demonstrator. Eclipse 4diac was extended to automatically generate the test code, which was subsequently ported to another IDE. The generation mechanisms consider platform-specific deviations from the IEC~61499-standard to promote portability. We demonstrated the feasibility of cross-platform testing for IEC~61499-based components, enabling further work in the area of test case generation.


\begin{comment}
Hence, executing FB tests in the target platform helps validate the correct functionality of an FB. % before deploying control software to a cyber-physical system. 
%The standardized visual service sequence model 
Interface models in IEC 61499 specify the expected input/output behaviour of a component. A test execution framework can use this information for model-based testing, but these tests are currently not portable. %However, the available approaches require the use of a certain development environment and do not allow specifying tests for control software in a platform-independent way.
In this paper, we use the modelled tests to automatically generate a platform-independent test application that allows validating the functionality of an FB on various platforms. The test automation helps detect any relevant variations in execution behavior among platforms. 
%First, service sequences as test specifications are generated manually or are derived from an existing (partial) implementation. They serve as unit tests for these FBs. We generate a complete 
The test application that is executable on any IEC 61499-compliant platform and automatically compares the results obtained from the FB under test with the expected results. After generation, the test code is fully compliant with the IEC~61499-standard and does not rely on any tool-specific extensions. Hence, it can be ported between platforms. We have evaluated our approach using a test suite as well as a set of FBs that control a demonstrator. We used two IEC~61499 platforms, Eclipse 4diac and SE EcoStruxure, to evaluate the portability of the generated test application. The generation approach takes into account any platform-specific deviations from the IEC~61499-standard. In order to further promote portability between platforms, we also list deviations from the IEC standard for each platform.
\end{comment}