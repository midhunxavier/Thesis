\section{Conclusion and Future Work}
\label{sec::conclusions}
In conclusion, the proposed testing methodology for IEC 61499 FBs offers systematic and reliable means to verify the correct behaviour of FBs across diverse RTEs. The generation of test FBs from a model-based specification, represented as a service sequence, was accomplished through semi-automated means. Engineers can manually create the test specification (e.g., for test-driven development), or derive it from an existing implementation via the IDE. Our created test FBs are portable across platforms to allow platform-independent testing. This paper presented the overall approach and a first proof-of-concept implementation. We provide an initial set of transformation rules and the corresponding tool support for automating part of the process.

In future work, we aim to support all kinds of FB implementations, provide also the test applications automatically, and evaluate our approach based on a realistic use case to evaluate the scalability and feasibility of the approach in practice. Additionally, developing a runtime comparison tool to analyse and compare the behaviour and performance of FBs across different platforms would enable control engineers to identify and address discrepancies. Finally, integrating the testing approach with further model-based development techniques, such as formal methods or simulation, would provide a holistic approach to system verification and enhance the overall reliability of developed systems.


