IEC 61499 is a standard for modular and event-driven industrial automation, enabling distributed control through function blocks (FBs). This architecture enhances reusability, interoperability, and scalability, making it well-suited for cyber-physical automation systems. One of the major challenges in this context is balancing dependability with flexibility. As systems evolve, rapid revalidation becomes essential. Automatic testing plays a crucial role in addressing this by enabling quick verification after changes. However, when deploying this architecture in safety-critical systems, automatic testing alone is insufficient. To ensure correctness and reliability, formal verification techniques are required. Closed-loop verification helps mitigate state-space explosion by integrating plant models with the control logic, allowing for more rigorous analysis. Another key challenge lies in obtaining appropriate models of the physical plant for verification. One practical solution is to leverage existing simulation models, discretize them, and inject non-determinism to represent execution uncertainties. Process mining techniques facilitate the construction of plant models by analyzing event logs from digital twins, providing an accurate representation of system behavior. This approach ensures robust validation, verifying system performance under diverse conditions and operational uncertainties. 

IEC 61499 provides a modular framework for designing control systems, enabling reconfigurable and flexible manufacturing. Blockchain-based traceability enhances security and ensures verification in flexible production system. AI-driven automation further optimizes industrial control by enabling intelligent decision-making, real-time adjustments, and process adaptation. AI agents, leveraging large language models (LLMs) and knowledge graphs (KGs), enhance human-machine collaboration by analyzing tasks and executing actions via OPC UA. These agents can interpret operator instructions, generate and validate execution sequences, and ensure conformance with specified requirements to support reliable and adaptive industrial automation.