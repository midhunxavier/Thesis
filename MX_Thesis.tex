%-------------------------------------------------------------------
% Document class and package definitions
%-------------------------------------------------------------------
\documentclass[12pt,a4paper,openright,final,twoside]{cseethesis}

%Included for Gather Purpose only:
%input "thesisreferences.bib"

% Additional packages for Paper 2
\usepackage{listings}
\usepackage{xcolor}

% Additional packages for Paper 4
\usepackage{float}
\usepackage{tabularx}

% Additional packages for Paper 5
\usepackage{url}

% Additional packages for Paper 6
\usepackage{algorithm2e}

\makeglossaries

\newacronym{nlp}{NLP}{Natural Language Processing}

\begin{document}

\defaultbibliography{thesisreferences,MX_Papers/Paper2/INDIN2021,MX_Papers/Paper3/refrencias_sobraep,MX_Papers/Paper1/sns,MX_Papers/Paper1/main,MX_Papers/Paper4/bibliography/Bibliography,MX_Papers/Paper5/INDIN2022,MX_Papers/Paper6/ETFA2022,MX_Papers/Paper7/bibliography/mybibfile,MX_Papers/Paper8/conference,MX_Papers/Paper9/refs,MX_Papers/Paper10/bibliography/mybibfile}     %% Change this only.
\defaultbibliographystyle{plain}        %% Could be changed if you like 
                                           %% references typeset differently.
%-------------------------------------------------------------------
% Define title, author, etc.
%-------------------------------------------------------------------
\def\thesistitle{Example Thesis updated by Tosin Adewumi}

\begin{figure}[h]
 \centering
 \includegraphics[width=0.5\textwidth]{bg_wall.jpg}
\end{figure}

\def\theauthor{Johan E.\ Carlson}
\def\theaddress{Dept.\ of Computer Science, Electrical and Space Engineering\\
Lule{\aa} University of Technology\\ Lule{\aa}, Sweden}

\def\supervisors{Name of your supervisor(s)}
\def\supervisorstring{Supervisors:} % Edit here if you have only one supervisor
\def\dedication{To my surprise...}

% Read abstract and preface from separate files.
% Make sure these exist. See example files.
\def\theabstract{IEC 61499 is a standard for modular and event-driven industrial automation, enabling distributed control through function blocks (FBs). This architecture enhances reusability, interoperability, and scalability, making it well-suited for cyber-physical automation systems. One of the major challenges in this context is balancing dependability with flexibility. As systems evolve, rapid revalidation becomes essential. Automatic testing plays a crucial role in addressing this by enabling quick verification after changes. However, when deploying this architecture in safety-critical systems, automatic testing alone is insufficient. To ensure correctness and reliability, formal verification techniques are required. Closed-loop verification helps mitigate state-space explosion by integrating plant models with the control logic, allowing for more rigorous analysis. Another key challenge lies in obtaining appropriate models of the physical plant for verification. One practical solution is to leverage existing simulation models, discretize them, and inject non-determinism to represent execution uncertainties. Process mining techniques facilitate the construction of plant models by analyzing event logs from digital twins, providing an accurate representation of system behavior. This approach ensures robust validation, verifying system performance under diverse conditions and operational uncertainties. 

IEC 61499 provides a modular framework for designing control systems, enabling reconfigurable and flexible manufacturing. Blockchain-based traceability enhances security and ensures verification in flexible production system. AI-driven automation further optimizes industrial control by enabling intelligent decision-making, real-time adjustments, and process adaptation. AI agents, leveraging large language models (LLMs) and knowledge graphs (KGs), enhance human-machine collaboration by analyzing tasks and executing actions via OPC UA. These agents can interpret operator instructions, generate and validate execution sequences, and ensure conformance with specified requirements to support reliable and adaptive industrial automation.}
\def\thepreface{\input{preface.tex}}


% Change here if you want to remove the logo printed on the first page

%\def\thelogo{\includegraphics[width=2.5cm]{bg_wall.jpg}} % old EU logo
\def\thelogo{} % no logo

% The definitions above could be put directly in the function call below,
% but is here defined explicitly, for the purpose of clarity.

\startpreamble
  {\thesistitle}
  {\theauthor}
  {\theaddress}
  {\supervisors}
  {\dedication}
  {\theabstract}
  {\thepreface}
  {\thelogo}

%%%%%%%%%%%%%%%%%%%%%%%%%%%%%%%%%%%%%%%%%%%%%%%%%%%%%%%%%%%%%%%%%%%%
%% Begin Part I
%%%%%%%%%%%%%%%%%%%%%%%%%%%%%%%%%%%%%%%%%%%%%%%%%%%%%%%%%%%%%%%%%%%%
\makepartpage{Part I} 

%% Initialize part containing the thesis introduction chapters
\startchapters
%------------------------ Start chapter 1 --------------------------
% The \makechapter command takes three arguments
%  1) An abbreviated version of the chapter name,
%     to be used as page header
%  2) String to be added to the table of contents
%  3) The chapter name, possibly split in to lines,
%     as in Chapter 2 below.
%
%  The different arguments can have different line breaks.
%
% The actual contents of the chapter is included by removing the
% comment from the \input line below. Make sure the file
% chapter1.tex exists.
%-------------------------------------------------------------------

\def\myquote{``This report, by its very length, defends itself against the risk of being read."\\[.5\baselineskip] Winston Churchill}

\makechapter[\myquote]{Thesis Introduction}{Thesis Introduction}{Thesis
Introduction\label{ch1}}
% Chapter 1: Thesis Introduction
% This chapter provides an introduction to the thesis, including key definitions,
% research questions, and a summary of included papers.

\section{Introduction}
\subsection{Overview}

Industrial automation is undergoing a fundamental transformation driven by the shift from mass production to mass customization, requiring unprecedented levels of flexibility, reconfigurability, and adaptability in manufacturing systems. This evolution has been catalyzed by the emergence of modern computing technologies and their integration into industrial processes, leading to significant improvements in how industries operate. While early industrial automation focused primarily on cost reduction and productivity enhancement, contemporary systems must address complex challenges including quality assurance, system flexibility, reconfigurability, and component reusability. The control architecture of industrial automation systems has consequently evolved from centralized, monolithic designs to modular, decentralized component-based architectures that can adapt to changing production requirements.

The IEC 61499 standard has emerged as a cornerstone technology in this transformation, providing a vendor-independent, component-based framework for designing distributed control systems. This standard addresses the critical need for modular, reusable, flexible, extensible, scalable, and reconfigurable automation solutions. The IEC 61499 architecture is particularly well-suited for cyber-physical automation systems, where the integration of computational and physical processes creates complex interdependencies that traditional centralized control approaches cannot effectively manage. By enabling the development of distributed control applications through function blocks (FBs), IEC 61499 facilitates the creation of intelligent mechatronic components that can be seamlessly integrated and reused across different automation systems.

However, the increased complexity and distributed nature of these systems pose significant verification and validation (V\&V) challenges that traditional approaches cannot adequately address. While simulation remains a valuable tool for initial system testing and behavior assessment, it becomes increasingly inadequate as system complexity grows. Simulation alone cannot guarantee comprehensive validation of automation systems, particularly for safety-critical applications where the consequences of system failures can be severe. The limitations of simulation-based approaches include their inability to explore the entire state space of complex systems, their reliance on predefined test scenarios, and their inability to provide formal guarantees of system correctness.

To address these limitations, formal verification techniques have emerged as essential tools for ensuring system reliability and safety. Formal verification provides mathematically rigorous methods for proving or disproving the correctness of systems with respect to specified properties. Among these techniques, model checking has proven particularly valuable for industrial automation systems due to its automated nature, ability to generate counterexamples when properties are violated, and support for temporal logic specifications. Model checking systematically explores all reachable states of a system model to verify whether it satisfies given formal specifications, typically expressed in Computational Tree Logic (CTL) or Linear Temporal Logic (LTL).

Despite the theoretical advantages of formal verification, its practical adoption in industrial automation has been limited by several factors. The state-space explosion problem, where the number of possible system states grows exponentially with system complexity, presents a significant computational challenge. Additionally, the creation of formal models requires specialized expertise that is often not available in industrial engineering teams. The lack of user-friendly tools that integrate seamlessly into existing engineering workflows further hinders adoption. Furthermore, the gap between formal verification theory and practical industrial applications has created a barrier that prevents many automation engineers from leveraging these powerful techniques.

This thesis addresses these challenges by presenting a comprehensive framework that bridges the gap between formal verification theory and industrial practice. The research demonstrates how formal verification can be made accessible to automation engineers through integrated toolchains, automated model generation, and intuitive analysis tools. The framework leverages the IEC 61499 standard not only for controller design but also for modeling the system's physical environment (the "plant"), creating comprehensive closed-loop models suitable for rigorous verification.

A key innovation in this research is the introduction of non-deterministic transitions (NDTs) in plant models to enhance verification realism. Traditional deterministic plant models may fail to uncover timing-related bugs that occur only under specific timing conditions. By incorporating NDTs, the framework enables more realistic verification scenarios that can detect subtle, timing-dependent design flaws that would be difficult to identify through simulation alone. This approach allows verification engineers to selectively inject non-determinism where appropriate, enabling targeted formal verification under specific stress conditions.

The research also addresses the critical challenge of verifying runtime safety monitors, which are essential components in safety-critical systems. Runtime monitors observe system behavior during operation and signal errors when safety properties are violated. However, a monitor only improves system safety if its own correctness is guaranteed. The thesis presents a methodology for closed-loop model checking of monitors using non-deterministic twins of the controllers they supervise, ensuring that online safety mechanisms are themselves reliable.

Beyond formal verification, this research explores the application of process mining techniques to industrial control systems, representing a significant advancement in system understanding and model generation. Process mining, traditionally applied in business process management, has been adapted to address the unique challenges of industrial automation. The research demonstrates how process mining can extract control logic from event logs, enabling automated model generation for verification and analysis. This approach provides data-driven methods for understanding complex system behaviors and generating formal models from recorded event traces.

The research presents three complementary process mining approaches: process model extraction and conformance checking for anomaly detection, interactive learning for automatic controller generation, and automatic plant model generation for formal verification. These methodologies collectively address critical challenges in modern industrial automation, including the need for automated system understanding, the complexity of controller development, and the requirement for formal verification in safety-critical applications.

Another significant contribution of this research is the development of comprehensive cross-platform testing methodologies for IEC 61499 applications. Despite the standard's goal of enabling portability across different vendor platforms, practical implementation has revealed significant challenges due to variations in execution semantics and vendor-specific interpretations. The research addresses these challenges through systematic approaches to testing function blocks across different development and runtime environments, including comprehensive test function blocks for data types, boundary conditions, standard functions, and adapter support.

The research also explores model-based testing methodologies using service sequence models for test specification and automated generation. This approach provides a systematic method for ensuring consistent behavior across diverse vendor platforms, addressing a fundamental challenge in IEC 61499 adoption. The development of portable test applications that can execute across different IEC 61499 runtime environments represents a practical solution for comprehensive cross-platform validation.

The most transformative aspect of this research lies in the integration of emerging technologies—blockchain, large language models, and knowledge-driven AI agents—into industrial automation systems. The blockchain-enabled product traceability framework demonstrates how distributed ledger technology can enhance transparency and compliance in flexible manufacturing environments. By providing immutable, tamper-proof records of production processes, blockchain technology addresses the critical need for product traceability in modern manufacturing.

The integration of large language models (LLMs) into industrial control systems represents a significant advancement in human-machine interaction. LLMs enable operators to interact with complex systems using natural language, significantly reducing technical barriers and enhancing system accessibility. The research demonstrates how multi-actor LLM frameworks can provide sophisticated analysis and visualization capabilities for IEC 61499 control systems, enabling conversational AI interfaces that can handle complex queries and provide contextual recommendations.

The most significant advancement in this research is the development of knowledge-driven AI agents for planning, reasoning, and testing. These agents represent a paradigm shift in industrial automation, moving from rule-based systems to intelligent, adaptive systems that can understand natural language instructions, generate optimized operational plans, and autonomously validate system behavior. The agents leverage semantic reasoning with knowledge graphs to make context-aware decisions that balance operational efficiency with sustainability goals.

The AI agent framework includes specialized agents for planning, validation, and testing, demonstrating how complex industrial tasks can be decomposed and executed through coordinated AI systems. The planning agent generates optimized skill sequences to fulfill user-defined objectives by querying knowledge graphs and reasoning over operational constraints. The validating agent verifies whether proposed plans adhere to operational rules and resource constraints, while the testing agent supports requirement-based system testing by transforming specified requirements into executable test actions.

The integration of OPC UA communication enables these agents to interact directly with industrial control systems, reading real-time variables and executing control actions. This capability allows for dynamic decision-making based on sensor feedback, enabling systems to operate adaptively and improve efficiency by reducing unnecessary operations. The agents can handle complex scenarios such as conditional execution based on sensor feedback, enabling systems to adapt their behavior based on real-time conditions.

The research presented in this thesis addresses critical challenges in modern industrial automation while providing practical solutions that can be implemented in real-world systems. The integrated framework for formal verification makes powerful mathematical techniques accessible to automation engineers, while the process mining approaches provide data-driven methods for system understanding and model generation. The cross-platform testing methodologies ensure consistent behavior across diverse vendor platforms, and the AI-driven automation approaches enable intelligent, adaptive systems that can operate autonomously while maintaining high levels of reliability and efficiency.

The convergence of these technologies creates a comprehensive framework for the next generation of industrial automation systems. By integrating formal verification, process mining, cross-platform testing, and AI-driven automation, this research provides both theoretical foundations and practical tools for addressing the complex challenges facing modern manufacturing. The framework enables the development of intelligent, adaptive, and reliable industrial automation systems that can meet the demands of mass customization while maintaining high levels of safety and performance.

The research demonstrates that the integration of formal methods, AI, and industrial automation represents a promising direction for creating more efficient, reliable, and sustainable manufacturing systems. As industrial systems become increasingly complex and interconnected, the need for intelligent automation solutions will continue to grow. The research presented here provides the foundation for addressing these challenges while opening new avenues for future investigation and development.

The future of industrial automation lies in the seamless integration of human intelligence, artificial intelligence, and physical systems, creating cyber-physical systems that are not only efficient and reliable but also intelligent and adaptive. This thesis contributes to this vision by providing the theoretical foundations, practical tools, and experimental validation needed to realize truly intelligent industrial automation systems that can adapt to changing requirements and operate autonomously while maintaining high levels of safety and performance.

\section{Key Definitions}

Since many terms are used interchangeably by researchers in the field of industrial automation and formal verification, this section provides clear definitions of key technical terms used throughout this thesis. These definitions ensure non-ambiguous understanding of the terminology employed in this research.

\textbf{Abstract State Machine (ASM):} A mathematical model for describing the behavior of discrete dynamic systems. ASMs provide a formal foundation for modeling the execution semantics of IEC 61499 function blocks and serve as an intermediate representation in the translation process from IEC 61499 to formal verification languages.

\textbf{Architecture:} The fundamental organization of a system embodied in its components, their relationships to each other and to the environment, and the principles governing its design and evolution. In the context of industrial automation, architecture refers to the structural design of control systems and their constituent components.

\textbf{Blockchain:} A distributed, decentralized digital ledger technology that records transactions across multiple computers in a way that ensures security, transparency, and immutability. In industrial automation, blockchain technology is used for product traceability, process verification, and secure recording of manufacturing events.

\textbf{Closed-Loop Modeling:} A verification approach that includes both the controller and a model of the physical plant (environment) in the formal verification process. This approach constrains the inputs to the controller to only those that are physically possible, leading to more meaningful verification results.

\textbf{Conformance Checking:} A process mining technique that compares observed event logs against reference process models to identify deviations and ensure that system behavior conforms to expected specifications.

\textbf{Cyber-Physical System (CPS):} A system that integrates computational and physical processes, where embedded computers and networks monitor and control physical processes, usually with feedback loops where physical processes affect computations and vice versa.

\textbf{Digital Twin:} A virtual representation of a physical system that mirrors its real-world counterpart in real-time, enabling simulation, analysis, and optimization of system behavior.

\textbf{Event-Driven Architecture:} A software architecture pattern that promotes the production, detection, consumption, and reaction to events. In IEC 61499, function blocks communicate through events, making this architecture fundamental to distributed control systems.

\textbf{Execution Control Chart (ECC):} A state machine that defines the execution logic of a basic function block in IEC 61499. The ECC determines the sequence of algorithm execution in response to input events.

\textbf{Finite State Machine (FSM):} A mathematical model of computation that can be in exactly one of a finite number of states at any given time. FSMs are used to model the behavior of control systems and are fundamental to formal verification.

\textbf{Flexibility:} The ability of a system to adapt to new or different environments, requirements, or operational conditions without significant redesign or reconfiguration.

\textbf{Formal Methods:} Mathematical techniques for the specification, development, and verification of software and hardware systems. These methods provide rigorous, mathematical foundations for ensuring system correctness and reliability.

\textbf{Formal Verification:} The process of using mathematical methods to prove or disprove the correctness of a system with respect to a set of formal specifications. This includes techniques such as model checking, theorem proving, and static analysis.

\textbf{Function Block (FB):} A reusable software component in IEC 61499 that encapsulates data and algorithms. Function blocks interact through well-defined interfaces consisting of event and data inputs and outputs.

\textbf{Industrial Cyber-Physical System (iCPS):} A cyber-physical system specifically designed for industrial applications, integrating computational and physical processes in manufacturing and automation environments.

\textbf{Interoperability:} The ability to use multiple control hardware and software components from various vendors in a single application, ensuring seamless communication and integration.

\textbf{Knowledge Graph:} A structured representation of knowledge that captures relationships between entities in the form of nodes and edges. In AI-driven automation, knowledge graphs encode system configuration, available skills, and operational constraints.

\textbf{Large Language Model (LLM):} An artificial intelligence model trained on vast amounts of text data to understand and generate human language. LLMs are used in industrial automation for natural language processing, human-machine interaction, and intelligent analysis.

\textbf{Model Checking:} An automated formal verification technique that systematically explores all reachable states of a system model to verify whether it satisfies properties expressed in temporal logic.

\textbf{Multi-Agent System:} A system composed of multiple interacting intelligent agents that can work together to achieve complex objectives. In industrial automation, multi-agent systems enable distributed, flexible, and adaptive control.

\textbf{Non-Deterministic Transition (NDT):} A transition in a state machine that can fire at an arbitrary time after it becomes enabled. NDTs are used in plant models to represent execution uncertainties and enable more realistic verification scenarios.

\textbf{OPC UA (OPC Unified Architecture):} A machine-to-machine communication protocol for industrial automation that enables secure, reliable, and platform-independent data exchange between devices and systems.

\textbf{Plant Model:} A formal representation of the physical system or environment that a controller interacts with. Plant models are essential for closed-loop verification and enable comprehensive system analysis.

\textbf{Portability:} The usability of software across various environments, platforms, or runtime systems. In IEC 61499, portability refers to the ability to deploy function blocks across different vendor platforms.

\textbf{Process Discovery:} A process mining technique that automatically generates process models from event logs, revealing the actual behavior and workflow patterns of systems.

\textbf{Process Mining:} A data-driven approach to process analysis that extracts knowledge from event logs to understand, monitor, and improve processes. In industrial automation, process mining is used for system modeling, verification, and enhancement.

\textbf{Reconfigurability:} The ability of a system to be easily modified or adapted to accommodate changes in requirements, functionality, or operational conditions.

\textbf{Safety-Critical System:} A system whose failure could result in loss of life, significant property damage, or environmental harm. These systems require rigorous verification and validation to ensure reliability and safety.

\textbf{Service Sequence Model:} A standardized way to specify the expected input/output behavior of IEC 61499 function blocks, defining expected event occurrences, input values, and expected output values for test scenarios.

\textbf{Smart Contract:} A self-executing contract with predefined rules and conditions encoded in blockchain technology. Smart contracts automate predefined processes and ensure consistency and reliability in industrial applications.

\textbf{Specification:} A formal or informal description of system requirements, behavior, or properties. Specifications can be requirements specifications (used in requirements phases) or system specifications (used in design/development and testing phases).

\textbf{State Space Explosion:} A problem in formal verification where the number of possible system states grows exponentially with system complexity, making exhaustive verification computationally infeasible.

\textbf{Temporal Logic:} A formal system for reasoning about propositions qualified in terms of time. Linear Temporal Logic (LTL) and Computational Tree Logic (CTL) are commonly used in formal verification to specify system properties.

\textbf{Test-Driven Development (TDD):} A software development methodology that emphasizes writing tests before implementing functionality, ensuring that implemented behavior matches specified requirements from the beginning.

\textbf{Verification and Validation (V\&V):} Following IEEE Standard Glossary definitions, verification corresponds to determining if artifacts produced in the current phase fulfill requirements established during the previous phase. Validation is the process of evaluating software at the end of development to ensure compliance with requirements. Verification answers "Am I building the product right?" while validation answers "Am I building the right product?"

\section{List of Abbreviations}

This section provides a comprehensive list of abbreviations used throughout the thesis to ensure clarity and consistency in terminology.

\begin{itemize}
\item \textbf{4DIAC:} Framework for Distributed Industrial Automation and Control
\item \textbf{ASM:} Abstract State Machine
\item \textbf{BFB:} Basic Function Block
\item \textbf{CFB:} Composite Function Block
\item \textbf{CFBM:} Composite Function Block Module
\item \textbf{CORBA:} Common Object Request Broker Architecture
\item \textbf{CPS:} Cyber-Physical System
\item \textbf{CTL:} Computational Tree Logic
\item \textbf{ECC:} Execution Control Chart
\item \textbf{FB:} Function Block
\item \textbf{FBDK:} Function Block Development Kit
\item \textbf{FSM:} Finite State Machine
\item \textbf{HMI:} Human Machine Interface
\item \textbf{ICS:} Industrial Control System
\item \textbf{IPFS:} InterPlanetary File System
\item \textbf{KG:} Knowledge Graph
\item \textbf{LLM:} Large Language Model
\item \textbf{LTL:} Linear Temporal Logic
\item \textbf{NCES:} Net Condition/Event Systems
\item \textbf{NDT:} Non-Deterministic Transition
\item \textbf{NLP:} Natural Language Processing
\item \textbf{NuSMV:} New Symbolic Model Verifier
\item \textbf{OPC UA:} OPC Unified Architecture
\item \textbf{PLC:} Programmable Logic Controller
\item \textbf{RAG:} Retrieval Augmented Generation
\item \textbf{RMCSES:} Reference Model of Control/Sensor Events Sequencing
\item \textbf{RTE:} Runtime Environment
\item \textbf{SDLC:} Software Development Life Cycle
\item \textbf{SIFB:} Service Interface Function Block
\item \textbf{SMC:} State Machine Component
\item \textbf{SMV:} Symbolic Model Verifier
\item \textbf{SQL:} Structured Query Language
\item \textbf{TDD:} Test-Driven Development
\item \textbf{UML:} Unified Modeling Language
\item \textbf{V\&V:} Verification and Validation
\item \textbf{XES:} eXtensible Event Stream
\end{itemize}

\section{Research Question}
\subsection{Aims \& Objectives}
The objective of this research is to enhance formal verification in industrial automation by incorporating modular nondeterministic transitions (NDTs) to improve realism and complexity management. It aims to develop modular applications with automatic verification procedures and automate plant model generation within IEC 61499 for formal verification. Additionally, the research focuses on creating portable test codes for cross-platform validation to ensure consistency in different runtime environments. To further optimize industrial automation, the study integrates AI agents to enable reasoning, planning, and real-time adaptability of IEC 61499-based control systems.

\subsection{Hypothesis}
This research hypothesizes that integrating formal verification techniques with modular automation frameworks enhances system reliability and scalability, while non-deterministic transitions enable realistic simulation and validation of complex processes. Process mining can extract control logic from event logs, enabling automated model generation for verification and analysis. Automated test generation improves system portability and validation, ensuring reliable cross-platform performance. AI-driven agents, leveraging knowledge graphs and LLMs, enhance industrial automation by enabling sustainable planning, dynamic execution, and requirement-based validation of IEC 61499 control systems through OPC UA integration.

\subsection{Problem Statement}
Integrating formal verification into the design of IEC 61499-based automation systems remains a challenge in ensuring reliability and correctness. Leveraging process mining to generate formal models from event logs can improve system analysis, but effective methods for this transformation are needed. Detecting portability issues before deployment is critical to prevent failures due to execution environment differences. AI-driven reasoning, planning and decision-making can enhance system adaptability, but efficient implementation methods need further exploration.

\subsection{Research Questions}
\subsubsection{Main Research Question}
How can best practices in formal verification, process mining, model-based testing, and AI-driven automation be combined to enhance the safety, conformance, portability and adaptability of industrial control systems?

\subsubsection{Sub-questions}
\begin{enumerate}
    \item How can formal verification techniques be applied to modular industrial automation systems to ensure their safety and correctness?
    
    \item How can the plant model creation be automated by process mining? How could such models be integrated into the formal verification toolchain to verify and optimize industrial control systems based on IEC 61499?
    
    \item How can model-based testing improve the portability and reliability of IEC 61499 control applications across multiple platforms?
    
    \item How can a knowledge-driven AI agent interpret natural-language operator instructions to generate sustainable execution plans for IEC 61499-based industrial control systems? How can AI agents automate requirement-based testing and validate system behavior to ensure conformance in industrial control systems?
\end{enumerate}

\section{Included Papers' Summary}
The following section lists all publications included in Part II of the thesis. A short summary of each paper is presented and my contribution is highlighted. Figure~\ref{fig:ch1:research_mapping} shows the mapping between papers appended and what research question they address and Figure~\ref{fig:ch1:paper_relationship} shows how the appended papers are linked to achieving the thesis objective.

\begin{figure}[htbp]
    \centering
    \includegraphics[width=0.9\textwidth]{chapters/images/chapter1/relationship_papers_researchquestions.png}
    \caption{Mapping between research questions and appended papers.}
    \label{fig:ch1:research_mapping}
\end{figure}

\begin{figure}[htbp]
    \centering
    \includegraphics[width=0.9\textwidth]{chapters/images/chapter1/Literature Review.png}
    \caption{Relation between appended papers and the thesis topics.}
    \label{fig:ch1:paper_relationship}
\end{figure}

\subsection{Paper A}
\textbf{\textit{Title:}} Formal Modelling, Analysis, and Synthesis of Modular Industrial Systems Inspired by Net Condition/Event Systems\\
\textbf{\textit{Authors:}} Midhun Xavier, Sandeep Patil, Victor Dubinin, and Valeriy Vyatkin\\
\textbf{\textit{Published in:}} Lecture Notes in Computer Science (LNCS), 2023\\
\textbf{\textit{Summary:}} This paper summarises recent developments in the application of modular formalisms to model-based verification of industrial automation systems. The paper is a tribute to the legacy of Professor Hans-Michael Hanisch who invented Net Condition/Event Systems (NCES) and passionately promoted the closed-loop modelling approach to modelling and analysis of automation systems. The paper surveys the related works and highlights the impact NCES has made on the current progress of modular automation systems verification.\\
\textbf{\textit{My Contribution:}} [To be added]

\subsection{Paper B}
\textbf{\textit{Title:}} Cyber-physical automation systems modelling with IEC 61499 for their formal verification\\
\textbf{\textit{Authors:}} Midhun Xavier, Sandeep Patil, and Valeriy Vyatkin\\
\textbf{\textit{Published in:}} 2021 IEEE 19th International Conference on Industrial Informatics (INDIN), Palma de Mallorca, Spain, 2021, pp. 1-6, doi: 10.1109/INDIN45523.2021.9557416\\
\textbf{\textit{Summary:}} Distributed industrial automation systems pose a significant challenge for their efficient verification and validation due to their heterogeneous structure, use of wireless communication and decentralised logic. The inherent inter twinning of computational and communication processes with complex physical dynamics has called for the term cyber-physical systems (CPS) to emphasize the challenges and the need for new development approaches. The {IEC 61499} architecture is getting increasingly recognised as a powerful mechanism for engineering such systems. It has been proven also as an efficient way of modeling CPS in automation. The challenge of IEC 61499 verification has been well-recognized from the early stages of the standard's development and evaluation. Closed-loop modelling has been proposed for the most comprehensive verification, which implies the need for modelling the plant. In quite many works, the plant modelling was done in the same formalism, which was used eventually to represent the model for the model-checker. Graphical modelling languages of finite-state machines and Petri nets were used in particular, and the models were prepared using the corresponding graphical editors. However, the IEC 61499 itself provides a graphical engineering interface and supports programming in terms of state machines. Therefore, a problem-oriented notation could be proposed to take advantage of the existing tools and avoid using additional ones in the process of modelling. This paper proposes such an approach by introducing a tool chain.\\
\textbf{\textit{My Contribution:}} [To be added]

\subsection{Paper C}
\textbf{\textit{Title:}} Formal verification of observers supervising a cyber-physical system implemented using IEC 61499\\
\textbf{\textit{Authors:}} Polina Ovsiannikova, Etienne Le Priol, Vincent Perret, Pranay Jhunjhunwala, Midhun Xavier, and Valeriy Vyatkin\\
\textbf{\textit{Published in:}} 2023 IEEE 32nd International Symposium on Industrial Electronics (ISIE), Helsinki, Finland, 2023, pp. 1-6, doi: 10.1109/ISIE51358.2023.10228148\\
\textbf{\textit{Summary:}} A rigorous check is a significant phase in the design process of control programs of safety-critical cyber-physical systems. Here, we consider such programs to be implemented using IEC~61499 standard for industrial automation. After the check is performed (for example, using formal verification), the engineer needs to ensure that even in unexpected situations, the system will not fail during the runtime, and for this online verification methods can be utilized. In this work, we consider attaching monitors implemented as basic function blocks to the interface of the controller, thus having a property being monitored represented in the form of a state machine. Now, monitors make the system safer only if their quality is also ensured. Since their complexity is far lower than the complexity of the controller, they can be model checked, however, in the case of IEC~61499 function blocks, open-loop model checking will produce spurious counterexamples as it will allow combinations that are not possible according to the IEC~61499 function blocks semantics (e.g., data transferred without firing the event). The current work addresses this issue and proposes a method for close-loop model checking of monitors, using the non-deterministic twin of a controller under supervision. We present our approach using the system of two orthogonal pneumatic cylinders.\\
\textbf{\textit{My Contribution:}} [To be added]

\subsection{Paper D}
\textbf{\textit{Title:}} Formal Verification of the Control Software of a Radioactive Material Remote Handling System, Based on IEC 61499\\
\textbf{\textit{Authors:}} Giordano Lilli, Midhun Xavier, Etienne Le Priol, Vincent Perret, Tatiana Liakh, Roberto Oboe, and Valeriy Vyatkin\\
\textbf{\textit{Published in:}} IEEE Open Journal of the Industrial Electronics Society, vol. 4, pp. 417-431, 2023, doi: 10.1109/OJIES.2023.3321084\\
\textbf{\textit{Summary:}} Automation systems within nuclear laboratories are intended to work under harsh operating conditions. SPES (Selective Production of Exotic Species) is a nuclear research facility currently under construction by INFN (Istituto Nazionale di Fisica Nucleare), dedicated to the production and study of Radioactive Ion Beams (RIBs). Isotopes are produced within the Target Ion Source (TIS) unit, a vacuum vessel that must be replaced on a regular basis. The highly radioactive environment necessitates the deployment of a set of automated systems dedicated to the unit's remote management. To meet high-level security standards, the design of such instrumentation and control systems must include extensive verification. Based on specific safety requirements, model checking can be used to assess the systems' correctness. This paper describes how to employ an integrated tool-chain to design, simulate, formally verify, and deploy the control software for the Horizontal Handling Machine, a safety-critical remote handling system in operation at SPES. The IEC 61499 standard's adoption led to a redesign of the control logic. Following a preliminary online simulation, the closed-loop system has been formally verified using the NuSMV symbolic model checker, with the help of the FB2SMV converter. Additionally, the FBME tool was used for automating verification and analyzing counterexamples.\\
\textbf{\textit{My Contribution:}} [To be added]

\subsection{Paper E}
\textbf{\textit{Title:}} Process mining in industrial control systems\\
\textbf{\textit{Authors:}} Midhun Xavier, Victor Dubinin, Sandeep Patil, and Valeriy Vyatkin\\
\textbf{\textit{Published in:}} 2022 IEEE 20th International Conference on Industrial Informatics (INDIN), Perth, Australia, 2022, pp. 1-6, doi: 10.1109/INDIN51773.2022.9976111\\
\textbf{\textit{Summary:}} In this paper, we discuss how process mining techniques can be applied in industrial control systems for modeling, verification, and enhancement of the cyber-physical system based on recorded data logs. Process mining is used for extracting the process models in different notations from the recorded behavioral traces of the system. The output model of the system's behavior is mainly derived using an open-source tool called ProM. The model can be used for such applications as anomaly detection, detection of cyber-attacks and alarm analysis in industrial control systems with the help of various control flow discovery algorithms. The extracted process model can be used to verify how the event log deviates from it by replaying the log on Petri net for conformance analysis.\\
\textbf{\textit{My Contribution:}} [To be added]

\subsection{Paper F}
\textbf{\textit{Title:}} An interactive learning approach on digital twin for deriving the controller logic in IEC 61499 standard\\
\textbf{\textit{Authors:}} Midhun Xavier, Victor Dubinin, Sandeep Patil, and Valeriy Vyatkin\\
\textbf{\textit{Published in:}} 2022 IEEE 27th International Conference on Emerging Technologies and Factory Automation (ETFA), Stuttgart, Germany, 2022, pp. 1-7,\\ doi: 10.1109/ETFA52439.2022.9921602\\
\textbf{\textit{Summary:}} In this paper, we describe a method to automatically derive the controller for an automated process by an interactive learning approach using a simulation model developed in Visual Components 3D simulation software. The latter is used to record the events of the processes and the controller is generated as an IEC 61499 function block. To create different process scenarios, the actuator signals are triggered manually in appropriate order. The controller logic in Petri net is derived by process discovery algorithms with help of recorded events and conversion of Petri net to IEC 61499 function blocks is done by a software tool configured with a set of transformation rules.\\
\textbf{\textit{My Contribution:}} [To be added]

\subsection{Paper G}
\textbf{\textit{Title:}} A Framework for the Generation of Monitor and Plant Model From Event Logs Using Process Mining for Formal Verification of Event-Driven Systems\\
\textbf{\textit{Authors:}} Midhun Xavier, Victor Dubinin, Sandeep Patil, and Valeriy Vyatkin\\
\textbf{\textit{Published in:}} IEEE Open Journal of the Industrial Electronics Society, vol. 5, pp. 517-534, 2024, doi: 10.1109/OJIES.2024.3406059\\
\textbf{\textit{Summary:}} This paper proposes a method for the automatic generation of a plant model and monitoring using process mining algorithms based on recorded event logs. The behavioural traces of the system are captured by recording event logs during plant operation in either manual control mode or with an automatic controller. Process discovery algorithms are then applied to extract the logic of the process behaviour properties from the recorded event logs. The result is represented as a Petri net, which is used to construct the state machine of the plant model and monitor and is in accordance with the IEC 61499 standard. The monitor is implemented as a function block and can be deployed in real time to trigger an error signal whenever there is a deviation from the actual process scenario. The plant model and controller are connected in a closed loop and are used for the formal verification of the system with the help of the 'fb2smv' converter and symbolic model checking tool NuSMV.\\
\textbf{\textit{My Contribution:}} [To be added]

\subsection{Paper H}
\textbf{\textit{Title:}} Developing a Test Suite for Evaluating IEC 61499 Application Portability\\
\textbf{\textit{Authors:}} Midhun Xavier, Sandeep Patil, and Valeriy Vyatkin\\
\textbf{\textit{Published in:}} 2023 IEEE 32nd International Symposium on Industrial Electronics (ISIE), Helsinki, Finland, 2023, pp. 1-4, doi: 10.1109/ISIE51358.2023.10228154\\
\textbf{\textit{Summary:}} This paper presents the creation of a series of function blocks with the specific aim of testing the portability of IEC 61499 applications across diverse development and runtime environments. These function blocks have been developed to cover a wide range of test scenarios, including basic data types, functions, boundary conditions, and adapter features. The function blocks can be conveniently exported or imported through the use of XML files, thus facilitating seamless testing. By testing the runtime environment of different IEC 61499 systems, these function blocks help to identify and highlight any possible issues that may arise related to portability.\\
\textbf{\textit{My Contribution:}} [To be added]

\subsection{Paper I}
\textbf{\textit{Title:}} Generating Portable Test Cases for IEC 61499 FBs from Interface Behaviour Specifications\\
\textbf{\textit{Authors:}} Midhun Xavier, Sandeep Patil, and Valeriy Vyatkin\\
\textbf{\textit{Published in:}} 2023 IEEE 28th International Conference on Emerging Technologies and Factory Automation (ETFA), Sinaia, Romania, 2023, pp. 1-4, doi: 10.1109/ETFA54631.2023.10275633\\
\textbf{\textit{Summary:}} IEC 61499 is an executable, event-based language for control software that allows visual and textual implementation of individual software components (Function Blocks, FBs). The standardized visual service sequence model specifies the expected input/output behaviour of a component, thus supporting model-based testing. We present our approach for testing an FB on various platforms, which helps manage the variations in execution semantics between different vendors. First, service sequences are generated manually or derived from an existing (partial) implementation. Then, these service sequences serve as unit tests for this implementation. Finally, we create a test application that is executable on any IEC 61499-compliant platform. Executing tests directly in the target platform helps validate the correct functionality of an FB before deploying the control software to a cyber-physical system.\\
\textbf{\textit{My Contribution:}} [To be added]

\subsection{Paper J}
\textbf{\textit{Title:}} Develop Once, Test Everywhere: Cross-Platform Development of Distributed Control Software\\
\textbf{\textit{Authors:}} Bianca Wiesmayr, Melanie Winter, Midhun Xavier, Sandeep Patil, Valeriy Vyatkin, and Alois Zoitl\\
\textbf{\textit{Published in:}} IEEE Open Journal of the Industrial Electronics Society, 2025 [Not Yet Published]\\
\textbf{\textit{Summary:}} Realising flexible industrial automation systems requires an approach for autonomous and distributed designs. Programmable Logic Controllers (PLCs) are the established platform for real-time control software that accesses sensors and actuators. Standards play an important role in distributed automation, for instance, IEC~61131-3 and IEC~61499, which define programming paradigms for control software development. Multiple interacting PLCs form a distributed control system. Providing the respective engineering methodologies and models is a goal of IEC~61499. Heterogeneous systems can even be composed of PLCs from various vendors and programmed with different tools. Furthermore, a single development tool can distribute control code across multiple runtime environments (RTEs), motivating the need to execute component tests in each of these RTEs. Developers of IEC~61499 library modules will also need to provide their modules to users of various development environments. Despite the focus on portability and the standardized XML format for data exchange, IEC~61499-based software components must often be modified during the porting process. Due to varying execution behaviour, the ported software may behave differently on each platform, possibly leading to malfunctions of the distributed control system. Therefore, it is crucial to thoroughly test an IEC 61499 application on each relevant target platform before using the software in a real-world system. A platform-independent test specification has the potential to greatly reduce the involved effort.\\
\textbf{\textit{My Contribution:}} [To be added]

\subsection{Paper K}
\textbf{\textit{Title:}} Enhancing Traceability in Flexible Production System: A Blockchain-Powered Approach in IEC 61499 Multi-Agent Control System\\
\textbf{\textit{Authors:}} Midhun Xavier, Sandeep Patil, and Valeriy Vyatkin\\
\textbf{\textit{Published in:}} 2024 IEEE 33rd International Symposium on Industrial Electronics (ISIE), Ulsan, Korea, Republic of, 2024, pp. 1-6, doi: 10.1109/ISIE54533.2024.10595680\\
\textbf{\textit{Summary:}} This paper presents a novel approach for tracking products and processes within industrial multi-agent control systems by leveraging blockchain technology. The suggested solution facilitates the recording and validation of every step involved in creating a tailored product through the utilization of Ethereum-based smart contracts. The OWL ontology is used to describe agents and their capabilities and these software agents interact with IEC 61499 function blocks for process execution. The software agents record process events at each stage on the blockchain and the latter smart contract helps to trace and verify these process sequences of the customised product.\\
\textbf{\textit{My Contribution:}} [To be added]

\subsection{Paper L}
\textbf{\textit{Title:}} LLM-Powered Multi-Actor System for Intelligent Analysis and Visualization of IEC 61499 Control Systems\\
\textbf{\textit{Authors:}} Midhun Xavier, Sandeep Patil, and Valeriy Vyatkin\\
\textbf{\textit{Published in:}} IECON 2024 - 50th Annual Conference of the IEEE Industrial Electronics Society, Chicago, IL, USA, 2024, pp. 1-8, doi: 10.1109/IECON55916.2024.10905502\\
\textbf{\textit{Summary:}} This paper introduces an innovative multiactor framework that harnesses the potential of LLMs to augment the functionalities of ICS. By integrating conversational AI technologies, this framework significantly improves human-machine interactions, enabling sophisticated analysis and visualization of intricate data sets. The core of the system comprises specialized LLM actors that interact through a LangGraph-based multi-actor framework, addressing various aspects of IEC 61499 control systems including PLC code analysis, SQL query execution, and data visualization. This integration enables operators to interact with the control system using natural language, significantly reducing technical barriers and enhancing the accessibility and usability of complex industrial systems.\\
\textbf{\textit{My Contribution:}} [To be added]

\subsection{Paper M}
\textbf{\textit{Title:}} ReACT - Gen AI Agents for Reasoning, Planning, and Testing in Industrial Automation Systems\\
\textbf{\textit{Authors:}} Midhun Xavier, Sandeep Patil, and Valeriy Vyatkin\\
\textbf{\textit{Published in:}} [Not Submitted]\\
\textbf{\textit{Summary:}} This paper proposes a novel AI agent for reasoning, planning, and testing industrial automation applications. The AI agent accepts operator instructions in natural language and performs semantic reasoning over functional requirements to generate an optimized cost-effective action plan. This plan comprises a sequence of executable actions that can be deployed in industrial control systems (ICS) to enable efficient and sustainable machine operation. Then the AI agent automates the testing of control systems by comparing the agent's planned and executed actions against expected outputs, ensuring requirement conformance and enhancing system reliability. Experimental results demonstrate the effectiveness of the AI agent in generating sustainable operational plans and validating control system behavior for laboratory scale case studies, underscoring its potential in the future of intelligent industrial automation.\\
\textbf{\textit{My Contribution:}} [To be added]


\makechapter{Running header}{Table of contents entry}{Title
appearing on the\\ chapter start page\label{ch2}}
\section{Introduction}

Industrial automation is undergoing a significant transformation, driven by the need for mass customization, increased flexibility, and reconfigurability. This has led to the adoption of modular, distributed architectures, often termed Component-Based industrial Automation Systems (CBAS) or, more broadly, industrial Cyber-Physical Systems (iCPS) \cite{lee2017introduction}. The International Electrotechnical Commission (IEC) 61499 standard has emerged as a key enabler for engineering these complex systems, providing a vendor-independent, component-based model for distributed control applications \cite{iec61499part12012, dai2017discrete}.

However, the increased complexity and distributed nature of these systems pose significant verification and validation (V\&V) challenges \cite{vyatkin1999modeling}. Traditional V\&V techniques, such as simulation and testing, are often inadequate for ensuring the high levels of dependability, safety, and reliability required, particularly in safety-critical domains like nuclear material handling or power grid protection \cite{clarke1999, schneider2004}. While simulation is invaluable for assessing general system behavior, it cannot explore the entire state space of complex industrial software, meaning it can show the presence of bugs but not their absence \cite{clarke2000, biere2003bounded}.

To address this gap, formal methods offer a mathematically rigorous approach to system verification \cite{clarke1999}. Model checking, in particular, is an automated technique that systematically explores all reachable states of a system model to verify whether it satisfies a given set of formal specifications, typically expressed in temporal logic \cite{emerson1985decision}. Despite its power to uncover subtle design flaws that simulation might miss, the adoption of model checking in industrial practice has been limited. This is often due to challenges such as the state-space explosion problem, the need for specialized expertise to create formal models, and a lack of user-friendly tools that integrate seamlessly into the existing engineering lifecycle \cite{Buzhinsky2020, Cimatti2012}.

This chapter presents an integrated, model-based framework designed to bridge this gap \cite{vyatkin2008closed}. It leverages the IEC 61499 standard not only for controller design but also for modeling the system's environment (the "plant"), creating a comprehensive closed-loop model suitable for rigorous verification \cite{xavier2021cyber}. The framework incorporates a seamless toolchain that automates the translation of IEC 61499 designs into a formal representation for a powerful model checker, and provides intuitive tools for analyzing the results \cite{patil2015formal}. Key contributions of this framework include a novel notation for introducing non-determinism into plant models to make verification more realistic and a specific methodology for verifying the correctness of runtime safety monitors \cite{17jhunjhunwala2022monitoring}.

The chapter is structured as follows. Section \ref{sec:iec61499} provides an overview of the IEC 61499 standard. Section \ref{sec:formal_verification} introduces the principles of formal verification and the closed-loop modeling paradigm. Section \ref{sec:toolchain} details the components of the integrated toolchain and the associated methodologies for enhancing verification. Section \ref{sec:case_studies} demonstrates the practical application of the framework through several case studies, from laboratory-scale systems to a safety-critical industrial machine. Finally, Section \ref{sec:conclusion} concludes the chapter and discusses future research directions.

\section{The IEC 61499 Standard for Distributed Automation}\label{sec:iec61499}

The IEC 61499 standard provides a reference architecture for the development of distributed control systems \cite{iec61499}. It was conceived as a successor to the prevalent IEC 61131-3 standard \cite{tiegelkamp1995iec}, moving beyond the limitations of centralized, PLC-centric logic to support the demands of modern, distributed iCPS \cite{zoitl2014}.

At the core of the standard is the Function Block (FB), a reusable software component that encapsulates data and algorithms \cite{drozdov2021}. FBs interact through a well-defined graphical interface consisting of event and data inputs and outputs. The execution of logic within an FB is event-driven; an input event triggers the execution of associated algorithms and may result in the emission of one or more output events, propagating the flow of control through the network \cite{christensen2000design}.

The standard defines several types of FBs, with the most fundamental being the Basic Function Block (BFB). The internal logic of a BFB is defined by an Execution Control Chart (ECC), which is a state machine that dictates the sequence of algorithm execution in response to input events \cite{hanisch2009one}. A Composite Function Block (CFB), in contrast, contains an internal network of other interconnected FBs, enabling hierarchical system design \cite{sonnleithner2021iec}. This component-based, event-driven nature makes IEC 61499 inherently suited for modeling modular, reconfigurable, and distributed systems \cite{patil2018}.

Despite these advantages, the verification of IEC 61499 applications is a recognized challenge \cite{sinha2019survey}. The concurrent and distributed nature of the systems can lead to complex emergent behaviors that are difficult to predict. Furthermore, ambiguities in the standard's original execution semantics led to different interpretations by tool vendors, creating portability and interoperability issues that formal modeling can help to identify and resolve \cite{patil2015neutralizing}.

\section{Formal Verification for IEC 61499 Systems}\label{sec:formal_verification}

Formal verification is the process of using mathematical methods to prove or disprove the correctness of a system with respect to a set of formal specifications \cite{clarke1999}. Model checking is a prominent formal verification technique that automates this process \cite{baier2008}. It operates on a finite-state model of the system, systematically exploring all possible execution paths to check if they satisfy properties expressed in a formal language, such as Linear Temporal Logic (LTL) or Computation Tree Logic (CTL) \cite{emerson1985decision}. If a property is violated, the model checker produces a counterexample—a specific trace of execution that demonstrates the failure \cite{beer2012}.

\subsection{The Closed-Loop Modeling Paradigm}

A crucial aspect of applying model checking effectively to control systems is the adoption of a closed-loop modeling approach \cite{vyatkin2008closed}. Verifying a controller in isolation (open-loop) requires making assumptions about all possible inputs it might receive from its environment, which often leads to an unmanageably large state space and can produce spurious counterexamples that are not possible in the real system \cite{buzhinsky2016plant}.

A closed-loop model, by contrast, includes a formal model of the plant—the physical process or environment the controller interacts with \cite{xavier2021cyber}. This model constrains the inputs to the controller to only those that are physically possible, drastically reducing the reachable state space and leading to more meaningful verification results \cite{xavier2022plant}. This architecture, encompassing both the controller and the plant, allows for the verification of system-level properties that involve interactions between the two \cite{xavier2023formal}.

\begin{figure}[h]
\centering
\includegraphics[width=0.8\textwidth]{chapters/images/chapter2/wholesystem_withhmi.png}
\caption{An illustration of the closed-loop modeling concept, where a controller (or a non-deterministic twin, right) is verified in conjunction with a monitor (left), representing an interconnected system.}
\label{fig:closed_loop}
\end{figure}

\subsection{From Simulation to Verification Models}

The IEC 61499 standard is well-suited for closed-loop modeling because the same component-based language can be used to model both the controller and the plant \cite{vyatkin2003verification}. This allows for a seamless transition from simulation to formal verification \cite{patil2015formal}.

Initially, a detailed plant model, often equivalent to a hybrid automaton with continuous dynamics, can be created for simulation-in-the-loop to test the overall system behavior and provide visualization \cite{vyatkin2008closed}. For formal verification, this detailed model is then abstracted into a simpler, discrete-state model that captures the essential behaviors relevant to the properties being checked, but omits complexities like continuous dynamics that are computationally expensive for model checkers to handle \cite{drozdov2016formal}. This dual-model approach allows engineers to use familiar tools for design and simulation, while enabling a pathway to rigorous formal verification \cite{xavier2022process}.

\section{An Integrated Verification Toolchain and Methodology}\label{sec:toolchain}

To make formal verification practical for automation engineers, a seamless and highly automated toolchain is required \cite{xavier2021}. The framework presented here integrates industry-standard design environments with a powerful open-source model checker, facilitated by a custom model generator and advanced analysis tools \cite{fb2smv}.

\subsection{Overview of the Toolchain}

The proposed workflow connects the design, verification, and analysis phases into a cohesive process.

\begin{figure}[h]
\centering
\includegraphics[width=0.9\textwidth]{chapters/images/chapter2/system_functional_model.png}
\caption{The proposed workflow for the design and validation of safety-critical automation systems, integrating design, simulation, formal model generation, model checking, and counterexample analysis.}
\label{fig:workflow}
\end{figure}

The toolchain consists of the following key components:

\begin{itemize}
\item \textbf{IEC 61499 Design Environment}: The process begins in an IEC 61499-compliant IDE, such as EcoStruxure™ Automation Expert or the open-source Function Blocks Modelling Environment (FBME) \cite{FBME}, where the controller and plant FBs are designed and simulated.

\item \textbf{Formal Model Generator (fb2smv)}: The fb2smv tool automatically translates the IEC 61499 application, described in its standard XML format, into a formal model in the SMV (Symbolic Model Verifier) language \cite{drozdov2021formal}. This tool uses Abstract State Machines (ASM) as an intermediate model to handle the translation of the ECCs, algorithms, and network connections \cite{gurevich1995evolving}.

\item \textbf{Symbolic Model Checker (NuSMV)}: The generated SMV model is fed into NuSMV, an open-source symbolic model checker \cite{Cimatti2002}. NuSMV verifies the model against the specified LTL or CTL properties and, if a violation is found, generates a counterexample trace \cite{nusmv}.

\item \textbf{Counterexample Analysis Environment (FBME)}: Analyzing the raw text output of a model checker can be difficult and unintuitive \cite{Ovsiannikova2021}. Tools like FBME have been enhanced with trace visualization capabilities that import the counterexample from NuSMV and map it back onto the original graphical IEC 61499 model \cite{liakh2022formal}. This allows engineers to step through the failure trace visually, see the state of ECCs and variable values at each step, and use causal analysis to pinpoint the root cause of the error.
\end{itemize}

\subsection{Enhancing Plant Models with Non-Deterministic Transitions (NDT)}

A key challenge in creating abstracted plant models for verification is ensuring they are realistic enough to capture potential failures \cite{patil2011closed}. A purely deterministic model, where an action takes a fixed amount of time, may fail to uncover timing-related bugs, such as collisions that occur only when one axis moves faster or slower than another \cite{patil2015counterexample}.

To address this, this framework introduces a problem-oriented notation for Non-Deterministic Transitions (NDT) directly within the ECC of an IEC 61499 plant model \cite{xavier2021cyber}. An NDT represents a transition that can fire at an arbitrary time after it becomes enabled. For example, in a model of a linear axis, an NDT can be used to model the unknown duration of the motion between two points \cite{toolchain}.

\begin{figure}[h]
\centering
\includegraphics[width=0.7\textwidth]{chapters/images/chapter2/ECC-moniteur.png}
\caption{An example ECC for an elevator plant model. The transition from the GO state to the END state is triggered by an NDT event, modeling an unspecified amount of time for the motion to complete.}
\label{fig:ecc_ndt}
\end{figure}

When the fb2smv tool encounters an NDT event, it translates it into a non-deterministic choice in the SMV model \cite{drozdov2016formal}. This instructs the model checker to explore all possible timings for that transition, effectively simulating a wide range of real-world scenarios (e.g., different relative speeds between concurrent movements) and enabling the detection of subtle, timing-dependent design flaws that would be difficult to find through simulation alone \cite{patil2015counterexample}. This ability to govern and selectively inject non-determinism is a powerful feature for performing targeted formal verification under specific stress conditions \cite{patil2011closed}.

\subsection{A Methodology for Verifying Runtime Monitors}

In addition to offline verification at the design stage, many safety-critical systems utilize online verification through runtime monitors \cite{17jhunjhunwala2022monitoring}. These monitors, often implemented as FBs themselves, observe the system during operation and signal an error if a safety property is violated \cite{1falcone2022runtime}. However, a monitor only improves system safety if its own correctness is guaranteed \cite{9wiesmayr2022supporting}.

Verifying a monitor presents a unique challenge \cite{15blech2016comparison}. Traditional open-loop model checking, where the monitor's inputs are allowed to change arbitrarily, will often produce spurious counterexamples \cite{12yoong2010verifying}. This is because it allows input combinations that are semantically impossible in a real IEC 61499 system, such as data being transferred without a corresponding event firing \cite{13yoong2015verification}.

To solve this, a methodology is proposed for the closed-loop model checking of monitors using a non-deterministic twin (ND twin) of the controller it supervises \cite{17jhunjhunwala2022monitoring}. The ND twin is a simplified FB that abstracts the controller's logic but is designed to produce every possible valid combination of its outputs. It achieves this by using NDTs to allow arbitrary transitions between its key states (e.g., EXTEND, RETRACT, STOP) \cite{toolchain}.

By connecting the monitor to this ND twin, the model checker can exhaustively test the monitor's logic against all valid controller behaviors \cite{nusmv}. The verification is then guided by two main properties:

\begin{enumerate}
\item The monitor must report a failure when a failure actually occurs. (e.g., G ((m.hext ∧ m.vext) → (m.hext ∧ m.vext U m.collision)))
\item The monitor must not report a spurious failure when the controller is functioning as expected. (e.g., G (¬(m.hext ∧ m.vext) → F ¬m.collision))
\end{enumerate}

This approach ensures that the monitor is trustworthy before it is deployed for online verification, forming a critical link between offline design-time checks and online runtime safety assurance \cite{17jhunjhunwala2022monitoring}.

\section{Application to Case Studies}\label{sec:case_studies}

The effectiveness of the proposed framework and methodologies has been demonstrated on several case studies, ranging from laboratory-scale systems to real-world safety-critical applications \cite{xavier2023formal}.

\subsection{System of Two Pneumatic Cylinders}

This system involves two orthogonal pneumatic cylinders that must move without colliding \cite{17jhunjhunwala2022monitoring}. A runtime monitor, NoCollisionMonitor, was implemented as a BFB to ensure this safety property holds during operation. To verify the monitor, an ND twin of the cylinder controller was created. The ND twin was designed to non-deterministically produce EXTEND, RETRACT, and STOP commands, allowing it to generate the collision scenario (EXTEND from both twins simultaneously) as well as all non-collision scenarios. Using NuSMV, the monitor was successfully verified against the two key properties: it correctly reported the collision when it occurred and did not report false positives, proving its reliability \cite{nusmv}.

\begin{figure}[h]
\centering
\includegraphics[width=0.8\textwidth]{chapters/images/chapter2/2cylindre_hmi.JPG}
\caption{The two pneumatic cylinder system with HMI interface, demonstrating the collision detection and prevention mechanism.}
\label{fig:two_cylinders}
\end{figure}

\subsection{Drilling Station}

This case study features a drilling station with a drill and a rotating table \cite{fb2smv}. The control logic, implemented in IEC 61499, was designed to prevent the table from rotating while the drill is active. The system was modeled in a closed loop, with NDTs used in the plant models for the drill and table to represent the unknown duration of their movements \cite{drozdov2016formal}. During model checking with NuSMV, a CTL specification G !(DrillCTL\_RET = TRUE \& ActuatorGen\_EO = TRUE) (the table cannot rotate while drilling) was found to be false \cite{emerson1985decision}. The model checker generated a counterexample trace which, after analysis, revealed a flaw in the table controller's ECC that failed to check for a blocking signal before initiating rotation. After fixing the logic and re-running the verification, the property was satisfied, demonstrating the toolchain's ability to find and help correct non-trivial design errors \cite{patil2015counterexample}.

\subsection{Safety-Critical Horizontal Handling Machine (HHM)}

The framework was applied to the refactoring and verification of the control software for the HHM, a remote handling robot used to transport highly radioactive material at the SPES nuclear research facility \cite{Marchietal.2020}. The original control software, based on IEC 61131, was redesigned using a modular IEC 61499 architecture \cite{zoitl2014}.

A critical safety requirement is to prevent mechanical collisions between the machine's axes \cite{khan2014}. To test this, a design flaw was deliberately introduced into the controller's FSM, changing a sequence of three axis movements from sequential to parallel execution. This type of error is hard to detect via simulation, as a collision may or may not occur depending on the relative speeds of the axes \cite{patil2015counterexample}.

The plant models for the linear axes were created with NDTs to model variable movement times \cite{xavier2021cyber}. The entire closed-loop system was translated to SMV and verified with NuSMV \cite{fb2smv}. The model checker successfully found a violation of the LTL specification designed to prevent collisions \cite{emerson1985decision}. The resulting counterexample trace was imported into FBME for analysis \cite{liakh2022formal}.

\begin{figure}[h]
\centering
\includegraphics[width=0.9\textwidth]{chapters/images/chapter2/vizu_jumeauV1.png}
\caption{Analysis of the NuSMV counterexample trace in the FBME environment. The tool visualizes the system state at each step, highlights changes in the graphical FB network, and allows for causal analysis to trace the error back to its source.}
\label{fig:counterexample_analysis}
\end{figure}

The visual trace analysis in FBME made it possible to step through the failure scenario and clearly identify how the parallel execution, under a specific timing sequence explored by NuSMV due to the NDTs, led to the collision \cite{Ovsiannikova2021}. This case study demonstrates the scalability and practical value of the integrated toolchain in ensuring the safety of complex, real-world industrial systems \cite{xavier2023formal}.

\section{Conclusion and Future Outlook}\label{sec:conclusion}

This chapter presented an integrated framework for the model-based design and formal verification of iCPS developed with the IEC 61499 standard \cite{xavier2021cyber}. By combining a seamless toolchain with advanced modeling methodologies, the framework makes the power of formal verification more accessible to automation engineers, enabling the early detection of subtle and critical design flaws that traditional testing methods may miss \cite{patil2015formal}.

The main contributions are threefold. First, the integrated toolchain provides a practical workflow from design and simulation in standard IDEs to automated formal model generation (fb2smv) \cite{fb2smv}, verification (NuSMV) \cite{Cimatti2002}, and intuitive counterexample analysis (FBME) \cite{liakh2022formal}. Second, the proposed notation for Non-Deterministic Transitions (NDT) within IEC 61499 plant models allows for the creation of more realistic verification models that can uncover timing-dependent bugs \cite{xavier2021cyber}. Third, the methodology for verifying runtime monitors using a non-deterministic twin of the controller ensures that online safety mechanisms are themselves reliable \cite{17jhunjhunwala2022monitoring}.

The application of this framework to multiple case studies, including a safety-critical nuclear handling system \cite{Marchietal.2020}, has validated its effectiveness in improving the dependability of complex distributed automation systems \cite{xavier2023formal}. By lowering the barrier to entry for formal methods, this work contributes to building more robust, reliable, and safer industrial systems \cite{clarke1999}.

Future work will focus on extending the framework's capabilities \cite{xavier2022interactive}. This includes automating the generation of ND twins for monitor verification and further exploring the scalability of the approach for even larger systems \cite{xavier2022plant}. Another promising direction is the extension of the modeling notation and toolchain to support timed automata, which would allow for the verification of quantitative real-time properties, moving beyond the current focus on logical correctness and untimed non-determinism \cite{Time-AwareComputations1}. Finally, automatically generating specifications from informal requirements and generating plant models from system data logs are active research areas that could further enhance the automation and power of the verification process \cite{xavier2022process}.


\makechapter{Nonsense chapter}{Nonsense chapter}{Nonsense chapter, here only to verify that some issues in previous versions are really resolved!}
\section{Process Mining Applications in Industrial Control Systems}

The integration of process mining techniques into industrial control systems represents a significant advancement in the field of cyber-physical systems, offering novel approaches to system modeling, verification, and control generation. Process mining, traditionally applied in business process management, has evolved to address the unique challenges of industrial automation, providing data-driven methods for understanding complex system behaviors and generating formal models from recorded event traces.

This chapter presents three complementary approaches that demonstrate the versatility and effectiveness of process mining in industrial control systems. The first approach focuses on process model extraction and conformance checking for anomaly detection and system monitoring. The second approach introduces an interactive learning methodology for automatic controller generation through simulation-based event recording. The third approach extends these capabilities to automatic plant model generation and real-time monitoring for formal verification.

These methodologies collectively address critical challenges in modern industrial automation: the need for automated system understanding, the complexity of controller development, and the requirement for formal verification in safety-critical applications. By leveraging process mining algorithms and the IEC 61499 standard, these approaches provide systematic methods for transforming recorded behavioral traces into formal models that can be used for system analysis, control generation, and verification.

\begin{figure*}[!t]
	\centering
	\includegraphics[width=0.9\textwidth]{chapters/images/chapter4/workflow1.png}
	\caption{Workflow and use cases for process mining applications in industrial control systems. The workflow illustrates the transformation from recorded event logs to various applications including monitor generation (D1), plant model generation (D2), control model generation (D3), and control program generation (D4).}
	\label{workflowUsecase}
\end{figure*}

Figure \ref{workflowUsecase} illustrates the comprehensive workflow and potential use cases enabled by process mining in industrial control systems. Initially, signals from distributed control systems are recorded to construct behavioral models using process mining techniques. These models then serve as the basis for generating monitors for real-time error detection, plant models for formal verification, control models for automatic controller generation, and control programs for system implementation. This integrated approach provides a systematic framework for addressing the complex requirements of modern industrial automation systems.

\section{Process Model Extraction and Conformance Analysis}

The application of process mining techniques in industrial control systems begins with the fundamental challenge of understanding system behavior through recorded event logs. This approach leverages the inherent data-rich nature of modern automation systems to extract meaningful process models that can be used for system analysis, anomaly detection, and performance optimization.

\subsection{Process Mining Fundamentals in Industrial Contexts}

Process mining in industrial control systems differs fundamentally from its business process applications by focusing on the temporal and causal relationships between sensor and actuator signals rather than human activities. The core principle involves analyzing event logs that capture the sequence of control and sensor events occurring during system operation, then applying discovery algorithms to extract formal process models that represent the system's behavioral patterns.

The process mining workflow in industrial contexts typically involves three main phases: process discovery, conformance checking, and process enhancement. Process discovery algorithms analyze event logs to generate process models in various formalisms, most commonly Petri nets, which provide a mathematical foundation for representing concurrent and sequential behaviors. Conformance checking compares observed behavior against expected models to identify deviations, while process enhancement focuses on improving existing models based on new observations.



The selection of appropriate process discovery algorithms is crucial for effective model extraction. Alpha algorithm, as a representative of abstraction-based methods, generates models by analyzing ordering relations between events in the log. This algorithm creates dependency graphs based on the sequence of events, making it suitable for systems with well-defined, deterministic behaviors. However, its sensitivity to noise in event logs can lead to overly complex models when dealing with systems that exhibit significant variability.

Heuristic-based algorithms, such as the fuzzy miner, offer an alternative approach that considers the frequency of event occurrences. These algorithms generate models based on the relative importance of activities and the strength of their relationships, making them more robust to noise and better suited for systems with complex, variable behaviors. The fuzzy miner, in particular, provides interactive representations that help understand system behavior in complex logs, though it may be more challenging to convert to other process modeling languages.

\begin{figure}[!t]
	\centering
	\includegraphics[width=0.4\textwidth]{chapters/images/chapter4/Methodology.png}
	\caption{Tool chain and data flow for generating FSMs from Event Logs and their implementation in the form of IEC 61499 FBs. The process involves preprocessing event logs, applying process discovery algorithms, generating Petri nets, decomposing into state machine components, and finally implementing as IEC 61499 function blocks.}
	\label{monitorFlowchart}
\end{figure}

The process of generating formal models from event logs involves several systematic steps, as illustrated in Figure \ref{monitorFlowchart}. The workflow begins with event log preprocessing, where raw data is cleaned and converted to appropriate formats such as eXtensible Event Stream (XES). Process discovery algorithms are then applied to extract behavioral patterns, typically represented as Petri nets. These Petri nets are subsequently decomposed into state machine components (SMCs) and composed into finite state machines (FSMs). Finally, the FSMs are transformed into IEC 61499 function blocks for implementation in industrial control systems. This systematic approach ensures the accurate representation of system behavior while maintaining compatibility with industrial automation standards.

\subsection{Event Log Structure and Preprocessing}

The quality and structure of event logs significantly influence the effectiveness of process mining applications. Industrial control systems generate event logs that typically include case identifiers, timestamps, component information, signal names, and signal values. The case identifier represents unique process executions, often corresponding to complete operational cycles in cyclic manufacturing processes.

Event log preprocessing is essential for ensuring model quality and accuracy. This includes data cleaning to remove irrelevant events, format conversion to ensure compatibility with process mining tools, and attribute selection to focus on the most relevant information for the analysis. The conversion from CSV format to eXtensible Event Stream (XES) format is particularly important, as most process discovery algorithms require XES input.


The preprocessing phase also involves mapping standard XES attributes to the industrial context. Case columns typically correspond to process execution identifiers, while event columns combine component, signal, and value information to create meaningful activity descriptions. This mapping ensures that the process mining algorithms can correctly interpret the industrial data and generate appropriate models.

\subsection{Conformance Checking and Anomaly Detection}

Conformance checking represents one of the most valuable applications of process mining in industrial control systems, providing systematic methods for detecting deviations from expected behavior. This capability is particularly important for anomaly detection, cyber-attack identification, and quality assurance in safety-critical applications.

The conformance checking process involves comparing observed event logs against reference process models to identify discrepancies. Several algorithms support this analysis, including causal footprint checking, token-based replay, and alignment-based methods. Causal footprint checking compares dependency matrices between the event log and reference model, providing a quick assessment of model fitness. However, this method does not consider event frequencies and may miss subtle deviations.

Token-based replay offers a more detailed analysis by simulating the execution of event traces on the reference model. This approach tracks missing and remaining tokens after each transition, providing quantitative measures of conformance. While effective, token replay has limitations when dealing with non-uniquely labeled transitions and can suffer from token flooding in complex models.

Alignment-based methods provide the most sophisticated approach to conformance checking, offering optimal alignment between observed and modeled behavior using user-defined cost functions. These methods are independent of process model notation and can handle complex scenarios that challenge other approaches. The alignment process identifies the optimal sequence of moves that transforms the observed trace into one that conforms to the model, providing detailed insights into the nature and location of deviations.

The application of conformance checking in industrial control systems extends beyond simple deviation detection to include performance analysis and system optimization. By analyzing conformance metrics over time, operators can identify trends in system behavior, detect gradual degradation, and optimize operational parameters. This capability is particularly valuable in predictive maintenance applications, where early detection of behavioral changes can prevent equipment failures and reduce downtime.

\subsection{Monitor Implementation for Real-Time Conformance Checking}

The implementation of monitors for real-time conformance checking represents a critical application of process mining in industrial control systems. These monitors are implemented as IEC 61499 function blocks that can be deployed in operational systems to provide continuous monitoring and early warning of potential issues.

\begin{figure*}[!t]
	\centering
	\includegraphics[width=1\textwidth]{chapters/images/chapter4/ConformaceCheckingApp.jpg}
	\caption{Application for conformance checking. The structure shows a closed-loop system consisting of a plant and controller, with the RMCSES (Reference Model of Control/Sensor Events Sequencing) connected for conformance checking. The monitor function block receives input signals from both controller and sensor components and generates OK/ERROR outputs based on conformance to the reference model.}
	\label{conformanceCheckingApp}
\end{figure*}

The structure of the application for conformance checking is depicted in Figure \ref{conformanceCheckingApp}. The application is based on a closed-loop system that includes a plant and a controller. The RMCSES obtained from the event log is connected to the closed-loop system for the purpose of conformance checking. The RMCSES is implemented as a finite state machine that uses input data to drive its logic. The monitor function block is capable of detecting errors and verifying the correct sequence of steps in the process flow by monitoring events from sensors and actuators in real time. When the monitor detects a valid transition, it generates an "OK" output indicating successful operation. If an unexpected event occurs, it transitions to an error state and produces an "ERROR" event with details about the event that caused the error.

The transformation from finite state machines to monitor ECCs follows a systematic algorithmic approach that ensures complete coverage of all possible system states and transitions. This transformation is essential for implementing real-time monitoring capabilities in industrial control systems. The algorithm takes a finite state machine representing the expected system behavior and produces an ECC with comprehensive error handling capabilities.

Algorithm \ref{FSMtoMonitor} provides the formal description of this transformation process, which involves creating both conforming and non-conforming ECC transitions. The algorithm operates in two main phases: first, it creates conforming ECC transitions that represent valid system behavior, generating "OK" outputs when expected events occur. Second, it creates non-conforming ECC transitions that capture all possible error scenarios, generating "ERROR" outputs with detailed information about the specific event and state where the error occurred.

\begin{algorithm}[T]
    \caption{FSM to Monitor ECC }
    \label{FSMtoMonitor}
\SetKwInOut{Input}{Input}
\SetKwInOut{Output}{Output}

\Input{ FSM = \{ Q: $\{q_0, q_1, \ldots, q_{n-1}\}$  
is a set of \\ states, where  $q_0$ is an initial state;\\
CS:  $\{X_1, X_2, \ldots, X_{p+r}\}$ 
is a set of input \\ symbols, where $X_i \in C \cup S$, where \\ C= $\{c_1,c_2,..., c_p\}$ is a set of control \\ signals, and S= $\{s_1,s_2,..., s_r\}$ is a set of \\ sensor signals; \\
$\delta \subseteq Q \times CS \times Q$ \} is a transition relation;
}
\Output{ ECC = \{ QC: $\{q_0, q_1, \ldots, q_{n-1}, {SE_n, SE_{n+1}} \dots {SE_{n+p+r-1}}\} \supset Q $ 
is a set of ECC states; \\
CS is a set of input events (see above);\\
Out: $\{\text{OK}, \text{ERROR}\}$  is a set of output events;\\
Alg= $\{a_0, a_1, \ldots, a_{n+p+r-1} \} $ is a set of \\ algorithms; \\ 
$\phi: \{ \text{Q} \times \text{CS} \rightarrow \text{QC} \times \text{Alg} \times \text{Out}$ \}  is an \\ ECC  transition function\} 

}
StateID=$q_0$  \\
{/* Creating the conforming (right) ECC transitions */}

\ForEach{$(q_i, X_k, q_j) \in \delta$ }{ 
Create ECC Transitions:
$(q_i, X_k) \rightarrow ( q_j, a_j, \text{OK})$
$a_j.StateID=j$  
} 

{/* Creating the non-conforming (erroneous) ECC transitions */}

\ForEach{$ q_w \in Q$ }{ 
\ForEach{$ X_e \in CS$ - $T(q_w)$}{ 
Create ECC Transitions:
$(q_w, X_e) \rightarrow (SE_e, a_e, \text{ERROR})$
$a_e.EventID=e$  
} 
} 
\Return{ECC}\; 

\textbf{ *Note* : Used variables and sets} \\

 StateID is an  output variable of the FB Monitor that stores the identifier of the current ECC  state;  \\

$a_j$.StateID is an occurrence of the  StateID  variable in the $a_j$ algorithm  associated with an ECC state;   \\

EventID is an  output variable of the FB Monitor that stores the identifier of the erroneous (not expected) input event; \\

$a_e$.EventID is an  occurrence of the EventID variable in the $a_e$ algorithm associated with an ECC state; \\

$T(q_w) \subseteq CS$ is a set of signals labelling the transitions outgoing from the ECC state $q_w \in Q$.

\end{algorithm}

The monitor implementation provides several key advantages for industrial control systems. First, it offers real-time error detection with immediate feedback on system state and error conditions. Second, it maintains detailed error information including both the state where the error occurred (StateID) and the specific event that caused the error (EventID). Third, it ensures complete coverage of all possible input events from each state, guaranteeing that any deviation from expected behavior will be detected. This comprehensive monitoring capability is essential for safety-critical applications where early detection of system malfunctions can prevent equipment damage and ensure operational safety.

The monitor continues operation until the first error is detected, after which it becomes non-responsive to prevent further processing. This behavior ensures that the system can be immediately stopped when a deviation is detected, preventing cascading failures and maintaining system integrity. The detailed error information provided by the monitor enables rapid diagnosis and correction of system issues, significantly reducing downtime and improving system reliability.

\section{Interactive Learning for Automatic Controller Generation}

The development of control logic for industrial automation systems traditionally requires significant domain expertise and manual programming effort. The interactive learning approach addresses this challenge by leveraging process mining techniques to automatically generate controllers from recorded behavioral traces, significantly reducing development time and improving system reliability.

\subsection{Simulation-Based Event Recording}

The foundation of interactive learning lies in the systematic recording of system behavior through simulation models. This approach uses 3D simulation environments, such as Visual Components, to create virtual representations of industrial systems that can be manipulated and observed without the risks and costs associated with physical experimentation.

The simulation environment provides a controlled setting where actuator signals can be manually triggered in appropriate sequences to generate desired process scenarios. Each interaction with the simulation model produces events that are recorded in chronological order, creating comprehensive behavioral traces that capture the complete system dynamics. This approach enables the generation of diverse process scenarios that might be difficult or dangerous to create in physical systems.



The recorded event logs capture the complete interaction between the simulation model and the operator, including sensor readings, actuator commands, and system state information. This comprehensive data collection ensures that the generated control logic accurately reflects the intended system behavior and can handle the full range of operational scenarios.

The simulation-based approach offers several advantages over traditional controller development methods. First, it eliminates the need for extensive domain knowledge in control system design, as the controller logic emerges directly from the recorded behavior. Second, it reduces development time by automating the transformation from behavioral requirements to executable control code. Third, it improves system reliability by ensuring that the controller logic is consistent with the observed system behavior.

\subsection{Process Discovery and Petri Net Generation}

The transformation from recorded event logs to executable control logic involves several systematic steps, beginning with process discovery to extract formal models from the behavioral traces. The alpha algorithm serves as the primary process discovery method, generating Petri nets that represent the system's behavioral patterns in a mathematically rigorous format.

The generated Petri nets capture the causal relationships between events, representing the system's state transitions and control flow. These models provide a formal foundation for understanding system behavior and serve as the basis for controller generation. The Petri net representation is particularly valuable because it can handle concurrent activities, sequential dependencies, and complex synchronization requirements that are common in industrial automation systems.


The Petri net generation process includes several important considerations. First, the addition of supplementary transitions, such as "Repeat" transitions, enables cyclic operation that is typical in manufacturing processes. Second, the stepwise simulation of the Petri net validates the model's correctness and ensures that it accurately represents the intended system behavior. Third, the conversion to reachability graphs provides a finite state machine representation that can be directly implemented in control systems.

The reachability graph represents all possible states and transitions of the system, providing a complete behavioral model that can be analyzed for properties such as deadlock freedom, liveness, and reachability. This analysis ensures that the generated controller will behave correctly under all operational conditions and can handle unexpected situations gracefully.

\subsection{Transformation to IEC 61499 Function Blocks}

The final step in the interactive learning process involves transforming the formal process models into executable IEC 61499 function blocks. This transformation requires careful mapping between the mathematical representation of the Petri net and the practical implementation requirements of industrial control systems.

The transformation process begins with the conversion of the reachability graph to a deterministic finite state machine (FSM). This conversion involves handling non-deterministic transitions, often represented as spontaneous or lambda transitions in the original Petri net. The determinization process ensures that the resulting FSM has unique transitions for each input combination, making it suitable for implementation in control systems.


The transformation from FSM to IEC 61499 function blocks involves several key steps. First, the FSM states are mapped to ECC states in the function block. Second, the FSM transitions are converted to ECC transitions with appropriate conditions and actions. Third, the input and output events are defined based on the sensor and actuator signals identified in the original event log.

The resulting function block interface includes input events for all sensor signals and output events for all actuator signals. The ECC implements the control logic by defining state transitions that respond to sensor inputs and generate appropriate actuator outputs. This implementation ensures that the generated controller behaves exactly as recorded during the interactive learning process.


The interactive learning approach provides several significant advantages for controller development. First, it reduces development time by automating the transformation from behavioral requirements to executable code. Second, it improves system reliability by ensuring that the controller logic is consistent with observed system behavior. Third, it enables rapid prototyping and testing of different control strategies without extensive programming effort.

\section{Automatic Plant Model Generation and Real-Time Monitoring}

The extension of process mining techniques to automatic plant model generation represents a significant advancement in formal verification capabilities for industrial control systems. This approach addresses the critical challenge of creating accurate plant models for closed-loop system verification, enabling comprehensive analysis of system behavior and real-time monitoring of operational compliance.

\subsection{Reference Model of Control/Sensor Events Sequencing}

The foundation of automatic plant model generation lies in the Reference Model of Control/Sensor Events Sequencing (RMCSES), a formal model that represents the complete behavior of a closed-loop industrial automation system. This model is derived from process mining of event logs recorded during error-free system operation over extended periods.

The RMCSES provides a condensed representation of vast event logs, encompassing all possible signals from all system components. Conceptually, if the event log represents a collection of sentences describing system behavior, then RMCSES functions as a sentence generator for this formal language. This representation enables the model to be readily implemented in software or hardware, making it suitable for real-time applications.

The RMCSES interface focuses exclusively on control signals originating from the controller and informative signals originating from sensors, representing the interface between the controller and the plant. Internal signals circulating within the control system and plant are not considered, though control signals from external sources such as operators can be included. This focus on the controller-plant interface makes the model particularly suitable for closed-loop system analysis and verification.

The model assumes that each component in the system operates in a cyclical, meaningful, and locally complete manner. Components are considered to have meaningful behavior when they have specific goals and actively work toward achieving them. Each component follows specific scenarios that begin with initialization and end with termination, with multi-functional devices potentially executing different scenarios depending on their current task or operation.

\subsection{Petri Net Decomposition and State Machine Generation}

The transformation from event logs to plant models involves several sophisticated steps, beginning with Petri net construction and proceeding through decomposition and state machine generation. This process addresses the complexity of industrial systems by breaking down complex behaviors into manageable components while preserving essential behavioral characteristics.

The Petri net decomposition process utilizes reachability graph methods to identify subsets of states and transitions, facilitating the partitioning of complex Petri nets into manageable modules. This approach provides a structured and systematic method for exploring all reachable states and transitions within the system, ensuring completeness in capturing all potential behaviors.

The reachability graph method, while comprehensive, can suffer from state space explosion in complex systems. To address this challenge, several strategies are employed: state space reduction using symbolic representations such as Binary Decision Diagrams (BDDs), abstraction and aggregation methods that focus on essential behavioral properties, partial order reduction techniques that avoid redundant interleavings, and on-the-fly exploration that generates states as needed.

The composition of State Machine Components (SMCs) into a cohesive FSM requires careful integration to ensure synchronization and coordination. Parallel composition and synchronization techniques are preferred for IEC 61499 systems, as they align with the distributed and modular architecture of these systems. This approach enables concurrent execution of SMCs while maintaining proper coordination and communication between distributed components.

\subsection{Plant Model Transformation and Implementation}

The transformation from finite state machines to plant model ECCs represents a critical step in the automatic plant model generation process. This transformation requires careful handling of the different types of signals and transitions that characterize plant behavior in industrial control systems.

\begin{figure*}[!t]
	\centering
	\includegraphics[width=0.9\textwidth]{chapters/images/chapter4/VerificationApp1.png}
	\caption{Application for verification. The structure shows a closed-loop system formed by connecting a plant model obtained from the RMCSES with either a new or existing controller. The plant model interface comprises control signals and non-deterministic transitions (NDT) as input signals, while sensor signals are designated as output signals. This configuration enables formal verification using tools such as NuSMV with CTL/LTL specifications.}
	\label{verificationApp}
\end{figure*}

The presented application for verification follows a structure illustrated in Figure \ref{verificationApp}, where a closed-loop system is formed by connecting a plant model obtained from the RMCSES with either a new or an existing controller. An existing controller is utilized to construct the RMCSES, whereas a new controller is connected with RMCSES for the purpose of verification. The verification process involves conformance checking to ensure that the newly developed controller operates in accordance with the previous controller. The plant model interface comprises control signals and non-deterministic transitions (NDT) as input signals, while sensor signals are designated as output signals. This configuration enables formal verification using tools such as NuSMV with CTL/LTL specifications.

The transformation from RMCSES to plant model involves specific rules for handling sensor and control signals. Transitions triggered by sensor signals are replaced with non-deterministic transitions (NDT) that can fire at arbitrary time intervals, with the output of the next state serving as a sensor event signal. Control signal transitions remain unchanged, as the plant should respond directly to control inputs. This transformation ensures that the plant model accurately represents the uncontrolled behavior of the physical system while maintaining compatibility with formal verification tools.

The transformation process follows a systematic algorithmic approach that handles the complexity of plant behavior modeling. This transformation is essential for creating accurate plant models that can be used for formal verification and closed-loop system analysis. The algorithm takes a finite state machine representing the overall system behavior and produces a plant model ECC that accurately represents the uncontrolled plant behavior.

Algorithm \ref{FSMtoPlant} provides the formal description of this transformation process, which involves processing control signals, sensor signals, and handling converging sensor signals through the introduction of intermediate states. The algorithm operates in three main phases: first, it processes control signal transitions, maintaining them unchanged as the plant should respond directly to control inputs. Second, it processes sensor signal transitions that are not converged, replacing them with non-deterministic transitions (NDT) that can fire at arbitrary time intervals. Third, it handles converging sensor signals by introducing intermediate states that produce appropriate sensor signals before converging to the target state.

\begin{algorithm}[t!]
    \caption{FSM to Plant model ECC }
    \label{FSMtoPlant}
\SetKwInOut{Input}{Input}
\SetKwInOut{Output}{Output}

\Input{ FSM = \{ Q: $\{q_0, q_1, \ldots, q_{n-1}\}$ \\
is a set of states, where $q_0$ is an initial state; \\
CS =  $ C \cup S$ is a set of input signals, where \\  C = $\{c_1, c_2, \ldots, c_p \}$ is a set of control \\  signals, \\  S = $\{s_1, s_2 \ldots, s_r \}$ is a set of sensorsignals; \\
$\delta \subseteq Q \times CS \times Q$ is a transition  relation; \}
}
\Output{ ECC = \{ $QC= Q \cup D$ is a set of ECC states, where D is a set of additional ECC states created dynamically;
$Cond= TC \cup \{1\}$ is a set of ECC transition  conditions, where $TC= C \cup \{NDT\}$ is a set  of input events;
S is a set of output events (see above);\\
$ \phi: QC \times Cond \rightarrow QC \cup QC\times S $ \\ is an ECC  transition function
\}
}

/* Processing the FSM transition labelled by control signals */

\ForEach{$(q_i, \sigma, q_j) \in \delta$}{
    \If{$\sigma \in $ C}{
        Create ECC Transition:
        $(q_i,  \sigma ) \rightarrow (q_j)$;\
    }
}
/* Processing the FSM transition labelled by sensor signals which are not converged */

\ForEach{$(q_i, \sigma, q_j) \in \delta$}{
    \If{$\sigma \in  S $ \text{AND}  $q_i \notin QJ$}{
        Create ECC Transition:
        $(q_i,  NDT ) \rightarrow (q_j, \sigma)$;\
    }
}

/* Processing the FSM transition labelled by converging sensor signals */

\ForEach{$(q_w \in QJ) $}{
\ForEach{$(q_i, \sigma, q_w) \in \delta'}{
    \If{$\sigma \in  S $}{
        Create ECC State $d \in D$ \\
        Create ECC Transition:
    $(q_i,  NDT ) \rightarrow (d, \sigma)$ ; \ \\
        $(d,  1 ) \rightarrow (q_w)$ ;\
    }
}

}
    \Return{ECC}\;

\textbf{ *Note* : Used variables and sets} \\
\BlankLine

$QJ= { \{ q_j \in Q , | \{(q_i, s_k, q_j) \in \delta'\}|>1} \} $, where $\delta' = \delta \cap  Q \times S \times Q$ is a set of ECC states which have converging sensor signals in the FSM

\end{algorithm}

The plant model implementation addresses several important challenges in industrial control system modeling. First, it handles diverging sensor signals by creating separate NDT transitions for each possible sensor output, ensuring that all possible plant behaviors are captured. Second, it manages converging sensor signals by introducing intermediate states that produce appropriate sensor signals before converging to the target state, maintaining the non-deterministic nature of plant behavior. Third, it maintains the non-deterministic nature of plant behavior while providing a deterministic interface for integration with control systems, enabling formal verification using tools such as NuSMV and CTL/LTL specifications.

This algorithmic approach ensures that the generated plant model accurately represents the uncontrolled behavior of the physical system while maintaining compatibility with formal verification tools. The plant model can be used for formal verification using tools such as NuSMV and CTL/LTL specifications, enabling comprehensive analysis of system properties including safety, liveness, and reachability. This verification capability is particularly valuable for safety-critical applications where formal guarantees of system behavior are required.

\subsection{Monitor Implementation for Real-Time Conformance Checking}

The automatic generation of monitors represents a critical application of the RMCSES, enabling real-time detection of deviations from expected system behavior. These monitors are implemented as IEC 61499 function blocks that can be deployed in operational systems to provide continuous monitoring and early warning of potential issues.

The monitor implementation involves transforming the RMCSES into a function block that receives input signals from both controller and sensor components. The monitor's ECC is generated using transformation rules that map FSM states and transitions to appropriate ECC states and transitions. When the monitor detects a valid transition, it generates an "OK" output indicating successful operation. If an unexpected event occurs, the monitor transitions to an error state and produces an "ERROR" event with details about the event that caused the error.

The monitor's error detection capabilities are comprehensive, covering all possible input events from each state. For each state in the FSM, the monitor defines transitions to error states for all input events that are not specified in the reference model. This complete coverage ensures that any deviation from expected behavior is detected immediately, providing robust protection against system malfunctions and unauthorized operations.

The monitor implementation includes several important features. First, it provides detailed error information including the state where the error occurred and the specific event that caused the error. Second, it continues operation until the first error is detected, after which it becomes non-responsive to prevent further processing. Third, it can be easily integrated with existing control systems without requiring significant modifications to the operational infrastructure.

\subsection{Plant Model Generation for Formal Verification}

The automatic generation of plant models extends the capabilities of process mining to formal verification applications. These plant models represent the uncontrolled behavior of the physical system and can be integrated with control models to create complete closed-loop system representations suitable for formal analysis.

The plant model generation process involves extracting the uncontrolled plant behavior from the overall system model represented by the RMCSES. This extraction requires careful analysis to distinguish between control-driven and sensor-driven transitions, as the plant model should only respond to control signals and generate sensor signals based on its internal state.

The transformation from FSM to plant model involves specific rules for handling sensor and control signals. Transitions triggered by sensor signals are replaced with non-deterministic transitions (NDT) that can fire at arbitrary time intervals, with the output of the next state serving as a sensor event signal. Control signal transitions remain unchanged, as the plant should respond directly to control inputs.

The plant model implementation addresses several important challenges. First, it handles diverging sensor signals by creating separate NDT transitions for each possible sensor output. Second, it manages converging sensor signals by introducing intermediate states that produce appropriate sensor signals before converging to the target state. Third, it maintains the non-deterministic nature of plant behavior while providing a deterministic interface for integration with control systems.

The generated plant model can be used for formal verification using tools such as NuSMV and CTL/LTL specifications. The model can be converted to SMV format using the fb2smv tool, enabling comprehensive analysis of system properties including safety, liveness, and reachability. This verification capability is particularly valuable for safety-critical applications where formal guarantees of system behavior are required.

\section{Integration and Applications}

The three approaches presented in this chapter represent complementary aspects of process mining applications in industrial control systems, each addressing different challenges in system understanding, control generation, and verification. Their integration provides a comprehensive framework for addressing the complex requirements of modern industrial automation.

\subsection{Comprehensive System Analysis Framework}

The integration of process model extraction, interactive learning, and automatic plant model generation creates a comprehensive framework for industrial control system analysis and development. This framework enables end-to-end system understanding, from initial behavior recording through formal verification and operational monitoring.

The framework begins with process model extraction, which provides fundamental understanding of system behavior through analysis of recorded event logs. This understanding serves as the foundation for both controller generation and plant model development, ensuring consistency across all system representations.

Interactive learning builds upon this foundation by enabling automatic controller generation from recorded behavioral traces. This approach reduces development time and improves system reliability by ensuring that control logic is consistent with observed system behavior. The generated controllers can be immediately deployed or used as starting points for further refinement.

Automatic plant model generation extends the framework to formal verification applications, enabling comprehensive analysis of closed-loop system behavior. The generated plant models can be used with existing or newly developed controllers to verify system properties and ensure compliance with safety and performance requirements.

\subsection{Industrial Applications and Case Studies}

The presented approaches have been validated through several industrial case studies, demonstrating their effectiveness in real-world applications. The gripper and conveyor system case study illustrates the application of process model extraction for system understanding and anomaly detection. This case study shows how process mining can be used to analyze complex manufacturing processes and identify potential improvements in system operation.

The interactive learning case study demonstrates the application of simulation-based controller generation for a production system with multiple components including conveyors, grippers, and autonomous guided vehicles. This case study shows how the approach can handle complex, multi-component systems and generate controllers that accurately reflect desired system behavior.

The pneumatic cylinder case study illustrates the application of automatic plant model generation and real-time monitoring. This case study shows how the approach can be used for formal verification of safety-critical systems and real-time monitoring of operational compliance.

\begin{figure*}[!t]
	\centering
	\includegraphics[width=1\textwidth]{chapters/images/chapter4/CLMC.PNG}
	\caption{Closed-loop system of plant and controller model with the monitor. The figure shows the integration of a plant model, controller, and monitor in a closed-loop configuration. The monitor function block receives input signals from both controller and sensor components, generating OK/ERROR outputs based on conformance to the reference model. The ECC diagrams illustrate the internal logic of both the plant model and monitor function blocks.}
	\label{CL_plant_controller_monitor}
\end{figure*}

Figure \ref{CL_plant_controller_monitor} illustrates a practical implementation of the closed-loop system integration with monitor for a pneumatic cylinder case study. The system consists of a plant model, controller, and monitor connected in a closed-loop configuration. The monitor function block receives input signals from both controller and sensor components, generating OK/ERROR outputs based on conformance to the reference model. The Execution Control Chart (ECC) diagrams show the internal logic of both the plant model and monitor function blocks, demonstrating how the transformation from finite state machines to IEC 61499 function blocks is implemented in practice.

In this implementation, the monitor generates an 'OK' event with the corresponding StateID when the system process executes in the correct order. In the case of any deviation from the expected process behavior, an 'ERROR' signal is emitted along with the 'EventID' indicating the specific event responsible for the deviation. The monitor continues operation until the first error is detected, after which it becomes non-responsive to prevent further processing. This real-time monitoring capability is essential for ensuring system safety and reliability in industrial applications.

These case studies demonstrate the versatility of the presented approaches and their applicability to a wide range of industrial applications. The approaches can be adapted to different system types, from simple single-component systems to complex multi-component manufacturing lines, providing flexible solutions for various industrial automation challenges.

\subsection{Future Research Directions}

The integration of process mining with industrial control systems opens several promising research directions that could further enhance the capabilities and applicability of these approaches.

First, the development of more sophisticated process discovery algorithms could improve the quality and accuracy of generated models. Current algorithms may struggle with complex, highly concurrent systems or systems with significant noise in their event logs. Advanced algorithms that can handle these challenges would expand the applicability of the approaches to a broader range of industrial systems.

Second, the integration of machine learning techniques with process mining could enable adaptive models that improve over time based on new observations. This capability would be particularly valuable for systems that evolve or change their behavior over time, enabling continuous improvement of system understanding and control.

Third, the development of more efficient algorithms for handling large-scale event logs could enable the application of these approaches to complex, multi-facility manufacturing systems. Current approaches may face computational challenges when dealing with very large event logs or complex system architectures.

Fourth, the integration of these approaches with emerging technologies such as digital twins and edge computing could enable real-time process mining and control generation. This integration would enable dynamic adaptation of control systems based on real-time analysis of system behavior, providing more responsive and adaptive automation solutions.

The research presented in this chapter demonstrates the significant potential of process mining techniques for addressing critical challenges in industrial control systems. The three complementary approaches provide comprehensive solutions for system understanding, control generation, and verification, while the integration of these approaches creates a powerful framework for industrial automation development and analysis.

As industrial systems become increasingly complex and interconnected, the need for automated methods for system understanding and control generation will continue to grow. The approaches presented in this chapter provide a solid foundation for addressing these challenges and enable the development of more intelligent, adaptive, and reliable industrial automation systems.

The future of industrial automation lies in the integration of data-driven approaches with formal methods, creating systems that can learn from their environment, adapt to changing requirements, and provide formal guarantees of their behavior. The process mining approaches presented in this chapter represent important steps toward this vision, providing practical methods for realizing intelligent industrial automation systems that can meet the demands of modern manufacturing.


\makechapter{Nonsense chapter}{Nonsense chapter}{Nonsense chapter, here only to verify that some issues in previous versions are really resolved!}
%\section{Process Mining Applications in Industrial Control Systems}

The integration of process mining techniques into industrial control systems represents a significant advancement in the field of cyber-physical systems, offering novel approaches to system modeling, verification, and control generation. Process mining, traditionally applied in business process management, has evolved to address the unique challenges of industrial automation, providing data-driven methods for understanding complex system behaviors and generating formal models from recorded event traces.

This chapter presents three complementary approaches that demonstrate the versatility and effectiveness of process mining in industrial control systems. The first approach focuses on process model extraction and conformance checking for anomaly detection and system monitoring. The second approach introduces an interactive learning methodology for automatic controller generation through simulation-based event recording. The third approach extends these capabilities to automatic plant model generation and real-time monitoring for formal verification.

These methodologies collectively address critical challenges in modern industrial automation: the need for automated system understanding, the complexity of controller development, and the requirement for formal verification in safety-critical applications. By leveraging process mining algorithms and the IEC 61499 standard, these approaches provide systematic methods for transforming recorded behavioral traces into formal models that can be used for system analysis, control generation, and verification.

\begin{figure*}[!t]
	\centering
	\includegraphics[width=0.9\textwidth]{chapters/images/chapter4/workflow1.png}
	\caption{Workflow and use cases for process mining applications in industrial control systems. The workflow illustrates the transformation from recorded event logs to various applications including monitor generation (D1), plant model generation (D2), control model generation (D3), and control program generation (D4).}
	\label{workflowUsecase}
\end{figure*}

Figure \ref{workflowUsecase} illustrates the comprehensive workflow and potential use cases enabled by process mining in industrial control systems. Initially, signals from distributed control systems are recorded to construct behavioral models using process mining techniques. These models then serve as the basis for generating monitors for real-time error detection, plant models for formal verification, control models for automatic controller generation, and control programs for system implementation. This integrated approach provides a systematic framework for addressing the complex requirements of modern industrial automation systems.

\section{Process Model Extraction and Conformance Analysis}

The application of process mining techniques in industrial control systems begins with the fundamental challenge of understanding system behavior through recorded event logs. This approach leverages the inherent data-rich nature of modern automation systems to extract meaningful process models that can be used for system analysis, anomaly detection, and performance optimization.

\subsection{Process Mining Fundamentals in Industrial Contexts}

Process mining in industrial control systems differs fundamentally from its business process applications by focusing on the temporal and causal relationships between sensor and actuator signals rather than human activities. The core principle involves analyzing event logs that capture the sequence of control and sensor events occurring during system operation, then applying discovery algorithms to extract formal process models that represent the system's behavioral patterns.

The process mining workflow in industrial contexts typically involves three main phases: process discovery, conformance checking, and process enhancement. Process discovery algorithms analyze event logs to generate process models in various formalisms, most commonly Petri nets, which provide a mathematical foundation for representing concurrent and sequential behaviors. Conformance checking compares observed behavior against expected models to identify deviations, while process enhancement focuses on improving existing models based on new observations.



The selection of appropriate process discovery algorithms is crucial for effective model extraction. Alpha algorithm, as a representative of abstraction-based methods, generates models by analyzing ordering relations between events in the log. This algorithm creates dependency graphs based on the sequence of events, making it suitable for systems with well-defined, deterministic behaviors. However, its sensitivity to noise in event logs can lead to overly complex models when dealing with systems that exhibit significant variability.

Heuristic-based algorithms, such as the fuzzy miner, offer an alternative approach that considers the frequency of event occurrences. These algorithms generate models based on the relative importance of activities and the strength of their relationships, making them more robust to noise and better suited for systems with complex, variable behaviors. The fuzzy miner, in particular, provides interactive representations that help understand system behavior in complex logs, though it may be more challenging to convert to other process modeling languages.

\begin{figure}[!t]
	\centering
	\includegraphics[width=0.4\textwidth]{chapters/images/chapter4/Methodology.png}
	\caption{Tool chain and data flow for generating FSMs from Event Logs and their implementation in the form of IEC 61499 FBs. The process involves preprocessing event logs, applying process discovery algorithms, generating Petri nets, decomposing into state machine components, and finally implementing as IEC 61499 function blocks.}
	\label{monitorFlowchart}
\end{figure}

The process of generating formal models from event logs involves several systematic steps, as illustrated in Figure \ref{monitorFlowchart}. The workflow begins with event log preprocessing, where raw data is cleaned and converted to appropriate formats such as eXtensible Event Stream (XES). Process discovery algorithms are then applied to extract behavioral patterns, typically represented as Petri nets. These Petri nets are subsequently decomposed into state machine components (SMCs) and composed into finite state machines (FSMs). Finally, the FSMs are transformed into IEC 61499 function blocks for implementation in industrial control systems. This systematic approach ensures the accurate representation of system behavior while maintaining compatibility with industrial automation standards.

\subsection{Event Log Structure and Preprocessing}

The quality and structure of event logs significantly influence the effectiveness of process mining applications. Industrial control systems generate event logs that typically include case identifiers, timestamps, component information, signal names, and signal values. The case identifier represents unique process executions, often corresponding to complete operational cycles in cyclic manufacturing processes.

Event log preprocessing is essential for ensuring model quality and accuracy. This includes data cleaning to remove irrelevant events, format conversion to ensure compatibility with process mining tools, and attribute selection to focus on the most relevant information for the analysis. The conversion from CSV format to eXtensible Event Stream (XES) format is particularly important, as most process discovery algorithms require XES input.


The preprocessing phase also involves mapping standard XES attributes to the industrial context. Case columns typically correspond to process execution identifiers, while event columns combine component, signal, and value information to create meaningful activity descriptions. This mapping ensures that the process mining algorithms can correctly interpret the industrial data and generate appropriate models.

\subsection{Conformance Checking and Anomaly Detection}

Conformance checking represents one of the most valuable applications of process mining in industrial control systems, providing systematic methods for detecting deviations from expected behavior. This capability is particularly important for anomaly detection, cyber-attack identification, and quality assurance in safety-critical applications.

The conformance checking process involves comparing observed event logs against reference process models to identify discrepancies. Several algorithms support this analysis, including causal footprint checking, token-based replay, and alignment-based methods. Causal footprint checking compares dependency matrices between the event log and reference model, providing a quick assessment of model fitness. However, this method does not consider event frequencies and may miss subtle deviations.

Token-based replay offers a more detailed analysis by simulating the execution of event traces on the reference model. This approach tracks missing and remaining tokens after each transition, providing quantitative measures of conformance. While effective, token replay has limitations when dealing with non-uniquely labeled transitions and can suffer from token flooding in complex models.

Alignment-based methods provide the most sophisticated approach to conformance checking, offering optimal alignment between observed and modeled behavior using user-defined cost functions. These methods are independent of process model notation and can handle complex scenarios that challenge other approaches. The alignment process identifies the optimal sequence of moves that transforms the observed trace into one that conforms to the model, providing detailed insights into the nature and location of deviations.

The application of conformance checking in industrial control systems extends beyond simple deviation detection to include performance analysis and system optimization. By analyzing conformance metrics over time, operators can identify trends in system behavior, detect gradual degradation, and optimize operational parameters. This capability is particularly valuable in predictive maintenance applications, where early detection of behavioral changes can prevent equipment failures and reduce downtime.

\subsection{Monitor Implementation for Real-Time Conformance Checking}

The implementation of monitors for real-time conformance checking represents a critical application of process mining in industrial control systems. These monitors are implemented as IEC 61499 function blocks that can be deployed in operational systems to provide continuous monitoring and early warning of potential issues.

\begin{figure*}[!t]
	\centering
	\includegraphics[width=1\textwidth]{chapters/images/chapter4/ConformaceCheckingApp.jpg}
	\caption{Application for conformance checking. The structure shows a closed-loop system consisting of a plant and controller, with the RMCSES (Reference Model of Control/Sensor Events Sequencing) connected for conformance checking. The monitor function block receives input signals from both controller and sensor components and generates OK/ERROR outputs based on conformance to the reference model.}
	\label{conformanceCheckingApp}
\end{figure*}

The structure of the application for conformance checking is depicted in Figure \ref{conformanceCheckingApp}. The application is based on a closed-loop system that includes a plant and a controller. The RMCSES obtained from the event log is connected to the closed-loop system for the purpose of conformance checking. The RMCSES is implemented as a finite state machine that uses input data to drive its logic. The monitor function block is capable of detecting errors and verifying the correct sequence of steps in the process flow by monitoring events from sensors and actuators in real time. When the monitor detects a valid transition, it generates an "OK" output indicating successful operation. If an unexpected event occurs, it transitions to an error state and produces an "ERROR" event with details about the event that caused the error.

The transformation from finite state machines to monitor ECCs follows a systematic algorithmic approach that ensures complete coverage of all possible system states and transitions. This transformation is essential for implementing real-time monitoring capabilities in industrial control systems. The algorithm takes a finite state machine representing the expected system behavior and produces an ECC with comprehensive error handling capabilities.

Algorithm \ref{FSMtoMonitor} provides the formal description of this transformation process, which involves creating both conforming and non-conforming ECC transitions. The algorithm operates in two main phases: first, it creates conforming ECC transitions that represent valid system behavior, generating "OK" outputs when expected events occur. Second, it creates non-conforming ECC transitions that capture all possible error scenarios, generating "ERROR" outputs with detailed information about the specific event and state where the error occurred.

\begin{algorithm}[T]
    \caption{FSM to Monitor ECC }
    \label{FSMtoMonitor}
\SetKwInOut{Input}{Input}
\SetKwInOut{Output}{Output}

\Input{ FSM = \{ Q: $\{q_0, q_1, \ldots, q_{n-1}\}$  
is a set of \\ states, where  $q_0$ is an initial state;\\
CS:  $\{X_1, X_2, \ldots, X_{p+r}\}$ 
is a set of input \\ symbols, where $X_i \in C \cup S$, where \\ C= $\{c_1,c_2,..., c_p\}$ is a set of control \\ signals, and S= $\{s_1,s_2,..., s_r\}$ is a set of \\ sensor signals; \\
$\delta \subseteq Q \times CS \times Q$ \} is a transition relation;
}
\Output{ ECC = \{ QC: $\{q_0, q_1, \ldots, q_{n-1}, {SE_n, SE_{n+1}} \dots {SE_{n+p+r-1}}\} \supset Q $ 
is a set of ECC states; \\
CS is a set of input events (see above);\\
Out: $\{\text{OK}, \text{ERROR}\}$  is a set of output events;\\
Alg= $\{a_0, a_1, \ldots, a_{n+p+r-1} \} $ is a set of \\ algorithms; \\ 
$\phi: \{ \text{Q} \times \text{CS} \rightarrow \text{QC} \times \text{Alg} \times \text{Out}$ \}  is an \\ ECC  transition function\} 

}
StateID=$q_0$  \\
{/* Creating the conforming (right) ECC transitions */}

\ForEach{$(q_i, X_k, q_j) \in \delta$ }{ 
Create ECC Transitions:
$(q_i, X_k) \rightarrow ( q_j, a_j, \text{OK})$
$a_j.StateID=j$  
} 

{/* Creating the non-conforming (erroneous) ECC transitions */}

\ForEach{$ q_w \in Q$ }{ 
\ForEach{$ X_e \in CS$ - $T(q_w)$}{ 
Create ECC Transitions:
$(q_w, X_e) \rightarrow (SE_e, a_e, \text{ERROR})$
$a_e.EventID=e$  
} 
} 
\Return{ECC}\; 

\textbf{ *Note* : Used variables and sets} \\

 StateID is an  output variable of the FB Monitor that stores the identifier of the current ECC  state;  \\

$a_j$.StateID is an occurrence of the  StateID  variable in the $a_j$ algorithm  associated with an ECC state;   \\

EventID is an  output variable of the FB Monitor that stores the identifier of the erroneous (not expected) input event; \\

$a_e$.EventID is an  occurrence of the EventID variable in the $a_e$ algorithm associated with an ECC state; \\

$T(q_w) \subseteq CS$ is a set of signals labelling the transitions outgoing from the ECC state $q_w \in Q$.

\end{algorithm}

The monitor implementation provides several key advantages for industrial control systems. First, it offers real-time error detection with immediate feedback on system state and error conditions. Second, it maintains detailed error information including both the state where the error occurred (StateID) and the specific event that caused the error (EventID). Third, it ensures complete coverage of all possible input events from each state, guaranteeing that any deviation from expected behavior will be detected. This comprehensive monitoring capability is essential for safety-critical applications where early detection of system malfunctions can prevent equipment damage and ensure operational safety.

The monitor continues operation until the first error is detected, after which it becomes non-responsive to prevent further processing. This behavior ensures that the system can be immediately stopped when a deviation is detected, preventing cascading failures and maintaining system integrity. The detailed error information provided by the monitor enables rapid diagnosis and correction of system issues, significantly reducing downtime and improving system reliability.

\section{Interactive Learning for Automatic Controller Generation}

The development of control logic for industrial automation systems traditionally requires significant domain expertise and manual programming effort. The interactive learning approach addresses this challenge by leveraging process mining techniques to automatically generate controllers from recorded behavioral traces, significantly reducing development time and improving system reliability.

\subsection{Simulation-Based Event Recording}

The foundation of interactive learning lies in the systematic recording of system behavior through simulation models. This approach uses 3D simulation environments, such as Visual Components, to create virtual representations of industrial systems that can be manipulated and observed without the risks and costs associated with physical experimentation.

The simulation environment provides a controlled setting where actuator signals can be manually triggered in appropriate sequences to generate desired process scenarios. Each interaction with the simulation model produces events that are recorded in chronological order, creating comprehensive behavioral traces that capture the complete system dynamics. This approach enables the generation of diverse process scenarios that might be difficult or dangerous to create in physical systems.



The recorded event logs capture the complete interaction between the simulation model and the operator, including sensor readings, actuator commands, and system state information. This comprehensive data collection ensures that the generated control logic accurately reflects the intended system behavior and can handle the full range of operational scenarios.

The simulation-based approach offers several advantages over traditional controller development methods. First, it eliminates the need for extensive domain knowledge in control system design, as the controller logic emerges directly from the recorded behavior. Second, it reduces development time by automating the transformation from behavioral requirements to executable control code. Third, it improves system reliability by ensuring that the controller logic is consistent with the observed system behavior.

\subsection{Process Discovery and Petri Net Generation}

The transformation from recorded event logs to executable control logic involves several systematic steps, beginning with process discovery to extract formal models from the behavioral traces. The alpha algorithm serves as the primary process discovery method, generating Petri nets that represent the system's behavioral patterns in a mathematically rigorous format.

The generated Petri nets capture the causal relationships between events, representing the system's state transitions and control flow. These models provide a formal foundation for understanding system behavior and serve as the basis for controller generation. The Petri net representation is particularly valuable because it can handle concurrent activities, sequential dependencies, and complex synchronization requirements that are common in industrial automation systems.


The Petri net generation process includes several important considerations. First, the addition of supplementary transitions, such as "Repeat" transitions, enables cyclic operation that is typical in manufacturing processes. Second, the stepwise simulation of the Petri net validates the model's correctness and ensures that it accurately represents the intended system behavior. Third, the conversion to reachability graphs provides a finite state machine representation that can be directly implemented in control systems.

The reachability graph represents all possible states and transitions of the system, providing a complete behavioral model that can be analyzed for properties such as deadlock freedom, liveness, and reachability. This analysis ensures that the generated controller will behave correctly under all operational conditions and can handle unexpected situations gracefully.

\subsection{Transformation to IEC 61499 Function Blocks}

The final step in the interactive learning process involves transforming the formal process models into executable IEC 61499 function blocks. This transformation requires careful mapping between the mathematical representation of the Petri net and the practical implementation requirements of industrial control systems.

The transformation process begins with the conversion of the reachability graph to a deterministic finite state machine (FSM). This conversion involves handling non-deterministic transitions, often represented as spontaneous or lambda transitions in the original Petri net. The determinization process ensures that the resulting FSM has unique transitions for each input combination, making it suitable for implementation in control systems.


The transformation from FSM to IEC 61499 function blocks involves several key steps. First, the FSM states are mapped to ECC states in the function block. Second, the FSM transitions are converted to ECC transitions with appropriate conditions and actions. Third, the input and output events are defined based on the sensor and actuator signals identified in the original event log.

The resulting function block interface includes input events for all sensor signals and output events for all actuator signals. The ECC implements the control logic by defining state transitions that respond to sensor inputs and generate appropriate actuator outputs. This implementation ensures that the generated controller behaves exactly as recorded during the interactive learning process.


The interactive learning approach provides several significant advantages for controller development. First, it reduces development time by automating the transformation from behavioral requirements to executable code. Second, it improves system reliability by ensuring that the controller logic is consistent with observed system behavior. Third, it enables rapid prototyping and testing of different control strategies without extensive programming effort.

\section{Automatic Plant Model Generation and Real-Time Monitoring}

The extension of process mining techniques to automatic plant model generation represents a significant advancement in formal verification capabilities for industrial control systems. This approach addresses the critical challenge of creating accurate plant models for closed-loop system verification, enabling comprehensive analysis of system behavior and real-time monitoring of operational compliance.

\subsection{Reference Model of Control/Sensor Events Sequencing}

The foundation of automatic plant model generation lies in the Reference Model of Control/Sensor Events Sequencing (RMCSES), a formal model that represents the complete behavior of a closed-loop industrial automation system. This model is derived from process mining of event logs recorded during error-free system operation over extended periods.

The RMCSES provides a condensed representation of vast event logs, encompassing all possible signals from all system components. Conceptually, if the event log represents a collection of sentences describing system behavior, then RMCSES functions as a sentence generator for this formal language. This representation enables the model to be readily implemented in software or hardware, making it suitable for real-time applications.

The RMCSES interface focuses exclusively on control signals originating from the controller and informative signals originating from sensors, representing the interface between the controller and the plant. Internal signals circulating within the control system and plant are not considered, though control signals from external sources such as operators can be included. This focus on the controller-plant interface makes the model particularly suitable for closed-loop system analysis and verification.

The model assumes that each component in the system operates in a cyclical, meaningful, and locally complete manner. Components are considered to have meaningful behavior when they have specific goals and actively work toward achieving them. Each component follows specific scenarios that begin with initialization and end with termination, with multi-functional devices potentially executing different scenarios depending on their current task or operation.

\subsection{Petri Net Decomposition and State Machine Generation}

The transformation from event logs to plant models involves several sophisticated steps, beginning with Petri net construction and proceeding through decomposition and state machine generation. This process addresses the complexity of industrial systems by breaking down complex behaviors into manageable components while preserving essential behavioral characteristics.

The Petri net decomposition process utilizes reachability graph methods to identify subsets of states and transitions, facilitating the partitioning of complex Petri nets into manageable modules. This approach provides a structured and systematic method for exploring all reachable states and transitions within the system, ensuring completeness in capturing all potential behaviors.

The reachability graph method, while comprehensive, can suffer from state space explosion in complex systems. To address this challenge, several strategies are employed: state space reduction using symbolic representations such as Binary Decision Diagrams (BDDs), abstraction and aggregation methods that focus on essential behavioral properties, partial order reduction techniques that avoid redundant interleavings, and on-the-fly exploration that generates states as needed.

The composition of State Machine Components (SMCs) into a cohesive FSM requires careful integration to ensure synchronization and coordination. Parallel composition and synchronization techniques are preferred for IEC 61499 systems, as they align with the distributed and modular architecture of these systems. This approach enables concurrent execution of SMCs while maintaining proper coordination and communication between distributed components.

\subsection{Plant Model Transformation and Implementation}

The transformation from finite state machines to plant model ECCs represents a critical step in the automatic plant model generation process. This transformation requires careful handling of the different types of signals and transitions that characterize plant behavior in industrial control systems.

\begin{figure*}[!t]
	\centering
	\includegraphics[width=0.9\textwidth]{chapters/images/chapter4/VerificationApp1.png}
	\caption{Application for verification. The structure shows a closed-loop system formed by connecting a plant model obtained from the RMCSES with either a new or existing controller. The plant model interface comprises control signals and non-deterministic transitions (NDT) as input signals, while sensor signals are designated as output signals. This configuration enables formal verification using tools such as NuSMV with CTL/LTL specifications.}
	\label{verificationApp}
\end{figure*}

The presented application for verification follows a structure illustrated in Figure \ref{verificationApp}, where a closed-loop system is formed by connecting a plant model obtained from the RMCSES with either a new or an existing controller. An existing controller is utilized to construct the RMCSES, whereas a new controller is connected with RMCSES for the purpose of verification. The verification process involves conformance checking to ensure that the newly developed controller operates in accordance with the previous controller. The plant model interface comprises control signals and non-deterministic transitions (NDT) as input signals, while sensor signals are designated as output signals. This configuration enables formal verification using tools such as NuSMV with CTL/LTL specifications.

The transformation from RMCSES to plant model involves specific rules for handling sensor and control signals. Transitions triggered by sensor signals are replaced with non-deterministic transitions (NDT) that can fire at arbitrary time intervals, with the output of the next state serving as a sensor event signal. Control signal transitions remain unchanged, as the plant should respond directly to control inputs. This transformation ensures that the plant model accurately represents the uncontrolled behavior of the physical system while maintaining compatibility with formal verification tools.

The transformation process follows a systematic algorithmic approach that handles the complexity of plant behavior modeling. This transformation is essential for creating accurate plant models that can be used for formal verification and closed-loop system analysis. The algorithm takes a finite state machine representing the overall system behavior and produces a plant model ECC that accurately represents the uncontrolled plant behavior.

Algorithm \ref{FSMtoPlant} provides the formal description of this transformation process, which involves processing control signals, sensor signals, and handling converging sensor signals through the introduction of intermediate states. The algorithm operates in three main phases: first, it processes control signal transitions, maintaining them unchanged as the plant should respond directly to control inputs. Second, it processes sensor signal transitions that are not converged, replacing them with non-deterministic transitions (NDT) that can fire at arbitrary time intervals. Third, it handles converging sensor signals by introducing intermediate states that produce appropriate sensor signals before converging to the target state.

\begin{algorithm}[t!]
    \caption{FSM to Plant model ECC }
    \label{FSMtoPlant}
\SetKwInOut{Input}{Input}
\SetKwInOut{Output}{Output}

\Input{ FSM = \{ Q: $\{q_0, q_1, \ldots, q_{n-1}\}$ \\
is a set of states, where $q_0$ is an initial state; \\
CS =  $ C \cup S$ is a set of input signals, where \\  C = $\{c_1, c_2, \ldots, c_p \}$ is a set of control \\  signals, \\  S = $\{s_1, s_2 \ldots, s_r \}$ is a set of sensorsignals; \\
$\delta \subseteq Q \times CS \times Q$ is a transition  relation; \}
}
\Output{ ECC = \{ $QC= Q \cup D$ is a set of ECC states, where D is a set of additional ECC states created dynamically;
$Cond= TC \cup \{1\}$ is a set of ECC transition  conditions, where $TC= C \cup \{NDT\}$ is a set  of input events;
S is a set of output events (see above);\\
$ \phi: QC \times Cond \rightarrow QC \cup QC\times S $ \\ is an ECC  transition function
\}
}

/* Processing the FSM transition labelled by control signals */

\ForEach{$(q_i, \sigma, q_j) \in \delta$}{
    \If{$\sigma \in $ C}{
        Create ECC Transition:
        $(q_i,  \sigma ) \rightarrow (q_j)$;\
    }
}
/* Processing the FSM transition labelled by sensor signals which are not converged */

\ForEach{$(q_i, \sigma, q_j) \in \delta$}{
    \If{$\sigma \in  S $ \text{AND}  $q_i \notin QJ$}{
        Create ECC Transition:
        $(q_i,  NDT ) \rightarrow (q_j, \sigma)$;\
    }
}

/* Processing the FSM transition labelled by converging sensor signals */

\ForEach{$(q_w \in QJ) $}{
\ForEach{$(q_i, \sigma, q_w) \in \delta'}{
    \If{$\sigma \in  S $}{
        Create ECC State $d \in D$ \\
        Create ECC Transition:
    $(q_i,  NDT ) \rightarrow (d, \sigma)$ ; \ \\
        $(d,  1 ) \rightarrow (q_w)$ ;\
    }
}

}
    \Return{ECC}\;

\textbf{ *Note* : Used variables and sets} \\
\BlankLine

$QJ= { \{ q_j \in Q , | \{(q_i, s_k, q_j) \in \delta'\}|>1} \} $, where $\delta' = \delta \cap  Q \times S \times Q$ is a set of ECC states which have converging sensor signals in the FSM

\end{algorithm}

The plant model implementation addresses several important challenges in industrial control system modeling. First, it handles diverging sensor signals by creating separate NDT transitions for each possible sensor output, ensuring that all possible plant behaviors are captured. Second, it manages converging sensor signals by introducing intermediate states that produce appropriate sensor signals before converging to the target state, maintaining the non-deterministic nature of plant behavior. Third, it maintains the non-deterministic nature of plant behavior while providing a deterministic interface for integration with control systems, enabling formal verification using tools such as NuSMV and CTL/LTL specifications.

This algorithmic approach ensures that the generated plant model accurately represents the uncontrolled behavior of the physical system while maintaining compatibility with formal verification tools. The plant model can be used for formal verification using tools such as NuSMV and CTL/LTL specifications, enabling comprehensive analysis of system properties including safety, liveness, and reachability. This verification capability is particularly valuable for safety-critical applications where formal guarantees of system behavior are required.

\subsection{Monitor Implementation for Real-Time Conformance Checking}

The automatic generation of monitors represents a critical application of the RMCSES, enabling real-time detection of deviations from expected system behavior. These monitors are implemented as IEC 61499 function blocks that can be deployed in operational systems to provide continuous monitoring and early warning of potential issues.

The monitor implementation involves transforming the RMCSES into a function block that receives input signals from both controller and sensor components. The monitor's ECC is generated using transformation rules that map FSM states and transitions to appropriate ECC states and transitions. When the monitor detects a valid transition, it generates an "OK" output indicating successful operation. If an unexpected event occurs, the monitor transitions to an error state and produces an "ERROR" event with details about the event that caused the error.

The monitor's error detection capabilities are comprehensive, covering all possible input events from each state. For each state in the FSM, the monitor defines transitions to error states for all input events that are not specified in the reference model. This complete coverage ensures that any deviation from expected behavior is detected immediately, providing robust protection against system malfunctions and unauthorized operations.

The monitor implementation includes several important features. First, it provides detailed error information including the state where the error occurred and the specific event that caused the error. Second, it continues operation until the first error is detected, after which it becomes non-responsive to prevent further processing. Third, it can be easily integrated with existing control systems without requiring significant modifications to the operational infrastructure.

\subsection{Plant Model Generation for Formal Verification}

The automatic generation of plant models extends the capabilities of process mining to formal verification applications. These plant models represent the uncontrolled behavior of the physical system and can be integrated with control models to create complete closed-loop system representations suitable for formal analysis.

The plant model generation process involves extracting the uncontrolled plant behavior from the overall system model represented by the RMCSES. This extraction requires careful analysis to distinguish between control-driven and sensor-driven transitions, as the plant model should only respond to control signals and generate sensor signals based on its internal state.

The transformation from FSM to plant model involves specific rules for handling sensor and control signals. Transitions triggered by sensor signals are replaced with non-deterministic transitions (NDT) that can fire at arbitrary time intervals, with the output of the next state serving as a sensor event signal. Control signal transitions remain unchanged, as the plant should respond directly to control inputs.

The plant model implementation addresses several important challenges. First, it handles diverging sensor signals by creating separate NDT transitions for each possible sensor output. Second, it manages converging sensor signals by introducing intermediate states that produce appropriate sensor signals before converging to the target state. Third, it maintains the non-deterministic nature of plant behavior while providing a deterministic interface for integration with control systems.

The generated plant model can be used for formal verification using tools such as NuSMV and CTL/LTL specifications. The model can be converted to SMV format using the fb2smv tool, enabling comprehensive analysis of system properties including safety, liveness, and reachability. This verification capability is particularly valuable for safety-critical applications where formal guarantees of system behavior are required.

\section{Integration and Applications}

The three approaches presented in this chapter represent complementary aspects of process mining applications in industrial control systems, each addressing different challenges in system understanding, control generation, and verification. Their integration provides a comprehensive framework for addressing the complex requirements of modern industrial automation.

\subsection{Comprehensive System Analysis Framework}

The integration of process model extraction, interactive learning, and automatic plant model generation creates a comprehensive framework for industrial control system analysis and development. This framework enables end-to-end system understanding, from initial behavior recording through formal verification and operational monitoring.

The framework begins with process model extraction, which provides fundamental understanding of system behavior through analysis of recorded event logs. This understanding serves as the foundation for both controller generation and plant model development, ensuring consistency across all system representations.

Interactive learning builds upon this foundation by enabling automatic controller generation from recorded behavioral traces. This approach reduces development time and improves system reliability by ensuring that control logic is consistent with observed system behavior. The generated controllers can be immediately deployed or used as starting points for further refinement.

Automatic plant model generation extends the framework to formal verification applications, enabling comprehensive analysis of closed-loop system behavior. The generated plant models can be used with existing or newly developed controllers to verify system properties and ensure compliance with safety and performance requirements.

\subsection{Industrial Applications and Case Studies}

The presented approaches have been validated through several industrial case studies, demonstrating their effectiveness in real-world applications. The gripper and conveyor system case study illustrates the application of process model extraction for system understanding and anomaly detection. This case study shows how process mining can be used to analyze complex manufacturing processes and identify potential improvements in system operation.

The interactive learning case study demonstrates the application of simulation-based controller generation for a production system with multiple components including conveyors, grippers, and autonomous guided vehicles. This case study shows how the approach can handle complex, multi-component systems and generate controllers that accurately reflect desired system behavior.

The pneumatic cylinder case study illustrates the application of automatic plant model generation and real-time monitoring. This case study shows how the approach can be used for formal verification of safety-critical systems and real-time monitoring of operational compliance.

\begin{figure*}[!t]
	\centering
	\includegraphics[width=1\textwidth]{chapters/images/chapter4/CLMC.PNG}
	\caption{Closed-loop system of plant and controller model with the monitor. The figure shows the integration of a plant model, controller, and monitor in a closed-loop configuration. The monitor function block receives input signals from both controller and sensor components, generating OK/ERROR outputs based on conformance to the reference model. The ECC diagrams illustrate the internal logic of both the plant model and monitor function blocks.}
	\label{CL_plant_controller_monitor}
\end{figure*}

Figure \ref{CL_plant_controller_monitor} illustrates a practical implementation of the closed-loop system integration with monitor for a pneumatic cylinder case study. The system consists of a plant model, controller, and monitor connected in a closed-loop configuration. The monitor function block receives input signals from both controller and sensor components, generating OK/ERROR outputs based on conformance to the reference model. The Execution Control Chart (ECC) diagrams show the internal logic of both the plant model and monitor function blocks, demonstrating how the transformation from finite state machines to IEC 61499 function blocks is implemented in practice.

In this implementation, the monitor generates an 'OK' event with the corresponding StateID when the system process executes in the correct order. In the case of any deviation from the expected process behavior, an 'ERROR' signal is emitted along with the 'EventID' indicating the specific event responsible for the deviation. The monitor continues operation until the first error is detected, after which it becomes non-responsive to prevent further processing. This real-time monitoring capability is essential for ensuring system safety and reliability in industrial applications.

These case studies demonstrate the versatility of the presented approaches and their applicability to a wide range of industrial applications. The approaches can be adapted to different system types, from simple single-component systems to complex multi-component manufacturing lines, providing flexible solutions for various industrial automation challenges.

\subsection{Future Research Directions}

The integration of process mining with industrial control systems opens several promising research directions that could further enhance the capabilities and applicability of these approaches.

First, the development of more sophisticated process discovery algorithms could improve the quality and accuracy of generated models. Current algorithms may struggle with complex, highly concurrent systems or systems with significant noise in their event logs. Advanced algorithms that can handle these challenges would expand the applicability of the approaches to a broader range of industrial systems.

Second, the integration of machine learning techniques with process mining could enable adaptive models that improve over time based on new observations. This capability would be particularly valuable for systems that evolve or change their behavior over time, enabling continuous improvement of system understanding and control.

Third, the development of more efficient algorithms for handling large-scale event logs could enable the application of these approaches to complex, multi-facility manufacturing systems. Current approaches may face computational challenges when dealing with very large event logs or complex system architectures.

Fourth, the integration of these approaches with emerging technologies such as digital twins and edge computing could enable real-time process mining and control generation. This integration would enable dynamic adaptation of control systems based on real-time analysis of system behavior, providing more responsive and adaptive automation solutions.

The research presented in this chapter demonstrates the significant potential of process mining techniques for addressing critical challenges in industrial control systems. The three complementary approaches provide comprehensive solutions for system understanding, control generation, and verification, while the integration of these approaches creates a powerful framework for industrial automation development and analysis.

As industrial systems become increasingly complex and interconnected, the need for automated methods for system understanding and control generation will continue to grow. The approaches presented in this chapter provide a solid foundation for addressing these challenges and enable the development of more intelligent, adaptive, and reliable industrial automation systems.

The future of industrial automation lies in the integration of data-driven approaches with formal methods, creating systems that can learn from their environment, adapt to changing requirements, and provide formal guarantees of their behavior. The process mining approaches presented in this chapter represent important steps toward this vision, providing practical methods for realizing intelligent industrial automation systems that can meet the demands of modern manufacturing.


%%%%%%%%%%%%%%%%%%%%%%%%%%%%%%%%%%%%%%%%%%%%%%%%%%%%%%%%%%%%%%%%%%%%
%% Begin Part II - Collection of papers
%%%%%%%%%%%%%%%%%%%%%%%%%%%%%%%%%%%%%%%%%%%%%%%%%%%%%%%%%%%%%%%%%%%%

\makepartpage{Part II}%
\startpapers

% Bibliography files for papers

%-------------------------------------------------------------------
\def\paperheader{Paper A}
\def\papertitle{Formal Modelling, Analysis, and Synthesis of Modular Industrial Systems inspired by Net Condition/Event Systems}
\def\paperauthorstring{Midhun Xavier, Sandeep Patil, Victor Dubinin, Valeriy Vyatkin}
\def\referencestring{Proceedings of the IEEE International Conference on Industrial Informatics (INDIN), 2021.}
\def\copyrightstring{2021, IEEE, Reprinted with permission.}

% The definitions above could just as well be put directly into the function
% call below, but were explicitly defined to more clearly illustrate the
% use of the function \makepaper.

\makepaperaccepted
  {\paperheader}
  {\papertitle}
  {\paperauthorstring}
  {\referencestring}
  {\copyrightstring}

% The actual contents is imported by un-commenting the \input line below.
% Make sure the file exist.
\begin{bibunit}
\thispagestyle{plain}

\section*{Abstract}
This paper summarises recent developments in the application of modular formalisms to model-based verification of industrial automation systems. The paper is a tribute to the legacy of Professor Hans-Michael Hanisch who invented Net Condition/Event Systems (NCES) and passionately promoted the closed-loop modelling approach to modelling and analysis of automation systems. The paper surveys the related works and highlights the impact NCES has made on the current progress of modular automation systems verification.

\section{Introduction}
Modularity is a fundamental feature of technical systems, in particular in industrial automation and cyber-physical systems.
On the other hand, modular systems is a good example of distributed systems.
Petri nets (PN) have been known as a formal language specifically focused on modelling of distributed state systems. That suggests a clear overlap and the need to address modularity in formal modelling. Petri nets inspired an uncountable number of derivatives.

Modularity in the context of PN has been discussed for a long time. 
According to \cite{davidrajuh2019new}, the concept of Modular Petri Nets has been through four generations of development. 

On the other hand, the concept of Condition/Event Systems (C/ES)
 \cite{sreenivas1991condition} was invented to model modular systems composed of communicating modules and study their generic properties.

Net Condition/Event Systems (NCES) \cite{rausch1995net} is a particular case of C/ES where modules are defined as (extended) Petri nets. It was proposed to model more efficiently distributed systems that are modular. 

One should note that computational analysis of NCES is in general undecidable as shown by Starke and Hanisch \cite{starke1997analysis}. Nevertheless, the formalism fits very well with the emerging engineering concepts for CPS such as service-oriented architecture (SOA) and IEC 61499 function blocks due to the properly addressed event-driven semantics. The initial effort of NCES application to IEC 61499 modelling is summarised in \cite{hanisch2009one}.

In this paper, we attempt to observe the developments related to the modelling and analysis of distributed modular industrial automation systems from the particular perspective of how the modular derivatives of Petri nets influence them.

The rest of the paper is structured as follows. In Section \ref{sec:def} the necessary definitions of NCES are provided. It is followed by a brief illustration of some features NCES provide for modelling distributed modular systems in Section \ref{sec:ncesmod}. Section \ref{sec:61499} contains some observations of similarities between IEC 61499 and NCES. Section \ref{sec:survey} attempts to overview the related research works on the modelling of modular systems. The recently developed modelling framework for modular systems based on IEC 61499 and influenced by the Condition/Event Systems paradigm is described in Section \ref{sec:framework}. The paper is concluded with a short summary and outlook in Section \ref{sec:summary} and acknowledgements. 

\section{Some definitions}\label{sec:def}

Net Condition/Event Systems (NCES) is a finite state formalism that preserves the graphical notation and the non-interleaving semantics of Petri nets \cite{Petri62}, and extends them with a clear and concise notion of signal inputs and outputs. The formalism was introduced in \cite{RaHA95} in 1995 and has been used in dozens of applications, especially in embedded industrial automation systems. 

\begin{figure}
    \centering
    \includegraphics[width=0.5\textwidth]{MX_Papers/Paper1/images/nces_module.jpg}
    \caption{Graphical notation of an NCES module.}
    \label{fig:nces}
\end{figure}

Given a place/transition net $N=(P,T,F,m_0)$, the
Net Condition/Event System (NCES) is defined as a tuple
$\mathcal{N}=(N,\theta_N,\Psi_N, Gr)$, where $\theta_N$ is an internal
structure of signal arcs, $\Psi_N$ is an input/output structure,
and $Gr \subseteq T$ is a set of so-called "obliged" transitions that fire as soon as it is enabled.
Fig. \ref{fig:nces} shows an example of an NCES module. 
The structure $\Psi_N$ consists of
condition and event inputs and outputs ($ci,ei,eo,co$). The
structure $\theta_N$ is formed from two types of {\it signal}
arcs. Condition arcs lead from places and condition inputs to
transitions and condition outputs. They provide additional
enableness conditions of the recipient transitions. Event arcs
from transitions and event inputs to transitions and event
outputs provide one-sided synchronization of the recipient
transitions: the firing of the source transition forces the firing of the recipient if the latter is enabled by the marking and conditions.

The NCES modules can be interconnected by the condition and event arcs, forming thus distributed and hierarchical models as illustrated in Fig.\ref{fig:composition}.
NCES having no inputs can be analyzed without any additional
information about its external environment.

\begin{figure}
    \centering
    \includegraphics[width=0.5\textwidth]{MX_Papers/Paper1/images/composition.jpg}
    \caption{Composition of NCES modules.}
    \label{fig:composition}
\end{figure}

The semantics of NCES cover both asynchronous and synchronous behaviour (required to model plants and controllers respectively). NCES are supported by a family of model-developing and model-checking tools, such as a graphic editor, SESA and ViVe (\cite{vive}).

The state of an NCES module is completely determined by the
current marking $m: P \rightarrow \mathbb{N}_0$ of places and values of
inputs. A state transition is determined by the subset $\tau
\subseteq T$ of simultaneously fired transitions, called {\it
step}. The transitions having no incoming event arcs are called
{\it spontaneous}, otherwise {\it forced}. The step fully determines the
values of the event outputs of the module. In the original NCES version the step is formed by
choosing some\footnote{This means the step in NCES is non-deterministic.} of the enabled spontaneous transitions, and all the enabled transitions forced by the transitions already included in the step. 

A state of NCES is fully described by the marking of all its places (in the timed version also by clocks). A transition step specifies a state transition. 
When used for system analysis, a set of all reachable states (complete or partial) 
of NCES model is generated and then analyzed.

For describing the execution model of function blocks we use a deterministic dialect 
of NCES and the modeling approach that guarantee certain properties of the models as follows: 
\begin{enumerate}
\item In the chosen dialect a step is formed from all enabled spontaneous transitions and all forced transitions;
\item The models are designed so that there is no conflicts (i.e. deficient marking in some places) leading to non-deterministic choice of some of the enabled transitions;
\item The models also guarantee bounded marking in all places.
\end{enumerate}

\section{Modelling distributed systems with NCES}\label{sec:ncesmod}

To illustrate the key features of NCES modelling for distributed systems, let us consider an example of a simple distributed control system. In the system of two cylinders in Fig. \ref{fig:two_cylinders} each cylinder pushes a workpiece to the destination hole. The process starts when the workpiece appears in front of the corresponding cylinder as indicated by sensors P1 and P2 respectively. 
As it is clear from the Figure, cylinders can collide in the middle point, therefore the goal of controller design is to avoid such a situation. 

\begin{figure}
    \centering
    \includegraphics[width=0.5\textwidth]{MX_Papers/Paper1/images/two_cylinders.jpg}
    \caption{Two cylinders example of a distributed system.}
    \label{fig:two_cylinders}
\end{figure}

There are many possible ways to achieve the desired behaviour, which can be done by designing a "central" controller of both cylinders or a protocol ensuring that distributed controllers collaborate correctly. Distributed control is of interest for many practical reasons, for example for the case when control logic is "embedded" in each cylinder, so they can start working as soon as powered on.

NCES model of the two cylinders system with distributed control is presented in Fig. \ref{fig:nces_2_cyl}.

\begin{figure}
    \centering
    \includegraphics[width=0.9\textwidth]{MX_Papers/Paper1/images/nces_two_cylinders.jpg}
    \caption{NCES model of the two cylinders system.}
    \label{fig:nces_2_cyl}
\end{figure}

An abstract model of two processes interacting with each other with the help of buffer is presented in Fig.\ref{fig:nces_interprocess}. Here Process 1 adds a token to the Buffer, and Process 2 sees it and removes it from the buffer.

\begin{figure}
    \centering
    \includegraphics[width=0.7\textwidth]{MX_Papers/Paper1/images/interprocess.jpg}
    \caption{NCES model of interprocess communication.}
    \label{fig:nces_interprocess}
\end{figure}

A more sophisticated synchronous communication mechanism between clock-driven processes through a rendezvous channel is modelled by means of NCES formalism in Fig. \ref{fig:rendezvous}. To verify the correctness of the channel's operation the model-checking tool ViVe can be applied. The reachability graph of the model is presented in Fig.\ref{fig:verification}. 

\begin{figure}
    \centering
    \includegraphics[width=0.8\textwidth]{MX_Papers/Paper1/images/rendezvous2.jpg}
    \caption{NCES model of the process synchronisation.}
    \label{fig:rendezvous}
\end{figure}

\begin{figure}
    \centering
    \includegraphics[width=0.9\textwidth]{MX_Papers/Paper1/images/verification.jpg}
    \caption{Reachability graph of the model (left) and the behaviour along the $S1\rightarrow S2\rightarrow S4$ trace (right), where the rendezvous occurs at the state transition  $S2\rightarrow S4$.}
    \label{fig:verification}
\end{figure}

\section{IEC 61499 based modular engineering of automation systems}\label{sec:61499}

The {IEC 61499} architecture \cite{iec61499} is getting increasingly recognised as a powerful mechanism for engineering cyber-physical systems. 
In IEC 61499, the basic design construct is called function block (FB). Each FB consists of a graphical event-data interface and a set of executable functional specifications (algorithms), represented as a state machine (in basic FB), as a network of other FB instances (composite FB), or as a set of services (service interface FB). FBs can be interconnected into a network using event and data connections to specify the entire control application. The execution of an individual FB in the network is triggered by the events it receives. This well-defined event-data interface and the encapsulation of local data and control algorithms make each FB a reusable functional unit of software.

A basic Function Block (FB)  consists of a signal interface (left-hand side) and an Execution Control Chart (ECC) state machine (right-hand side). The algorithms executed in the ECC states determine the behavior of the FB in response to changes in its inputs and its internal state.

A function block application is a network of FBs connected by event and data links as illustrated in the upper part of Fig. \ref{fig:similarity}, which illustrates models of the same one pneumatic cylinder system with IEC 61499 (top) and NCES (bottom). The structural similarity is supported by the semantic similarity since both modelling languages are event-based. Connections between modules in both modelling languages are passing events and data. 
This simplifies the modelling of IEC 61499 with NCES and several modelling and analysis tools were developed to explore it. 

\begin{figure}
    \centering
    \includegraphics[width=0.8\textwidth]{MX_Papers/Paper1/images/similarity.jpg}
    \caption{Similarity of IEC 61499 and NCES models.}
    \label{fig:similarity}
\end{figure}

In 1998, way before the IEC 61499 was formally accepted as a standard by IEC, using an early draft, Hans-Michael Hanisch observed this stunning similarity and wrote a research proposal together with Peter Starke, supported by the German Research Council (DFG), on formal verification of IEC 61499 applications by means of NCES. That gave rise to a number of developments summarised in \cite{hanisch2009one}.

In particular, in 2001, Vyatkin and Hanisch developed a software package called "Verification Environment for Distributed Applications" (VEDA) for model-based simulation and verification \cite{vyatkin2001formal}. NCES is used for modelling and IEC 61499 function blocks are automatically converted with the help of VEDA for efficient simulation and verification. 

But, surprisingly, the NCES-IEC 61499 similarity helped develop a modelling approach in which IEC 61499 itself was directly used as a modelling language as it will be illustrated in Section \ref{sec:framework}. 

\section{Survey of works on modular engineering and modelling}\label{sec:survey}

To put the above-referenced developments on NCES and IEC 61499 to the broader context, in this section we present a brief survey of other related works on formal modelling and analysis of modular automation systems. 

\subsection{Modelling of flexible reconfigurable systems}

Reconfigurable Manufacturing Systems (RMS) are flexible and adaptable to manufacture various products to meet changing market demands. Meng \textit{et al.} explain how complex RMS can be hierarchically modularized for modelling reconfigurability using coloured Object Oriented Petri nets \cite{MENG201081}. The  RMS model is developed with the help of the macro-level Petri net and the changes in RMS drive the change in Petri net.

Later,  Wu \textit{et al.} introduced Intelligent Token Petri Net (ITPN) for modelling reconfigurable Automated Manufacturing Systems (AMS) \cite{wu2011intelligent}. The ITPN model captures dynamic changes in the system and the deadlock-free policy makes the model always deadlock-free and reversible. The change in configuration modifies only changed part of the current model and the deadlock-free policy remains the same.

In real-time systems temporal constants are inevitable and these systems need to be modelled to ensure that it satisfies functional and non-functional requirements. Recently, Kaid \textit{et al.} developed  Intelligent Colored Token Petri Net (ICTPN) and it models dynamic changes in a modular manner and produces a compact model which ensures PN behavioural properties like boundness, liveness and reversibility but the ITCPN model lacks a conversion method to industrial control languages.

Reconfigurable Discrete Event System (RDES) such as reconfigurable manufacturing systems (RMS) has the ability to change the configuration of the system to adapt to changes in conditions and requirements.
Reconfigurable discrete event control systems  (RDECS) are an important part of RDESs.  Reconfiguration done at the run time is called Dynamic reconfiguration and it should occur without influencing the working environment and with no deadlock. Zhang \textit{et al.} introduced the reconfiguration based on the Timed Net Condition/Event system (R-TNCES) and it is a formalism for the modelling and verification of RDECSs. SESA model checker does the layer-by-layer verification of R-TNCES \cite{zhang2013r}.  

Modern manufacturing systems switch energy-intensive machines between working and idle mode with the help of dynamic reconfiguration to save energy. The later works of Zhang \textit{et al.} developed how formal modelling and verification of reconfigurable and energy-efficient manufacturing systems can be done using R-TNCES formalism and SESA tool is applied to check functional, temporal and energy efficient properties \cite{zhang2015modeling,zhang2018simulation}.

System reconfiguration in run-time is inevitable and a discrete event system with dynamic reconfigurability is called (DRDES). NCES is widely applied in DRDESs in the past decade. NCES are a modular extension of PN and it is used for modelling, analysis and control of DRDES. Many researchers worked on the modelling, analysis and verification of reconfigurable RMS. 

The system reconfiguration should be completed before the permissible reconfiguration delay. Whenever a reconfiguration event is triggered then DRDES should be able to go to the target state within the permissible reconfiguration delay otherwise it should reject the reconfiguration requirement. Zhang \textit{et al.} developed to compute a shortest legal firing sequence (SLFS) of an NCES using Integer Linear Programming (ILP) under a given maximum permissible reconfiguration delay \cite{zhang2018shortest}.

Interpreted time Petri net (ITPN) is used to model real-time systems, which helps to increase the modelling power and expressiveness compared to (Timed Petri net) TPN's. Hadjidj \textit{et al.} proposed  RT-studio (Real-time studio) for  modelling, simulation and automatic verification. \cite{hadjidj2013rt}. RT-studio tries to tighten the gap with the UPPAAL model checker by modularizing the ITPN model.  

Dehnert \textit{et al.} introduced a new probabilistic model checker \cite{dehnert2017storm, hensel2022probabilistic} called Storm that can analyze both discrete- and continuous-time variants of Markov chains and Markov decision processes  (MDPs), using the Prism and JANI modelling languages, probabilistic programs, dynamic fault trees and generalized stochastic Petri nets. It has a flexible design that allows for easy exchange of solvers and symbolic engines, and it offers a Python API for rapid prototyping. Benchmark experiments have shown that Storm has competitive performance.

\subsection{Modelling of IEC 61499}\label{sec:mod61499}

Another approach to verify the application of IEC 61499 was presented by Schnakenhourg \textit{et al.}, who explained the method to verify IEC 61499 function blocks by converting to the SIGNAL model \cite{schnakenbourg2002towards}. The specification also converts to a SIGNAL model and verifies using SILDEX from the TNI society.

In order to formally model function blocks in IEC 61499, it is necessary to first define their complete execution semantics. The semantic ambiguities in IEC 61499, can lead to different interpretations of function blocks. To address this, the Sequential Hypothesis can be used, which defines a more intuitive and clear sequential execution model of function blocks. Pang \textit{et al.} \cite{pang2007towards},  developed IEC 61499 basic function  blocks using the sequential hypothesis, which assumes that blocks within a network are activated sequentially. They used NCES and verified the behaviour of the model using model-checking tools such as iMATCH and SESA. They later proposed a model generator \cite{pang2008automatic} that automatically translates IEC 61499 function blocks into the NCES formal model for the purpose of verification. The function blocks developed using the FBDK (Function Block Development Kit) are translated into functionally and semantically equivalent NCES models following the sequential execution model. This NCES model can be opened in ViEd and properties are verified using the ViVe tool.

Cengic \textit{et al.} \cite{cengic2006formal} introduced a new runtime environment called Fuber, which uses a formal execution model to make the behaviour of IEC 61499 applications deterministic and predictable. They developed a tool to translate IEC 61499 function blocks into a set of finite-state automata and used the Supremica tool for supervisor verification and synthesis. After that, they  introduced a software tool to automatically generate formal closed-loop system models between control code and process models expressed as IEC 61499 function blocks, using extended finite automata (EFA) and Supremica for formal verification \cite{vcengic2008control}. They further extended this by defining the buffered sequential execution model (BSEM) and its formal verification using Supremica by analyzing the EFA model \cite{cengic2008definition}. In another study, Yoong \textit{et al.} developed a tool to translate IEC 61499 function blocks to Esterel for verification \cite{yoong2010verifying}.  Existing verification tools for Esterel help to analyze the safety properties of IEC 61499 function block programs.

Formal verification of embedded control systems using closed-loop plant-controller models is becoming more popular.  However, the use of non-determinism in the model of the plant can lead to the complexity explosion in the model-checking process and make it difficult to verify the correctness of the plant model itself before it can be used in the closed-loop verification process. The paper \cite{patil2011closed} describes the integration of modelling principles into the Veridemo toolchain, and it also explains the implementation of controlled non-determinism in NCES systems. The controlled non-determinism limits the state space and eventually results in better verification times. This approach provides better model-checking performance with ViVe and SESA compared to NuSMV and UPPAAL model checkers with fully deterministic state machines. Later they introduced \cite{patil2015counterexample}  a framework for model checking and counter-example playback in simulation models used to verify the system. The control logic and dynamics of the plant are modelled using Net Condition/Event Systems formalism and ViVe/SESA toolchain is used for model checking. The counter examples for failures during model checking are played back in simulation models for a better understanding of the failures.

The IEC 61499 standard is used for the development of distributed control systems, but it has limited support for reconfigurable architectures. To address this limitation, Guellouz \textit{et al.}  proposed a new model called reconfigurable function blocks (RFBs) in their study \cite{guellouz2016reconfigurable}. They use GR-TNCES, a derivative of NCES, to model the system and applied the proposed approach to a medical platform called BROS. Further studies \cite{guellouz2018designing, fkaier2021modeling} proposed translating RFBs to GR-TNCES in order to verify their correctness and alleviate state space explosion in model checking. Additionally, the latter paper aimed to analyze probabilistic properties and used a smart-grid system as a case study to demonstrate the approach. The study also developed a visual environment called ZiZo v3 for modelling reconfigurable distributed systems.

 The formal verification technique is a reliable approach to ensure the correctness of instrumentation and control (I\&C) systems. It mentions that model-checking is widely used in avionics, the automotive industry, and nuclear power plants but has some  difficulties in locating errors in the model. The Oeritte tool, presented in the first study of Ovsiannikova \textit{et al.} \cite{ovsiannikova2021oeritte}, is a solution for assisting analysts in the debugging process of formal models of instrumentation and control systems. It uses a method for automatic visual counterexample explanation and includes reasoning for both the falsified LTL formula and the NuSMV function block diagram of the formal model of the system. The tool addresses the challenges of counterexample visualization, LTL formulae, and counterexample explanation by providing methods, visual elements, and user interface. The second study,\cite{ovsiannikova2021towards}, presents the development of a model-checking plugin for IEC 61499 systems in the FBME graphical development environment. The plugin automates the process of converting the system to a formal model, model-checking, and providing a visual explanation of counterexamples.

 The next step to verification is the formal synthesis of correct-by-design systems with ensured safe operation. Missal and Hanisch \cite{missal2008modularA,missal2008modularB} present a modular synthesis approach. It is based on the modular backward search in order to avoid the complexity of generating all states and state transitions of the plant model.
It uses modular backward steps that describe the trajectories leading to forbidden states. The generation of these trajectories is stopped as soon as a preventable step is found. From this information, the models of the controllers are generated. 
Each controller has decision functions and communications functions. Together they establish a network of local, interacting controllers with communication. It is assumed that the plant is completely observable, i.e. the local controllers have complete information about the local states of the partial plants they are supposed to control. 
The paper also contributes with the definition of the behaviour of the plant without its complete composition. This means that the behaviour can be studied by means of modular steps within the modules and their interaction across module boundaries.

Dubinin \textit{et al.} in \cite{dubinin2015synthesis} demonstrate safety controller synthesis  using the description of the plant and forbidden behaviour, proposing a method of synthesis of adaptive safety controller models for distributed control systems based on reverse safe Net Condition/Event Systems (RsNCES). The method allows for the generation of prohibiting rules to prevent the movement of closed-loop systems to forbidden states. The method is based on a backward search in the state space of the model.

\section{Use IEC 61499 for Condition/Event Modelling: A Comprehensive Tool Chain} \label{sec:framework}

The works on formal modelling and verification of IEC 61499 systems by means of NCES and its analysis tools have confirmed the benefits of exploring their structural and semantic similarities. On the other hand, applying the verification to systems of industrial scale has raised several questions:
\begin{itemize}
\item Model-checking tools for NCES require constant support and improvement, which was lacking. A bridge to industrially supported powerful tools was desirable.
\item Verification should be a part of the regular engineering and testing process that includes testing by simulation, and analysis of results. 
\end{itemize}
 
 Towards the first goal, Patil \textit{et al.} \cite{patil2015formal} introduce a method for formally modelling and verifying IEC 61499 function blocks, a component model used in distributed embedded control system design, using the Abstract State Machines (ASM) as an intermediate model and the SMV model checker. The ASM model is translated into the input format of the SMV model checker, which is used to formally verify the properties. The proposed verification framework enables the formal verification of the IEC 61499 control systems, and also highlights other uses of verification such as the portability of IEC 61499-based control applications across different implementation platforms compliant with the IEC 61499 standard. Their other work \cite{patil2015neutralizing} proposes a general approach for neutralizing semantic ambiguities in the IEC 61499 standard by the formal description of the standard in ASM.

Another study \cite{patil2015formal}, highlights the importance of formally verifying function block applications in different execution semantics and the benefits of verifying the portability of component-based control applications across different platforms compliant with the IEC 61499 standard. The paper applies the formal model to an example IEC 61499 application and compares the verification results of the two-stage synchronous execution model with the earlier cyclic execution model, to verify the portability of the IEC 61499 applications across different platforms.

After that, they addressed the SMV modelling of the IEC 61499 specific timer function block types, particularly in hierarchical function block systems with timers located at different levels of hierarchy \cite{drozdov2016formal}. The paper also introduces plant abstraction techniques to reduce the complexity of cyber-physical systems models using discrete-timed state machine models implemented in UPPAAL. The approach is demonstrated with an example of formal verification of a modular mechatronic automated system and is shown to extend the abilities in the validation of real cyber-physical automation systems. A toolchain was developed to support the described modelling method, including an automatic FB-to-SMV converter for the transformation of IEC 61499 FB applications to the control part of SMV models. This approach can be used for the verification of newly developing industrial safety-critical systems such as smart grids.

Addressing the second goal, the road map on the creation of a tool-chain connecting engineering with verification seamlessly was outlined in  \cite{vyatkin2008closed}.  A problem-oriented notation within the IEC 61499 syntax for creating formal closed-loop models of cyber-physical automation systems \cite{xavier2021cyber} is proposed. The notation enables the creation of a comprehensive toolchain for the design, simulation, formal verification, and distributed deployment of automation software. The toolchain includes an IEC 61499-compliant engineering environment, a converter for functions blocks to SMV code, the NuSMV model-checker and utilities for interpreting counterexamples. The proposed method aims to overcome the hurdle of verifying and analyzing function blocks implemented in IEC 61499 standard by providing a toolchain for continuous development and testing of distributed control systems. 

\begin{figure*}
    \centering
    \includegraphics[width=0.5\textwidth]{MX_Papers/Paper1/images/Two_Cylinder.PNG}
    \caption{Visualisation of the Two Cylinder system produced by the model of the plant implemented in IEC 61499.}
    \label{fig:two_cylinder_HMI}
\end{figure*}

The two-cylinder system  consists of two orthogonal pneumatic cylinders controlled by a switch button shown in figure \ref{fig:two_cylinder_HMI}. It is built using five basic function blocks, including a controller function block (Button FB) that triggers the movement of the cylinders when pressed, plant function blocks (HorCyl and VerCyl FBs) that model the physical device of each cylinder, and controller function blocks (HorCTL FB and VerCTL FB) that control the plant by analyzing sensor signals and triggering actuator signals. These blocks receive information from the switch FB and send orders to the plant FB.

\begin{figure*}
    \centering
    \includegraphics[width=1\textwidth]{MX_Papers/Paper1/images/NDT_linear.PNG}
    \caption{a) Deterministic discrete state linear motion process model without NDT, b) Discrete state linear motion process model with NDT.}
    \label{fig:NDT_in_Plant}
\end{figure*}

To implement the closed-loop approach to system modelling, the model of the plant needs also to be modelled using function blocks. A discrete state linear motion of a cylinder for a linear motion, for example, a linear axis, can be represented by a LinearDDtrA function block with two states (sHOME and sEND) that transition between them based on input signals (BACK or FWD) Fig. \ref{fig:NDT_in_Plant}, a. However, this minimal approach may not capture all possible errors that can occur during transitions between states. 

By using NDT (Non-deterministic transition), a more comprehensive model can be created by adding two dynamic states (ddMOVETO and ddRETURN) to capture potential errors during transitions Fig. \ref{fig:NDT_in_Plant},b.

The axis moves from the stHOME state to the stEND state via the motion state ddMOVETO when the FWD signal is TRUE. The use of NDT (Non-Deterministic Transition) in the transition from the ddMOVETO state to the stEND state models the unknown duration of the motion from one state to another. The NDT event input of the LinearDA function block, which was unassigned in the application, is reserved for non-deterministic transitions in the proposed modelling notation. This approach can provide a more detailed and accurate representation of the system, allowing for more thorough formal verification. 

The (multi-) closed-loop model of the two cylinders system using this extension of the IEC 61499 language is shown in Fig. \ref{fig:twocylindersfb}. This is nothing else, but a Condition/Event discrete-state model represented by means of IEC 61499. 

\begin{figure*}
    \centering
    \includegraphics[width=0.8\textwidth]{MX_Papers/Paper1/images/twocylindersfb.PNG}
    \caption{Complete two cylinders model in the modified FB language.}
    \label{fig:twocylindersfb}
\end{figure*}

The fb2smv tool is a model generator that is used to generate SMV (Symbolic Model Verifier) models of function block systems in IEC 61499. It is part of a formal verification tool-chain that includes the model checker NuSMV and a tool for analyzing counterexamples in terms of the original FB system. The tool uses the Abstract State Machine (ASM) as an intermediate model to convert IEC 61499 function blocks expressed in XML format into a formal model. The generated SMV code has a structure that consists of a declaration part and an ASM rules part. The tool can convert both basic and composite function blocks, and also includes additional features such as limiting variable boundaries to reduce the state space, changing the execution order of FBs, and deciding the input event priority by changing its order. Additionally, the tool has been updated to include a proposed non-deterministic transitions notation.

Closed-loop modelling is a powerful approach for the verification of distributed industrial automation systems, as it allows for a comprehensive evaluation of the system's behaviour. However, it requires the creation of a model of the plant, which can be a complex and resource-intensive task, typically done manually. In these papers \cite{xavier2021plant,xavier2022process,xavier2022interactive,xavier2022plant}, authors show how to generate the plant and controller models automatically using a data-driven approach. The above-mentioned toolchain has been effectively used in these experiments to verify, simulate and analyse counterexamples. 

\section{Summary and Open Problems}\label{sec:summary}

Systems with dynamically created and terminated modules or dynamic connections between modules cannot be efficiently and naturally modelled within the C/ES paradigm and require complicated workarounds.

The idea of modular analysis of NCES has not been developed although the absence of token flow between the NCES modules could potentially facilitate it.

\begin{figure}
    \centering
    \includegraphics[width=0.35\textwidth]{MX_Papers/Paper1/images/Hanisch.jpg}
    \caption{Hans-Michael Hanisch (1957-2022).}
    \label{fig:Hanisch}
\end{figure}

\section*{Acknowledgments}
This paper attempts to be a tribute to Professor Hans-Michael Hanisch who has been a co-inventor and a great enthusiast and proponent of NCES as a part of the closed-loop modelling concept. 

\putbib
\end{bibunit}



%-------------------------------------------------------------------
\def\paperheader{Paper B}
\def\papertitle{Cyber-physical automation systems modelling with IEC 61499 for their formal verification}
\def\paperauthorstring{Midhun Xavier, Sandeep Patil, Valeriy Vyatkin}
\def\referencestring{Proceedings of the IEEE International Conference on Industrial Informatics (INDIN), 2021.}
\def\copyrightstring{2021, IEEE, Reprinted with permission.}

% The definitions above could just as well be put directly into the function
% call below, but were explicitly defined to more clearly illustrate the
% use of the function \makepaper.

\makepaperaccepted
  {\paperheader}
  {\papertitle}
  {\paperauthorstring}
  {\referencestring}
  {\copyrightstring}

% The actual contents is imported by un-commenting the \input line below.
% Make sure the file exist.
\begin{bibunit}
\thispagestyle{plain}

\section{Introduction}

Distributed industrial automation systems pose a significant challenge for their efficient verification and validation due to their heterogeneous structure, use of wireless communication and decentralised logic. The inherent inter twinning of computational and communication processes with complex physical dynamics has called for the term cyber-physical systems (CPS) \cite{lee2017introduction} to emphasize the challenges and the need for new development approaches.   

The {IEC 61499} architecture \cite{iec61499part12012} is getting increasingly recognised as a powerful mechanism for engineering such systems. 
It has been proven also as an efficient way of modeling CPS in automation \cite{dai2017discrete}. 

The challenge of IEC 61499 verification has been well-recognized from the early stages of the standard's development and evaluation \cite{vyatkin1999modeling}. Closed-loop modelling has been proposed for the most comprehensive verification \cite{vyatkin2008closed}, which implies the need for modelling the plant. 

In quite many works, the plant modelling \cite{buzhinsky2016plant} was done in the same formalism, which was used eventually to represent the model for the model-checker. Graphical modelling languages of finite-state machines and Petri nets \cite{berthomieu1991modeling} were used in particular, and the models were prepared using the corresponding graphical editors.
However, the IEC 61499 itself provides a graphical engineering interface and supports programming in terms of state machines. Therefore, a problem-oriented notation could be proposed to take advantage of the existing tools and avoid using additional ones in the process of modelling. This paper proposes such an approach by introducing a tool chain.

The paper is structured as follows: Section \ref{sec:related_work} discusses the related work and problem statement. Section \ref{sec:illustrative_example} and \ref{sec:simulation_model} illustrates an example and simulation model in detail. Section \ref{sec:discrete_state} describes the discrete-state modelling approach including the implementation of non-deterministic transition in smv. Section \ref{sec:fb2smv_tool} gives an overview of the fb2smv tool's functionalities and features. Section \ref{sec:results} presents the results and analysis of the work. Finally, Section \ref{sec:conclusion} concludes the paper and outlines future goals.

\section{Related works and problem statement}\label{sec:related_work}

Christensen suggested a model-driven development approach \cite{christensen2000design} for distributed automation systems that is based on the use of the model-view-control object-oriented design pattern. The approach supports several development stages, from simulation in the loop to the deployment~\cite{patil2018adapting}. 
In \cite{vyatkin2008closed} that approach was extended to also include formal verification into the verification and validation of function block systems.

The suggested framework is heavily based on the closed-loop architecture of the model, where the plant part is explicitly represented in the overall system model. This architecture allows for easy integration with a simulation model for the virtual commissioning purposes, which can be seamlessly converted to the deployment configuration. In addition, the closed-loop configuration could be transformed to a structurally similar formal model, appropriate for more exhaustive verification by means of model-checking.   
The concept of an integrated environment VEDA presented in \cite{vyatkin2003verification} had already supported closed-loop verification of IEC 61499 function block systems. While the controller parts of the closed-loop models were automatically translated into the corresponding formal model in a Petri-net like language, the model of the plant parts had to be developed manually in the same formalism. 
This made the process of model creation quite difficult, not allowing for systematic use of the tool by control engineers. 
The emergence of the automatic model generator fb2smv \cite{fb2smv} that is capable of creating SMV models from IEC 61499 function block systems, promises increased potential of formal verification on account of using the industry grade model-checkers of the NuSMV \cite{cimatti2000nusmv} and NuXMV family. 
However, the problem of creating the model of plant in SMV remains to be the limiting factor~\cite{sinha2019survey} for industrial application of the corresponding verification tool-chain.

In this paper, we attempt to overcome this hurdle by proposing a CPS modelling method that is entirely based on IEC 61499. By means of the same modelling language, we represent both simulation models of the plant, which are equivalent to hybrid automata, and the models for formal verification, which are equivalent to discrete-state automata with non-determinism. 
The latter model can be derived from the former by applying a sequence of transformation steps. 

Both kinds of plant models, implemented as IEC 61499 function blocks can be included in the multi-closed-loop model connected to the real controllers. This opens the opportunity of checking the distributed control logic of CPS in simulation and model-checking.  

\section{Illustrative example}\label{sec:illustrative_example}

The modelling method and tool-chain used for CPS verification are illustrated in this paper using a laboratory-scale distributed automation system "Drilling station", described in the next subsection. In this case study, we selected and created a formal model for the same system. Implementation of formal models of real systems as well as verification is done with the help of the tool chain. 

\begin{figure}[h]
    \centering
    \includegraphics[scale = 0.2]{MX_Papers/Paper2/images/DT_REAL.png}
    \caption{Drilling station system.}
    \label{figure:RealImageofDrillTable}
\end{figure}

The drilling station system in Fig. \ref{figure:RealImageofDrillTable} is composed of several mechatronic components, among which, in this study we selected only the Drill and rotating Table. 
It is assumed that the mechatronic components are smart, i.e. they are equipped with their own control devices, implementing their basic operations, which are as follows. 

The Drill moves in an upward or downward direction. Whenever a workpiece is detected by the sensor under the drill, it moves downward and starts drilling. Once it completes drilling, it moves upwards and rests at the home position. 

The Table rotates from one fixed position to another. The cycle is completed when it rotates six times. 
When a workpiece is placed in the loading positions, the table rotates to bring it under the drill.

The control logic of each mechatronic component is implemented as a function block which follows the IEC 61499 standard. The function block diagram shown in Fig. \ref{figure:RealFBDiagram} consists of the two function blocks orchestrated to work together by means of event and data connections between each other and sources and sinks of sensor inputs and actuator outputs (function blocks on the left hand side and right hand side respectively). A close-up on the interacting controllers is presented in Fig. \ref{figure:RealFBControllers}.

It is assumed that the smart mechatronic components are delivered by their vendors together with the software components for implementing their control logic. They are integrated to the drilling station in a way, assuming that the control of internal operations in each mechatronic component is implemented by its predefined function block, and the integrator tries to minimise its software development effort by reusing the software components received from the vendors.
This approach requires exhaustive testing of the orchestrated system on compliance with functional and non-functional requirements. 

Hence, we will use both simulation in the loop with the plant model and formal verification in closed loop for more exhaustive exploration of the state space. 

\begin{figure}[h]
    \centering
    \includegraphics[scale = 0.4]{MX_Papers/Paper2/images/fig2updated.PNG}
    \caption{Function block representation of the distributed automation of drilling station.}
    \label{figure:RealFBDiagram}
\end{figure}

\begin{figure}[h]
    \centering
    \includegraphics[scale = 0.20]{MX_Papers/Paper2/images/DT_REAL_FB_CONTROLLERS.png}
    \caption{Close-up on the decentralised controllers' interaction.}
    \label{figure:RealFBControllers}
\end{figure}

\section{Simulation Model}\label{sec:simulation_model}

The simulation-in-the-loop environment is shown in Fig. \ref{figure:SimulationFBDiagram}. The original function blocks containing the autonomous control logic of drill and table are connected with the function blocks implementing simulation models of the drill and table respectively. The controllers are also connected with each other in exactly the same way as in the real configuration in Fig. \ref{figure:RealFBControllers}. 
For example, the function block TableMod11 of type TableModTop represents a model of the table that is driven by one control signal \texttt{fwd}. When this signal receives the value \texttt{TRUE}, the simulated table starts continuous rotation clockwise. The rotation is stopped if \texttt{fwd} resets to \texttt{FALSE}.

The model outputs the Boolean value \texttt{FixPos} which becomes \texttt{TRUE} when the table comes to one of the six fixed positions. To keep the table in the fixed position, the controller has to stop motion by resetting the control signal \texttt{fwd} to \texttt{FALSE}. Besides, the model produces the \texttt{WP\_DRILL} signal, which is the reading from the sensor indicating the presence of a workpiece under the drill.

The simulation environment reproduces the working behavior of the real plant which helps to visually identify the behavior of the system before deploying the distributed control to the real hardware. Besides, the errors identified during the formal verification can be represented in simulation. 

\begin{figure}[h]
    \centering
    \includegraphics[scale = 0.3]{MX_Papers/Paper2/images/SimulationFBsystem.png}
    \caption{The Function Block representation of the simulation-in-the-loop configuration.}
    \label{figure:SimulationFBDiagram}
\end{figure} 

The simulation models, used in this configuration, have continuous dynamics, which is the time-domain implementation of hybrid state machine as discussed in \cite{vyatkin2008closed}. 
The core part of the plant simulation function block is based on the hybrid automaton model of the process, as illustrated in Fig. \ref{figure:Hybrid}.
The state machine has three states corresponding to the static position of the moving object, such as the vertical axis of the drill, or the rotating table. These are \texttt{stHOME}, \texttt{stEND} and \texttt{stSTOP}. There are also two dynamic states, when the coordinate of the object is changing: \texttt{dMOVETO} and \texttt{dRETURN}. 

When the state machine is in the one of the dynamic states (say, \texttt{dMOVETO}), it emits the START output event which invokes the external \texttt{E\_CYCLE} FB, which starts emitting periodic events, activating the function block with the state machine. 
The state machine remains in the \texttt{dMOVETO} state until the position reaches the end position, i.e. Pos=DIST. Until then the loopback transition condition is true, so the state-machine remains in the \texttt{dMOVETO} state. Every time the loopback transition is executed, the event CHG is emitted which also invokes the external "Integrator" FB, which recalculates the new value of the process variable based on the current value. The duration of the time interval between recalculations is determined by the \texttt{DT} parameter of the \texttt{E\_CYCLE} and speed of the motion. 

\begin{figure}[h]
    \centering
    \includegraphics[scale = 0.36]{MX_Papers/Paper2/images/hybrid.pdf}
    \caption{Computational implementation of a hybrid automaton in function block.}
    \label{figure:Hybrid}
\end{figure}

Given the ever changing process values, the evolution of the model can be visually displayed using internal or external means. The development reported in this paper was done using the NxtStudio of NxtControl, which offers a proprietary visualisation technology called \texttt{CAT}. The plant model blocks were implemented as the \texttt{CATs}, therefore the model behaviour was implemented internally, within the same development environment. The interactive system visualisation by means of CATs is shown in Fig. \ref{figure:SimulationDiagram}.

\begin{figure}[h]
    \centering
    \includegraphics[scale = 0.30]{MX_Papers/Paper2/images/DT_HMI5.PNG}
    \caption{The visual representation of the simulation process.}
    \label{figure:SimulationDiagram}
\end{figure}

\section{Discrete-state Modelling Approach}\label{sec:discrete_state}

The discrete-state model of the system is created in IEC 61499 based on the simulation model described in the previous section. 

The discrete-state equivalent of the simulation configuration is shown in Fig. \ref{figure:FBver}. Here the function blocks simulating the drill and table are substituted by their analogs operating in the discrete state domain, instead of modelling the continuous process parameters, such as the drill's and table's numeric position.
The model can be then simulated in the IEC 61499 IDE with values of function block inputs and outputs displayed and modified interactively. It can be translated to the SMV model using the fb2smv tool and exposed to the formal verification by model-checking. 

\subsection{Notation for Plant Modelling}

\begin{figure}[h]
    \centering
    \includegraphics[scale = 0.65]{MX_Papers/Paper2/images/VerificationFB.png}
    \caption{Discrete state function block model of the Drilling station.}
    \label{figure:FBver}
\end{figure}

In the current example, the drill model is represented by an instance of a basic discrete motion model LinearDA. 
Its state-machine implementation is shown in Fig. \ref{figure:DrillECC}. The execution semantics of the state-machine follows the rules of IEC 61499, i.e. the function block is activated by an input event, and the state machine evolution is following the rules for execution control chart (ECC) of basic function blocks. 

Similarly to the hybrid state-machine in Fig. \ref{figure:Hybrid}, the discrete state model of the drill specifies three static and two dynamic states, but it does not model the position as a numeric value. The drill moves from \texttt{stHOME} to \texttt{stEND} state via a motion state called \texttt{ddMOVETO}. The state transition occurs from stHOME to ddMOVETO whenever FWD signal is TRUE. 
It is remarkable to note the NDT event input of the LinearDA function block, which remained unassigned in the application in Fig.\ref{figure:FBver}.
The NDT is reserved in the proposed modelling notation for Non-Deterministic Transition. 
Whenever the formal model generator encounters NDT in the state machine, it will interpret it accordingly. For example, the SMV modelling of NDT will be described in section \ref{sec:NDTinSMV} 

In terms of our model, the use of NDT in the transition from the motion state \texttt{ddMOVETO} to the static state \texttt{stEND} models the unknown duration of the motion from one state to another. Plant may fail whenever the \texttt{FWD} and \texttt{BACK} signals become \texttt{TRUE} simultaneously, so we introduced an ERROR state which helps to identify whether the controller creates this scenario.

\begin{figure}[h]
    \centering
    \includegraphics[scale = 0.4]{MX_Papers/Paper2/images/LinearDAFinal.PNG}
    \caption{Discrete state linear motion process model with NDT.}
    \label{figure:DrillECC}
\end{figure}

\subsection{Non Deterministic Transitions in controllers}

Non-deterministic transition can be also helpful for simplification of controller models containing timers. For example, in our case study, the controllers were developed as state machines with timeouts, therefore they are implemented in composite function blocks. In the drill, drilling process needs to be done for several durations, which is achieved with the help of The \texttt{E\_DELAY} function block. The composite function block consists of a real controller and \texttt{E\_DELAY} function block as shown in Fig. \ref{figure:DrillInterfaceControllers}(a). 

\begin{figure}[h]
    \centering
    \includegraphics[scale = 0.33]{MX_Papers/Paper2/images/Fig9.png}
    \caption{a) The real drill controller with external timeout. b) The interface of modified drill Controller with non-deterministic transition input.}
    \label{figure:DrillInterfaceControllers}
\end{figure}

However, formal modelling of the timers in SMV is computationally hard. It can be avoided if the concrete delay duration was substituted by non-deterministic transitions with the help of NDT signal. Therefore, the controllers can be modified this way in order to reduce complexity of model-checking. Therefore, we removed the timeout \texttt{E\_DELAY} substituted the corresponding input of the function block and added NDT input. The execution control chart of the real drill controller and the modified drill controller are shown in Fig. \ref{figure:DrillECCControllers}.

\begin{figure}[h]
    \centering
    \includegraphics[scale = 0.24]{MX_Papers/Paper2/images/Fig10.png}
    \caption{a) The ECC of the real drill controller; b) The ECC of the modified drill controller with a non-deterministic transition modelling the time delay.}
    \label{figure:DrillECCControllers}
\end{figure}

\subsection{Modelling of non-determinism in SMV}\label{sec:NDTinSMV}

SMV provides a way to accomplish non-deterministic choice by providing a set of values to the signal. The first statement is used to declare the variable NDT as a Boolean type and the second statement is used to initialize the NDT variable to either TRUE or FALSE value.

\begin{lstlisting}[breaklines,basicstyle=\small]
1 | VAR NDT:= boolean;
2 | init(NDT):= { TRUE, FALSE };
\end{lstlisting}

In every transition we are giving a provision to choose either TRUE or FALSE. This makes the NDT variable unpredictable in each transition.

\begin{lstlisting}[breaklines,basicstyle=\small]
3 | next(NDT):={ TRUE, FALSE };
\end{lstlisting}

Implementing non-determinism in every transition can be limited by introducing conditions in the next statement. If it is not required for the NDT variable to choose values in every transition then we can design like below:

\begin{lstlisting}[breaklines,basicstyle=\small]
4 | next(NDT):= case
5 | 	Condition: { TRUE, FALSE };
6 |     TRUE : NDT;
7 | Esac;
\end{lstlisting}

\section{fb2smv tool}\label{sec:fb2smv_tool}

The \textbf{fb2smv} tool \cite{fb2smv} is a model generator for generating SMV models of function block systems in IEC 61499. It is a part of formal verification tool-chain, that includes the model checker NuSMV and the tool for counterexample analysis in terms of the original FB system. 

The tool implements the formal model of IEC 61499 as per the modelling method present in \cite{drozdov2021formal}. In order to construct \texttt{SMV} code, the \textbf{fb2smv} tool uses Abstract State Machine (ASM) \cite{gurevich1995evolving} as an intermediate model. The tool takes IEC 61499 function blocks expressed in XML format as input and generates a formal model with the help of ASM semantics. According to \cite{drozdov2021formal}, the structure of \texttt{SMV} code consists of the declaration part and the rules part, which are the ASM rules.

The tool converts basic and composite function blocks and also includes more additional features like, limiting the boundaries of variables to reduce the state space, changing execution order of FBs, deciding the input event priority by changing its order etc. The proposed non-deterministic transitions notation has been added to the tool as a result of this work. 

\section{Results and Analysis}\label{sec:results}

In this paper, we demonstrated on a case study example, the use of IEC 61499 language for design and formal verification of cyber-physical automation systems. First, we designed the control logic of each mechatronic component in a drilling station and then implemented it as a function block in IEC 61499 standard. The simulation environment of the drilling station is developed with the help of the same controllers which are used in the real configuration. We reproduced the same plant behavior in the simulation model using the function blocks. In order to create a formal model of the system, we transformed existing function blocks of the plant models and controllers by adding non-deterministic transitions, using the proposed NDT notation. Using fb2smv we converted the functional blocks to SMV code, which was verified using NuSMV on a machine with Intel(R) core(TM) i7-10510U CPU @1.80GHz 2.30GHz with 32GB RAM. The simulate feature of NuSMV was used to check the formal model's working behavior. Simulating the SMV code in interactive mode gives us the provision to go through each state by giving appropriate sensor inputs. We can try all possible input combinations to verify whether the system behaves as we expected. We can also randomly simulate the traces or we can manually go through each state by selecting different input combinations.

The main drawback of simulation is that the number of possible behaviors can be too large or even infinite. Simulation can show the presence of bugs, not their absence. 

More comprehensive verification can also be done by model-checking various properties, i.e. checking whether a requirement is true or not in all possible execution traces of control logic. This will allow us to test critical scenarios where there can be failures in some combination of inputs. The formal model of the system was verified with help of CTL \cite{emerson1985decision} specifications. While testing the CTL specifications in NuSMV, we found the following statement is false so it is possible that the table can rotate while the drilling process is going on.

\begin{lstlisting}[breaklines,basicstyle=\small]
-- specification  G !(DRILL_TABLE_CFB3_inst.DrillCTL_RET = TRUE & DRILL_TABLE_CFB3_inst.ActuatorGen_EO = TRUE)  
\end{lstlisting}

While executing the above specification, the NuSmv gave a counterexample that contradicts this statement. The counterexample generation for the above specification took 26000 seconds to complete. The counterexample helps to identify the error in the controller's design. We tested the same logic in the simulation model, the table rotated from its current position to another while drilling the workpiece. The real system also exhibits the same issue.

To fix this issue we need to analyze the counterexample provided by the NuSMV, but identifying the variables changed in each iteration is difficult. The Nutrac tool provides a better way to understand each state. The Nutrac tool converts the counterexample to a CSV file. This CSV file's columns represent states and rows represent input/output events or data input/output variables. We analyzed the CSV file and identified the issue. The issue was present in the execution control chart of the table's controller. It is required to add the BLOCK signal and it should be checked before moving from DRILLED state to REMOVED state. We verified the CTL specification again and this time NuSMV gave the TRUE result. The modified controller is tested in the real object, as well as in a simulation system and it was behaving as we expected i.e table was not able to rotate while drilling is going on.

In order to identify whether the controller produces the \texttt{FWD} and \texttt{BACK} signals true at the same time, the following specification is used.

\begin{lstlisting}[breaklines,basicstyle=\small]
-- specification  G !(DRILL_TABLE_CFB3_inst.DRILL.Q_smv = ERROR_ecc)
\end{lstlisting}

After executing the above specification, the NuSMV produced the output as TRUE. The ECC of the drill model never goes to ERROR\_ecc state i.e. the controller never produces \texttt{FWD=TRUE} and \texttt{BACK=TRUE} condition simultaneously.  

This paper \cite{vyatkin2003verification} proposes a software tool called VEDA (verification environment for distributed application), which is used for closed loop modelling and verification of distributed control systems in intelligent manufacturing. Manually developed plants and automatically generated controllers are combined for model-based verification in IEC 61499 standard but it requires higher integration effort. The paper \cite{Time-AwareComputations1} \cite{Time-AwareComputations2} describes a closed loop system with a simulation model used to autogenerate SMV but the resulting complexity was prohibitive. In this paper, we propose a methodology to create a closed-loop model staying within IEC 61499 standard which produces lower complexity of model-checking than \cite{Time-AwareComputations1} \cite{Time-AwareComputations2} and reasonable engineering effort that is less than in \cite{vyatkin2003verification}.  

\section{Conclusion and Future Work}\label{sec:conclusion}

In this paper, we proposed a tool chain which helps to verify and analyze the function blocks implemented in IEC 61499 standard. This tool chain can be used for continuous development and evaluation of distributed control systems. With the help of this tool chain, it is possible to test the system quickly and efficiently. The accurate implementation formal model is necessary to identify all possible flaws in the system. The existing functionalities of fb2smv along with the non-deterministic transitions in function blocks help to provide a similar formal model of the system. Previously, the counterexample analysis was complicated but now the Nutrac tool solves this issue by giving a better representation of counterexample in CSV format. The developers working on complex system design can use this tool chain for continuous development and testing.

The non-deterministic transitions in function blocks can be extended by introducing NDT as a variable to the FBs. This NDT variable can be any of any IEC 61499 data type but the NDT variable should be able to choose one value from set values. For example, if we introduce NDT as a variable of integer data type and its values limited 0 to 5 then it should be able to randomly select one value from 0 to 5. In order to introduce more randomness to our formal model, it's better to implement NDT as a variable instead of using NDT as an event signal. An interesting field to explore is if we can generate the specification as well as the plant model automatically then existing manual interventions can be avoided. The tool chain which identifies all possible errors and fixes them automatically could be the next step in the future.

The models used in this paper are not based on time so the timing problems due to different set of time scales of controller and plant cannot be identified by this approach. The extension of notation to timed automata can be added for future work.

\section{Acknowledgements}
This work was sponsored, in part, by the H2020 project 1-SWARM co-funded by the European Commission (grant agreement: 871743).  

%%% Put references here
\putbib
\end{bibunit}

%-------------------------------------------------------------------
\def\paperheader{Paper C}
\def\papertitle{Formal verification of observers supervising a cyber-physical system implemented using IEC~61499}
\def\paperauthorstring{Polina Ovsiannikova, Etienne Le Priol, Vincent Perret, Pranay Jhunjhunwala, Midhun Xavier, and Valeriy Vyatkin}
\def\referencestring{Proceedings of the IEEE International Conference on Industrial Informatics (INDIN), 2021.}
\def\copyrightstring{2021, IEEE, Reprinted with permission.}

% The definitions above could just as well be put directly into the function
% call below, but were explicitly defined to more clearly illustrate the
% use of the function \makepaper.

\makepaperaccepted
  {\paperheader}
  {\papertitle}
  {\paperauthorstring}
  {\referencestring}
  {\copyrightstring}

% The actual contents is imported by un-commenting the \input line below.
% Make sure the file exist.
\begin{bibunit}
\thispagestyle{plain}

\section*{Abstract}
A rigorous check is a significant phase in the design process of control programs of safety-critical cyber-physical systems. Here, we consider such programs to be implemented using IEC~61499 standard for industrial automation. After the check is performed (for example, using formal verification), the engineer needs to ensure that even in unexpected situations, the system will not fail during the runtime, and for this online verification methods can be utilized.

In this work, we consider attaching monitors implemented as basic function blocks to the interface of the controller, thus having a property being monitored represented in the form of a state machine. Now, monitors make the system safer only if their quality is also ensured. Since their complexity is far lower than the complexity of the controller, they can be model checked, however, in the case of IEC~61499 function blocks, open-loop model checking will produce spurious counterexamples as it will allow combinations that are not possible according to the IEC~61499 function blocks semantics (e.g., data transferred without firing the event). The current work addresses this issue and proposes a method for close-loop model checking of monitors, using the non-deterministic twin of a controller under supervision. We present our approach using the system of two orthogonal pneumatic cylinders.

\section{Introduction}
%\PO{Continuing the topic of monitors in 61499 systems and their verification started in (Pranay, Monitoring design pattern for distributed automation systems in IEC 61499 and its formal modelling )}

%\PO{oflfine (static) verification of online (runtime) means for ensuring safety. }

%\PO{Works rarely cover testing of their observing mechanisms}

%\PO{There exist methods for monitoring IEC 61499 and methods for verifuinf IEC 61499, however there's none on verifying the monitors, which requires an interaction withthe system (?) }

%A rigorous check of safety-critical cyber-physical systems is crucial to industrial automation. 
Safety-critical cyber-physical systems must always comply with their requirements before they become operational. One of the important parts of the process of ensuring compliance is checking whether the control program (or controller) of such a system works as expected. This can be done using verification or validation approaches, which are united by the fact that none of them has a hundred percent coverage when dealing with complex industrial-sized systems. In the case of conventional testing or simulation, even if they are automated, some operational environment-related events might be left out and the industrial-sized system might be too complex to have all the possible test cases generated. Formal verification techniques allow the engineer to check the whole state space of the system; however, they suffer from a state-space explosion problem and of being computationally demanding overall when verifying complex systems. To combat this issue, various model abstraction techniques were developed to decrease model complexity (for example, to verify some particular functionality)~\cite{clarke2000,burch1992symbolic}. Another approach is bounded model checking, in which the executions of particular lengths are checked~\cite{biere2003bounded}. This, in turn, brings us back to the problem of missing a longer scenario that leads to failure.

Nevertheless, both approaches avail in finding the issues in the pre-operational stage, and what has to be added is an entity that would observe whether the particular property of the system holds during the runtime and communicate to an error handling system if the malfunction occurs. Such entities are called monitors or observers~\cite{17jhunjhunwala2022monitoring}. They can be internal or external to the system and perform the \emph{online verification}. Online verification has another advantage, i.e., if the changes have to be applied to the control program fast and there is a lack of resources to perform the global re-check, observers will maintain the safety state of the system by communicating the critical errors during the runtime.

In this work, our control programs are implemented following the IEC~61499 standard for industrial automation, and we consider internal monitors represented with basic function blocks, meaning that the properties of the system to be monitored are expressed as individual finite-state machines. This makes our monitors a better target for formal verification, and model checking, in particular, than the control program as a whole, since checking their whole state space requires less computational resources.

Model checking~\cite{clarke1999} is an approach for formal verification, where the formal model of the system is checked during the pre-operational stage, which is called \emph{offline verification}. In addition to a formal model of the system, model checking requires a formal representation of the properties of the system (for example, using linear temporal logic~(LTL)) as input. The model checker then derives all possible execution scenarios and produces counterexamples if the requirements do not hold. The counterexample for a requirement is such an execution scenario (or a system trace) where the requirement fails. 

Now, we propose to perform an offline verification (model checking) over the functional unit developed for online verification. Here, we face the issue that the traditional open-loop approach for model checking will produce spurious counterexamples due to the semantics of IEC~61499 function blocks~(FBs), where, for example, data cannot be sent or received without corresponding events being fired. 
We address this by modeling a non-deterministic twin of the controller and verifying a closed-loop model instead. We demonstrate our approach on a run-through example of two orthogonal pneumatic cylinders moving forward and backward. For model checking, we use NuSMV verifier~\cite{nusmv} and translate our FBs to its programming language (SMV) using the FB2SMV tool~\cite{drozdov2015fb2smv}.

The remainder of this paper is structured as follows. Section~\ref{sec:prelim} gives an overview of IEC~61499 standard and the verification methods of FBDs implemented according to it. Section~\ref{sec:method} describes a methodology for designing a supervised system, which is described in detail on a run-through example in Section~\ref{sec:methodappl}. Section~\ref{sec:concl} concludes the paper.

\section{Preliminaries}
\label{sec:prelim}
\subsection{IEC~61499}
IEC~61499 defines a design paradigm for distributed automation and control systems. The systems are implemented using the graphical language of function block diagrams~(FBDs). An FBD is a set of various interconnected FBs that can be of a basic, complex, or service interface type. In this work, we do not consider the latter. All FBs have their data and event input and output interfaces. The bricks of an FBD are basic FBs that represent atomic functional units. The logic of a basic FB is defined by an execution control chart~(ECC), which essentially is a Moore state machine and consists of states, transitions, and actions. In each state, an algorithm can be executed and/or an event (defined in the output interface of the FB) emitted. Complex FBs are nets of interconnected FBs of any type. The final FBD is assembled using the available FBs.

IEC~61499 systems are event-driven, unlike, for example, \mbox{IEC~61131-3}~\cite{tiegelkamp1995iec} systems that follow a cyclic execution pattern and \emph{event} is a key concept for the standard. Any event input or output can be bound to a subset of data inputs or outputs, respectively, which means that the corresponding data will be received and processed or sent only if the particular event fires. Intuitively, any update of the event variable opens the gates for the data connected to it.

In this paper, we create our observers using basic FBs, incorporating the logic of the condition to be monitored in their ECCs. As in~\cite{toolchain}, we use Non-Deterministic Transitions~(NDT) to create a non-deterministic twin of the controller of two pneumatic cylinders to verify the observers in a closed loop.



\subsection{IEC~61499 verification}
There exists a sufficient amount of literature on the topics of online (dynamic) and offline (static) verification of IEC~61499 FBDs. 

An overview of both static and dynamic verification approaches is given in~\cite{15blech2016comparison}.
\cite{12yoong2010verifying} proposes an approach of converting IEC~61499 FBs to Esterel and its subsequent verification.
In \cite{13yoong2015verification,14bhatti2011observer}, the monitors expressed as IEC~61499 FBs are added not for run-time verification but to better understand the counterexamples produced by the static verification approach. The system represented as a Kripke structure, together with observers and (possibly) computation tree logic~(CTL) formula are provided as input to the verification module. The authors suggest that, if the counterexample is received, the debugging process is simplified, as observers are inserted into the system structure.
\cite{11lindgren2016contract} considers static verification by enriching FBs with formal contracts and addressing verification on the component, algorithm, and ECC levels. In the works~\cite{agn_case_study} and~\cite{agnostic} the authors translate FBDs to SMV closed-loop formal models, \cite{toolchain} continues in this direction and presents a notation within IEC~61499 syntax for the subsequent generation of closed-loop formal models of FBDs and their verification by means of NuSMV. 


Examples of dynamic verification include, for example, \cite{1falcone2022runtime} that proposes adding enforcers to the application, which will not only monitor but adjust the supervised values in case of the property failure to ensure that the correct values are emitted by the controller. In~\cite{3do2020towards, 8ng2019contract}, the authors add assume and guarantee contracts in the form of FBs to the application to monitor the system during the run-time. \cite{4wenger15behavior} introduces behavioral runtime monitors into the 4DIAC framework. These monitors are generated automatically using service sequences extended with behavioral types. Combining various control and verification techniques, a reconfiguration architecture for fault handling in industrial systems is designed in~\cite{10leitao2020fault}. 


Approaches for dynamic and static verification, both, have their advantages and serve their purposes, hence, probably, the best way to ensure the system's correctness is to complement one with the other. However, despite the fact that there are numerous approaches for online verification, very few articles mention that the diagnosis units themselves must undergo a sanity check, which is especially important in relation to IEC~61499 FBs, where an occasional missing event in a transition condition can make a monitor erroneous. The point is outlined in~\cite{9wiesmayr2022supporting} and in~\cite{17jhunjhunwala2022monitoring} an approach to verify the monitors using Timed Net Condition-Event Systems was mentioned. We continue the work~\cite{17jhunjhunwala2022monitoring} and elaborate on the static verification of monitors expressed as IEC~61499 FBs and the challenges it brings. 
 
\begin{figure*}[htb]
    \centering
    \includegraphics[width=0.98\textwidth]{MX_Papers/Paper3/pic/wholesystem_withhmi.png}
    \caption{The system of two pneumatic orthogonal cylinders. \texttt{HorCyl} and \texttt{VerCyl} simulate the plant (horizontal and vertical cylinder correspondingly), while \texttt{HorCylCTL} and \texttt{VerCylCTL} compose a control program. On the right, the HMI of the system is shown with the situation that should be avoided~-- a simultaneous extension of two cylinders.}
    \label{system}
\end{figure*}
%Tu t'es trompé de réseau de Pétri, le réseau où les deux cylindres ne se touchent pas c'est l'autre, tu l'as ou pas? 

\section{Design methodology for a monitored control program}
\label{sec:method}

For an automation system to function according to its specification, it must be checked using offline verification methods and equipped with online verification mechanisms to ensure that the requirements hold in unexpected circumstances. For this, the runtime verification means should also be checked.
Thus, in this paper, we propose a design methodology for the control programs implemented in form of IEC~61499 FBDs with internal observers (or monitors) that solves the aforementioned problem. Given a plant model (plant simulation program), for a single monitor, the methodology involves the following steps:
\begin{enumerate}
    \item Implementation of the control program.
    \item Verification of the control program in a closed loop with the plant model provided. Depending on its size, the plant model can be simplified and the heuristics for reducing the resulting closed-loop formal model state space may be applied. If the issues are found, the engineer has to address them and repeat this step until the verification result is positive.
    \item Formulation of a desired property to be observed in the form of a finite state machine and its implementation as a basic FB, i.e., implementation of the monitor.
    \item Implementation of the simplified non-deterministic twin of the controller that can produce any combination of outputs based on any combination of inputs. Verification of the created monitor in a closed loop with the twin. The properties to be formulated for the verification procedure, intuitively, are the following: the monitor indicates a failure when the failure occurs and the monitor does not report a failure when the controller functions as expected. If problems are found, they should be eliminated, and the system re-checked. This step should be repeated until the verification yields no counterexamples.
    \item Verification of the monitor in a closed loop together with the real control program.
    \item Verification of the monitor in a closed loop together with the control program where the fault was injected. Steps~5 and~6 may employ bounded model checking techniques or heuristics for state space reduction. The steps must be repeated if there were found issues to be addressed.
\end{enumerate}

The main contribution is concentrated in step~4.
In the next sections, we demonstrate our methodology step by step on a run-through example of a system of two pneumatic cylinders, especially focusing on step~4.


%As a base for our work, only one model is needed : an existing system. The first aim is to add a monitor to this model. The monitor added is not trustworthy, it has to be verified. 

%In order to avoid a kind of dependence to the current version of the system which needs to be checked, the monitor has to be verified independently. These formal verification have to ensure that the monitor would be able to output the correct result based on its inputs.\\
%In this paper, different kinds of formal verification of monitor will be approached. To prove and demonstrate that the method used is correct, we will use two systems with their behaviour verified. The first system is fully functional, while the second one is a dysfunctional analog of the first, with the error known. Those system are described in the Paragraph \ref{study_case}~~.

%For formal verification we used LTL expressions.

%The aim of this study is to reduce the number of checks of the entire system due to slight/little/minor modifications. Our idea is relies in the follows. Once the monitor is fully checked, it becomes trustworthy and able to check the behavior of the system instead of the formal verification.
%
%v2
%As a base for our work, only one model is needed: an existing system. The first aim is to add a monitor to this model. The monitor added is not trustworthy; it has to be verified. Different kinds of verification are needed to ensure that the monitor would work and would be able to verify our system. 
%The first verification considers the internal functioning of the monitor. A formal verification is applied to the monitor involving only its internal variables. The objective is to check the monitor's ability to output the correct result.
%The second verification is about its externals links. A second formal verification is applied to the monitor. This verification involves some of its internals variables and variable from the system. %pas clair
%The aim is to verify the functioning connections between the system and the monitor : its ability to correctly 'read' the working inputs.
%Once the monitor is fully checked, it is trustworthy and able to check the system's behaviour. This method allows us to avoid formal verification of entire systems and subsystems which would cost a significant amount of time especially for minor modifications.
%These formal verification are ensured thanks to LTL expressions.
%In this paper, to simplify the study at first, the system studied is a verified working system in order to ensure that mistakes only come from the monitor and not from the system. 
%To demonstrate the reliability of a formally verified monitor, this method will be applied a second time on the same system with behaviour modifications.
%The method described above will be applied a second time/once more with a dysfunctional system in order to prove the efficiency of the method. The dysfunctional system is the same model with voluntary dysfunctions that will be described in the following paragraph.
%
%v1
%As a base of our work, two models are needed : an existing model working as expected and the same existing model with voluntary dysfunctions. The first aim of adding same monitors to those models is to ensure that monitors work and are able to detect bugs that we have implemented. The systems are checked by Online verification.
%The second aim is to do formal verification of the models. LTL expressions have to contain variables from the working system and from the monitor. For the working model, the issue is to ensure that the monitor doesn't detect error which don't exist. For the dysfunctional model, LTL expression issue is to ensure that the monitor only detect errors. Once specification of LTL expressions are respected, it means that the formal verification from the system is the same as the formal verification of the monitor. The monitor is trustable.\\
%To confirm again the trustworthiness of monitors, systems studied here will be modified. The modifications applied do not have to change the main function of the system. Same process then will be applied to our modified system : multiple formal verifications with different LTL expressions. Finally, in the case of monitors and formal verifications outputting same 'things', it would proved the reliability of monitors checked with formal verification.
%This  verification is useful for more complex system. Offline verification is a real loss of time in the case of minor modifications. Limit of formal verification are not known thus this method implies to work with a simple model in order to avoid first endless calculations. 

\section {Methodology application}
\label{sec:methodappl}
\subsection{Steps 1 and 2: control program implementation and verification}

Our task was to create a control program for two orthogonal pneumatic cylinders that move towards each other and back by pressing a switch button. The plant simulation model was implemented as an IEC~61499 FBD in EcoStruxure Automation Expert (EAE).

The aim of the controller was to allow both cylinders to extend and retract once the button was pressed. The following safety requirement was to prevent the cylinders from colliding.
Thus, we implemented the priority system so that the horizontal cylinder can start moving only when the vertical cylinder is retracted. The complete system is presented in Figure~\ref{system}. Here, the control program consists of two controllers (for vertical and horizontal cylinders) of the same FB type that are connected in a closed loop with their plant models (cylinders). When the button is pressed, the command \texttt{START} is sent to the controllers and the cylinder with the higher priority gets the command to start its moving cycle (extend and retract). After the controller recognizes that its cylinder is retracted, it sends the priority giving event (\texttt{give\_PRIO}) to the vertical cylinder and sets the variable \texttt{LET} to true, allowing the vertical cylinder to take its turn.




%As it is stated in the Section~\ref{method}, there exist two versions of this system, i.e., correct and erroneous. Both of those systems are built using five Basic FBs. The first one is the controller(prev. switch?), once "pressed", it allows cylinders to move. Then, there are blocks that model the physical device of each cylinder (alternatively, we can call them plant blocks). They execute order of the command bloc such as "EXTEND", "RETRACT" or "STOP". Finally, there are command blocks of the horizontal cylinder and of the vertical cylinder. Those blocs receive information from the switch FB and send orders to the plant FB.
%The two cylinders are controlled and described the same way by the same FBs (same class but different objects).

%The most crucial requirement for our system is that the cylinders must never collide. Figure~\ref{system} shows the FB network of our correct system and its function is illustrated with a Petri net.\\
%Each cylinder has a priority, in our case, the priority of the vertical cylinder is higher than of the horizontal one. Therefore, when the button is pressed and both of variables \texttt{START} become \texttt{true}, the cylinder with a higher priority moves first. 
%In the end of its motion, the horizontal cylinder sends the priority giving event (\texttt{give\_PRIO}) to the vertical cylinder and sets the variable \texttt{LET} to true, allowing the vertical cylinder to take its turn.
%Thus, the system works in an infinite cycle. The system is forced in such a way that they cannot collide but neither are they able to extend a the same time no matter how many times the button is pressed.

In our case, the system is of a moderate size and we can apply model checking as is, without abstracting the system or using bounded model checking.
%To ensure that those models behave as expected, we model checked them using NuSMV verifier. 
Thus, we translated our system to SMV code using the FB2SMV tool and checked the following LTL formula with NuSMV verifier: \texttt{G $\neg$(Ver.EXTEND $\land$ Hor.EXTEND)} (where \texttt{Ver} and \texttt{Hor} are \texttt{VerCylCTL} and \texttt{HorCylCTL} correspondingly), 
%{\small$$G!(Ver.EXTEND=TRUE ~~\& ~~Hor.EXTEND=TRUE)$$}
which means that at every discrete time step, the two cylinders never get the command to extend simultaneously. The verification result was successful. 






%v1
%The method was implemented in the following case study: two perpendicular double-action cylinders commanded by a switch press button. The model is represented in Fig. 1. It is designed with IEC61499 standard through \textit{EcoStruxure Automation Expert (EAE)}.
%
%There are many different kinds of model of this system. First, we have the FB network which allows us to simulate with the HMI interface (Human Machine Interface) represented in Fig. \ref{petri_hmi_iec}(c). This model uses CAT FBs. CAT FBs are used to connect the Canvases objects in Fig. \ref{petri_hmi_iec}(b) and defined behaviours as forward and backward motion. They are plant FBs. However CAT FBs can not generate fbt files which are needed to do formal verification.
%Thus, we have a second FB network model allowing for a formal verification. CAT FBs are simply just substituted with basic FB also describing, thanks to ECCs, the forward and backward motion. These plant basic FBs' ECCs are built with NDT events which can be emitted at any time. It allows the formal verification to create many different combinations. \cite{toolchain} %of variables
%\\
%In both models, command of cylinders is controlled by basic FB. It is used to receive information from the switch press button and send information to the plant FB.
%The two cylinders are controlled and described by the same FBs (same class), with only objects being different.
%
%As said in the Method paragraph, we need two systems : the first one have to work perfectly and the other have to be dysfunctional. 
%The correct behaviour here is the case where cylinders never collide. Fig. \ref{petri_hmi_iec}(a) and \ref{petri_hmi_iec}(b) shows the FB network of our working system and it global operation illustrated with a Petri net.
%\\
%An order of priority has been defined upstream. During initialisation, the vertical cylinder gives priority to move to the horizontal cylinder. Once the switch button is pressed a 'START' signal is sent to the horizontal and to the vertical command FBs. To move, first, both cylinders need to have priority. Cylinder then will be in a waiting 'START' event state as it is shown in Fig. \ref{petri_hmi_iec}(b).

%Thus, the horizontal cylinder goes first. Just before the end of its motion, the command FB of the horizontal cylinder gives priority to the vertical cylinder letting it in the waiting 'START' event state and, at the end of its motion, sends a 'START' event: 'GO'\footnote{\label{GO}IEC 61499 rules does not allow two different outputs for a single data input. Another must therefore be created.} to the vertical cylinder, allowing it to move. 


%The system describes an infinite cycle where the horizontal cylinder moves first. Once the button is pressed it cannot be stopped. The system is forced in such a way that they cannot collide but neither are they able to extend a the same time no matter how many times the button is pressed.

%The dysfunctional system is different in terms of command FBs. There is no more priority rule. Once the switch button is pressed, the cylinders are both able to move at the same time and thus to collide as shown in Fig. \ref{petri_col}.
%both systems have
%The system has been checked firstly with HMI first and then with formal verification. Formal verification is described with the following LTL expression: 
%{\small$$G!(Ver.EXTEND=TRUE ~~\& ~~Hor.EXTEND=TRUE)$$}
%This means that for every combination of value of variables (G), the two cylinders never (!) extend at the same time. Expected behaviour has been observed, the LTL expression is always true in our assumed working system .
%and multiple counter-example have been found in our assumed dysfunctional system. 


\subsection{Step 3: Monitor implementation as a basic FB}


The next step was to create a monitor that would observe whether the safety property (i.e., two cylinders should never receive the command to extend simultaneously) is maintained by the system during the runtime. We implemented the monitor as a basic FB and assumed that it would be connected to the controllers as in Figure~\ref{fig:monitorplacement} (other connections are not depicted for the sake of clarity).

%The monitor has to observe the system the same way (as) the previous formal verification (did). This means that it has to check that the two cylinders never extend at the same time. 

\begin{figure}[h!]
    \centering
    \includegraphics[width=0.35\textwidth]{MX_Papers/Paper3/pic/monitorplacement.png}
    \caption{Monitor \texttt{NoCollisionMonitor} connected to the controllers (other connections are not shown for clarity).}
    \label{fig:monitorplacement}
\end{figure}

%This monitor was implemented as a basic function block. 
The monitor receives the \texttt{EXTEND} outputs from both controllers together with their events \texttt{CNF} and communicates whether the collision is occurring by setting its output \texttt{collision} to \texttt{true} and triggering the \texttt{CNF} event. The ECC of the monitor is presented in Figure~\ref{fig:monitorecc}.

%Its ECC, Figure~\ref{ECC} is able to know (when and) which cylinder is extending. 'OBSERVER' FB only returns 'collision := TRUE' when the 'HextANDVext' state is reached.

%After multiple online tests with the functional systems, the monitor seems able to output a consistent result with our system behaviour.
%The monitor shown here can present mistakes, so it is susceptible to change depending on results of its formal verification.

\begin{figure}[h!]
    \centering
    \includegraphics[width=0.48\textwidth]{MX_Papers/Paper3/pic/monitorecc.png}
    \caption{Interface of the monitor and its ECC. Algorithms \texttt{OK} and \texttt{COLLISION} set output variable \texttt{collision} to \texttt{false} and \texttt{true} correspondingly.}
    \label{fig:monitorecc}
\end{figure}

\subsection{Step 4: monitor verification with a non-deterministic twin of the controller }
\label{sec:monitorver}
The next step is to verify the implemented monitor in a closed loop with the controller (which we decompose into two controllers~-- for vertical and horizontal cylinders). In our case, the controllers are represented with basic FBs, and the closed-loop model checking will produce the result within an acceptable time interval. However, we will have to close the loop on the inputs of the controllers as well, meaning that we will either have to model simplified plants or provide other means for the controllers to produce all the possible combinations of their output values. Moreover, even if we succeed in closing the loop on the controllers, in case issues are found during the model checking, counterexamples will be hard to decipher as they will elongate and dozens of additional variables will be added to the system state.

Thus, we propose creating a \emph{non-deterministic twin} of the controller, which will be represented with two FBs of the same type just like our individual cylinder controllers. Now, let us disclose the notion of a non-deterministic twin~(ND twin). 

To model check the monitor, it is important that the model checker would infer all the possible combinations of its inputs, meanwhile triggering the events associated with them. Therefore, the goal of an ND twin is to abstract the logic of the controller by saving its key states, where the variables that are monitored change and add the non-deterministic transitions~(NDT) between them. Such events in an FBD turn into non-deterministic inputs in the SMV module of the corresponding FB, generated by FB2SMV as described in~\cite{agn_case_study}. 

The ECC of our individual controller of the cylinder together with its ND twin are presented in Figure~\ref{fig:ndtwinecc}. Two states that change the supervised variable \texttt{EXTEND} are \texttt{EXT}, \texttt{RETRACT}, and \texttt{STOP}. We kept them in the ND~twin and added NDTs between them so that at any time any output could be produced. From the algorithms, we remove everything that does not influence the variable \texttt{EXTEND} and leave only the \texttt{CNF} event associated with it. Figure~\ref{fig:ndtwinloop} shows two ND~twins of the controllers connected to our monitor.

\begin{figure}[t]
    \centering
    \includegraphics[width=0.5\textwidth]{MX_Papers/Paper3/pic/ndtcontrollereccs.png}
    \caption{ECCs of the cylinder controller (to the left) on of its ND~twin (to the right).}
    \label{fig:ndtwinecc}
\end{figure}


\begin{figure}[b]
    \centering
    \includegraphics[width=0.34\textwidth]{MX_Papers/Paper3/pic/ndtwinloop.png}
    \caption{Connection of the monitor to ND~twins of the cylinder controllers.}
    \label{fig:ndtwinloop}
\end{figure}

When the system is converted to SMV, the first step is to verify that we can indeed get all the values of the monitor input, that is, our ND twins function according to their specifications. The set of properties that make up the specification can be created according to the following template. 

Assume that we have a set of all the supervised variables in a single ND~twin $U$.
$\forall u \in U, \exists D_u$, where $D_u$ is a domain of $u$. Then, $\forall u \in U, \exists V_u = \{(u,v) \:|\: v \in D_u\}$, where $V_u$ a set of all possible assignments of $u$. Now, a set of all the possible combinations of the variables values is $C = V_{u_1}\times\ldots\times V_{u_n}, n = |U|$. Having $C$, we can now formulate the CTL specifications to be checked as a set $S$:
$$S = \{\textbf{AG} \:\neg(\bigwedge_{p \in P} u(p) = v(p)) \:|\: P \in C\},$$
where $P$ a set of variable and value pairs, $u(p)$ returns the variable name of pair $p$ and $v(p)$ its value. Thus, if any of the specifications from $S$ are satisfied, it means that their corresponding combination cannot be generated by a formal model of an ND~twin.
Similarly, we check if the observer can get all the possible combinations of input values.

In our case, we checked the following specifications for the ND~twins (here, \emph{twin} is replaced with the corresponding names of the FBs for horizontal and vertical cylinders controllers twins): \texttt{AG twin.EXTEND} and \texttt{AG $\neg$twin.EXTEND},~-- and four specifications for the monitor of type \texttt{AG $\neg$(NoCollisionMonitor.hext $\land$ NoCollisionMonitor.vext)} (we omit others to save space). All the specifications failed, which means that all the combinations were possible in our model. 

Now, that we know that our model produces all the combinations of inputs, we can formulate the properties of the monitor to be checked. The first one tells that whenever the monitor gets the signals that both of the cylinders extend simultaneously, it produces the warning. Its LTL equivalent is: \texttt{\textbf{G} ((m.hext $\land$ m.vext) $\rightarrow$ (m.hext $\land$ m.vext \textbf{U} m.collision))}, where \texttt{m.} is short for \texttt{NoCollisionMonitor.}. Here we use operator \textbf{U}~-- ''until'' instead of \textbf{X} (''next'') or no temporal operator due to the specifics of the generated SMV model, where several model evaluations should occur to obtain the result of the FB output.

The second property tells that if the cylinders are not extending simultaneously, the collision is not reported, which is in LTL: \texttt{\textbf{G} ($\neg$(m.hext $\land$ m.vext) $\rightarrow$ \textbf{F} $\neg$ m.collision)}.

Both properties were satisfied by our monitor.

%A new solution has been provided in this paper. Formal verification of the monitor alone could not deal with connections and it is hard to understand how formal verification emits the different event and datas of the monitor. To meet the needs of it, a new system is created and modeled. This system would be a kind of a twin of the original system able to create every possible case. A first version is designed.\\

%\begin{figure}[h!]
%    \centering
%    \includegraphics[width=0.48\textwidth]{MX_Papers/Paper3/pic/jumeau_V1.png}
%    \caption{First version of the twin two cylinders system}
%    \label{twin1}
%\end{figure}

%The twin system is composed of two FBs. The first one is the monitor and the other one is a simple basic FB able to output, thanks to a 'NDT' event, a 'EXTEND' event directly connected to the event 'REQ' of the monitor. moreover, datas of the monitor are forced as TRUE as seen in Figure~\ref{twin1}. Those 2 blocks are implemented in a composite FB and composed its FB network.\\
%During the formal verification, the event 'NDT' will be emitted randomly which will activate the 'REQ' event and thus the datas. LTL expression used here is the~(\ref{LTLmonitor}) one. A counterexample is found.

%As seen in the counterexample visualizer (Figure~\ref{vizu_twin1}), the monitor is still stuck in the 'OnlyH' state whereas Hext and Vext datas become TRUE at the same time. The twin system has to be improve but it already allows us to find some mistakes in the monitor. In facts, if the two datas become TRUE at the same time, in the ECC, it does not exist any transition for this case. Transitions are added. Then, if the system is stuck it's also because it can' read correctly transitions. The 'REQ' event is emitted once for two datas. It needs to be split into two different input events each will be link to one of the two datas. 'REQ\_h' is created and connected to 'Hext' and 'REQ\_v' is created and connected to 'Vext', Fig. \ref{monitorV2}.\\
%Now that two 'REQ' events have been created, a another bloc is added to the system in order to also trigger one of the 'REQ\_h/v ' event. A second version of the twin system is designed (Figure~\ref{twin2}).

%\begin{figure}[h!]
%    \centering
%    \includegraphics[width=0.48\textwidth]{MX_Papers/Paper3/pic/mointorV2.png}
%    \caption{Monitor and its ECC modified}
%    \label{monitorV2}
%\end{figure}

%A new formal verification is done, same LTL expression. Another counterexample, Figure~\ref{vizu_twin2}, is found. However, if watched more closely, the monitor is able to output that a collision happen but after few steps. Those few steps represent only steps needed to reach the good state and the time to activate the action allowing the monitor to output 'collision=TRUE'. The monitor here seems to work but to be more specific and make the formal verification founds no counterexample, the LTL expression needs to be modify. Using the function 'F', for future, would allows to reach our goal. The LTL expression now becomes :
%\begin{center}
%    $G(~(OBS.Hext = TRUE~~\&~~OBS.Vext = TRUE)$\\
%\end{center}
%\begin{equation}
%    \xrightarrow~~\textit{\textbf{F}}~~OBS.collision = TRUE)
%    \label{LTLmonitor2}
%\end{equation}


\subsection{Steps 5 and 6: monitor verification with the real and erroneous controller}

Now, we add the monitor to the system and check whether we connected it properly by verifying the system with the monitor as a whole. If the system at step~2 was checked using state-space reduction techniques, they can be applied here as well.

To check whether the monitor works when there is an error in the system, we inject the fault manually. In our case, we remove the priority mechanism from the controllers so that, by pressing the button, both cylinders extend simultaneously. The ECC of the erroneous cylinder controller is shown in Figure~\ref{fig:errctlecc}.

With both systems, we partially verify the same properties as in Section~\ref{sec:monitorver}. In both cases, we check whether all combinations of input values are possible for the monitor. The correct system reports that there is no possibility for two cylinders to get the extension command simultaneously, while only this is possible in the malfunctioned system. Then, we check the monitor properties, i.e., whether it reports the collision when it happens and does not report spurious collisions. Since the collision is not allowed in the first system, we omit the first monitor property, when checking the correct system, similarly, we omit the second property, checking the erroneous one. In both systems the checked properties hold, thus we can conclude that the monitor and its connections are correct.


\begin{figure}[t]
    \centering
    \includegraphics[width=0.34\textwidth]{MX_Papers/Paper3/pic/errctlecc.png}
    \caption{The ECC of the erroneous cylinder controller.}
    \label{fig:errctlecc}
\end{figure}

%Now it's to see if the monitor is still working while implemented in the correct model and the erroneous one. A new formal verification is made. No counterexample has been found in both model. The monitor is verified formally.




%The dysfunctional version of the described system is different in terms of command FBs. There is no more priority rule. Once the switch button is pressed, the cylinders are both able to move at the same time and thus to collide as shown in Figure~\ref{HMI_collide}.

%\begin{figure}[h!]
%    \centering
%    \includegraphics[scale=0.5]{MX_Papers/Paper3/pic/HMI_collide.png}
%    \caption{HMI representation of the two cylinders system colliding}
%    \label{HMI_collide}
%\end{figure}

%Monitor works online but there could be a transition path where a failure could appear. These paths can be determined with formal verification. The main aim of the monitor formal verification is to ensure that its internals are correctly designed. Misconceptions inside the monitor FB could be detected, i.e., an unreachable state.

%As said in the Section~\ref{method}, the monitor has to be checked independently from the system it has to observe. The first way to do it is to do the formal verification of the monitor alone. Indeed, from its own .fbt file, it is possible to do a formal verification. This formal verification would only involve internal inputs and outputs events and/or datas of the monitor ECC.\\
%The LTL expression has to verify the ability of the monitor to output the correct result depending on inputs. The LTL expression and its opposed should verify all the different cases.
%\begin{center}
%    \textit{G( (OBS.Hext = TRUE~~\&~~OBS.Vext = TRUE)}
%\end{center}
%\begin{equation}
%    \xrightarrow~~OBS.collision = TRUE)
%    \label{LTLmonitor}
%\end{equation}\\
%It means every time (G), if inputs \textit{Hext} and \textit{Vext} are true then the output \textit{collision} must be true.

%As a result, the LTL expression is always true, no counterexample has been found. This result allows to make a conclusion : the monitor works as expected.

%To see if it's true and if this kind of formal verification of monitors is enough to demonstrate that monitors are trustworthy, the monitor was added to both versions of the system (correct one and erroneous ones).\\
%When implemented in the functional system, checked with the same LTL expression~(\ref{LTLmonitor}), no counterexample is found. However, the system is functional so collision never happens. CHECK THIS It means that only the opposed of the LTL expression \ref{LTLmonitor} has been checked here, which is LTL expression~(\ref{LTLopposed}).
%\begin{center}
%    $G(OBS.collision = FALSE)$
%\end{center}
%\begin{equation}
%    ~~\xrightarrow~~(OBS.Hext = FALSE~~OR~~OBS.Vext = FALSE)
%    \label{LTLopposed}
%\end{equation}

%When implemented in the erroneous system, formal verification founds mistakes. Indeed, when the two cylinders are extending, the monitor is not able to output that there is a collision. Thanks to the counterexample of the visualizer, Figure~\ref{vizu_dys}, it is possible to understand what happen. Actually, the monitor is stuck in the 'OnlyH' state which outputs "no collision". Lots of hypothesis about the source of the problem but from now only a conclusion about this formal verification can be done: the formal verification of monitor only based on their .fbt file is not enough.



%The monitor works online but there could be a transition path where a failure could appear. These paths can be determined with formal verification.
%\\
%The main aim of the monitor verification is to ensure that its internals are correctly designed. Misconceptions inside the monitor FB could be detected by formal verification ie an unreachable state.
%
%In our case, the monitor FB is a basic FB so it is defined and composed by ECC.
%
%\subsubsection{Internal behaviour}
%First, only doing the formal verification of the monitor has been thought.\\
%This formal verification would only involve internal inputs and outputs events and/or datas of the monitor ECC and would be done directly from the \textit{fbt} file  of the monitor.
%\\
%The LTL expression has to verify the ability of the monitor to output the correct result. 
%LTL expression (ECC monitor alone)
%\begin{center}
%    {\footnotesize $G ((OBS.H\_ext = TRUE~~\&~~OBS.V\_ext = TRUE)$}\\
%    {\footnotesize$\xrightarrow~~OBS.collision = TRUE)$}
%\end{center}
%It means every time (G), inputs \textit{H\_ext} and \textit{V\_ext} are true then the output \textit{collision} must be true.
%\\
%As a result, the LTL expression is always true. It means that the ECC of the monitor works as expected.
%\\

%\subsection{Connections verification}
%However, the aim of a monitor is not to work alone, it has to be implemented. Moreover, the previous verification can not permit to conclude about the real behaviour of the monitor. In facts, to be fully checked, the monitor has to be simulated in a system which able to submit all the different possibilities of input combinations.
%
%A new system is created Fig. \ref{twin}. The aim is to recreate a simplified copy of the original system allowed to generate non deterministic events. Thanks to those non deterministic events and the NuSMV tool, the system will be able to generate all the different possibilities of input combinations.
%It composed of three basic FBs. One of them is the monitor. The two other FBs are twin of the cylinders command FBs. They are based on the same FB.
%\\
%As a basic FB, it is described by an ECC shown in Fig. \ref{ECC_twin}. Non deterministic events allows the state machine to evolve between its several states. The transition are called non deterministic transitions (NDT).
%\\
%When the \textit{'NDstate'}, for non deterministic state, is reached, the FB outputs the variable \textit{'EXTEND'} as true. Those outputs are directly connected to the observer bloc, one to \textit{'H\_ext'} and one to \textit{'V\_ext'}. The \textit{'EXTEND'} variable value is reset after that a second ND event (NDT2) is emitted.
%\\
%A quick online verification is made to ensure that the result is the same as the first online check in \textit{IV.B}.
%
%The system is verified offline with the following LTL expression :
%\begin{center}
%    {\footnotesize $G ((HorTwinCTL.EXTEND = TRUE~~\&~~VERTwinCTL.EXTEND = TRUE)$}
%    {\footnotesize$\xrightarrow~~F~~obsTEST.collision = TRUE)$}
%\end{center}
%
%This LTL means that every time (G) the \textit{'EXTENT'} variable of both basic FBs is true then the variable %\textit{'collision'} of the observer must be true instantly or in some states later (F).
%
%As a result, NuSMV finds counter-examples of this LTL expression. Which means that our monitor displays misconceptions.
%After being tested with all the different possibilities of inputs in a similar system, We thought our monitor could be considered as a certified working one.



%%%%%%%%%%%%%

%The monitor is 
%A first LTL expression has been test : 
%\begin{center}
%     {\footnotesize $G ((HorCTL.EXTEND=TRUE~~\&~~VerCTL.EXTEND=TRUE)
%     \xrightarrow ~~  (obs.collision=TRUE))$}
%\end{center} 
%But it return counters example. In fact, this LTL expression pretends that the detection of the monitor is instantaneous. In the detail of counter-example file, it is because the monitor detects the error several state after the two cylinders extend(ed/ing) at the same time. Therefore, we tried another approach :
%\begin{center}
%    {\footnotesize $G ((HorCTL.EXTEND=TRUE~~\&~~VerCTL.EXTEND=TRUE)
%     \xrightarrow~~$}
%     {\small $F(obs.collision=TRUE))$}
%\end{center}

%It means that the detection will arrive in the future (F) or, in other words, several states after the both EXTEND signals (verical and horizontal) came true . The potential drawback is that the detection arrive too late, but this can't be quantify in this case. Nevertheless, it is a starting point to avoid the previous issue. After implement this formula in Nusmv, it return that the LTL statement is TRUE.  

%%%%%%%%%%%%%%%%



%\subsection{Monitor implemented in the working system}
%\vspace{0.2cm}
%Now, the aim is to ensure that the monitor is correctly implemented in the system. The formal verification permits to test the connections of the monitor with the system. 
%\\
%The verification condition of our system is simple so are the connections between the system and the monitor. As said in the section \textit{B. Adding monitor} from the paragraph \textit{IV. STUDY CASE}, the monitor only reclaims two outputs from the command FB of each cylinder. A simple connections is needed (fig. \ref{monitor}). With the same LTL expression schema, this condition is checked.
%
%LTL expression (system+monitor in S)
%\begin{center}
%     {\footnotesize $G ((HorCTL.EXTEND=TRUE~~\&~~VerCTL.EXTEND=TRUE)
%     \xrightarrow ~~  F (obs.collision=TRUE))$}
%\end{center} 
%
%
%explanation of the expression
%This expression means : "IF the horizontal cylinder is extending AND (\&) if the vertical cylinder is also extending THEN ($\rightarrow$) the monitor must always (G) return "collision=TRUE". 
%\\
%It return that this LTL expression is always true.
%
%However, the system known as a working system, the variable 'EXTEND' of both cylinder can not be TRUE at the same time. Thus the LTL expression is not directly verified, only is its contraposition :
%\begin{center}
%    {\footnotesize $G(obs.collision=FALSE) \rightarrow~~$}\\ 
%    {\footnotesize $!~(HorCTL.EXTEND=TRUE~~\&~~VerCTL.EXTEND=TRUE)$}
%\end{center}
%
%\begin{figure}[h!]
%    \centering
%    \includegraphics[width=0.45\textwidth]{MX_Papers/Paper3/pic/IEC_syst_col.JPG}
%    \caption{IEC model design of the two cylinders able to collide}
%    \label{IEC_collide}
%\end{figure}

%\subsection{Monitoring with modifying the system} 
%\vspace{0.2cm}
%ajouter une transition pour dire que ca marche avec un system modifié en tout cas inchallah
%
%Now that monitor ECC is working and connections with the system are correctly established, our monitor is trustworthiness and able to fully diagnose online systems errors. To prove it, let's implement our monitor in a slightly different system.   
%\\
%The previous part assure that the monitor does not detect false error. However, we don't know is behaviour when it's linked to a system which include misconceptions.   
%
%Thus, we introduced a different system.
%it's actually the same as previously but with behaviour modifications.
%
%idée d'Etienne : il vient pas de poitier, permettre au système d'ateindre tout les états possible (ie même ceux "interdits") 
%
%Il n'a pas detecté d'erreur inexistante mais on ne l'a pas confronté à un système pouvant comporter des erreurs. Certe, ses entrèes sont cohérentes avec ses sorties (cf première LTL) mais (sinon on a juste à s'arréter à la vérif du ECC tout seul)) pas ouf 
%
%Nous avons vérifié les connections avec un système qui marche, nous avons vu que ces connections ne créent pas de fausses erreurs mais permettent-elles d'en detecter? c'est ce que nous allons voir en l'ajoutant à un système comportant des erreurs.
%
%
%
%%It is different in terms of command FBs. There is no more priority rule. Once the switch button is pressed, the cylinders are both able to move or not at the same time and thus to collide. As seen in Fig. \ref{IEC_collide}, the model design is almost the same and Fig. \ref{petri_collide} represented its Petri net.
%\\
%\begin{figure}[h!]
%    \centering
%    \includegraphics[width=0.3\textwidth]{MX_Papers/Paper3/pic/systeme_col.JPG}
%    \caption{Petri net  model of the two cylinders able to collide}
%    \label{petri_collide}
%\end{figure}
%
%% implementation of the monitor and online verification of the system
%The monitor is implemented in this system. It is connected the same way as it was in the previous %system : the monitor reclaims the 'EXTEND' variable from the horizontal and the vertical %cylinders (Fig. \ref{monitor} and Fig. \ref{IEC_collide}).
%%The monitor and its connections with the system, identically as before,  does not need any %formal verification. Pas compris
%
%Online verification may be run with the modified system. The monitor finds out that, at some %point, the two cylinders are extending at the same time. 
%\\
%If it sticks to the method plan, mistakes should come from the system - the monitor tells the %truth. To confirm this hypothesis, the system and the system with the monitor have to be %checked thanks to formal verification.
%
%FV of the system but no HMI verification just offline
%The system has been first checked alone - including only variables from the system. Formal %verification is described with the same LTL expression as the first system :
%{\small$$G!(Ver.EXTEND=TRUE ~~\& ~~Hor.EXTEND=TRUE)$$}
%
%Expected behaviour has been observed, the LTL expression is false, a counter-example file have %been returned with the expected mistake.
%\\
%
%Online, with adding the monitor, it can detect the simultaneous extend.
%\\
%However, when the monitor was added to the formal verification with the following LTL %expression : (the same as the working system)
%
%\begin{center}
%     {\footnotesize $G ((HorCTL.EXTEND=TRUE~~\&~~VerCTL.EXTEND=TRUE)
%     \xrightarrow ~~  F (obs.collision=TRUE))$}
%\end{center} 
%
%Then, NuSMV return a counter-example, not as expected. Therefore, our simplified system with all the possibility as to be reconsidered.

\section{Conclusion and future work}
\label{sec:concl}

In this paper, we presented a design methodology for the supervised system implemented in form of IEC~61499 FBD, putting special attention to the verification of the supervision mechanisms, that is, the monitors. This methodology allows using formal verification with state-space reduction or abstraction techniques while verifying the system as a whole, while creating reliable means for online observation of its function. Our main contribution is the approach for extensive closed-loop verification of monitors implemented in form of IEC~61499 FBs using the simplified non-deterministic twins of the controllers.

Having the system monitored gives an additional advantage while introducing small changes to the system when, for instance, the time does not permit performing the whole recheck. If the connections to the monitor do not change, with the proper setting of an error handling unit, the system will stay safe even if the new change leads to a failure.

The manual implementation of the proposed methodology may seem to cost time and effort; however, the process can be automated. Our future work in this direction focuses on creating the plugin to an open-source IDE for IEC~61499 programs, FBME~\cite{fbme}, that would implement the algorithm that distinguishes the key states of the controller, which change the variables under supervision, and generate a FB with NDTs between such states (i.e, ND~twin). The properties that should be checked on an SMV model of an ND~twin can also be generated automatically following the definition that we used in the current work. Thus, if we assume such a plugin, the input data for it would be a monitor, a supervised controller, and a set of temporal properties with which the monitor should comply. The subsequent check of the monitor injected into the system as a whole can be done semi-automatically, depending on the size of the SMV model of the system. 


Another direction of future work is to explore the scalability of our monitor-checking approach and verify more complex properties that include not only inputs of the monitors but any variable in the system, as well as monitors and controllers implemented as composite FBs.

%In our case, formal verification of the twin system is from 5.000 to 10.000 times faster than the formal verification of the entire system
\section*{Acknowledgments}
%\PO{This work was supported?} 

This work was supported, in part, by the HORIZON 2020 project 1-SWARM funded by the European Commission (grant agreement: 871743) and Horizon Europe project Zero-SWARM funded by the European Commission (grant agreement: 101057083).

We also wish to thank Dr. Igor Buzhinsky for suggesting us the missing piece of the puzzle during the work on Section~\ref{sec:monitorver}.

\putbib
\end{bibunit}


%-------------------------------------------------------------------
\def\paperheader{Paper D}
\def\papertitle{Formal Verification of the Control Software of a Radioactive Material Remote Handling System, based on IEC 61499}
\def\paperauthorstring{Giordano Lilli, Midhun Xavier, Etienne Le Priol, Vincent Perret, Tatiana Liakh, Roberto Oboe, and Valeriy Vyatkin}
\def\referencestring{IEEE Open Journal of the Industrial Electronics Society, 2023.}
\def\copyrightstring{2023, IEEE, Reprinted with permission.}

% The definitions above could just as well be put directly into the function
% call below, but were explicitly defined to more clearly illustrate the
% use of the function \makepaper.

\makepaperaccepted
  {\paperheader}
  {\papertitle}
  {\paperauthorstring}
  {\referencestring}
  {\copyrightstring}

% The actual contents is imported by un-commenting the \input line below.
% Make sure the file exist.
\begin{bibunit}
\thispagestyle{plain}

% Define commands used in the paper
\newcommand{\codeword}[1]{\texttt{\detokenize{#1}}}
\newcommand{\quotes}[1]{``#1''}

% Ensure proper figure placement
\setcounter{topnumber}{2}
\setcounter{bottomnumber}{2}
\setcounter{totalnumber}{4}
\renewcommand{\topfraction}{0.85}
\renewcommand{\bottomfraction}{0.65}
\renewcommand{\textfraction}{0.15}
\renewcommand{\floatpagefraction}{0.7}



\section*{Abstract}
    Automation systems within nuclear laboratories are intended to work under harsh operating conditions. SPES (Selective Production of Exotic Species) is a nuclear research facility currently under construction by INFN (Istituto Nazionale di Fisica Nucleare), dedicated to the production and study of Radioactive Ion Beams (RIBs). Isotopes are produced within the Target Ion Source (TIS) unit, a vacuum vessel that must be replaced on a regular basis. The highly radioactive environment necessitates the deployment of a set of automated systems dedicated to the unit's remote management. To meet high-level security standards, the design of such instrumentation and control systems must include extensive verification. Based on specific safety requirements, model checking can be used to assess the systems' correctness. This paper describes how to employ an integrated tool-chain to design, simulate, formally verify, and deploy the control software for the Horizontal Handling Machine, a safety-critical remote handling system in operation at SPES. The IEC 61499 standard's adoption led to a redesign of the control logic. Following a preliminary online simulation, the closed-loop system has been formally verified using the NuSMV symbolic model checker, with the help of the FB2SMV converter. Additionally, the FBME tool was used for automating verification and analyzing counterexamples.

\section{Introduction}
\label{sec:introduction}
Automation systems in nuclear laboratories must comply with strict safety requirements to avoid any potential risk to personnel or equipment.
The critical operating environment generally discourages innovation in the design of control software, leading to an old-fashioned approach still today in the Industry 4.0 era. However, the introduction of distributed control systems based on modern standards would be advantageous for operational and safety challenges. This paradigm shift will lead to the development of smarter systems based on flexible and reconfigurable automation architectures. In this context, the evolution from applications based on \mbox{IEC 61131-3} \cite{iec61131-3} towards IEC 61499 \cite{iec61499, zoitl2014} solutions would provide key tools to face the design and verification challenges typical of complex distributed control systems. 
The main advantages of this migration include:

\begin{itemize}
    \item Flexible, reconfigurable, and scalable architecture.
    \item Modular design, standardized Function Blocks.
    \item Simulations, offline and online verification. 
    \item Formal model checking techniques.
\end{itemize}

% Figure: HHM
\begin{figure}[h!]
    \centering
    \includegraphics[width=0.8\columnwidth]{MX_Papers/Paper4/pictures/lilli01.eps}
    \caption{The Horizontal Handling Machine (HHM) is the primary remote handling vehicle in operation at the SPES facility.} 
    \label{fig:hhm}
\end{figure}

\noindent The SPES (Selective Production of Exotic Species) facility \cite{Marchietal.2020} can be considered an attractive use case to demonstrate the advantages of implementing safety-critical control systems based on IEC 61499. INFN (Istituto Nazionale di Fisica Nucleare) is currently developing an experimental plant at Legnaro National Laboratories (LNL) for multidisciplinary research on Radioactive Ion Beams (RIBs) produced through the ISOL (Isotope Separation On-Line) technique \cite{Andrighetto2018}. Isotopes are generated, as fission reaction products, from the collision of a high-energy proton beam (40 MeV, 200 $\mu$A) with a multi-foil uranium carbide target \cite{Andrighetto2019} consisting of seven UCx disks \cite{Corradetti2021}. The Target Ion Source (TIS) unit \cite{Monetti2015} is identified as the core of the process. Here the isotopes are produced and extracted for further studies in the field of nuclear physics and for medical applications. Unfortunately, aging of target and source materials requires regular replacement of the TIS unit to maintain high efficiency. This is difficult in view of the specific operational conditions. Indeed, the highly radioactive environment precludes any human operation.
For this reason, the management of the TIS unit is entrusted to a remote handling framework \cite{lilli2023}, conceived to fulfill its specific life cycle. The safe operation and reliability of mobile robots operating in complex scientific facilities are essential features that need to be assessed \cite{khan2014}. In the SPES case, the automated systems involved in the replacement procedure face an intense radiation field generated by various contributions, such as the TIS unit \cite{Monetti2015}, the residual Front-End activation \cite{Donzella2020}, and the isotopes deposition along the RIB line \cite{Centofante2021Study}. 
Operational safety is the outcome of an integrated strategy that combines the formal verification of control software with the deployment of inherently safe design principles to the hardware.
In our work, we focused on the most critical remote handling task: the automated removal of a radioactive TIS unit from the SPES Front-End.
The main objectives of the study are to demonstrate the benefits of the migration of \mbox{IEC 61131-based} software to an IEC 61499 architecture and to implement offline and online software verification techniques. This contribution describes the development of flexible and reconfigurable control software based on IEC 61499, along with its formal verification through an integrated tool-chain, for a safety-critical remote handling system: the Horizontal Handling Machine (HHM) depicted in Fig. \ref{fig:hhm}. 
The provided implementation demonstrates how to incorporate modular Non-Deterministic Transitions (NDTs) in formal verification to improve the mode's realism while limiting complexity. 
The paper is structured as follows: Section \ref{sec:problem_statement} discusses the related works and the problem statement. Section \ref{sec:case_study} explains the design and formal verification approach adopted for a radioactive material remote handling system. Sections \ref{sec:simulation} and \ref{sec:formal_verification} describe the approach used to perform online simulations and formal verification, respectively. Section \ref{sec:results} presents the main results of the work, whereas Section \ref{sec:conclusion} concludes the article and outlines future goals.

\section{Related works and problem statement}
\label{sec:problem_statement}
Designing control systems for industrial applications, particularly those involving nuclear-based materials, is of utmost importance. It is essential to adopt an approach that offers a flexible, portable, reconfigurable, and scalable architecture. 
In this context, the use of design patterns within the field of industrial Cyber-Physical Systems (iCPS) employing the IEC 61499 standard emerges as a favorable strategy for designing such systems \cite{dai2017discrete, patil2018}. 
These design patterns, originating from experienced system designers, hold significant value in the field of software engineering \cite{gamma1995design}.

\subsection{IEC 61499: A model-driven approach to build complex control systems}
IEC 61499 \cite{iec61499} is a standardized framework employed in the field of industrial automation for the modeling of distributed control systems \cite{drozdov2021}. The language's flexibility greatly contributed to its popularity for this type of projects. 
The use of advanced design patterns is crucial during the development of applications, which are defined as platform-independent models composed of modular components, in order to properly exploit this potential, achieving a high level of reusability, and ensuring greater reconfigurability of the underlying systems \cite{sonnleithner2021iec}.
Christensen et al. \cite{christensen2000design} introduced a model-driven approach for distributed control systems, employing the Model-View-Control (MVC) design pattern \cite{Model-view-controller} within the context of IEC 61499 systems. 
The existing body of literature has extensively investigated a variety of model-driven software design and engineering approaches that can be employed for the development of control systems, with a particular emphasis on their application to iCPS \cite{sandeep2018}.
In \cite{bonfe2013design}, the authors examine the packaging industry practices and present design patterns specifically tailored for model-driven software design and implementation. Moreover, similar design patterns are proposed in \cite{cengic2006framework, vyatkin2005architecture, hametner2010automation}.

Safety of control software is an additional factor that becomes crucial in specific applications, such as laboratories dealing with nuclear-based materials.
In these contexts, comprehensive testing is imperative to mitigate the risk of potentially catastrophic issues arising from even minor errors. 
The verification challenge of IEC 61499 has been acknowledged since the early stages of the standard's development and evaluation \cite{vyatkin1999modeling,hanisch2009one}. To achieve the most thorough verification, closed-loop modeling has been suggested, necessitating the inclusion of plant modeling \cite{vyatkin2008closed}. The implementation of a model-driven approach facilitates simulation-in-the-loop \cite{hegny2010iec, yang2012transformation}, enabling thorough testing and identification of system errors. 

During the design process, simulation plays a crucial role in assessing the overall behavior of the control system, ensuring its compliance with expected outcomes and assisting in the process of virtual commissioning \cite{galkin2023automatic}. However, despite being advantageous in identifying flaws, this tool does not provide a guarantee of system reliability. 
Formal verification techniques have thus been proposed, as a promising approach to automatically verify the correctness and safety of automation systems.
Specifically, the model checking integration in closed-loop verification \cite{xavier2023formal} assists in the identification of design weaknesses in the model through the use of counterexamples. The closed-loop architecture, encompassing both the controller and the plant, depicts the overall behavior of the system. This design approach works successfully for both simulation and formal modeling applications \cite{sinha2019survey}. 

% Figure: workflow
\begin{figure*}
    \centering
    \includegraphics[width=\textwidth]{MX_Papers/Paper4/pictures/lilli02.pdf}
    \caption{Proposed workflow for the validation of safety-critical automation systems.} 
    \label{fig:workflow}
\end{figure*}

\subsection{Formal Verification Techniques for  Modular Industrial Control Systems: Challenges and Strategies}

Model checking includes formal verification methods that are used as trustworthy tools to grant the correctness of Instrumentation and Control (I\&C) systems \cite{clarke1999, schneider2004, baier2008}.
These techniques have been employed in a wide range of disciplines, including avionics \cite{gelman2013, wang2019}, automotive \cite{todorov2018, kim2015, Filipovikj2016}, and Nuclear Power Plants (NPP) \cite{pakonen2021, Jee2010, nemeth2009, adiego2015}.
Despite their remarkable benefits for the validation of complex automation control logics, the computing requirements of model checking techniques may frequently represent a bottleneck in their use \cite{Buzhinsky2020}. Several approaches have thus been proposed to reduce their computational complexity \cite{Cimatti2012, Biere2003, burch1992}.
Furthermore, special care should be taken to guarantee that the system model's architecture reflects the behaviors of actual systems \cite{Cordeiro2020}. 
The entire verification procedure is completed in three phases. Creating a formal version of the actual system is the first step. At the second stage, the model and temporal logic specifications are fed into a verification tool.
NuSMV \cite{Cimatti2002} is a symbolic model checker based on  Binary Decision Diagrams (BBDs), whereas SPIN \cite{Holzmann1997} is an explicit state software verification suite.
In the third step the tool reports whether or not the specification was met. A sequence of the model's states where the specification does not hold will also be provided, if possible, as a counterexample \cite{beer2012}.
Unfortunately, despite the values of the model variables being included in each element of the sequence, counterexamples are unable to reveal their inherent dependencies or internal structure \cite{Ovsiannikova2021}.
In an effort to make model checking more user-friendly, a number of visualization tools \cite{pakonen2018} have been created in recent years to assist users in understanding system behavior during specification violations and to identify the source of the problem \cite{Loer2006, bochot2010, Patil2015}. This method's ultimate goal is to identify design flaws in the controller. 
In \cite{vyatkin2011iec}, a novel framework is presented for the design and validation of industrial automation systems using formal methods, specifically leveraging the IEC 61499 architecture and the automation object concept. The proposed approach enables the comprehensive process of designing, simulating, formally verifying, and deploying pick-and-place systems. 
Several technologies are available for formally modeling and validating industrial control systems \cite{vyatkin2001formal}.
The VEDA tool, for instance, is designed to support the formal verification of IEC 61499 systems within a closed-loop context. Although VEDA supports the modeling of the controller using a Petri net representation, representing the plant component requires manual effort. 
Recent research demonstrates the automatic generation of plant models from event logs \cite{xavier2022plant, xavier2022process}, simplifying the arduous task of constructing formal models for control engineers. Furthermore, interactive learning techniques have played a crucial role in enabling the automatic generation of controller function blocks \cite{xavier2022interactive}, thereby greatly facilitating the process of formal verification.
The introduction of FB2SMV \cite{fb2smv} enabled the creation of SMV models as formal representation derived from the IEC 61499 function blocks within the system. 
This process allows to take advantage of the great potential provided by NuSMV industry-grade model-checker \cite{Cimatti2002}.
Given the introduced benefits, a significant effort has been made into including formal verification tools within the design process.
The development of a seamless process that links engineering with verification was outlined in \cite{xavier2021}. The tool-chain includes an IEC 61499-compliant engineering environment, a converter for translating function blocks into SMV code, the NuSMV model-checker, and utilities for interpreting counterexamples. The presented approach aims to facilitate the design, simulation, formal verification, and distributed deployment of automation software for CPS. To achieve this, a problem-oriented notation within the IEC 61499 syntax is suggested, enabling the creation of comprehensive closed-loop models. 
The proposed methodology addresses the challenge of verifying and analyzing function blocks implemented in the IEC 61499 standard by providing a tool-chain that supports continuous development and testing of distributed control systems.
In \cite{liakh2022formal}, the authors detail the creation of a model-checking plugin designed for IEC 61499 systems within the FBME \cite{FBME} graphical development environment. This plugin automates various stages of the process, including the conversion of the system into a formal model, model-checking, and the provision of visual explanations for counterexamples.

Despite the undeniable benefits introduced by the IEC-61499 standard, which provides a reference architecture and models for distributed control system development, the lack of established methodologies and theoretical foundation for iCPS poses a challenge for the development of new CPS applications. 
In particular, the integration of the aforementioned technologies into real industrial applications might be challenging due to the actual complexity of the solution and the time required for both the CPS implementation and verification. 
Indeed, while some examples for basic systems are provided in \cite{xavier2021}, it is still not clear whether the described techniques can be applied to complex CPS. 
The goal of this work is to provide real-world strategies for developing modular applications, implementing automatic verification procedures, and reducing system complexity. In this paper, we demonstrate how the described techniques can be applied in the refactoring and verification of a safety-critical control system.
The following sections give a detailed explanation of each component of the implemented workflow, which is illustrated in Fig. \ref{fig:workflow}.
The development of a modular and portable system model, the reduction of verification complexity through the partial incorporation of NDTs, and the implementation of an automatic verification procedure are the fundamental novelties of the proposed solution.

\section{Case study}
\label{sec:case_study}
An illustrative example is used to describe the entire formal verification process of a CPS, which includes the IEC 61499 software remodeling, the partial introduction of NDTs to conduct symbolic model checking under realistic conditions and the visualization of counterexamples.
The case study covered in this article is a safety-critical remote handling system used to transport and store radioactive material within a nuclear research facility. 
In this work, we propose the refactoring of an IEC 61131 control software with a new flexible and reconfigurable architecture based on the IEC 61499 standard. Additionally, formal modeling and verification tools have been implemented to validate the effectiveness of the designed solution.

\subsection{The Horizontal Handling Machine (HHM)}

%Figure: fsm
\begin{figure}[h!]
    \centering
    \includegraphics[width=0.95\columnwidth]{MX_Papers/Paper4/pictures/lilli03.png}
    \caption{The TIS unit pick-up sequence's finite state machine.} 
    \label{fig:fsm}
\end{figure}

The primary remote handling vehicle used to manipulate and transfer the TIS unit within the SPES target area is known as the Horizontal Handling Machine (HHM). This system, depicted in Fig. \ref{fig:hhm}, enables the safe removal of an irradiated TIS unit from the SPES Front-End and transport to the Temporary Storage System (TSS), an automated storage rack designed to house up to 54 TIS units for long-term radioactive decay. Following the TIS unit removal, the HHM is employed to install a new TIS unit on the SPES Front-End in preparation for a new irradiation cycle. The machine consists of a cartesian manipulator located on the top of an Automated Guided Vehicle (AGV). While the vehicle allows movement between different areas, the manipulator is used for the collection of the TIS unit, its storage, and the upcoming installation. Here, brushless motors are used for the precise positioning of three linear axes. 
Two of them (\textit{trolley} and \textit{crane}) allow the TIS units to move along the longitudinal and vertical direction, respectively. The third (\textit{elevator}) allows for the vertical movement of a shielded box, which is used to secure the TIS unit during transport. The manipulator's end effector consists of a redundant pneumatic gripper able to engage both the TIS unit and the shielding box lid. The machine control software is based on IEC 61131 and runs on an onboard PLC (Schneider Electric\textsuperscript{\textregistered} M340). Since its conception as a safety-critical automation system, the design of the HHM has incorporated inherently safe principles. Additionally, the system has been assessed using specific Probabilistic Risk Assessment (PRA) techniques to evaluate the most severe failure scenarios and validate the implemented Independent Protection Layers (IPLs). 
In this context, software formal verification acts as a fundamental protection layer that can reduce the risk of system failure, potentially leading to unintended maintenance interventions in areas with a significant environmental dose rate.
The HHM software logic supports multiple operating modes and motion sequences based on the type of remote handling task.
Among the existing operational procedures, we focused on the most critical task: the removal of an irradiated TIS unit and subsequent storage inside the shielding box during transport. 
During this procedure, the HHM is facing the SPES Front-End and all actions are carried out by the cartesian manipulator.
The onboard PLC controls the sequence management, which includes the axes movements, the pneumatic gripper, and the reading of the various hard-wired signals from the limit switches demanded to detect the proper positioning of the radioactive TIS unit. This scenario has been considered as critical since a potential fault during the execution would necessitate a maintenance intervention under severe radiological conditions, leading to a significant personnel exposure. 

\newpage

\noindent The operation consists of the following steps:
\begin{itemize}
  \item the \textit{trolley} initially moves ahead to pick up the TIS unit;
  \item the \textit{crane} descends, engages the TIS unit, and rises to the top positions;
  \item the \textit{trolley} moves to the middle position on top of the open shield box while holding the TIS unit;
  \item the \textit{crane} lowers the TIS unit while the \textit{elevator} rises the box. Once in position, the \textit{gripper} releases the payload;
  \item the manipulator finally closes the box with the lid.
\end{itemize}
The Finite State Machine (FSM) is depicted in Fig. \ref{fig:fsm}.


\subsection{The IEC 61499 implementation}

% Figure: iec 61131
\begin{figure}[h!]
    \vspace*{-5mm}
	\centering
    \includegraphics[width=\columnwidth]{MX_Papers/Paper4/pictures/lilli04.png}
    \caption{The original HHM control program, based on IEC 61131-3 Structured Text (ST).} 
    \label{fig:iec61131}
\end{figure}

% Figure: model
\begin{figure*}
	\centering
    \includegraphics[width=\textwidth]{MX_Papers/Paper4/pictures/lilli05.eps}
    \caption{The IEC 61499 global composite Function Block of the HHM model.} 
    \label{fig:model}
\end{figure*}

The HHM control software was initially designed in accordance with the IEC 61131-3 standard, an overview of the software section implementing the main state machine as a case structure is reported in Fig. \ref{fig:iec61131}.
With the advancement of technology, the IEC 61499 introduction suggested a complete code refactoring in the direction of a more modern, modular, and flexible architecture, where it would be possible to change the behavior of the system by acting on a single Function Block. 
The remodeled application of the HHM control logic was developed using the EcoStruxure\textsuperscript{\texttrademark} Automation Expert tool. The software architecture is built on Function Blocks (FBs) linked to Moore-type finite state machines known as Execution Control Charts (ECC) \cite{Lee2017}. An overview of the global composite FB model is available in \mbox{Fig. \ref{fig:model}}.
One of the many benefits provided by the IEC 61499 refactoring, aside from supporting formal verification, is the introduction of a modular, standardized, and reusable architecture for the development of FBs.
This strategy results in improved code organization and the potential to \quotes{certify} the behavior of the FBs, thus reducing the verification complexity in subsequent applications. Additionally, the existing IEC 61131 design, which is based on Structured Text (ST), incorporates global variables within the program to track the program execution.
Since the software's behavior is not always evident this poses a serious concern. 
In contrast, IEC 61499 provides for the explicit specification of the dependencies and interactions between different FBs.
The \textit{elevator}, \textit{trolley}, \textit{crane}, and \textit{gripper} are the key actuation groups employed in this application. Each of these mechatronic systems, which work together to securely encase the TIS unit in the shielding box, is supervised by a dedicated controller. The following sections provide a detailed description of the main Function Blocks.

% Figure: fbs
\begin{figure}[h!]
    \centering
    \includegraphics[width=0.9\columnwidth]{MX_Papers/Paper4/pictures/lilli06.eps}
    \caption{Overview of the controllers dedicated to the HHM linear motion axes and gripper. (a) The AXE\_CMD FB, (b) the AXE\_CMD ECC.} 
    \label{fig:fbs}
\end{figure}

\subsubsection{Linear motion axes}
A standardized pair of controller and plant Function Blocks can be used to conceptually model the three linear axes.
Using a modular and reusable strategy, the development work can be significantly decreased. 
Additionally, it makes it possible for the system to be easily reconfigured in order to achieve alternative capabilities in the future.
The core Function Block \codeword{AXE_CMD}, which implements an absolute positioning control system, is shared by the three linear axes. The FB and the correspondent ECC are displayed in \mbox{Fig. \ref{fig:fbs}}.
The plant FB precisely sets the axis according to the destination coordinates and provides the \codeword{POS_REACHED} signal to the controller once the motion is completed.
The given target position directs the \codeword{AXE_CMD} to the preset coordinates. The FB acknowledges its arrival and stops it once it reaches the designated spot. A visual representation of the \textit{elevator} linear motion axis is reported as an example in Fig. \ref{fig:elevator}.

% Figure: elevator
\begin{figure}[h!]
    \centering
    \includegraphics[width=\columnwidth]{MX_Papers/Paper4/pictures/lilli07.eps}
    \caption{Visual representation of the elevator linear motion axis: (a) bottom position, (b) top position. } 
    \label{fig:elevator}
\end{figure}

\subsubsection{Gripper}
The operating mode of the HHM pneumatic \textit{gripper} differs from the above-mentioned systems due to its inherent discrete logic. 
\textit{Gripper} \codeword{CLOSE} or \codeword{OPEN} commands are processed when the \codeword{REQ} event is triggered. The FB provides two output signals to indicate when the relevant \quotes{closed} or \quotes{open} state has been reached. 

\subsubsection{Sequence Controller}
The \codeword{SEQUENCE} FB manages the integration of the various subsystems and the overall HHM behavior throughout the execution of the remote handling sequence. The precise list of tasks is defined within the correspondent ECC. 
Each FSM state is associated with a set of actions carried out by a specific algorithm. Motion actions are started by setting the desired position for a specific axis and sending the \codeword{GO} command to the appropriate controller. The reception of the \codeword{POS_REACHED} command from the plant Function Block causes the transition to the next state. In our case study, the sequence controller Function Block implements a state machine that refers to a single HHM task: the TIS unit pick-up sequence. This sequence has been examined as a representative example. The system's adaptable architecture will make it possible to incorporate more motion sequences in the future by updating a single Function Block.

\subsubsection{Support Function Blocks}
The \codeword{INIT} Function Block initializes the system and prepares it to perform the desired procedure at the start of software execution.
The user can then choose between manual and automatic HHM operating modes by using the \codeword{TRIGGER} and \codeword{MODE_SELECTION} Function Blocks.
While the first allows the user to direct the HHM behavior, the automatic mode forces the system to stick to the Sequence controller's state machine logic.
The \codeword{ESTOP} Function Block, as the last support FB, offers the ability to stop the execution at any time. This feature protects the system from internal or external failure caused by unfavorable conditions.

\section{Simulation Model}
\label{sec:simulation}
IEC 61499 applications can typically be tested using dynamic (online) or static (offline) techniques. In order to assure safety in a system that has already been deployed and is in use, the first group of techniques seeks to monitor it in its operating state. Conversely, offline safety measures are meant to reduce fault risk at the design stage and test the system before use \cite{Ovsiannikova2021b}.
In our work, we focused on offline verification methods aimed at fault removal. 
This process can be accomplished at the designed stage using formal verification tools or online testing techniques. 
Software simulation involves feeding the program with input sequences that replicate the behavior of the actual system and determining whether or not the program's outputs comply with specific requirements.
The adopted development suite includes a native Human Machine Interface (HMI), which may be used as a command center and to simulate system execution.
Composite Automation Types (CAT) were used to model a range of mechatronic components for the simulated plant. This feature facilitates testing of the system's simulation behavior in a common environment because CATs can be directly linked to both HMI objects and Function Blocks.
Inputs were used to link the controller Function Blocks to the relevant CAT blocks, replicating the real-world behavior of the mechatronic components in the system. The HHM representation implemented in the HMI is shown in \mbox{Fig. \ref{fig:hmi}}, where the three linear motion axes are linked to distinct CAT blocks. 
Each axis plant FB is connected to a dedicated \codeword{AXE_CMD} controller, which selects the desired position set-point from a predefined pool of coordinates and triggers the motion request. In response to the controller's inputs, the plant block validates the coordinates, performs the movement, and acknowledges its arrival at the predetermined location. The axis motion may be stopped at any time by activating a \codeword{STOP} input event.
The \codeword{GR_CMD} FB opens or closes the clamp based on the input signal from the controller. In order to interlock the option of releasing the payload only in particular positions, the \codeword{GR_CMD} is additionally provided with the axes' actual positions.

We should emphasize that the HMI CATs provide a more accurate representation of the system behavior when compared to the axis plant Function Blocks discussed in \mbox{Sec. \ref{sec:case_study}}. While in the basic implementation, the plant FB will only trigger the \codeword{POS_REACHED} signal after an arbitrary time, here an integrator simulates the linear axis movement and sends the actual position coordinates to the correspondent object in the HMI allowing the user to follow the motion while it is being executed. Further debugging tools, such as runtime monitoring blocks, can be also employed to detect specific critical conditions.
The software's modularity allows for the independent and concurrent development of the controller and plant Function Blocks. Each FB will be initially tested and debugged with the aid of custom mock-up blocks. As they reach maturity, they can then be interconnected to run the simulation. After the verification, the final stage will be to replace the simulation's plant FB with the actual system. 

Unfortunately, simulations often cannot explore all possible paths due to the huge size of state automata representing industrial control software. This bottleneck makes them insufficient as an exhaustive verification method since it prevents conclusive verification of program behavior in a reasonable amount of time. Furthermore, the quality of the output is also influenced by the automation engineer's knowledge and experience in selecting pertinent testing sequences that may correspond to typical dangerous circumstances of the controlled process \cite{Schnakenbourg2002}. 
To address these problems, formal verification techniques have been established, which provide methods for closed-loop (plant and controller) model checking able to analyze a program in its entirety.

% Figure: hmi
\begin{figure}[h!]
    \centering
    \includegraphics[width=\columnwidth]{MX_Papers/Paper4/pictures/lilli08.eps}
    \caption{Graphical representation of the HHM Composite Automation Type (CAT) used in the Human Machine Interface (HMI) for online monitoring and simulations.} 
    \label{fig:hmi}
\end{figure}


% Table: 1 
\begin{table*}[!hb]
    \centering\small
    \renewcommand{\arraystretch}{1.5}
    \begin{tabularx}{\textwidth}{ m{0.1\textwidth} m{0.4\textwidth} m{0.4\textwidth} }
        \hline
        No & Property & Comment                                                                         \\ 
        \hline
        1   & \texttt{G !(ELplant.POS\_OUT = 5)} &  
        The \textit{elevator} plant Function Block must never reach the \textit{error} state in any of the sequence elements.                                                 
        \\ 
        2   & \texttt{G !(CAplant.POS\_OUT = 5)} &  
        The \textit{trolley} plant Function Block must never reach the \textit{error} state in any of the sequence elements.                                                       
        \\ 
        3   & \texttt{G !(CRplant.POS\_OUT = 5)} &                                                    The \textit{crane} plant Function Block must never reach the \textit{error} state in any of the sequence elements.                                                       
        \\ 
        4   & \texttt{G !(CRplant.POS\_OUT in (2..4) \& CAcmd.moving = TRUE)} &                       
        The \textit{crane} must always be in the top position while the \textit{trolley} is moving to prevent mechanical collisions.
        \\      
        5   & \texttt{G !(ELplant.POS\_OUT = 2 \& CRplant.POS\_OUT = 4 \& CAcmd.moving = TRUE)} &     
        To avoid mechanical collisions, the \textit{trolley} must not move while the HHM is lowering the TIS unit inside the HHM shielding box.      
        \\ 
        6   & \texttt{G !(ELplant.POS\_OUT = 1 \& CRplant.POS\_OUT = 4 \& GRplant.GRO = TRUE)} &      
        The pneumatic \textit{gripper} shouldn't open until the \textit{elevator} is not in the top position, even if the \textit{crane} is in the lower position.       
        \\ 
        \hline
    \end{tabularx}
    \caption{Description of the LTL specifications verified with NuSMV in the HHM model.}
    \label{table:1}
\end{table*}

% Figure: par_seq
\begin{figure*}
    \centering
    \includegraphics[width=\textwidth]{MX_Papers/Paper4/pictures/lilli09.eps}
    \caption{A comparison of the two investigated control sequences. (a) illustrates the ECC of the Controller FB implementing the parallel movement of the three linear axes, whereas (b) shows the ECC in which the three movements are executed sequentially.} 
    \label{fig:par_seq}
\end{figure*}

\newpage

\section{Formal verification}
\label{sec:formal_verification}
The formal verification of finite state systems, such as \mbox{closed-loop} control algorithms, has been effectively accomplished in the last ten years thanks to symbolic model checking based on Binary Decision Diagrams (BDDs). These tools have been developed in the past to overcome the state explosion problem in finite automata \cite{burch1992}. 
Model checking is the process of exploring the reachable states of a model, which is described as a finite state machine, in order to validate temporal logic specifications.
When a property is violated, the tool provides a counterexample in the form of a sequence of states \cite{Biere2003}. As previously mentioned, the most well-known open-source model detection tools among the available solutions are NuSMV and SPIN. In particular, because of its extensive core capabilities and good scalability, NuSMV is frequently used for reliability and security verification of industrial designs \cite{xu2018}. This tool supports the representation of synchronous and asynchronous finite state systems and it allows for the verification of both Linear Temporal Logic (LTL) and Computation Tree Logic (CTL) specifications using implicit methods. In more detail, it compares a model against a property using a symbolic representation of the specification \cite{frappier2010}. 

Accurate modeling of the real system is essential in order to validate the  intended behavior of the device and detect potentially undesirable states. This enables simulation and verification of the apparatus prior to its actual operation. Since the model is an abstraction, it may not include all relevant characteristics of the real-world system or the context in which it is embedded. Hence, a condensed version of the plant FB can be used to create a reduced formal model, which can then be verified utilizing symbolic model checking techniques thanks to the NuSMV tool.
As an illustration, in the presented use case, the \codeword{AXE_CMD} FB only takes into account the beginning, intermediate, and final states rather than the motion dynamic considered in the real system. 
This approximation is still acceptable because the goal at this stage is to assess the possible blockage within two locations instead of the specific stop positioning. 

Table \ref{table:1} describes a collection of LTL expressions that have been developed to identify potential critical problems.
Specifications 1-3 are meant to ensure that none of the three linear motion axis plants enters the error state during system execution. On the other hand, requirements 4 and 5 deal with potential collision detection. More in detail, the first verify that \textit{trolley} movements are inhibited when the \textit{crane} is not fully raised in the top position, and the second focuses on the system configuration occurring while positioning the TIS unit within the shielding box. Similarly to the last scenario, the \textit{trolley} must not move while the \textit{elevator} is raised and the \textit{crane} is lowered.
Specification 6 aims to confirm that the \textit{gripper} only opens in a specific location: when the TIS unit is lowered within the box (\textit{elevator} up and \textit{crane} down). 
A batch script, detailed in Appendix \ref{appA:NuSMV script}, has been developed to examine all the aforementioned requirements with NuSMV and log data.

The LTL specifications were evaluated in two different scenarios to test the reliability of the formal verification. 
As described in Section \ref{sec:case_study}, the \codeword{SEQUENCE} FB implements a FSM where the HHM axis movements are executed sequentially to prevent any potential collision. 
If we specifically consider state \codeword{GRC_03_GRC_04} in Fig. \ref{fig:par_seq}, which corresponds to the TIS unit picked up by the HHM cartesian manipulator, the subsequent path towards the shielding box shall be carried out in three distinct steps: (1) Backward movement of the \textit{trolley} axis, (2) lowering of the \textit{crane} axis, and (3) rising of the \textit{elevator}. The described motion sequence and the correspondent states are visible in \mbox{Fig. \ref{fig:par_seq} (b)}.
In our research, we deliberately induced a design flaw in the control software to determine if NuSMV was able to identify it. 
Specifically, we updated the main FSM to launch the previously mentioned actions in a parallel execution, with the three motion axes moving simultaneously, as shown in \mbox{Fig. \ref{fig:par_seq} (a)}. 
Since it does not always result in a fault condition, this type of design error is particularly difficult to identify through conventional simulations. 
The relative motion axes speeds do, in fact, affect the likelihood of a collision. This implies that we may be able to perform multiple simulations without observing any failure event. The following section discusses how adding non-determinism to the model can make it more realistic by taking into account the impact of non-idealities found in the real world and allowing for the early identification of potential system defects.
With regard to the test under discussion, NDTs within axes plant FBs seek to change the amount of time required to get the \codeword{POS_REACHED} signal, directly impacting the relative speed between concurrent axis movements.
The violation of LTL specification No. 5 in Table \ref{table:1} allows for the detection of the collision occurrence. 

\subsubsection*{Discrete State Plant Modelling in Function Blocks with Non-Deterministic Transitions}
The original system, created to exploit the visualization and online verification capabilities provided by EcoStruxure\textsuperscript{\texttrademark} Automation Expert, needs to be reduced and adapted in order to apply formal verification methods. The \textit{elevator}, \textit{trolley}, and \textit{crane} components were modeled in this study by a simplified Function Block that embodies the intended behavior of the actual system while omitting the features used for visualization. As opposed to the simulation scenario where each component was modeled using a single Function Block, this global plant model has been implemented to capture the behavior of the three linear motion axes collectively.
Thus, by discretizing the plant model's Function Block while maintaining its functional capabilities, the original complex model can be reduced to a simpler representation.
The \codeword{AXE_PLANT} component features two data inputs (\codeword{GO} and \codeword{POS_IN}) and two data outputs (\codeword{POS_REACHED} and \codeword{POS_OUT}).
The system may simulate real-world behavior using the non-deterministic transition (NDT) event's random signal emission, which enables the discovery of previously undetected faults using CTL or LTL specifications.
As an example, Fig. \ref{fig:ndt} depicts a potential scenario in which a NDT has been introduced in the ECC associated with \textit{elevator} plant FB.
In this case the plant enters the \codeword{GO} state upon receiving the controller's \codeword{GO} signal and following the NDT event, it reaches the \codeword{END} state.
The physical meaning of this NDT is that the transition between the \codeword{GO} and \codeword{END} states, i.e. the axis motion towards a given position, might take an unspecified amount of time.
If a \codeword{NOT_GO} signal is generated while the plant is in the \codeword{GO} state, it enters the \codeword{STOP} state and remains there until another \codeword{GO} signal is activated.
In the \codeword{END} state, the plant notifies the controller that the task has been completed by setting the value of  \codeword{POS_REACHED} signal to \codeword{TRUE}. 
Following the deactivation of the \codeword{GO} signal, the plant returns to the \codeword{HOME} state.

% Table: 2
\begin{table}[b!]
    \centering\small
    \renewcommand{\arraystretch}{1.5}
    \begin{tabularx}{\columnwidth}{ m{0.1\columnwidth} m{0.9\columnwidth} }
        \hline
        No. & Scenario                                                                                   \\ 
        \hline
        1   & NDT in \textit{elevator} plant                                                            \\ 
        2   & NDT in \textit{trolley} plant                                                             \\  
        3   & NDT in \textit{crane} plant                                                               \\ 
        4   & NDT in \textit{gripper} plant                                                             \\ 
        5   & NDT in \textit{elevator} and \textit{trolley} plants                                      \\ 
        6   & NDT in \textit{elevator}, \textit{trolley} and \textit{crane} plants                      \\ 
        7   & NDT in \textit{elevator}, \textit{trolley}, \textit{crane} and \textit{gripper} plants    \\
        \hline
    \end{tabularx}
    \caption{Non-Deterministic Transition (NDT) scenarios analyzed in the study: NDTs are progressively included in the model.}
    \label{table:2}
\end{table}

% Figure: ndt
\begin{figure}[t!]
    \centering
    \includegraphics[width=0.95\columnwidth]{MX_Papers/Paper4/pictures/lilli10.eps}
    \caption{Example of injection of a Non-Deterministic Transition (NDT) within the elevator plant model.}
    \label{fig:ndt}
\end{figure}

% Figure: fbme
\begin{figure*}
    \centering
    \includegraphics[scale=0.5]{MX_Papers/Paper4/pictures/lilli11.png}
    \caption{FBME trace analysis.} 
    \label{fig:fbme}
\end{figure*}

\noindent The \textit{gripper} plant model features two data inputs, \codeword{OPEN} and \codeword{CLOSE}, and two data outputs, \codeword{GRO} and \codeword{GRC}. 
The model initially enters the \codeword{OPENING} state when the controller sets \codeword{OPEN} to \codeword{TRUE}. Second, it switches to the \codeword{OPEN} state in response to a NDT signal.
Similarly, when the controller sets \codeword{CLOSE} to \codeword{TRUE}, the plant reaches the \codeword{CLOSING} state and, following a random time delay caused by the NDT, enters the \codeword{CLOSED} state.
If the \codeword{CLOSE} command is activated during the \codeword{OPENING} state, the model transitions to the \codeword{CLOSING} state. If the \codeword{OPEN} command is activated during the \codeword{CLOSING} state, the plant returns to the \codeword{OPENING} state and awaits for the emission of the NDT signal.
The discrete state model of the HHM was converted into an SMV model using the FB2SMV tool. Subsequently, the verification has been carried out by NuSMV, using an Intel\textsuperscript{\textregistered} core\textsuperscript{\texttrademark} \mbox{i7-10510U} CPU@1.80 GHz 2.30 GHz with 32 Gb RAM. 
In an effort to mitigate the state space explosion problem, NDTs have gradually been introduced into different sections of the model according to the scenarios in Table \ref{table:2}.
The progressive integration of NDTs might be viewed as a feature of the proposed tool-chain.
While it is true that critical faults might occur as a result of multiple non-deterministic conditions acting simultaneously, in a first verification stage, distinct blocks can be assessed independently while maintaining the execution time within reasonable limits. 

\subsubsection*{Trace analysis using Function Blocks Modelling Environment (FBME)}
In our work, we used Function Blocks Modelling Environment (FBME) \cite{FBME}, enhanced with trace visualization and in-depth analysis capabilities, to examine counterexamples and find the causes of violations of requirements. 

One of the non-trivial tasks when using formal verification is to analyze the resulting counterexample. Verifiers frequently offer an output trail in a text format that is inconvenient and confusing to the user. As a result, the user has to make additional efforts to analyze the counterexample. 
The IEC 61499 code presents additional challenges due to the automatic generation of the model for the verifier, which leads the counterexample to utilize the notations of the input model. 
Given an output trace, the following natural step is to determine the location of the error and investigate the underlying causes. The non-imperative nature of IEC 61499 further complicates this process. Moreover, current common IEC 61499 software development tools do not provide in-depth trace analysis capabilities. 

FBME is an Integrated Development Environment (IDE) for IEC 61499 applications, currently under active development at Luleå University of Technology (LTU). FBME is a cross-platform, open-source, modular IDE that is based on IntelliJ IDEA and the Meta Programming System (MPS) \cite{mps}. MPS provides powerful tools for developing custom Domain-Specific Languages (DSL) and also provides a platform for creating custom IDEs. Modularity and extensibility are key features of FBME, so LTU has also extended FBME's functionality by adding enhanced capabilities to the visualization and automatic analysis of IEC 61499 program execution traces and counterexamples. 
The functionality to automatically call NuSMV and generate a model for verification was also seamlessly integrated into FBME.

\newpage

\noindent The trace analysis in FBME is shown in Fig. \ref{fig:fbme}. Trace is stored in unified format \cite{liakh2022formal} and can be obtained from different sources: either from a simulation of the verifier model, as a counterexample, or from the actual execution of the IEC 61499 program. The entire trace history is displayed on \mbox{panel 1}. The user selects the trace step of interest and can examine the state of the system. The values of all variables, message counters and other data are also displayed on the diagram itself (2), where changes that have occurred in the current step are highlighted. 
In addition to clear visualization of the trace, FBME uses powerful techniques for its in-depth analysis. 
By examining the preceding code, this method enables the user to visually locate the section of the code that resulted in the problem. 
Visual Explanation allows to establish causal relationships between different trace events (note that the term  \quotes{event} here is used in a broader sense, as a change in system state, rather than an event in IEC 61499 terms). For instance, it might be possible to determine whether a variable modification, an IEC 61499 event emission, or a change in an ECC state produced a specific event. 
In this way, the cause of the violation of a requirement can be identified. An example of the result of Visual Explanation technique is available in window 3.

% Figure: graph
\begin{figure*}
    \centering
    \includegraphics[width=0.84\textwidth]{MX_Papers/Paper4/pictures/lilli12.eps}
    \caption{Execution time required by NuSMV in different NDTs configuration.} 
    \label{fig:graph}
\end{figure*}

\section{Results and Discussion}
\label{sec:results}
The initial phase of the project, which followed the software remodeling based on IEC 61499, was devoted to validating the model by launching various simulations directly within the EcoStruxure\textsuperscript{\texttrademark} Automation Expert suite.
This was achieved through the use of CATs, which allow Function Blocks to be directly linked to HMI objects.
The application model and the HMI have been developed independently. Once sufficiently stable, the HMI plant FB was connected to the controller FBs, replacing the existing simplified version of plant FBs.
Launching the online simulation, the user can monitor the sequence execution. The software will begin in the initial state, progress through specific checkpoints, and eventually reach the final state.
Unfortunately, even if the simulation doesn't report any errors, this merely indicates that there exists a path where it crosses all the checkpoints.
Hence, using symbolic model checking tools will provide a more thorough level of investigation.
Prior to the verification procedure, it is crucial to verify the accuracy of the formal model. This can be accomplished by simulating the model in NuSMV, where various paths and random states are explored.
The simulation assists in demonstrating that the model properly covers all the ECC states of the behavioral Function Block by tracing the path of ECC states. It also helps to confirm that the generated formal model behaves in accordance with the discrete state model by providing information about the values of all the variables in each state.
The NuSMV simulation technique can detect changes in the ECC and their impact on system behavior.
Initially, using this method it will be possible to confirm that all paths leading from the beginning to the end will pass through the crucial checkpoint. As the second step, this assertion needs to be proven even in the presence of non-determinism.
Indeed, the introduction of NDTs may have resulted in the inclusion of certain additional pathways in the application, and this is reflected in a larger state space with multiple routes. In contrast to simulation, where we can test only one scenario, NDTs allow us to evaluate several possibilities. The evidence that the given specifications are validated in all of these paths will thus extend the results of the online simulation. The six formulated properties have been checked using a batch script that reads the supplied SMV model and performs the verification, logging both the execution time and result for each specification. 
The quantity of memory needed to store and manipulate BDDs is the primary limitation of model checking methods. In light of this, the proposed implementation allows for the gradual integration of NDTs into the model. This stepwise approach provides better control over the model and allows for faster specification analysis. The time required for NuSMV to execute the formal verification of all the described LTL specifications while altering the number of NDTs is depicted in Fig. \ref{fig:graph}. It is evident that the gradual inclusion of NDTs resulted in a global increase in execution time.
Because of the ample state space, it is feasible that with a larger number of NDTs, global verification of all pathways will fail.
Reducing the number of NDT points in this situation may be a viable option for squeezing the state space to a tolerable size and then gradually increasing it.
Bounded Model Checking (BMC) is an alternate strategy that searches for a counterexample in executions whose length is constrained by some number k. If no bug is discovered, k is increased until either a bug is discovered, the problem becomes unmanageable, or some predetermined upper bound is reached \cite{Biere2003}.
A key feature of the described engineering framework is the ability to govern non-determinism. NDTs can be injected into specific locations to perform formal verification in a particular configuration. This method allows us to validate the automation system under particular stress conditions.
As discussed in Section \ref{sec:formal_verification}, the IEC 61499 application was formally verified following the purposeful introduction of a design fault that might potentially lead to a collision occurrence.
Despite the difficulties in identifying this failure condition using conventional simulations, NuSMV was able to successfully accomplish this task, thus providing a counterexample which demonstrates the violation of LTL property No. 5 in Table \ref{table:1}.
In the case under study, the amount of time needed for the formal verification was comparable with what was required for the same LTL expression in Scenario No. 7 (see \mbox{Fig. \ref{fig:graph}}). However, it is difficult to formulate a generic statement because the duration depends on the particular paths that lead to the failure conditions. 
The evidence of the violation is provided by NuSMV in the form of a failure trace, which depicts a state sequence of system model transitions where the specification is not met. \mbox{Figure \ref{fig:counterexample}} shows how, through the use of specific visualization tools \cite{pakonen2018} it would be possible to decode the output trace and examine the path that led to the violation. 
This result is of great significance as it showcases how the presented set of tools can be employed in the verification of complex safety-critical control systems, enabling the early detection of potential failure conditions that would be extremely difficult to spot through traditional simulation and testing techniques. 

% Figure: counterexample
\begin{figure}[t!]
    \centering
    \includegraphics[width=\columnwidth]{MX_Papers/Paper4/pictures/lilli13.eps}
    \caption{Graphical visualization of the counterexample trace produced by NuSMV when a LTL specification is violated.} 
    \label{fig:counterexample}
\end{figure}

\section{Conclusions and Future Work}
\label{sec:conclusion}
In this paper, we've shown how to use an integrated tool-chain for the analysis and verification of the control software for a real, safety-critical automated system employed in the transport and storage of radioactive material in a nuclear research facility. 
The provided use case was intended to demonstrate the actual feasibility of integrating the phases of modeling, simulation, verification, and analysis in a complex system using an automatic procedure.
The study benefited from the software redesign based on the IEC 61499 standard for several kinds of reasons.
First, it enabled the optimization of code structure by defining standardized, modular, and reusable FBs based on specific ECCs. Second, it allowed for the explicit specification of the relationships and dependencies between FBs while eliminating the incorporation of global variables. Third, it supported the translation of the code into an SMV model, thereby enabling formal verification of LTL safety specifications. Finally, the incorporation of NDTs within different Function Blocks facilitated the simulation of sequence execution under realistic conditions. 
The developed IEC 61499 solution's portability promotes the system to be integrated into various tool-chains. In the proposed example, we investigated this feature by combining it with FB2SMV and FBME for the verification of a set of LTL safety specifications. 
While the first tool is used to extract the software formal model, model verification is subsequently carried out using NuSMV. FBME, on the other hand, is a comprehensive tool, capable of automating the entire verification process by incorporating automatic model generation, NuSMV verification, visualization, and analysis of counterexample trace.
The suggested tool-chain can be instrumental in the early identification of design flaws that could result in potential mechanical collisions. The presented results emphasize the validity of the tool-chain by demonstrating the benefits of formal system verification in detecting non-trivial design errors that may result in a failure event under specific circumstances.
A key feature of the proposed solution, in addition to modularity and portability, is the deep control over localized NDT introduction. This capability can be effective in reducing the process complexity, permitting independent testing of specific FBs, and keeping the time required by model checking within reasonable limits. 
One limitation of the presented methodology resides in the accuracy with which the IEC 61499 model represents the actual system. Indeed, the necessity for mitigating the state explosion problem ultimately led to the adoption of a simplified design, especially with regard to plant FBs. 
Ensuring a high level of accuracy between the model and its real-world equivalent is crucial during this phase. Furthermore, in the provided use case we investigated a single, albeit critically important, remote handling procedure. 
Further developments will allow the software model to be expanded to include more system motion sequences and plant details, thus finalizing the development of a digital twin of the primary SPES remote handling system.

%\section*{Acknowledgment}

\section*{Appendix A: NuSMV script}
\label{appA:NuSMV script}
\begin{figure}[h!]
    \centering
    \begin{lstlisting}[caption=Batch script used to check the LTL specifications with NuSMV.]
    time
    read_model -i HHM_FV_PART_1.smv
    flatten_hierarchy
    encode_variables
    build_model
    time
    check_ltlspec -p "G !(HHM_FV_PART_1_inst.ELplant.POS_OUT = 5)" -o spec1.txt
    time
    check_ltlspec -p "G !(HHM_FV_PART_1_inst.CAplant.POS_OUT = 5)" -o spec2.txt
    time
    check_ltlspec -p "G !(HHM_FV_PART_1_inst.CRplant.POS_OUT = 5)" -o spec3.txt
    time
    check_ltlspec -p "G  !(HHM_FV_PART_1_inst.CRplant.POS_OUT in (2..4) & HHM_FV_PART_1_inst.CAcmd.moving = TRUE)" -o spec4.txt
    time
    check_ltlspec -p "G  !(HHM_FV_PART_1_inst.ELplant.POS_OUT = 2 & HHM_FV_PART_1 _inst.CRplant.POS_OUT = 4 & HHM_FV_PART_1_inst.CAcmd.moving = TRUE)" -o spec5.txt
    time
    check_ltlspec -p "G  !(HHM_FV_PART_1_inst.ELplant.POS_OUT = 1 & HHM_FV_PART_1_inst.CRplant.POS_OUT = 4 & HHM_FV_PART_1_inst.GRplant.GRO = TRUE)" -o spec6.txt
    time
    \end{lstlisting}
    \label{lst:ltl specs}
\end{figure}

\clearpage
\putbib
\end{bibunit}


%-------------------------------------------------------------------
\def\paperheader{Paper E}
\def\papertitle{Process mining in industrial control systems}
\def\paperauthorstring{Midhun Xavier, Victor Dubinin, Sandeep Patil, Valeriy Vyatkin}
\def\referencestring{Proceedings of the IEEE International Conference on Industrial Informatics (INDIN), 2022.}
\def\copyrightstring{2022, IEEE, Reprinted with permission.}

% The definitions above could just as well be put directly into the function
% call below, but were explicitly defined to more clearly illustrate the
% use of the function \makepaper.

\makepaperaccepted
  {\paperheader}
  {\papertitle}
  {\paperauthorstring}
  {\referencestring}
  {\copyrightstring}

% The actual contents is imported by un-commenting the \input line below.
% Make sure the file exist.
\begin{bibunit}
\thispagestyle{plain}

% Add missing command definitions from original paper
\newcommand \op[1] {\ensuremath{\operatorname{\mathbf{#1}}}}
\newcommand \com[1] {\ensuremath{\mathtt{#1}}}
\newcommand{\RNum}[1]{\uppercase\expandafter{\romannumeral #1\relax}}

\section*{Abstract}
	In this paper, we discuss how process mining techniques can be applied in industrial control systems for modeling, verification, and enhancement of the cyber-physical system based on recorded data logs. Process mining is used for extracting the process models in different notations from the recorded behavioral traces of the system. The output model of the system’s behavior is mainly derived using an open-source tool called ProM. The model can be used for such applications as anomaly detection, detection of cyber-attacks and alarm analysis in industrial control systems with the help of various control flow discovery algorithms. The extracted process model can be used to verify how the event log deviates from it by replaying the log on Petri net for conformance analysis.
	
	\section{Introduction}
 
	Process mining \cite{van2012process1}  extracts the behavior of the system by analyzing the events in order and it consists of process discovery, conformance checking and process enhancement. Process models derived from event logs can be classified in different ways like how formal the model is, how the model is constructed, etc., and popular process modelling paradigms are Transition systems, Petri nets, Workflow nets, Business Process Modelling Notations (BPMN), Causal nets (C-nets),  etc. Data mining and process mining have some differences even if they used to predict patterns from data logs. Data mining is used to discover or predict the patterns by analyzing the data sets but process mining combines data analysis with modelling for extracting deep insights about the processes from recorded event logs  \cite{van2016process}. While data mining ignores the processes, process mining is really interested in processes using the data.
	
	There are three components in process mining technique \cite{van2012process2} i.e., process discovery, conformance checking and enhancement. The process discovery generates a model from a recorded event log with the help of several control flow discovery algorithms. The generated model and other event logs from the same system can be compared to identify the deviations this is called conformance checking. The generated model can be updated by analysing a new set of event logs called enhancement.
   
   Many process scenarios can be constructed by simulating the Petri net \cite{petri1962} and this method is called 'Play Out'. Instead of simulating the Petri net, we can use the simulation model or digital twins or even real systems to create the event log and inferring the model from many scenarios or traces is called process discovery or 'Play In'. It is possible to identify the deviation of the model by replaying a scenario on a built model. These features in process mining are helpful to identify bottlenecks in process and where machines deviate from expected process \cite{aalst2011process}.
   
   
	
	Cyber-physical systems (CPS) \cite{lee2017introduction} is a popular designation for complex industrial automation systems with decentralized control logic distributed across many communicating devices, often embedded into various mechatronic components. The IEC 61499 architecture \cite{iec61499part12012} is considered as a suitable method for modelling cyber-physical automation systems. In the modern automation industry world, mixed structure of distributed controllers in different mechatronic components introduces the verification and validation challenges, so formal modelling of CPS is necessary for their formal verification. The latter helps making the system less prone to errors by checking their behaviour comprehensively on compliance with specifications expressed in such formal languages, as temporal logic, e.g., LTL or CTL \cite{yoong2015verification}. Closed-loop modelling is considered beneficial for the verification but it requires the model of the plant. The implementation of the plant model is complex and resource consuming, and it is normally done by manually. The process mining approach opens a wide range of opportunities for modelling industrial control systems. In this paper, we discuss how process mining can be effectively applied in the field of industrial automation systems and it also describes the implementation of process model from event log by process discovery algorithm. Then we demonstrate how to check if the new event log deviates from the expected behaviour.
   
   
   The paper is structured as follows: Section \ref{sec:processmininginindustrialcontrolsystem}  discusses the related work and process mining in industrial control systems. Section \ref{sec:processmodelextraction} explains the event log structure, process mining tools and its advantages, selection of process discovery algorithms and conformance checking of process model in detail.  Finally, Section \ref{sec:conclusion} concludes the paper and outlines future goals.
   
   
   \section{Process mining in industrial control systems : Overview}
   \label{sec:processmininginindustrialcontrolsystem}
   \subsection{Process mining in factory automation}
   Process mining in industrial control systems can be applied in various directions. Process mining techniques applied in factory automation are used for model enhancement and conformance checking \cite{paper1}. The industrial control systems are used to record the events and this trace of events is called event log. The  event log can be in various formats like CSV, XES, MXML etc. The control work-flow model is discovered from the event log with the help process mining algorithms. There are several types of control flow discovery algorithms, but we need to select one of them according to the event log and depending on the goal of the process model. The model built by the process discovery algorithms is evaluated with the help of basic performance analysis which considers fitness, precision, over-fitting, and simplicity parameters. These parameters can be measured by replaying event log on derived model.
   
   \subsection{Anomaly detection using log data}
   
   Anomaly detection using log data found to be another application using process mining technique. Paper \cite{paper2} introduces a method for identifying anomalous behaviour of the industrial control system using device logs with the help of process discovery and conformance checking. Process mining techniques are mainly used in business related areas to improve the process by analysing the event log, but it is possible to detect cyber-attacks and anomalous behaviour of the industrial control system by analysing the event log using conformance checking. In the paper \cite{paper2}, process discovery algorithm generates process models which are used for conformance checking. The latter is done with the help of a token game which compares the new event log with the generated model. Myers found in \cite{paper3} that models created with the help of inductive miners give good fit compared to other miners, especially in the field of industry control system.
   
   \subsection{ Alarm analysis from the event-log database of an industrial plant}
   
   In industries alarm analysis can be done with the help of process mining techniques.  Abonyi and Dorgo explain in \cite{paper4} how process mining techniques can be used effectively for the alarm analysis from the event-log database of an industrial plant. Here, the process model is derived from fuzzy miner instead of the alpha miner \cite{van2004workflow} because alpha miner does not consider number of times the traces are repeated in event log while the fuzzy miner keeps the highly important behaviour of the system.
   
   \subsection{online parameter estimation for CPPS with process mining}
   
   In the cyber physical production systems (CPPS), the need to adjust in production entails the need to change the automation software which requires a lot of manual engineering effort. In order to fix this, paper \cite{paper5} provides online parameter estimation for a CPPS  using process mining. Alpha algorithm is one of the popular process discovery algorithms which creates dependency graphs from event logs. The dependencies between each event are created according to its order of events in the event log. If any noises present in the system, then the alpha algorithm produces a high difference from the expected behaviour \cite{paper5}.
   
   \subsection{PLC programming logic modelling and other applications in ICS }
   
   The alarm analysis, parameter estimation and detection of cyber-attacks in cyber physical systems based on outlier analysis in event logs were major application of the process mining. However,  Theis at al in \cite{paper6} propose a method to model PLC programming logic by analysing event log with the help of process mining. The paper uses split miner as a process discovery algorithm and uses DREAM-NAP (decay replay mining - next activity prediction) for predicting next activities in the running process. In the modern industry world robots help in the manufacturing system to increase the overall productivity of the processes. The data captured from the robots can be used to create a general model of the manufacturing system to understand and extract hidden behaviour of the system. The generated model and other event logs from the same system can be compared to identify the deviations. In this case, the process discovery algorithm inductive miner is used because it gives a more generalized model compared to other control flow discovery algorithms. Another paper \cite{paper8} which collects data from the factory floor and converts it to event stream in order to generate a model for conformance checking. The conformance checking detects deviation from the manufacturing floor.
   
   \subsection{Plant model generation from event log}
   
   In recent years, cyber physical systems are used in almost all wireless communication areas. These systems produce several flaws due to its decentralized and heterogeneous structure. Formal verification of these systems become more relevant in order to verify and detect possible errors in the system. Modelling of the controller is straightforward because logic is already known but the construction of plant model is difficult. Previously, the researchers manually constructed the model of the plant and used it for verification. The paper \cite{etfapaper} describes how to construct plant models automatically from event logs for formal verification. Formal model of the plant in SMV format is developed and verification is done with the help of a symbolic model checker tool called NuSMV.
   
   
   \section{Process model extraction}
   \label{sec:processmodelextraction}
   \begin{figure}[!t]
	   \centering
	   \includegraphics[width=0.3\textwidth]{MX_Papers/Paper5/images/structure.PNG}
	   \caption{Gripper and conveyor system structure}
	   \label{fig:structure}
   \end{figure}
   
   Consider a running example of  'Gripper and conveyor' system and see how the process flow of the system can be derived from the event log with the help of process discovery algorithms. The structure of the system is shown in the Figure \ref{fig:structure} consisting of a gripper and conveyor. Gripper can either move in 'upward' or 'downward' direction and the clamp attached to the gripper component can 'open' and 'close' to grab an object. The conveyor component moves in one direction when the actuator signal is triggered. There are five sensors and three actuators exists in the system and their description is given below:
   
   
   \subsubsection{Sensor signals} 
   \begin{itemize}
	 \item S1\_cup\_detected : Whenever a cup or an object appears in the sensor (S1) then its value will change to 'TRUE' otherwise its value remains the same as 'FALSE'. 
	 \item  S2\_conveyor\_running : If the conveyor is running then the sensor (S2) value becomes 'TRUE' otherwise 'FALSE'.
	 \item S3\_gripper\_at\_bottom : If the gripper's clamp reaches bottom position then the sensor (S3) value becomes 'TRUE' otherwise 'FALSE'.
	 \item S4\_gripper\_closed : Whenever clamp is closed then  sensor (S4) value will change to 'TRUE' otherwise its value remains same as 'FALSE'.
	 \item S5\_gripper\_at\_top : If the gripper's clamp reaches top position then the sensor (S5) value becomes 'TRUE' otherwise 'FALSE'.
   \end{itemize} 
   
   \subsubsection{Control signals}
   \begin{itemize}
	 \item C\_conveyor\_run : If the control signal becomes 'TRUE' then the conveyor starts running otherwise it stops.
	 \item  C\_gripper\_go\_down : If this signal becomes 'TRUE' then the gripper starts moving downwards otherwise it moves upward.
	 \item  C\_conveyor\_to\_close : If this signal becomes 'TRUE' then the clamp closes  otherwise it opens.
   \end{itemize}
   
   
   A simple process sequence is taken into account and it works cyclically for a particular period of time. Initially, the conveyor is moving, the gripper rests at top position and the clamp is already at open condition. Whenever 'CUP' (workpiece) is detected by sensor s1 then the conveyor stops running. Gripper comes down and grabs the 'CUP' with the help of its clamp. After that the gripper returns the 'CUP' on the conveyor and the conveyor moves again. This process runs cyclically and if there is no object then the conveyor keeps on running and the gripper does not do anything.
   
   \subsection{Event log structure, attribute selection and pre-processing}
   
   \begin{figure}[!t]
	   \centering
	   \includegraphics[width=0.3\textwidth]{MX_Papers/Paper5/images/EL.PNG}
	   \caption{Event log}
	   \label{fig:EL}
   \end{figure}
   
	An event description in the log consists of the following fields \cite{paper2}: case identifier, event id/name and attributes. The case identifier is a unique id for each execution of processes, Normally, industrial control system processes are cyclic so each cycle can be considered as a different process instance. The event name refers to the triggered activities that occur while running the system. The attributes’ part consist of resource, timestamp, etc. These attributes give additional information, i.e., ordering information, which component produced the event, etc. The attributes are not mandatory fields but if we get more information, then that would be useful to extract hidden features about the system.
   
	An event log of the 'Gripper Conveyor' system is shown in the Figure \ref{fig:EL} and it consists of five columns: CaseId, Timestamp, Components, Signal, Value. The case identifier in this log is denoted as 'caseId' which is a unique id for each process execution and here the event is composed of three columns: component, signal, and value. Timestamp represents the time at which the event occurred and it is considered as an attribute. This event log is taken as the input for further processing so it's necessary to record the correct information. The event log is sorted using 'Time Stamp' because the order of occurrence of events is a key factor and process discovery algorithms work mainly with the relation of these events.
	
	Event log should be cleaned to get a good quality event log. In many situations event log quality should not be compromised,  in order to produce an accurate model of the system. For anomaly detection, the log pre-processing step is ignored because these outliers or irrelevant events help to detect the cyber-attack while modelling the system with the help of process mining technique \cite{paper2}.
	
   
	\subsection{Process mining tools and its advantages}
   
   Process mining consists of process discovery algorithms and conformance checking can be done via programs, but process mining tools give wider option to use these all algorithms and it provides the result in different notation to give better visualization. ProM \cite{ProM} \cite{van2005prom} and Disco \cite{Disco} \cite{gunther2012disco} are the most popular tools used for process mining. ProM is an open-source tool which is is widely used because of the following features.
   
   
   ProM version 6 consists of 250+ plugins which are used for event log pre-processing, process discovery and visualizations. The representation of the process model can be expressed in different notations but most commonly represented as Petri nets. There are many process discovery algorithms \cite{van2009process} like Alpha, Alpha +, PL based, T alpha, Petrify miner, etc. to produce output models as Petri net and ProM tool which supports almost all of them. ProM also supports plugins for conformance checking, and LTL specification checking. The fitness of the model derived from event log can be analysed using conformance checking. The ProM framework has different plugins for basic performance analysis and these plugins help to identify how much the generated model deviated from event log. The event log analysis, log pre-processing, and conversion from one event log format to another like CSV to eXtensible Event Stream (XES) can be done easily with this tool. On the other hand, Disco, developed by Gunther in 2007, is based on the fuzzy miner algorithm.  The fuzzy miner gives a better interactive representation to understand the system behaviour of complex logs and it also works in the ProM tool.
   
   Most of the process discovery algorithms take input in XES format and produce process models. In order to convert the event log from CSV to XES format, the Standard XES attributes need to be mapped by selecting the ’Case’ columns, ’Event’ columns, ’Start Time’ column and ’Completion Time’ column from the CSV data log. In the ’Gripper Conveyor’ system the event log in CSV is converted by mapping the standard XES attributes as follows:
   
   
   \begin{itemize}
	 \item Selected Case Columns : CaseId
	 \item Selected Event Columns : Component, Signal and Value
	 \item Start Time \& Completion Time is not selected because timing information is not considered for this experiment.
   \end{itemize}
   
   \subsection{Selection of Process discovery algorithms }
   
   \begin{figure}[!t]
	   \centering
	   \includegraphics[width=0.5\textwidth]{MX_Papers/Paper5/images/PN.PNG}
	   \caption{Process model extracted using alpha algorithm in ProM}
	   \label{fig:PN}
   \end{figure}
   
   \begin{figure}[!t]
	   \centering
	   \includegraphics[width=0.37\textwidth]{MX_Papers/Paper5/images/disco.png}
	   \caption{Process model extracted using fuzzy miner in Disco}
	   \label{fig:fm}
   \end{figure}
   
   
	There are several types of process discovery algorithms: Abstraction based, Heuristic based, Search based, Region-based algorithms etc and each algorithm is used to extract different process models using event log. Abstract -based algorithms generate models by ordering relations of events in an event log, and on the other hand heuristic miner generates models where events are ordered based on frequency of events. Events happening in fewer thresholds are ignored so heuristic miners perform better with event logs containing noisy data. Search based algorithms (Genetic algorithm Miner (GA)) which try to mimic the process of evolution. Process discovery technique key factors is the balance between fitness, precision, generalisation and simplicity \cite{buijs2012role}. While selecting the process discovery algorithm, one should consider the following questions: how fast analysis technique produce result, how much memory it is used, what is representation of process discovery algorithm and whether it solves the related problems.
	
   
   The most commonly used process discovery algorithm is the alpha algorithm, which is an abstraction-based algorithm. The process model extracted from the event log \ref{fig:EL} of ’Gripper- Conveyor System’ using the alpha algorithm is shown in the Figure \ref{fig:PN} and it explains the process flow in the Petri net notation. The Petri net consists of 16 transitions and each transition denotes the activity from the event log. The alpha algorithm creates a dependency graph based on the order of events in the event log. It does not consider the frequency of trace, so noises present in the log makes a high difference from expected behaviour. In order to avoid this, we can use fuzzy miner which is a heuristic approach and it generates a model according to the frequency of the traces. According to  \cite{paper4}, ”the algorithm calculates the importance of the activities and how closely the events follow each other”. The process model extracted from the same ’Gripper Conveyor’ system using fuzzy miner in Disco is shown in the Figure \ref{fig:fm}. It is exactly similar to the Petri net obtained from the alpha algorithm. The fuzzy model can be approximated by changing its ’Activity’ and ’Path’ detail from 0 to 100 percentage and the thickness of edges in the graph explains the number of times the particular event to another event is occurred. The fuzzy model is difficult to convert to other process modelling languages but its representation is easy to understand system behavior.
   
   \subsection{Conformance checking}
   
   \begin{figure}[!t]
	   \centering
	   \includegraphics[width=0.3\textwidth]{MX_Papers/Paper5/images/stat1.PNG}
	   \caption{Conformance checking using ProM}
	   \label{fig:stat}
   \end{figure}
   
   
   Conformance checking is used to identify the deviation of the model by replaying a scenario on a built model. The obtained process model is compared with the event log of the same simulation model or real system. Conformance checking can be seen in two perspectives, i.e., how process model deviates from log or how event log deviates from the process model. The first one helps in fixing the process model and second one explains the error occurred in the simulation system or real system. There are several algorithms for conformance analysis like checking causal footprint, token-based replay, aligning observed and modelled behaviour, etc.
   
   
   Checking causal footprint method which is easy to compare footprint matrices of the event log with the existing reference model and fitness of the model can be measured by checking the deviation in two dependent events. This method does not consider the frequency of trace and it compares the event log with Petri net process model notation. Fitness of the model varies from value 0 to 1. If the fitness value of the process model is 1 then everything seen in the event log is possible. In order to consider frequency of trace to account, a basic token replay approach is used and it identifies the deviation and fitness by analyzing the missing and remaining token after replaying each trace on the reference process model. If there is no missing and remaining token on a modeled Petri net then there is no deviation otherwise it does not conform to the derived process model. Token replay consists of following disadvantages : All transitions in Petri net should be uniquely labeled otherwise may be it choose wrong path and give wrong measure of fitness, Token flooding is another problem because whenever there is no transition it adds more token and atlast every transition will be triggered and when local decisions about the path misleads the fitness measure won’t be reliable. Advanced method for analysing conformance is done using alignments and the problems which present in Basic token replay and checking causal footprints never occur in this method. Conformance analysis using alignments is independent of process model notation and it identifies optimal alignment using user defined cost function.
	
   
   The existing event log is added with noise and the de- rived process model is used for conformance analysis. The Replay event log on Petri net for conformance checking and its analysis given by the ProM is shown in the Figure \ref{fig:stat}. The places and transitions where the deviation occurred is explained by this method. It also provides elements statistics and global statistics which helps to get a complete picture of the deviations and other metrics like fitness of the process model. There are different plugins available in ProM which can be used for measuring precision (avoid under-fitting), generalization (avoid over-fitting), fitness (explain observed behaviour) and simplicity. There is no such thing as the best process model because each one will have a different better metric over one another. One need to identify the relevant metric for the analysis and the model which performs better in all those metrics can be selected as the appropriate model.
   
   \section{Conclusion and future work}
   \label{sec:conclusion}
   There are several applications based on process mining techniques in the field of industrial control systems. The different process discovery algorithms help to implement the model in various notations and each discovery algorithm has its own advantages and disadvantages. The appropriate model is selected by considering relevant metrics like fitness, precision, generalization, simplicity etc. The conformance checking in process mining is considered as the most important feature which identifies the deviations by comparing event log and reference model.
   
   The process model extraction using discovery algorithms and conformance checking opens a wide range of opportunities in industrial control systems. The extracted process model derived from event log can be used for the following purposes: generation of monitors in IEC 61499 standard can be used for embedding and monitoring closed-loop system in real-time \cite{wenger2015behavioral}, verification of the conformity of the control system to certified with the derived process model, real-time verification of event log to determine whether the system deviates from its actual process , re-implementing controller design and migration from legacy control systems to the IEC 61499 standard. It is possible to incorporate the process model extraction method with the IEC 61499 tool chain \cite{xavier2021cyber} for automatic verification and validation of closed-loop control systems using CTL or LTL specifications. 
   The automatic generation of plant model  from this derived reference process model could be considered as the next step in future.
   
   \section{Acknowledgements}
   This work was sponsored, in part, by the H2020 project 1-SWARM co-funded by the European Commission (grant agreement: 871743).  
   
%%% Put references here
\putbib
\end{bibunit} 

%-------------------------------------------------------------------
\def\paperheader{Paper F}
\def\papertitle{An interactive learning approach on digital twin for deriving the controller logic in IEC 61499 standard}
\def\paperauthorstring{Midhun Xavier, Victor Dubinin, Sandeep Patil, Valeriy Vyatkin}
\def\referencestring{Proceedings of the IEEE International Conference on Emerging Technologies and Factory Automation (ETFA), 2022.}
\def\copyrightstring{2022, IEEE, Reprinted with permission.}

% The definitions above could just as well be put directly into the function
% call below, but were explicitly defined to more clearly illustrate the
% use of the function \makepaper.

\makepaperaccepted
  {\paperheader}
  {\papertitle}
  {\paperauthorstring}
  {\referencestring}
  {\copyrightstring}

% The actual contents is imported by un-commenting the \input line below.
% Make sure the file exist.
\input{papers/paper6.tex}

%-------------------------------------------------------------------
\def\paperheader{Paper G}
\def\papertitle{A framework for the generation of monitor and plant model from event logs using process mining for formal verification of event-driven systems}
\def\paperauthorstring{Midhun Xavier, Victor Dubinin, Sandeep Patil, and Valeriy Vyatkin}
\def\referencestring{IEEE Open Journal of the Industrial Electronics Society, 2024.}
\def\copyrightstring{2024, IEEE, Reprinted with permission.}

% The definitions above could just as well be put directly into the function
% call below, but were explicitly defined to more clearly illustrate the
% use of the function \makepaper.

\makepaperaccepted
  {\paperheader}
  {\papertitle}
  {\paperauthorstring}
  {\referencestring}
  {\copyrightstring}

% The actual contents is imported by un-commenting the \input line below.
% Make sure the file exist.
\begin{bibunit}
\thispagestyle{plain}

\section*{Abstract}

This paper proposes a method for the automatic generation of a plant model and monitoring using process mining algorithms based on recorded event logs. The behavioural traces of the system are captured by recording event logs during plant operation in either manual control mode or with an automatic controller. Process discovery algorithms are then applied to extract the logic of the process behaviour properties from the recorded event logs. The result is represented as a Petri net, which is used to construct the state machine of the plant model and monitor and is in accordance with the IEC 61499 standard.  The monitor is implemented as a function block and can be deployed in  real time   to trigger an error signal whenever there is a deviation from the actual process scenario. The plant model and controller are connected in a closed loop and are used for the formal verification of the system with the help of the 'fb2smv' converter and symbolic model checking tool NuSMV.




\section{Introduction}

The adoption of the IEC 61499 \cite{iec61499part12012} Standard in Industry 4.0 to model distributed systems signifies a reliable and widely accepted approach \cite{dai2017discrete}. In the IEC 61499 standard, the distributed control system is expressed in terms of a network of function blocks (FB) executed in an event-driven way, while individual FBs are implemented as state machines \cite{dai2012redesign,lewis2001modelling,frey2006modeling} 
called Execution Control Charts (ECC).


The design of distributed control for systems is composed of several mechatronic components with complex interactions and is susceptible to design faults in the real system. In order to detect potential errors during run-time, monitors \cite{wenger2015behavioral,do2020towards} can be used as a monitoring technique. The monitor can observe the process sequence and provide an alert or even stop the process if it deviates from the expected process flow. The deployment of distributed control systems without proper verification can cause significant damage to system components. Simulation \cite{ hegny2010iec} is another widely used error-capturing method, This method does not guarantee the absence of errors in the system. 

Formal verification \cite{patil2015formal,blech2016comparison} can be utilized to identify errors in the designed system by using Computation tree logic (CTL) and linear temporal logic (LTL) specifications. The development of a formal model for the system requires the creation of models for both the plant and controller. While creating a formal model for the controller is relatively straightforward, modeling the plant can be complicated and it requires manual development. Formal verification \cite{cengic2010formal1} offers a more rigorous approach to verifying the correctness of distributed automation systems. The article \cite{ramdani2020rctl} introduces reconfigurable CTL, an extension of classical CTL, aimed at streamlining the formal verification process for complex reconfigurable discrete-event systems by reducing the number of properties needing verification through the classification and prioritization of properties based on dominance and equivalence relations. By computationally exploring a broad range of error-causing scenarios within the system, formal verification techniques can provide a more comprehensive analysis of potential errors than simulation-based methods.

The manual implementation of plant models \cite{hanisch2009one, hegny2010iec} is complex and time-consuming hence the automatic generation of plant models \cite{xavier2021plant, xavier2022plant}  helps in many cases. The plant model can be implemented by obtaining system information through the recording of event sequences that occur within the system. 


This study focuses on generating a monitor and a plant model by leveraging a process mining methodology on recorded event logs. While the monitor can be represented in different process model paradigms for the purpose of conformance checking, this study focuses on developing a monitor based on the IEC 61499 FB. Subsequently, this monitor is deployed to evaluate conformance.


The study addresses the challenge of classic process mining, which typically constructs a model of the entire system. In our approach, the focus is on developing a model specifically for the uncontrolled behavior of the plant, with the intention of later integrating it with the control model to create a closed-loop system model. This approach aligns with the requirements of discrete event system verification.



The paper is structured as follows:  Section \ref{rw} provides an in-depth literature review that compares the proposed approach with existing methods found in prior research. Section \ref{background} provides an overview of the foundational concepts of the Reference Model of Control/Sensor Events Sequencing and outlines the process of decomposing Petri nets into State Machine Components (SMCs). The workflow and potential use cases are detailed in Section \ref{workflow}. Section \ref{FSMfromEL} describes the process of Finite State Machine (FSM) generation from the event log, including the implementation of both the monitor and the plant model within IEC 61499 FBs. Section \ref{cs} offers a case study focusing on a pneumatic cylinder, presenting comprehensive results and analysis. Finally, Section \ref{cf} serves as the conclusion, summarizing the paper's findings and delineating future objectives.


\section{Related Work }
\label{rw}
\hfill 

IEC 61499 \cite{iec61499part12012} is an international standard for the design and implementation of industrial automation systems (IAS), particularly in the realm of distributed control systems.  Christensen et al. \cite{christensen2000design} introduced a model-driven approach for distributed control systems within the context of IEC 61499 systems. Providing a framework for modeling, designing, and deploying distributed control systems using a modular and event-driven approach. IEC 61499 is employed in industrial applications due to its enhanced modularity \cite{thramboulidis2013iec}, control logic reusability \cite{hegny2012towards}, flexibility in complex processes \cite{dai2015bridging}, and improved interoperability among automation components \cite{sunder2006usability}.

The paper \cite{vyatkin2009iec} describes how IEC 61499 standard itself ensures safety, but it is necessary to verify the system before its deployment to guarantee its reliability.  Hegny et. al \cite{hegny2010iec} presented  IEC 61499 based simulation framework for model-driven production systems development. While simulation is valuable for identifying bugs and assessing system behavior, it alone does not guarantee error-free operation. In \cite{blech2016comparison},  the comparison of formal verification approaches for IEC 61499 to enhance safety assurance is explained. Formal analysis of IEC 61499 applications \cite{cengic2010formal1, cengic2010formal2 } involves mathematically proving that the control system meets specified safety properties and requirements. By applying formal methods \cite{schnakenbourg2002towards,yoong2010modelling}, potential hazards and errors can be rigorously identified and corrected.



In \cite{ vyatkin2008closed, pang2009systematic}, systematic closed-loop modelling in IEC 61499 FBs and its  formal verification is presented. Closed-loop formal verification in IEC 61499, where the plant model and controller are connected in a closed-loop, is a better approach \cite{sinha2019survey, xavier2023formal} as it allows for the verification of system properties under realistic operating conditions, ensuring that the control system behaves correctly and safely in response to dynamic and unpredictable environmental changes \cite{ovsiannikova2023formal}.



In \cite{hanisch2009one}, the plant model is presented not only for simulation but also serves the purpose of formal verification. The paper \cite{malik2017emulation} presents an efficient approach wherein the plant model is constructed within an IEC 61499 FB, simplifying the integration of the controller and plant within a closed-loop system. The introduction of the fb2SMV tool \cite{fb2smv} facilitates the conversion of FB code into SMV format, enabling the verification of CTL or LTL specifications through NuSMV model checker \cite{cimatti2002nusmv}, thus streamlining the formal verification process. In  \cite{xavier2021cyber}, a plant model in IEC 61499 FB with non-deterministic transitions (NDT) is introduced for formal verification. While this method provides a more realistic means of implementing the plant model, it does entail manual work that is susceptible to errors and can be time-intensive. In our previous study\cite{xavier2021plant}, we proposed a semi-automatic plant model generation in SMV, but it posed challenges in integrating with the SMV model of the controller for closed-loop formal verification.



The research gap centers on the automatic generation of a plant model within IEC 61499 FBs to facilitate formal verification, and this paper aims to fill this void. In this study, we employ recorded events and leverage a process mining approach to develop a plant model. While process mining typically yields an overview of the entire system's behavior, this paper specifically details the extraction of the plant model using IEC 61499 FBs from this broader system behavior.



Process mining is a data-driven technique used to discover, monitor, and improve real-world processes by analyzing event logs and extracting valuable insights about how these processes are executed \cite{agrawal1998mining, van2012process1}. Beyond the realm of business process modeling, such methods are also applied in anomaly detection, cyber-attack detection, and alarm analysis within industrial control systems, utilizing a range of control flow discovery algorithms \cite{xavier2022process}. By analyzing event logs from various components and sensors, it can provide a comprehensive view of the system's operational behavior, enabling the identification of bottlenecks, deviations, and inefficiencies. This study applies similar principles to derive the process sequence of a controlled system, utilizing a mining algorithm for process discovery. ProM \cite{van2005prom} and DISCO \cite{gunther2012disco} are the widely used process mining tools for the preparation of event logs, process discovery, visualization, and conformance checking.  ProM \cite{ProM} which is used in this study, is an open-source tool, and comprises several process discovery algorithms and plugins for conformance checking and visualization. The \textbf{alpha} algorithm \cite{petri1962} is selected as the process discovery algorithm in this study due to its simplicity and transparency. Other process mining algorithms, such as the Heuristics Miner, Inductive Miner, or Genetic Miner are suggested when more complex or detailed insights are required \cite{dunzer2019conformance, naderifar2019review}. In recent research, process mining algorithms have shown the capability to handle multiple-concurrency short-loop structures effectively \cite{sun2021algorithm,sun2021process}. These algorithms can handle larger and more intricate event logs, making them suitable for in-depth analysis, optimization, and automation of processes. The selection of the algorithm should align with the objectives of the level of complexity in the process being analyzed \cite{xavier2022process}.



The study provides a comprehensive understanding of the system's behavior through event logs is available. This sets the stage for performing conformance checking  based on the model derived from these event logs \cite{van2012replaying, munoz2016conformance}.  This paper introduces a method for the automated generation of a monitor in IEC 61499 FBs, facilitating real time deployment for monitoring purposes. This IEC 61499 monitor FB analyzes all events to detect any deviations from the expected flow.  The combination of this monitor in the form of an IEC 61499 FB and the plant model represented similarly in IEC 61499 FBs becomes particularly advantageous when introducing new controllers or migrating to IEC 61499 FBs. The newly introduced controller can seamlessly connect in a closed-loop configuration with the plant model, and the developed IEC 61499 monitor FB allows us to monitor system behavior. Any deviation in events leading to an error state triggers the monitor, ensuring that the newly developed controller performs in accordance with the previous controller. The option to create a formal model of the system and perform formal verification adds an extra layer of safety assurance before deployment into the operational system.




\begin{figure*}[!t]
	\centering
	\includegraphics[width=0.9\textwidth]{MX_Papers/Paper7/images/workflow1.png}
	\caption{Workflow and use cases.}
	\label{workflowUsecase}
\end{figure*}

\section{Background}
\label{background}

\subsection{RMCSES Concept}
\label{rmcses}

The Reference Model of Control/Sensor Events Sequencing (RMCSES) is a formal model that developed from the process mining  \cite{cook1998discovering} of event logs describing the functioning of a closed-loop IAS in an "error-free" mode over an extended period. RMCSES offers a condensed representation of a vast event log, encompassing all possible signals from all components. Conceptually, if one imagines the event log as a collection of sentences or phrases of a formal language, describing the behavior of the IAS \cite{sinha2019survey}, then RMCSES functions as a sentence generator for this language. Due to its conciseness, RMCSES can be readily implemented in software or hardware.


The interface of the RMCSES is shown in Figure \ref{workflowUsecase} (D). The RMCSES only takes into account control signals originating from the controller and informative signals originating from sensors as events. These signals represent the interface between the controller and the plant. Internal signals circulating within the control system and the plant are not considered but it is possible to include control signals from external sources such as an operator.




In an IAS operating in an "error-free" mode, the functioning of each component is characterized by cyclical, meaningful, and locally complete processes. The insignificant operation of components, such as when an ejector piston moves back and forth indefinitely, is not considered. The component is identified to have meaningful behavior when it has a specific goal and actively works towards achieving it. Each component or device follows a specific scenario that begins with its initialization and ends with its termination. A multi-functional device may execute different scenarios depending on the specific task or operation it is performing.

Figure \ref{workflowUsecase}, (B) illustrates the cyclical operation of a component (device) and the possible cut-off points of the event log. As the functioning of a component (device) is cyclical, meaningful, and locally complete, the event log represents only a part of the process history of this component (device), leading to incomplete operation traces in the log that will be mapped onto the formal model. Therefore, a question arises as to how these incomplete traces, or "scraps," affect the extraction of an event log. Our experimental results indicate that the addition of such scraps to the event log does not result in significant changes.

Given the convenience of implementing the RMCSES as an IEC 61499 FB, we have opted to employ it as an FSM in our research; we do not preclude the use of alternative models. The RMCSES will constitute a comprehensive representation of the lower-level operation of the system; for intricate IAS, this model may become unwieldy.


%%As a rule, Petri nets \cite{van2013discovering} are used as a modeling formalism in Process Mining, since they are much more powerful than finite-automata (or, FSM) models. The state machine Petri net \VV{[REF?] - what is it?}} an be manually transformed using certain refactoring rules to a FSM representation and this approach is a practically “one-to-one” mapping into an ECC diagram of the basic FB. The goal of the refactoring is to get rid of constructs like "Fork into parallel branches" and "Join parallel branches" but such transformations are not always possible. 


Petri nets  \cite{van2013discovering} are frequently employed as a formalism for system modeling in the domain of process mining, as they offer greater expressive power in comparison to finite-automata models. The Petri nets are subsequently decomposed into SMCs and then composed to create an FSM through the application of specific refactoring rules. This FSM can be readily mapped onto an ECC diagram of a basic FB, with a near "one-to-one" correspondence. The primary goal of this refactoring procedure is to eliminate constructs such as "Fork into parallel branches" and "Join parallel branches". It should be noted that such transformations may not always be feasible.



\subsection{Overview of Petri net Decomposition Techniques}
\label{Petri net Decomposition}

Petri nets are essential for modeling concurrent systems \cite{liu2022petri} but can become complex as systems grow in scale. To manage this complexity, various decomposition techniques have been developed. These methods, including the Reachability Graph Method, Structural Decomposition Methods, and Linear Algebra-Based Methods, aim to break down Petri nets into more understandable components while preserving their essential behavioral characteristics. The Reachability Graph Method \cite{buchholz2002hierarchical} involves analyzing the reachability graph to identify subsets of states and transitions, facilitating the partitioning of the Petri net into manageable modules. Structural Decomposition Methods \cite{ye2017structural, hsieh2011robustness} leverage properties like perfect graph theory to systematically break down complex nets into simpler subnets. Linear Algebra-Based Methods \cite{wisniewski2019decomposition, best2013petri} use matrix manipulation to extract sequential components from place invariants, aiding in the identification of independent modules within the Petri net. These techniques offer distinct advantages in analyzing, verifying, and understanding concurrent systems modeled with Petri nets.


In this paper, the focus lies on accurately modeling a plant using IEC 61499 FBs to ensure the comprehensive capture of all its behaviors, vital for the proper functioning of the closed-loop system with a controller. To achieve this, the reachability graph \cite{miyamoto2013modular, giua1994petri} method is employed as a decomposition technique to derive SMCs from Petri nets. This method offers a structured and systematic approach to exploring all reachable states and transitions within the system, guaranteeing completeness in capturing all potential behaviors. By exhaustively mapping out all possible states and transitions, it ensures that no aspects are overlooked during analysis, providing a thorough understanding of the system's behavior dynamics and potential outcomes. It is susceptible to the state explosion problem \cite{kungas2005petri,buchholz2002hierarchical}, particularly in larger and more complex Petri nets, leading to computational inefficiency and difficulty in analysis. In contrast, structural decomposition methods leverage the Petri net's structural properties \cite{wisniewski2019analysis} to identify cohesive subsets that can be modeled as individual SMCs, providing flexibility in decomposition criteria. While these methods can offer more manageable representations and may be less prone to state explosion, their effectiveness heavily depends on the chosen structural properties and decomposition criteria, and may require domain-specific knowledge for optimal results. 

This paper discusses on the complete automatic generation of a plant model in IEC 61499 FB, identifying structural properties poses a challenge and necessitates domain-specific knowledge. Structural decomposition methods may offer a different approach by breaking down the system into smaller components based on structural properties \cite{kiviriga2023efficient}. This approach may overlook certain behaviors arising from interactions between components, unlike the comprehensive exploration provided by Reachability graphs. Reachability graphs provide a fine-grained analysis of the system's behavior, capturing individual states and transitions, which may be lacking in granularity in structural decomposition methods. Linear algebra-based methods \cite{wisniewski2014theoretical} provide a systematic approach to extracting sequential components from place invariants, offering insights into the structural properties of the Petri net. They may have limited applicability to Petri nets with complex or nonlinear dynamics and may be computationally intensive for large systems, requiring expertise in linear algebra \cite{cai1995modeling, wisniewski2019decomposition} for implementation and analysis. Overall, while the reachability graph method ensures completeness but suffers from state explosion, structural decomposition methods offer flexibility but depend on chosen criteria, and linear algebra-based methods provide insights but may have limited applicability and computational complexity, highlighting the trade-offs between completeness, flexibility, and computational efficiency in decomposing Petri nets into SMCs. 

Considering the trade-offs among different decomposition techniques, the need for completeness in capturing all system behaviors to ensure successful formal verification of closed-loop system properties is paramount. The challenge of state space explosion looms large, particularly in the Reachability graph method \cite{zambon2012graph, apel2016fly}. To address this, several strategies can be implemented \cite{xin2009methods}. State Space Reduction techniques involve the adoption of Symbolic Representations such as Binary Decision Diagrams (BDDs)  or Decision Diagrams to compactly represent sets of states \cite{shrestha2009decision}, thereby reducing memory requirements and mitigating state space explosion. A recent study \cite{he2022petri} presents an approach utilizing reduced ordered binary decision diagrams (ROBDD) to address state explosion issues in verifying privacy properties of multiagent systems modeled with knowledge-oriented Petri nets (KPNs) and computation tree logic of knowledge (CTLK), significantly reducing verification time even for large-scale systems such as the dining cryptographers protocol (DCP). Abstraction and Aggregation methods \cite{ludtke2018state} selectively abstract or aggregate unnecessary details in the reachability graph, focusing on essential behavioral properties while discarding less critical information, thus reducing the size of the state space and enhancing efficiency. Partial Order Reduction techniques \cite{godefroid1996partial, clarke1999state} aim to diminish redundant interleavings explored during state space traversal, avoiding the exploration of equivalent states and further reducing the size of the state space. clarke1999state,  a Hybrid Exploration Strategy is suggested. It involves Incremental Construction, where the reachability graph is dynamically constructed as needed during analysis, conserving memory and computational resources. On-the-Fly Exploration is employed to explore states as needed rather than generating all states upfront, efficiently managing state space explosion. Structural Decomposition techniques are integrated into the hybrid method, wherein the Petri net is decomposed into smaller subsystems, and reachability analysis is performed on each subsystem independently. This approach reduces the overall size of the state space that needs to be explored and analyzed, further mitigating state space explosion while ensuring comprehensive system behavior coverage.

Combining SMCs into a cohesive FSM necessitates careful integration to ensure synchronization and coordination between them. Several methods facilitate this process: Sequential Composition \cite{wisniewski2017prototyping}, where the output of one SMC becomes the input of another, suitable for systems with well-defined sequential behavior; Parallel Composition, involving concurrent execution of SMCs and combining their states and transitions to represent parallel behavior \cite{czerwinski2013finite}; Synchronization and Handshaking techniques \cite{cortadella2012logic} ensuring coordination between SMCs through protocols and clock domain crossing mechanisms; Global State Encoding \cite{el2006finite}, where individual SMC states are merged into a unified state space for comprehensive system representation; State Merging, identifying and collapsing equivalent states to manage complexity \cite{czerwinski2013synthesis}; and Interface Design and Refinement, establishing clear interfaces and refining interactions for modularity and maintainability. These methods offer practitioners a toolkit to effectively integrate SMCs, with choices guided by system characteristics like concurrency and synchronization needs, enabling the creation of a coherent FSM reflecting the overall system behavior.

For developing an FSM, a combination of parallel composition and synchronization techniques is preferred. Given the distributed nature and modularity requirements of IEC 61499 systems, parallel composition allows individual SMCs to represent concurrent activities effectively, aligning with the system's modular architecture. This approach facilitates easier maintenance, debugging, and resource utilization optimization. Synchronization techniques ensure proper coordination and communication between distributed SMCs, meeting real-time constraints and enhancing system correctness and reliability. By leveraging parallel composition and synchronization, developers can seamlessly integrate SMCs into a coherent FSM, enabling the development of robust and efficient ECC in IEC 61499 FB environments.

\section{Workflow}
\label{workflow}


This section outlines the workflow and potential use cases enabled by RMCSES, as illustrated in Figure \ref{workflowUsecase}. Initially, signals from the distributed control system are recorded to construct RMCSES using process mining techniques. RMCSES then serves as the basis for generating the monitor (D1), plant model (D2), control model (D3), and control program (D4). In our prior research, we presented (D3) an interactive learning approach for deriving controller logic following the IEC 61499 standard \cite{xavier2022interactive}. Our previous study detailed (D2) the methodology and implementation of a basic plant model (a horizontal cylinder) \cite{xavier2022plant}. In this study, we enhance the methodology by incorporating global state information to create a precise plant model, subsequently closing the loop with a controller for formal verification using a toolchain \cite{xavier2021cyber}. Developing the monitor in IEC 61499 FBs is instrumental for real time error identification and prediction within the controller. In this paper, we develop such a monitor from event logs for conformance checking. Lastly, future work is planned to re-implement control and facilitate platform migration by deriving the control program from RMCSES.



\section{FSM from event logs and their implementation in IEC 61499 FBs}
\label{FSMfromEL}

This section describes the steps involved  in generating  IEC 61499 FBs from the recorded event log.

\begin{figure}[!t]
	\centering
	\includegraphics[width=0.4\textwidth]{MX_Papers/Paper7/images/Methodology.png}
	\caption{Tool chain and data flow for generating FSMs from Event Logs and their implementation in the form of IEC 61499 FBs.}
	\label{monitorFlowchart}
\end{figure}
    
\subsection{Petri net construction from event logs}



\begin{figure*}[!t]
	\centering
	\includegraphics[width=1\textwidth]{MX_Papers/Paper7/images/PN2FSM.png}
	\caption{Transformation of Petri net to FSM for Monitor implementation.}
	\label{TR_PN2FSM}
\end{figure*}


The process of generating FSM from event logs and their implementation in the IEC 61499 FB is shown in Figure \ref{monitorFlowchart}. To start the process, the event log is given as input to the process mining tool called ProM \cite{ProM}, which is an open-source tool that offers various features such as representing process logic in different models, event log pre-processing, format conversion, conformance checking, LTL specification testing, and visualizations. The input event log is provided to ProM in CSV format, which then preprocesses the data and converts it into eXtensible Event Stream (XES) format. ProM then applies a process mining algorithm to the event log in XES format to extract processes in the form of Petri nets.


\subsection{ FSM generation from Petri net
}

\paragraph{From Petri Nets to SMCs: Utilizing Reachability Graphs for Decomposition}


To depict a Petri net within the ECC of IEC 61499 FBs, it's crucial to first decompose the Petri net into SMCss. Subsequently, these SMCs can be composed into a FSM.The extracted Petri net can be exported in the Petri net Markup Language (PNML) format using the ProM tool. The PNML-formatted Petri net is then given as input to the TINA tool  \cite{berthomieu2004tool}, which constructs a RG and can perform Petri net simulation. The finite RG is decomposed to SMC with spontaneous transitions as follows. 


Consider a running example "Repairing telephones" for the implementation of the IEC 61499 FB of the FSM for conformance checking. The description of the "Repairing telephones'' event log is given in ProM 6 tutorial, 2010 \cite{prom6tutorial}. The generated Petri net from the event log using an inductive miner is shown in Figure \ref{TR_PN2FSM} (A). A supplementary transition is added to the Petri net to make the process cyclic. In Figure \ref{TR_PN2FSM} (A), the Petri net is under stepwise simulation, and the 'Repeat' transition is added to make this process cyclic. Graphical representation of the RG is drawn manually using the RG text from TINA is outlined in Figure \ref{TR_PN2FSM} (B). The SMC is derived from the RG, and the transitions in the SMC are designated with the same names as the corresponding transitions in the Petri net. The "spontaneous" transitions are marked in blue color.


There appears to be a discrepancy regarding the potentially infinite nature of constructing a reachability set for a Petri net, which, in theory, could be infinite. It's essential to clarify that this set should not be infinite, as the event log, by its very nature, is finite. From a finite event log, a Petri net should indeed yield a finite number of reachable markings. The count of these markings should not exceed the number of entries in the event log, serving as an upper bound. This is grounded in the event log's representation as a very large RG with linear topology. Each event (entry) in the log distinctly defines a transition from one state (marking) to another. Modern tools for Petri net analysis can construct high-dimensional RGs, like those achieved through the Binary Decision Diagram (BDD) method. The challenge lies not in constructing the RG but rather in its implementation using IEC 61499 FBs. Nevertheless, even in cases where the RG comprises several thousand states, this challenge could be surmountable with the availability of appropriate CAD tools. It's worth noting that reduction methods for Petri nets may offer a viable strategy to significantly reduce the size of the RG, further enhancing the manageability of this complex process.

\paragraph{Tackling State Space Explosion: Effective Strategies for Petri Net Analysis}

Addressing the state space explosion issue in the reachability graph method for Petri nets involves employing various strategies to manage the exponential growth of states and transitions. Techniques such as abstraction and aggregation identify high-level patterns and structures for simplification, while partial order reduction reduces redundancy by exploring subsets of transitions together. Symmetry reduction exploits symmetrical properties to collapse equivalent states and transitions, effectively pruning redundant branches. On-the-fly exploration dynamically generates states and transitions only when needed, reducing memory consumption. State compression techniques compress state representations for efficient storage, and parallel and distributed exploration leverage parallelism to speed up analysis. By utilizing these methods, practitioners can navigate the complexities of larger Petri nets while efficiently managing computational resources, with the choice of approach guided by specific system characteristics and resource constraints.

\paragraph{Composing FSM from SMCs}
Developing a FSM from SMCs demands meticulous synchronization of decomposed components, a challenging task influenced by the system's clock domains. Within the framework of IEC 61499 FB, creating an ECC requires selecting an appropriate method for merging SMCs into an FSM, tailored to the specific requirements and characteristics of the distributed control system. The combined utilization of parallel composition and synchronization techniques emerges as the favored approach, offering concurrency management, modularity preservation, resource optimization, adherence to real-time constraints, and reinforcement of system correctness and reliability. These methodologies seamlessly align with the distributed and modular architecture of IEC 61499 systems, making them well-suited for consolidating SMCs into a coherent FSM.

To implement the strategy of parallel composition and synchronization techniques, the initial step involves identifying individual SMCs representing distinct functionalities within the distributed control system. Clear interfaces must be defined to delineate inputs, outputs, events, and communication protocols, facilitating seamless coordination among SMCs. Each SMC is then implemented as a separate module, enabling concurrent execution to handle various tasks efficiently. Event-driven communication serves as a pivotal synchronization mechanism, establishing communication channels between SMCs to facilitate information exchange and event triggering. Integration of SMCs into a unified system requires ensuring accurate interaction and synchronization based on defined interfaces and communication channels. Rigorous testing is essential to validate correctness, reliability, and real-time performance, with iterative refinement and optimization of the FSM to meet desired behavior and performance criteria. This approach, aligned with the distributed and modular nature of IEC 61499 systems, fosters the development of robust and efficient distributed control systems.

\subsection{Determinization of FSM with spontaneous transitions}

Non-determinism may occur in FSMs when employing "spontaneous" transitions within Petri nets. This means that the transitions of the Petri net are not related to any event. In the ProM system, they are indicated by black rectangles in Petri nets. The rule of mapping the transitions is as follows: A transition in an FSM  will be spontaneous if it corresponds to a "spontaneous" transition in the Petri net. In FSM, spontaneous transitions are denoted by the symbol $\lambda$  ("lambda")  or  $\epsilon$ ("epsilon").

The process of determinization can significantly simplify the implementation of a FSM in software. In this case, this process consists of getting rid of $\lambda$-arcs representing spontaneous transitions.


In this study, we use two approaches to the determinization of FSM that contain spontaneous transitions. The essence of the first approach is as follows: two vertices (states) connected by a $\lambda$-arc are combined into one vertex. In this case, the incoming and outgoing arcs of both vertices are combined, and the $\lambda$-arc is removed (see Figure \ref{TR_PN2FSM} (D) ). This rule is applied until there are no $\lambda$-arcs left in the transformed graph. This method is not universal and it is applicable only in the case of a tree-like topology of $\lambda$-arc relationships. In the second (universal) method, two vertices are contracted into one vertex if one vertex is reachable from another vertex through a chain of $\lambda$-arcs.


Determinization of an FSM can be omitted when there are no states with two or more spontaneous transitions outgoing from a state of the FSM. If there are no such situations, then a spontaneous transition can be interpreted in the ECC as a transition with an always true condition “1”.


\begin{figure*}[!t]
	\centering
	\includegraphics[width=1\textwidth]{MX_Papers/Paper7/images/ConformaceCheckingApp.jpg}
	\caption{Application for conformance checking.}
	\label{conformanceCheckingApp}

\end{figure*}


\subsection{Generation of Deterministic FSM }



The JFLAP tool \cite{rodger2015jflap} is used to implement the determinization of FSM. In this process, only one-letter designations of input symbols (signals) can be used. To encode the long names of input events used in the Petri net for the given example, we use a one-letter encoding (Figure \ref{TR_PN2FSM}).


The non-deterministic FSM  is shown in Figure \ref{TR_PN2FSM} (B), and spontaneous transitions are labeled with the symbol $\lambda$ (“lambda”). The transition labeled as J (i.e., "Repeat") from q9 to q0 will not take part in the determinization process. Instead, it will be replaced with the symbol "1" in the corresponding ECC. After determinization, the deterministic FSM is obtained as shown in Figure \ref{TR_PN2FSM} (C). GraphML format \cite{GraphMLSpec} was used to visualize the reachability graph with the  help of visualization tools like Gephi \cite{bastian2009gephi}, Yed \cite{yworks} etc. 

We employ a process discovery algorithm to generate a Petri net from the event log. Subsequently, we utilize the TINA tool to convert this Petri net into SMCs. This SMCs is further transformed into a deterministic FSM with the assistance of the JFLAP tool. These tools have gained widespread acceptance and are known for their accuracy and reliability in performing these conversions. By following this step-by-step approach and analyzing the model at each stage, we ensure the correctness of the transformation process. Attempting a direct conversion could be error-prone, making these integrated tools the preferred choice to prevent inaccuracies and streamline the workflow.




\subsection{Transformation of FSM to IEC 61499 FBs for monitor implementation}


The structure of the application for conformance checking is depicted in Figure \ref{conformanceCheckingApp}. The application is based on a closed-loop system that includes a plant and a controller. The RMCSES obtained from the event log is connected to the closed-loop system for the purpose of conformance checking. The RMCSES is implemented as a FSM that uses input data to drive its logic. The FSM is capable of detecting errors and verifying the correct sequence of steps in the process flow by monitoring events from the sensor and actuator in real time.







Figure \ref{conformanceCheckingApp} depicts the conversion of the RMCSES into the monitor. The monitor interface FB receives input signals in the form of controller signals $c_1 \dots c_n$ and sensor signals $s_1 \dots s_m$. The ECC is generated through the utilization of the transformation rule depicted in Figure \ref{conformanceCheckingApp}. An algorithm, based on this transformation rule and facilitating the conversion of FSM to ECC with error handling, is presented in Algorithm\ref{FSMtoMonitor}. When the monitor FB detects a valid transition from state $q_i$ to $q_j$ triggered by event $X_k$, it generates an output indicating a successful transition, denoted as "OK." If an unexpected event occurs, it transitions to the ERROR state, producing an ERROR event and providing details about the event that caused the error (EventID) and the StateID, which identifies the state where the error took place. This transformation rule ensures that the ECC reflects the behavior of the FSM  and is equipped to handle errors.









\begin{algorithm}[t]
    \caption{FSM to Monitor ECC }
    \label{FSMtoMonitor}
\SetKwInOut{Input}{Input}
\SetKwInOut{Output}{Output}

\Input{ FSM = \{ Q: $\{q_0, q_1, \ldots, q_{n-1}\}$  
is a set of \\ states, where  $q_0$ is an initial state;\\
CS:  $\{X_1, X_2, \ldots, X_{p+r}\}$ 
is a set of input \\ symbols, where $X_i \in C \cup S$, where \\ C= $\{c_1,c_2,..., c_p\}$ is a set of control \\ signals, and S= $\{s_1,s_2,..., s_r\}$ is a set of \\ sensor signals; \\
$\delta \subseteq Q \times CS \times Q$ \} is a transition relation;
}
\Output{ ECC = \{ QC: $\{q_0, q_1, \ldots, q_{n-1}, {SE_n, SE_{n+1}} \dots {SE_{n+p+r-1}}\} \supset Q $ 
is a set of ECC states; \\
CS is a set of input events (see above);\\
Out: $\{\text{OK}, \text{ERROR}\}$  is a set of output events;\\
Alg= $\{a_0, a_1, \ldots, a_{n+p+r-1} \} $ is a set of \\ algorithms; \\ 
$\phi: \{ \text{Q} \times \text{CS} \rightarrow \text{QC} \times \text{Alg} \times \text{Out}$ \}  is an \\ ECC  transition function\} 

}
StateID=$q_0$  \\
{/* Creating the conforming (right) ECC transitions */}

\ForEach{$(q_i, X_k, q_j) \in \delta$ }{ 
Create ECC Transitions:
$(q_i, X_k) \rightarrow ( q_j, a_j, \text{OK})$
$a_j.StateID=j$  
} 

{/* Creating the non-conforming (erroneous) ECC transitions */}

\ForEach{$ q_w \in Q$ }{ 
\ForEach{$ X_e \in CS$ - $T(q_w)$}{ 
Create ECC Transitions:
$(q_w, X_e) \rightarrow (SE_e, a_e, \text{ERROR})$
$a_e.EventID=e$  
} 
} 
\Return{ECC}\; 

\textbf{ *Note* : Used variables and sets} \\

 StateID is an  output variable of the FB Monitor that stores the identifier of the current ECC  state;  \\

$a_j$.StateID is an occurrence of the  StateID  variable in the $a_j$ algorithm  associated with an ECC state;   \\

EventID is an  output variable of the FB Monitor that stores the identifier of the erroneous (not expected) input event; \\

$a_e$.EventID is an  occurrence of the EventID variable in the $a_e$ algorithm associated with an ECC state; \\

$T(q_w) \subseteq CS$ is a set of signals labelling the transitions outgoing from the ECC state $q_w \in Q$.

\end{algorithm}

\begin{figure*}[!t]
	\centering
	\includegraphics[width=1\textwidth]{MX_Papers/Paper7/images/MR.PNG}
	\caption{The FB interface and ECC of Telephone Repairing Monitor.}
	\label{TR_MonitorFB}
\end{figure*}

The Monitor FB shown in Figure \ref{conformanceCheckingApp}  is a direct implementation of the FSM, representing the formal reference model. The FSM is mapped to the ECC practically "one-to-one" and the FB interface is formed as follows. Each event is assigned its own event input. The INIT input signal is intended to set the model to the initial state. The signal from the event output 'OK' is issued when the received input event is in conformance with the formal reference model. In this case, the number (id) of the ECC state is issued, to which the transition took place (the output variable StateID). When an error is detected, not only the state number (StateID) where the error occurred but also the number of the input event (EventID) that led to the error is issued. Thus, the error is captured. The FB works until the first error appears, after that the FB is blocked. i.e., it becomes not responsive to any inputs. The ECC model is complete in the sense that from each basic ECC state, there are outgoing arcs labeled by each of the input events. Input events that are not specified for a state of the formal reference model transition the ECC to one of the error states. The ECC representation for the running example "Repairing telephones" is shown in Figure \ref{TR_MonitorFB} (B). For the sake of simplicity transitions to error, states are shown for only one test state q11.



It should be noted that the implementation of the monitor may differ in different tools. For example, in FBDK \cite{holobloc} it is simpler than in NxtStudio. This is due to the difference in execution models of basic FBs in these tools. While in FBDK the input event is immediately cleared after the activation (firing) of the ECC transition, in NxtStudio it is not cleared during the entire execution time of the FB (run-to-complete). As a result, during the execution of the FB, several ECC transitions marked with the same input event can be triggered.



\subsection{Extraction of plant model from the overall system model}

The FSM derived from the event log serves as a representation of the overall system behavior. In this study, an innovative approach for extracting the model of the uncontrolled plant is presented. The system is viewed as a closed-loop configuration, encompassing both the plant and the controller. The objective, therefore, centers on obtaining the model of uncontrolled plant behavior, which can subsequently be integrated with the control model, resulting in a closed-loop system model. This approach is particularly tailored to meet the requirements of event-driven system verification. The process of transitioning from the FSM to the plant model in the form of IEC 61499 FBs is elaborated upon below, illustrating our methodology for achieving this integration.

\subsection{Transformation of FSM plant model to IEC 61499 FBs }

\label{subsec:transformationOfFSMtoFB}

\begin{figure*}[!t]
	\centering
	\includegraphics[width=0.9\textwidth]{MX_Papers/Paper7/images/VerificationApp1.png}
	\caption{Application for verification.}
	\label{verificationApp}
\end{figure*}


The presented application for verification follows a structure illustrated in Figure \ref{verificationApp}, where a closed-loop system is formed by connecting a plant model obtained from the RMCSES with either a new or an existing controller. An existing controller is utilized to construct the RMCSES, whereas a new controller is connected with RMCSES for the purpose of verification. The verification process involves conformance checking to ensure that the newly developed controller operates in accordance with the previous controller. NuSMV tool can be used for verification through CTL/LTL specifications.





\subsubsection{Transformation Rules from FSM to ECC }



The graphical depiction of the transformation of RMCSES to plant model in IEC 61499 FB is illustrated in Figure \ref{verificationApp}. The resulting plant model interface comprises control signals $c_1 \dots c_n$ and non-deterministic transitions (NDT) as input signals, while sensor signals $s_1 \dots s_m$ are designated as the output signals of the FB. In this transformation, any transition in the FSM triggered by a sensor signal is replaced by an NDT transition, with the output of the next state serving as a sensor event signal. This NDT serves as a mechanism for initiating transitions at arbitrary time intervals.  On the other hand, transitions within the FSM that are initiated by control signals remain unaltered when transitioning to the ECC model.




\paragraph{Case 1: Diverging Sensor Signals }



The transformation rule when sensor signal arcs diverge from a state is shown in Figure \ref{verificationApp}. When signals emanate from one state and branch out to multiple states, the sensor signals are substituted with NDT signals, and each output place generates the corresponding sensor signal as its output.



Given transitions in the FSM:


\[
\begin{aligned}
(q_x, s_i) & \rightarrow q_i \\
(q_x, s_j) & \rightarrow q_j \\
\end{aligned}
\]



In ECC, these transitions are replaced by NDT transitions with output events $s_i$ and $s_j$:



\[
\begin{aligned}
(q_x, \text{NDT}) & \rightarrow (q_i, s_i) \\
(q_x, \text{NDT}) & \rightarrow (q_j, s_j) \\
\end{aligned}
\]


\paragraph{Case 2: Converging Sensor Signals }



When signals come together into a single state from various states, they are altered according to the rule depicted in Figure \ref{verificationApp}. It's important to note that it's not feasible to bring multiple NDT signals together into the same state and generate two sensor signals as outputs. To address this, we introduced additional intermediary states that produce their respective sensor signals as outputs, and eventually, these states converge into a single state. The transformation rule for this scenario is as follows:



Given transitions in the FSM:
\[
\begin{aligned}
(q_i, s_i) & \rightarrow q_x \\
(q_j, s_j) & \rightarrow q_x \\
\end{aligned}
\]



In ECC, additional intermediate states $q_{im}$ and $q_{jm}$ are introduced, and NDT transitions are used to connect them to $q_i$ and $q_j$, respectively:


\[
\begin{aligned}
(q_{im}, \text{NDT}) & \rightarrow (q_i, s_i) \\
(q_{jm}, \text{NDT}) & \rightarrow (q_j, s_j) \\
\end{aligned}
\]


Finally, $q_i$ and $q_j$ converge to $q_x$:


\[
\begin{aligned}
(q_i, \text{1}) & \rightarrow q_x \\
(q_j, \text{1}) & \rightarrow q_x \\
\end{aligned}
\] 



\begin{algorithm}[t!]
    \caption{FSM to Plant model ECC }
    \label{FSMtoPlant}
\SetKwInOut{Input}{Input}
\SetKwInOut{Output}{Output}

\Input{ FSM = \{ Q: $\{q_0, q_1, \ldots, q_{n-1}\}$ \\
is a set of states, where $q_0$ is an initial state; \\
CS =  $ C \cup S$ is a set of input signals, where \\  C = $\{c_1, c_2, \ldots, c_p \}$ is a set of control \\  signals, \\  S = $\{s_1, s_2 \ldots, s_r \}$ is a set of sensorsignals; \\
$\delta \subseteq Q \times CS \times Q$ is a transition  relation; \}
}
\Output{ ECC = \{ $QC= Q \cup D$ is a set of ECC states, where D is a set of additional ECC states created dynamically;
$Cond= TC \cup \{1\}$ is a set of ECC transition  conditions, where $TC= C \cup \{NDT\}$ is a set  of input events;
S is a set of output events (see above);\\
$ \phi: QC \times Cond \rightarrow QC \cup QC\times S $ \\ is an ECC  transition function
\}
}

/* Processing the FSM transition labelled by control signals */

\ForEach{$(q_i, \sigma, q_j) \in \delta$}{
    \If{$\sigma \in $ C}{
        Create ECC Transition:
        $(q_i,  \sigma ) \rightarrow (q_j)$;\
    }
}
/* Processing the FSM transition labelled by sensor signals which are not converged */

\ForEach{$(q_i, \sigma, q_j) \in \delta$}{
    \If{$\sigma \in  S $ \text{AND}  $q_i \notin QJ$}{
        Create ECC Transition:
        $(q_i,  NDT ) \rightarrow (q_j, \sigma)$;\
    }
}

/* Processing the FSM transition labelled by converging sensor signals */

\ForEach{$(q_w \in QJ) $}{
\ForEach{$(q_i, \sigma, q_w) \in \delta$}{
    \If{$\sigma \in  S $}{
        Create ECC State $d \in D$ \\
        Create ECC Transition:
    $(q_i,  NDT ) \rightarrow (d, \sigma)$ ; \ \\
        $(d,  1 ) \rightarrow (q_w)$ ;\
    }
}

}
    \Return{ECC}\;

\textbf{ *Note* : Used variables and sets} \\
\BlankLine

$QJ= { \{ q_j \in Q , | \{(q_i, s_k, q_j) \in \delta’\}|>1} \} $, where $\delta’ = \delta \cap  Q \times S \times Q$ is a set of ECC states which have converging sensor signals in the FSM


\end{algorithm}

Algorithm \ref{FSMtoPlant} is defined by applying the transformation rules described earlier to facilitate the conversion of  RMSCES into a plant model in the IEC 61499 FB.


\subsection{MODEL-DRIVEN APPROACH AS A METHODOLOGICAL BASIS FOR DEVELOPMENT}

In this study, the model-driven approach (MDA) was used for development, according to which the design process is represented as a chain of model transformations, starting from the initial model and ending with the target one. In model-driven development the model transformation is “the heart and soul” of the approach \cite{sendall2003model}.
The event log as a source model, and the target models are: 1) the monitor model and 2) the plant model. In accordance with this, two different chains of transformations were used. The plant model is important as an independent result that can be used for various purposes in the design process, including certifying a new controller during transitions between different hardware and software platforms. This challenge of certifying is addressed in this study.
The certification process involves two essential activities: one is the creation of a model representing the plant through a series of transformations from event log to Petri net, RG, FSM of the plant, resulting in an IEC 61499 FB-based plant model. Next is the development of a comprehensive model for the new controller using IEC 61499 FBs. As the new controller is developed in IEC 61499 FBs, it is advisable to maintain consistency by expressing the plant model in the same notation. This design not only ensures uniformity but is also supported by the availability of the fb2smv, a tool that streamlines the conversion of IEC 61499 FBs into SMV code. These conversions, while intricate, are significant for systematic controller verification and migration, ensuring uniformity and reliability in the certification process. While many conversions are required primarily for certification and monitoring system implementation, a more direct approach is available when the sole goal is to verify an existing system based on event log data. In this scenario, the Petri net generated in ProM can be directly analyzed in dedicated platforms like TINA, which offers compatible tools for this purpose. There are different corresponding tools available within this system. There exist methods for directly converting Petri nets into SMV code \cite{wimmel1997bdd}. The resulting SMV code can be employed for analysis using queries rooted in LTL and CTL within systems like NuSMV.



\section{case study: A Pneumatic cylinder}
\label{cs}
\subsection{General Description}

Visualization of the cylinder's generated by its IEC 61499 simulation model in NxtStudio is shown in Figure \ref{VC_NXT_HMI}.  The cylinder has three sensors START, MID, and END indicating the position of the piston. The vertical cylinder's motion is controlled by EXT and RETR actuator signals'. EXT  to move downwards and RETR  to move upwards. The Actuator and Sensor signals of the pneumatic cylinder are represented in Figure \ref{VC_NXT_HMI}.





\subsection{Event log Description}


The event log of the different processing scenarios of the vertical cylinder is recorded in CSV format, which captures the activities in chronological order. The event log, depicted in Figure \ref{EL2FSM} (A), comprises of three columns: CaseId, State, and Activity. The CaseId is unique for each processing scenario, while the Activity column represents the events that occurred during the processing of a scenario. Signals (like events) happen instantaneously, and it is a good practice to use the Boolean vector to store them, with "1" indicating it is set and "0" indicating it is reset. The timestamp information is used solely for sorting the activities in the event log.



\begin{figure}[!t]
	\centering
	\includegraphics[width=0.35\textwidth]{MX_Papers/Paper7/images/CylinderHMI.png}
	\caption{Pneumatic cylinder HMI representation NXTSTUDIO.}
	\label{VC_NXT_HMI}
\end{figure}



\begin{figure*}[!t]
	\centering
	\includegraphics[width=1\textwidth]{MX_Papers/Paper7/images/EL2FSM.png}
	\caption{A) Vertical cylinder Event Log representation in CSV B) Stepwise simulation of Petri net in TINA C) FSM representation in Yed editor}
	\label{EL2FSM}
\end{figure*}


The event log captures various processing scenarios of the vertical cylinder, including movements from START to END via  MID,
  and returns to the START position. The event log also includes random movements of the cylinder captured by pressing the HMI buttons. These events are recorded using the OPC UA communication protocol.


\subsection{ FSM generation from event log}

The ProM process mining tool is utilized for constructing a Petri net from the event log. ProM offers several process discovery algorithms, and the \textbf{alpha} algorithm is utilized here for extracting the process from the event log. The event log, in CSV format, is first converted to XES format, and the 'Case' column is selected as 'CaseId', while the event column is a combination of 'Activity' and 'State' columns. This XES format is then provided as input to the alpha algorithm. The resulting Petri net is shown in Figure \ref{EL2FSM} (B).


The TINA tool is used to generate the RG for the Petri net. To achieve cyclic processing behavior, a new transition named 'Repeat' is added to the Petri net, connecting the 'END' state to the 'START' state. Stepwise simulation is performed to test the Petri net, as shown in Figure \ref{EL2FSM} (B). The RG is then used to generate the FSM in text format. The RG obtained from the Petri net can contain spontaneous transitions, rendering it a non-deterministic FSM. To convert this non-deterministic FSM to a deterministic one, the 'Converter of TINA RG to GraphML' software tool \cite{xavier2022interactive} is utilized. The resulting deterministic FSM is presented in Figure \ref{EL2FSM} (C) in GraphML format. The FSM is visualized using the 'Yed' GraphML editor \cite{yworks}, which provides a clear process logic behind the system. The state information on the edges of the FSM is removed since it is unnecessary to differentiate control and sensor signals.



\subsection{IEC 61499 representation of monitor}


\begin{figure*}[!t]
	\centering
	\includegraphics[width=1\textwidth]{MX_Papers/Paper7/images/CLMC.PNG}
	\caption{Closed-loop system of plant and controller model with the monitor.}
	\label{CL_plant_controller_monitor}
\end{figure*}


To implement the monitor in the IEC 61499 FBs, the RMCSES' FSM representation is taken as a starting point. The transformation rules are then applied to this FSM to create the monitor's FB interface. The interface, represented in Figure \ref{CL_plant_controller_monitor} (A), includes input signals for sensors and control signals. An additional event 'R' is included for the 'Repeat' event, which is used for the cyclic operation of the system's process.


The monitor FB generates an 'OK' event with the corresponding StateID when the system process executes in the correct order. In the case of any deviation from the expected process behavior, an 'ERROR' signal is emitted along with the 'EventID' indicating the specific event responsible for the deviation. The ECC for the monitor FB is represented partially in Figure \ref{CL_plant_controller_monitor} (E2). If any event other than 'VCEXT\_TRUE' and 'VCRETR\_FALSE' occurs in 'STATE4', an 'ERROR' event is triggered, along with the respective EventID. The ECC in Figure \ref{CL_plant_controller_monitor} (E2) only shows errors occurring from 'STATE 4', but in the actual scenario, there will be multiple events from different states (STATE1, STATE 2, etc.) leading to their corresponding ‘ERROR’ states (VCEXT\_TRUE, VCRETR\_FALSE, etc.).


\subsection{IEC 61499 representation of Plant Model}



The ECC of the plant model is generated using the deterministic FSM obtained from the Petri net. Transformation rules discussed in Section \ref{subsec:transformationOfFSMtoFB} are applied to the FSM to derive the ECC. The interface of the plant model mirrors the controller depicted in  Figure \ref{CL_plant_controller_monitor} (A). The plant model has an additional signal called NDT, which provides a non-deterministic delay before producing the sensor signals as output. The behavior of the system is represented by the ECC and is embedded inside a FB. The ECC of the plant model is illustrated in Figure  \ref{CL_plant_controller_monitor} (E1).


\section{Traces with only events, no global state}

\begin{figure}[!t]
	\centering
	\includegraphics[width=0.4\textwidth]{MX_Papers/Paper7/images/StatelessPN.PNG}
	\caption{Petri net constructed without state information.}
	\label{StatelessPN}
\end{figure}

\subsection{Global state}


The global state of the system is defined as a vector of boolean values that represents the combination of sensor and actuator signals. This vector is used to construct the global state context of the system. The global state context represents a unique frame of reference for the system. To ensure the accuracy of the global state, clock synchronization and time stamping are used across all distributed controllers in the network.


\iffalse
The global state can be constructed relying on clock synchronization and time stamping.
What to do with big and linear state graphs?
Build state graphs using process mining without global state information.

Try to simulate it against the original traces and see if the entire graph is covered. 
If not, consider pruning the state graph.

Check the auto-generated model of the plant with another controller. 
\fi


\subsection{Importance of State information}

In order to obtain accurate results from the process mining algorithm, it is essential to include state information in the event log. The activity in the event log should reflect the combination of the event signal value and state information. By doing so, the generated Petri net will consist of two loops, which will showcase the different actions of the pneumatic cylinder. This confirms that the system is capable of supporting multiple processing paths or options.

\paragraph{How new transition in Petri net is created when state information is considered?}

A new transition in Petri net is created only when the combination of signal value and state of the system is different. 

\begin{itemize}
  \item If the state of the system is the same and signals are different, then a new transition occurs.
  \item If the signals are the same but  occur 
  in different state conditions, then  a new transition occurs.
\end{itemize}

So, the number of events in the event log won't be the same as the number of transitions in the Petri net. The Petri net constructed without the state information for the same experiment is shown in  Figure \ref{StatelessPN}. When comparing the Petri net created using state information Figure \ref{EL2FSM} (B) and Figure \ref{StatelessPN}, the Petri net without state information Figure  \ref{StatelessPN} is inaccurate and shows misleading transitions which never occur in the process scenarios.

\section{Results and Analysis}

\subsection{Formal verification of the system using generated plant model}

The IEC 61499 FB plant model can be used in conjunction with the controller for formal verification purposes. The fb2smv tool is used to convert the IEC 61499 FBs into the SMV formal model of the closed-loop system, which is then verified with a symbolic model checker tool called NuSMV. Various CTL or LTL specifications can be used to check the formal model against the desired system behavior. The NuSMV tool allows one to interactively explore the states of the system and observe its behavior. The simulation mode in NuSMV is useful for validating the system works according to the correct process sequence. The closed-loop model of the system, illustrated in Figure \ref{CL_plant_controller_monitor}  (A) but without the monitor FB, was transformed into a formal model and simulated using NuSMV. The simulation confirmed that the system followed the correct path. To detect possible failure situations that could arise in critical scenarios, the system can be verified using CTL specifications. The SMV specification of the system guarantees that the specified property will never occur in the system at any time.

\begin{lstlisting}[breaklines,basicstyle=\small]
check_ltlspec -p "G ( ClosedLoopModel_inst.PLANT_VCPOSITION_START_TRUE  ->  F (ClosedLoopModel_inst.CONTROLLER_VCRETR_FALSE ))"
\end{lstlisting}

The above specification is checked using NuSMV proving that when  the vertical cylinder reaches the start position the controller always produces a RETRACT\_FALSE event signal in the future. Like this, it is possible to  verify the system properties by connecting it with the controller in closed-loop  with the help of CTL or LTL specifications.


\subsection{Closed-loop system integration with Monitor}

The closed-loop system of the plant model and new controller is integrated with a monitor, as depicted in Figure \ref{CL_plant_controller_monitor} (A). The integration of the closed-loop system is achieved by connecting the output sensor and output actuator signals to the input of the previously generated monitor \ref{CL_plant_controller_monitor}  (A). The monitor checks the conformance of the system by analyzing the signal values. If the signals occur according to the actual process scenario, then it produces an 'OK' event with its corresponding 'StateID'. In case of non-conformance, the monitor produces an 'ERROR' signal along with the 'StateID' and 'EventID'. The 'StateID' specifies the state in which the error occurred, while the 'EventID' represents the event that caused the error.

In this experiment, the model produces the right events and it follows the process scenario; therefore, there are no ERROR signals produced after one cycle. Figure \ref{CL_plant_controller_monitor} (A) shows that it reached 'StateID' = 13 and ran one and a half cycles (R=1) is completed without any occurrence of errors ('ERROR'=0). In order to check whether the monitor captures the error,  'VCEXT\_TRUE' event was added instead of 'VCRETR\_TRUE' at 'STATE51' of the controller in Figure \ref{CL_plant_controller_monitor} (E3). Then the monitor produced an 'ERROR' signal with 'EventID' = 1 and 'StateID' = 11. The monitor at 'STATE11' (Figure \ref{CL_plant_controller_monitor} (E2) ) checks if any event other than 'VCRETR\_TRUE' occurs then it directs towards the respective event error state, i.e., in this case, it goes to 'E1\_VCEXT\_TRUE' state and executes 'AE1' algorithm and produces ERROR signal as output along with the 'EventID'= 1 and 'StateID' = 11.

\section{Conclusion and future work}
\label{cf}


The study proposes a novel method for generating a monitor and plant model from behavioral traces. The monitor is used to detect any deviation in the process sequence, while the automatic generation of a plant model is a challenging task without adequate domain knowledge and system behavior. The study also demonstrates the automatic generation of a plant model solely based on recorded event signals. The resulting plant model is useful for the formal verification of closed-loop systems in compliance with IEC 61499.

Further testing with additional use cases is required to validate the effectiveness of this approach. Future work could involve connecting the monitor to the actual plant in order to detect deviations from the expected process scenario. The global state of the system is the essential information needed for this approach, but recording all sensor and actuator signal values whenever an event occurs makes this process difficult. It is possible to poll all sensors and actuators' signal values with the help of synchronisation and timestamp, but for a complex system, there will be a considerable amount of time delays. The retrieval of state information for a complex system needs to be analyzed. 

In future research, the significance lies in exploring the performance of the approach, particularly in handling different levels of system complexity. This entails conducting comprehensive analyses of time complexity and engaging in quantitative experiments to gain deeper insights into the effectiveness and scalability of the proposed methodology. Such efforts are essential for establishing a robust understanding of the approach's capabilities across various scenarios and ensuring its applicability in real-world contexts.

\section{Low Level Event log}
\label{llel}

This section describes the formal definition of the low-level event log. 
Figure \ref{workflowUsecase} (A) shows the closed-loop system consisting of the plant and the controller. The set of (abstract) signal lines, L in a closed-loop system is defined as:

\[ L = S\cup C; S\cap C = \phi \]
where $ S=\{s_1,s_2,\dots,s_m\} $ is a set of lines from Plant’s sensor to Controller and $ C=\{c_1,c_2,\dots,c_n\} $ is a set of lines from Controller to Plant’s actuator. 
Figure \ref{workflowUsecase}  (A)  shows a closed-loop system with indicated signal lines. The set L includes all (n + m) lines: 

\[L=\{l_1,l_2,\dots,l_{n+m}\}\]
Let there be a set of attributes or parameters that can be associated with signals transmitted over the lines.

\[ A=\{a_1,a_2,\dots,a_k\} \]
Each signal line can have its own set of attributes, defined by a function:

\[p:L\rightarrow 2^A\]
Thus, signals with the following set of attributes are transmitted over the line $ l_i \in L$. This set of attributes is assumed to be ordered. A function is defined to assign attribute values to each of the lines  $ l_i \in L$.

\[ z_i: li\rightarrow Dom(a_i^1) * Dom(a_i^2) * \dots * Dom(a_i^t) \]
where $ Dom(a_i^j) $ is a domain for the attribute  $ a_i^j $. 
In the case of a large dimension of these domains, the total number of possible signal values can be very large. For convenience, a generalized value function is introduced: 

\[ z=z_1 \cup z_2 \cup \dots \cup z_{n+m}  \]
A signal on the line $ l_i \in L$ is a set of attribute values on this line, i.e. $z(l_i)$.
$z(l_i)[t]$ denotes a set of values of attributes of the line $ l_i \in L$ at the moment of time $ t \in  N^+ $, where $N^+ $ is the set of positive integer numbers. The use of  discrete time does not change the general situation. 

An event on the line $ l_i \in L$ is the moment when the signal on this line changes. The signal changes if the value of at least one of its attributes changes. The condition that determines the occurrence of an event on the line  $ l_i \in L$ is the following: 


\[ \exists t (z(l_i)[t] \ne z(l_i)[t+1]) \]
Theoretically, at the same time, different events can occur on two or more lines. It is possible, but the probability of this is very small. All events in the system can be enumerated: 


\[ E=\{e_1,e_2,\dots, e_h\} \]
Each event $ e_i \in E  $ is defined by the following tuple:

\[ e_i = (t_i, ne_i) \]
Where $t_i$ is the time at which the event occurred and $ne_i$ is the name of the event. The name of the event $e_i \in E $ arising   at the line $l_j \in L$ is defined as follows:

\begin{equation} \label{eq1}
   ne_i = (l_j,z(l_j)[t_i]) 
\end{equation}   
From this formula, one can see that the event name consists of the line ID and the values of all line attributes. We define a function that maps a signal line to a component (device) with which it is connected: 

\[ r : L \rightarrow D \] 
where $ D=\{d_1,d_2,\dots, d_v\} $ is a set of components (devices) in the system. 

An event log entry for event $ e_i \in E $ at line  $ l_j \in L$ is defined by the following tuple: 

\[  w_i= (CaseID, t_i, ne_i, r(l_j)) \]

Where CaseID is the constant identifier for each cyclic operation of the processes, $t_i$ (that is, Timestamp) is the time when the event occurred, $ne_i$ (that is, Activity) is the name of the event according to the formula \ref{eq1} above, $r(l_j)$ is the component with which the corresponding line is associated.  It should be noted that the component $r(l_j)$ may not be used. The component $t_i$ may not be used if the event log is chronologically ordered or if timing is not taken into account.

\clearpage
\putbib

\end{bibunit}


%-------------------------------------------------------------------
\def\paperheader{Paper H}
\def\papertitle{Developing a Test Suite for Evaluating IEC 61499 Application Portability}
\def\paperauthorstring{Midhun Xavier, Tatiana Laikh, Sandeep Patil, Valeriy Vyatkin}
\def\referencestring{Proceedings of the IEEE International Conference on Industrial Informatics (INDIN), 2021.}
\def\copyrightstring{2021, IEEE, Reprinted with permission.}

% The definitions above could just as well be put directly into the function
% call below, but were explicitly defined to more clearly illustrate the
% use of the function \makepaper.

\makepaperaccepted
  {\paperheader}
  {\papertitle}
  {\paperauthorstring}
  {\referencestring}
  {\copyrightstring}

% The actual contents is imported by un-commenting the \input line below.
% Make sure the file exist.
\begin{bibunit}
\thispagestyle{plain}

% Add missing command definitions from original paper
\newcommand{\RNum}[1]{\uppercase\expandafter{\romannumeral #1\relax}}


\section*{Abstract}

    This paper presents the creation of a series of function blocks with the specific aim of testing the portability of IEC 61499 applications across diverse development and runtime environments. These function blocks have been developed to cover a wide range of test scenarios, including basic data types, functions, boundary conditions, and adapter features. The function blocks can be conveniently exported or imported through the use of XML files, thus facilitating seamless testing. By testing the runtime environment of different IEC 61499 systems, these function blocks help to identify and highlight any possible issues that may arise related to portability.
    
    
\section{Testing IEC 61499 Applications: Ensuring Portability Across Different Environments}

The IEC 61499 standard \cite{iec61499part12012} has been developed to address the increasing demands for decentralized control and the exponential growth of control complexity in industrial automation systems, with the aim of establishing an open, component-oriented, and platform-independent development framework to improve the re-usability, reconfigurability, interoperability, portability \cite{gerber2010does}, and distribution of control software for complex distributed systems. The IEC 61499 technology can be effectively used in the implementation of Intelligent Mechatronic Components and engineering processes, using multiple commercial tools and hardware platforms \cite{patil2013composition}. 

The paper \cite{pang2014portability} highlights the importance of addressing portability issues in existing engineering tools for the IEC 61499 standard in order to achieve better interoperability and efficiency in distributed control systems. Testing IEC 61499 applications for adequate functionality is crucial in guaranteeing their compatibility across various development and runtime environment \cite{hopsu2019portability}. To this end, a function blocks library has been designed for testing the basic data types, functions, and boundary conditions, among other key scenarios. A guideline document has been provided to help developers understand the desired results of testing and to modify the test with new values if necessary. 

By enabling developers to test their IEC 61499 applications for portability using the developed standard testing function blocks, the likelihood of errors can be minimized and the efficiency and effectiveness of distributed industrial control systems can be improved.



\section{Test Function Block Design and Development }
\label{sec:mainSec}



\subsection{Data Type testing FBs}
\begin{figure}[!b]
    \centering
    \includegraphics[width=\columnwidth]{MX_Papers/Paper8/Figures/BSDT.PNG}
    \caption{a) BitStringDataType FB interface b) ECC c) INIT Algorithm d) REQ Algorithm.} 
    \label{fig:BSDT}
\end{figure}



\subsubsection{BitStringDataType}

The BitStringDataType function block (FB) is used to test the support of data types and their association for Bit data types, including BOOL, BYTE, WORD, DWORD, and LWORD. The testing procedure for the BitStringDataType FB involves two events, INIT and REQ.

When the INIT event is triggered, the output values of the BitStringDataType FB are set to predefined values. If the output values are not the same as  predefined values then the test fails. This simple test ensures that FB can correctly handle each of the supported data types.

Alternatively, when input data is provided, the REQ event is triggered. The output values of the BitStringDataType FB should correspond to the input values. For example, Out1 should be the same as In1. The CNF event is also triggered by this test. This test ensures that the FB can correctly process the input data and produce the correct output values. The interface, ECC, INIT algorithm, and REQ algorithm for the BitStringDataType function block are depicted in figure \ref{fig:BSDT} a, b, c, and d, respectively.

The BitStringDataType FB is an important testing tool for ensuring that data types and their associations are properly supported in Bit data types. By following the testing procedure outlined above, developers can ensure that their software systems can correctly handle different data types and produce the expected output values.
\hfill \break
\subsubsection{IntegerDataType}

\begin{figure}[!b]
    \centering
    \includegraphics[width=\columnwidth]{MX_Papers/Paper8/Figures/IDT.PNG}
    \caption{a) IntegerDataType FB interface b) ECC c) INIT Algorithm d) REQ Algorithm.} 
    \label{fig:IDT}
\end{figure}

The IntegerDataType function block is used to test the support and association of different integer data types, namely SINT, INT, DINT, LINT, USINT, UINT, UDINT, and ULINT, in an IEC 61499 application. Testing of the IntegerDataType function block can be done in manual mode by triggering the INIT event. When this event is triggered, the output values of Out1 to Out8 should be as follows: 10, 20, 30, 16\#40, 16\#50, 60, 70, and 80. This indicates that the function block correctly supports and associates the specified integer data types.

Another way to test the function block is to add values to the input data and then trigger the REQ event. In this case, the output values of Out1 to Out8 should have the same corresponding values as the input data. For example, the value of Out1 should be the same as the value of In1. This will also trigger the CNF event. These testing procedures help ensure that the IntegerDataType function block works correctly and supports the specified integer data types as expected. IntegerDataType FB interface, ECC, INIT algorithm and REQ algorithm  is shown in figure \ref{fig:IDT} a, b, c and d respectively.
\hfill \break
\subsubsection{RealDataType}

RealDataType also tests similar like the above test FBs.
It is a function block designed for testing the data type support and association with real data types, REAL and LREAL. In the manual testing mode, triggering the INIT event would produce the following output and cause the INITO event to be fired: Out1 with a value of 10.100 and Out2 with a value of 20.2000. Alternatively, as in the case of the BitStringDataType and IntegerDataType test FBs, input data values can be added and the REQ event can be triggered. This would cause the output values to correspond to the input values, such as Out1 being the same as In1. This would also trigger the CNF event.

\subsubsection{StringDataType}

The StringDataType function block (FB) is designed to test the support and association of the STRING data type. In manual testing mode, initiating the INIT event will produce the following output and trigger the INITO event:

Out1 := 'STRING'

Alternatively, adding values to the input data and triggering the REQ event will cause the output values to correspond to the input values. For example, Out1 should match the value of In1. This will also trigger the CNF event.
\break
\subsubsection{TimeDataType, DateDataType, TimeOfDayDataType and DateAndTimeOfDayDataType}

The TimeDataType, DateDataType, TimeOfDayDataType, and DateAndTimeOfDayDataType function blocks (FBs) test the data type support and associations for their respective data types (TIME, LTIME, DATE, LDATE, TIME\_OF\_DAY, LTIME\_OF\_DAY, DATE\_AND\_TIME, and LDATE\_AND\_TIME).

In a manual testing mode, triggering the INIT event will produce output values, and the INITO event will be fired. For TimeDataType, the output values are Out1 := T\#90s15ms and Out2 := LT\#90s15ms542us15ns. For DateDataType, the output values are Out1 := D\#1970-01-01 and Out2 := LD\#2177-11-30. For TimeOfDayDataType, the output values are Out1 := TOD\#00:00:00 and Out2 := LTOD\#00:00:01. For DateAndTimeOfDayDataType, the output values are Out1 := DT\#1970-01-01-00:00:00 and Out2 := LDT\#1971-01-01-00:00:00.

Alternatively, input data can be added and the REQ event can be triggered. The output values should correspond to the input values, and the CNF event will be fired.


\subsection{BoundCheckTest FB}

\begin{figure*}[!t]
	\centering
	\includegraphics[width=1\textwidth]{MX_Papers/Paper8/Figures/BCT4Diac.PNG}
	\caption{BoundCheckTest FB a) Interface b) ECC}
	\label{fig:BCT4Diac}
\end{figure*}


BoundCheckTest FB \ref{fig:BCT4Diac} is a test function block that tests the boundary support for each data type such as BYTE, WORD, DWORD, LWORD, USINT, UINT, UDINT, ULINT, SINT, INT, DINT, LINT, LREAL, TIME, DATE, TOD, STRING. This function block can be used to check if the system under test can handle the maximum and minimum values of the specified data types correctly.

To test the BoundCheckTest function block, input data needs to be added with necessary values and then the bound check event for the specified data type should be triggered. The output values of the event should be the input value plus one. For example, if the input value is In1, the output value should be Out1 = In1 + 1. This will also trigger the CNF event, indicating that the test has been completed successfully.

The purpose of the BoundCheckTest is to ensure that the system under test can handle the maximum and minimum values of the specified data types without any errors or unexpected behavior. This is important because it ensures the reliability and robustness of the system, especially in situations where the system is required to handle extreme values.


\begin{figure}[!b]
    \centering
    \includegraphics[width=\columnwidth]{MX_Papers/Paper8/Figures/CFBTest.PNG}
    \caption{a) Composite Function Block a) FB interface b) Basic FB connection Diagram c) ECC d) INIT Algorithm and REQ Algorithm.} 
    \label{fig:CFBTest}
\end{figure}

\begin{figure}[!b]
    \centering
    \includegraphics[width=\columnwidth]{MX_Papers/Paper8/Figures/AdapterCFB.PNG}
    \caption{ shows the Function Block Interface of a) Adapter CFB b) AdapterCFB consists of two sub-CFBs i.e. AdapterCFB\_Test1 and AdapterCFB\_Test2 c) AdapterCFB\_Test1 consists of AdapterTest FB and an AdapterBF1 (basic FB) d)  AdapterCFB\_Test2 consists of AdapterTest FB and an AdapterBF1 (basic FB), e \& f) Represents the  AdapterTest FB  and its SOCKET and PLUG } 
    \label{fig:AdapterCFB}
\end{figure}

\subsection{StanadardFunctionTest FB}

The StandardFunctionTest FB is designed to evaluate the support of IEC 61499 standard functions and their corresponding association with input values. This FB is capable of testing over 75 standard functions by providing specific events for testing each function, which trigger the corresponding testing algorithm for that particular function.

To test the StandardFunctionTest function block, suitable input values are added to the input data, and the function event is triggered. The output values should contain the result of the corresponding function, which is then verified by the CNF event.

For instance, if In1 is the input value, the output value Out1 should be equal to the result of the function evaluated using In1. Therefore, the CNF event confirms that the expected output value is produced. Let's consider the following example to test the MUX function feature,


To test the Multiplexer (MUX) function, the MUX\_Func event should be triggered, and the corresponding algorithm will execute.

\begin{lstlisting} 
ALGORITHM MUX_Func IN ST:
(* Add your comment (as per IEC 61131-3) here
*)
MUX_Out:=MUX(int1,time1,time2,time3);
END_ALGORITHM
\end{lstlisting}




The MUX function is used to select one of several input signals based on an index value. The input signals can be of different types, such as time, integers, etc. The MUX function returns the selected input signal based on the index value. The MUX function can be tested by triggering the MUX\_Func event and verifying that the output value matches the expected output value.

If we provide input values int1=2, time1=TIME\#1m30s, time2=T\#15s, and time3=T\#2m30s, then the expected output is MUX\_Out:=MUX(int1,time1,time2,time3) = 2m30s.


\subsection{CompositeBlockTest}

The CompositeBlock FB \ref{fig:CFBTest} is used to test the functionality of composite function block support. The block has two events, INIT and REQ, which can be triggered in manual testing mode or by adding integer values to the input data and triggering the REQ event.

When the INIT event is triggered, the output values should be the same as the input values, and the INITO event is fired. On the other hand, when the REQ event is triggered, the output should be calculated as QO1 = QI1 + QI2 and QO2 = QI1 - QI2. After the calculation is complete, the CNF event is fired.

In summary, the CompositeBlock function block tests the functionality of composite function blocks by calculating the output values based on the input values and triggering the CNF event when the calculation is complete.

\subsection{AdapterTest CFB}


AdapterCFB \ref{fig:AdapterCFB} is a function block that tests the functionality of adapter support. The AdapterCFB is composed of two basic function blocks, AdapterBFB1 and AdapterBFB2. AdapterBFB1 has three output parameters, QO0, QO1, and QO2, which are initialized to FALSE, 1, and "AdapterBFB1", respectively. AdapterBFB2 also has three output parameters, QO0, QO1, and QO2, which are initialized to TRUE, 2, and "AdapterBFB2", respectively.

The input values from AdapterBFB1 are transferred to AdapterBFB2 and vice versa. This is made possible through the use of an adapter, which connects the two function blocks. When the AdapterCFB is triggered, the input values are set by the tester, and the output values are calculated by the adapter based on the input values.

In other words, AdapterCFB provides a way to connect two function blocks that have different inputs and outputs by mapping the inputs and outputs of one block to the inputs and outputs of the other block, using an adapter.

\section{Xml-import/export compatibility test
between 4DIAC  V2.0.1 and EAE 
V21.2}

As a part of the 1-Swarm project, a hackathon was conducted to analyze the compatibility of IEC 61499 between 4DIAC and Schinder Electric EcoStruxure (EAE). The objectives of the hackathon were to identify portability issues between 4DIAC and EAE-IDE, as well as between Forte and EcoRT-runtime, create a roadmap for addressing the identified issues, and discuss potential solutions for creating a single portable test application that can automatically test new versions of IDE and runtime and certify some key attributes. Some of the issues identified during the hackathon are given below: 

\begin{enumerate}

\item Issue: The EAE import dialog does not allow for the selection of a $<BitStringDataType.Basic.ZIP>$ named file with a subdirectory IEC61499 and a metafile named $<BitStringDataType.Basic.export>$. EAE requires a specific extended import format which is not described in the IEC61499 standard.

Recommendation: Develop a company conformance profile or allow for the import of pure .fbt files to address the issue.

\item  Issue: After importing a 4diac .xml file in the 4DIAC .BASIC.ZIP format, the algorithms are not imported. EAE does not accept the CDATA format for the textual contents of algorithms.

Recommendation: Modify EAE to accept CDATA format as it offers user format overall lines, instead of compressing the algorithm in one .xml line.

\item Issue: The compiler/parser does not generate an error when a function with two parameters is called with only one argument.

Recommendation: Modify the compiler/parser to generate an error when an argument is missing.


\item Issue: EAE does not support data types CHAR, WCHAR, and WSTRING. Only the STRING data type is supported.

Recommendation: Modify EAE to support character string literals that directly represent a character or character string value of data type CHAR, WCHAR, STRING, or WSTRING as documented in the IEC61131-3 standard.
\end{enumerate}

An automated testing approach for assessing the compatibility of software tools and runtime platforms with the IEC 61499 standard could be a promising direction for future research and development. This approach could leverage service sequence testing as an additional testing methodology to ensure comprehensive compliance with the IEC 61499 specification across all integrated development environments (IDEs) and runtime environments. 

\section*{Acknowledgment}
This work was sponsored, in part, by the H2020 project 1-SWARM co-funded by the European Commission (grant agreement: 871743). Thank you Artur Fritz and Alois Zoitl for their valuable contributions and participation in the hackathon for the development of the IEC 61499 portability test suite.

%%% Put references here
\clearpage
\putbib
\end{bibunit} 

%-------------------------------------------------------------------
\def\paperheader{Paper I}
\def\papertitle{Generating Portable Test Cases for IEC~61499 FBs from Interface Behaviour Specifications}
\def\paperauthorstring{Bianca Wiesmayr, Midhun Xavier, Sandeep Patil, Alois Zoitl, Valeriy Vyatkin}
\def\referencestring{Proceedings of the IEEE International Conference on Emerging Technologies and Factory Automation (ETFA), 2023.}
\def\copyrightstring{2023, IEEE, Reprinted with permission.}

% The definitions above could just as well be put directly into the function
% call below, but were explicitly defined to more clearly illustrate the
% use of the function \makepaper.

\makepaperaccepted
  {\paperheader}
  {\papertitle}
  {\paperauthorstring}
  {\referencestring}
  {\copyrightstring}

% The actual contents is imported by un-commenting the \input line below.
% Make sure the file exist.
\input{papers/paper9.tex}

%-------------------------------------------------------------------
\def\paperheader{Paper J}
\def\papertitle{Develop Once, Test Everywhere: Cross-Platform Development of Distributed Control Software}
\def\paperauthorstring{Bianca Wiesmayr, Melanie Winter, Midhun Xavier, Sandeep Patil, Valeriy Vyatkin, and Alois Zoitl}
\def\referencestring{IEEE Open Journal of the Industrial Electronics Society, 2024.}
\def\copyrightstring{2024, IEEE, Reprinted with permission.}

% The definitions above could just as well be put directly into the function
% call below, but were explicitly defined to more clearly illustrate the
% use of the function \makepaper.

\makepaperaccepted
  {\paperheader}
  {\papertitle}
  {\paperauthorstring}
  {\referencestring}
  {\copyrightstring}

% The actual contents is imported by un-commenting the \input line below.
% Make sure the file exist.
\begin{bibunit}
\thispagestyle{plain}

% Add missing command definitions from original paper
\newcommand{\RNum}[1]{\uppercase\expandafter{\romannumeral #1\relax}}

\section{Introduction}
\label{sec:introduction}
Realising flexible industrial automation systems requires an approach
 for autonomous and distributed designs \cite{Lyu.2021}. Programmable Logic Controllers (PLCs) are the established platform for real-time control software that accesses sensors and actuators \cite{Sehr.2021}.  Standards play an important role in distributed automation, for instance, IEC~61131-3 and IEC~61499, which define programming paradigms for control software development \cite{Lyu.2021}. %Despite the increasing complexity of automation systems, the programming models of IEC 61131-3 are still prevalent \cite{Sehr.2021}. 
 Multiple interacting PLCs form a distributed control system. Providing the respective engineering methodologies and models is a goal of IEC~61499 \cite{61499}. Heterogeneous systems can even be composed of PLCs from various vendors and programmed with different tools \cite{lyu2020towards} \cite{mazzolini2017structured}. Furthermore, a single development tool can distribute control code across multiple runtime environments (RTEs)~\cite{eclipse4diac}, motivating the need to execute component tests in each of these RTEs. Developers of IEC~61499 library modules  \cite{oberlehner.2022} will also need to provide their modules to users of various development environments. 
 Despite the focus on portability \cite{61499} and the standardized XML format for data exchange \cite{61499.2}, IEC~61499-based software components must often be modified during the porting process \cite{Hopsu.2019, Testing_Midhun}. Due to varying execution behaviour, the ported software may behave differently on each platform \cite{Testing_Midhun, Wiesmayr.2023}, possibly leading to malfunctions of the distributed control system. Therefore, it is crucial to thoroughly test an IEC 61499 application on each relevant target platform before using the software in a real-world system. A platform-independent test specification has the potential to greatly reduce the involved effort.

\begin{figure}[!htbp]
    \centering
    \includegraphics[width=0.95\linewidth]{MX_Papers/Paper10/Figures/portingProcessPNG.PNG}
    \caption{Process of porting test code for IEC~61499 Function Blocks between execution environments.}
    \label{fig:porting_process}
\end{figure}

Evaluating a system's correctness typically involves providing test data and observing the system's reaction. Such tests are specified manually, obtained from models, or generated using other techniques \cite{Sinha.2019}. The engineering processes of complex control systems additionally require that PLC platforms ensure reliability through simulation and verification, rather than relying on iterative testing of a cyber-physical system \cite{Sehr.2021}. Simulation methods evaluate only a system model and, thus, can provide results faster \cite{Sinha.2019}. In the context of IEC~61499, executing test cases within a run-time environment can be considered a simulation. A certain degree of test automation is required to efficiently evaluate this software. Existing test processes for IEC~61499-based software rely on a framework in the target platform (e.g., \cite{hametner2014}) and are therefore not applicable for cross-platform development of IEC~61499-based software. When (re-)using parts of IEC~61499 software in multiple platforms, extensions to the IEC~61499 standard may not be available in all platforms (equally). Hence, we are investigating an approach for test automation that does not require any extensions of IEC~61499.

This paper aims to address the challenge of evaluating the effect of porting software components to other platforms or configuring additional RTEs. The test framework allows us to execute tests in any IEC~61499-compliant RTE. Based on our initial concept presented in \cite{biancaMidhunETFAwip}, we generate an IEC~61499-compliant test application that automatically provides test events and data, compares the results of the software component under test with the expected observable behaviour and summarises the results so that they are accessible by the user. The general process is visualised in Figure~\ref{fig:porting_process}. 
Like Hametner et al.~\cite{hametner2014}, we use service sequence models as test specifications. Related work in the context of testing and porting IEC~61499 components is outlined in Section~\ref{sec::sota}. Section~\ref{sec::running_example} introduces a running example. Based on this example, Section~\ref{sec::methodology} outlines the envisioned methodology to test FBs on any IEC~61499-compliant platform. 
One of these IDEs, the 4diac IDE from the Eclipse 4diac open source project, was extended to generate the test application. 
The generation rules for a test application and their implementation are described in Section~\ref{sec::implementation}. Realised as a composite FB, the test application is portable across various IEC 61499 platforms and enables validation of the correct functionality before deployment in real-world machinery. We evaluated our approach using a demonstrator (Section~\ref{sec::casestudy}). Section~\ref{sec::results} lists the identified portability issues and programming errors that we could detect using our test suite. In addition, we discuss limitations of our approach before concluding our paper in Section~\ref{sec::conclusions}. 

\section{Related Work}
\label{sec::sota}

Portability, interoperability, configurability, and distribution across devices are key goals of IEC~61499 \cite{jhunjhunwala2024interoperability,hopsu2019portability, batchkova2013dynamic}. 
This has led to its application in use cases of flexible manufacturing systems, such as multi-agent systems \cite{lyu2023multi, xavier2024enhancing}. However, portability between vendors has not yet been achieved due to vendor-specific execution behaviour \cite{misperceptions}. In this section, we therefore review recent studies on porting software between engineering platforms. Furthermore, techniques for improving the reliability of control software are discussed with a focus on testing approaches. The approach proposed in this paper aims to test software components on different platforms, thus, promoting reusability of software as well as supporting the process of porting software components. 

\subsection{Portability of Control Software}
Both standards that define programming languages for control software, i.e., IEC~61131-3 \cite{61131.3} and IEC~61499-1 \cite{61499}, define XML formats for exchanging software between PLC environments \cite{61131.10,plcopenpaper, 61499.2, Testing_Midhun}. The standardised format contributes to code exchange between tools from different vendors~\cite{plcopenpaper}. As existing tools only support a varying subset of features, portability is limited in practice \cite{Testing_Midhun}. Certain vendors also use custom XML tags to store additional information in their projects which are not covered by the standard itself, such as namespaces in IEC~61499. The resulting software may still be portable if the parser in the target IDE discards unknown XML tags that carry additional information \cite{Hopsu.2019}.  The language specification focusses on control software without detailing applied communication standards or visualisation options. Language extensions, such as modelling elements for communication \cite{Bruns.2023}, are not widely supported and therefore not yet portable. One of the IDEs allows specifying the human-machine interaction (HMI) as part of the control code with dedicated blocks, so-called CAT elements \cite{Hopsu.2019}. Converter programs can help translate programs between syntactic variants of IEC~61499 \cite{Hopsu.2019}, but vendor-specific extensions can typically not be transferred. 
Even syntactic equivalence does not guarantee portability. Nowadays, cross-platform development does not require porting because a single IDE allows deploying software to multiple run-time environments \cite{aimirimi}. 
The IEC~61499 language semantics is not specified formally and is subject to interpretation. Different execution semantics have emerged, which can affect the behaviour of the cyber-physical system \cite{cengic_executionsemantics}. Hence, migrating IEC~61499-based software to different run-time environments can introduce errors that may cause damage to operators or equipment. While the syntactic portability has been addressed \cite{Hopsu.2019,Testing_Midhun}, different execution behaviours cannot yet be detected (semi-)automatically. It is therefore crucial to thoroughly test IEC~61499 applications in the target platform before deployment. Existing testing mechanisms enable systematically evaluating the implemented execution semantics of IEC~61499 runtime environments \cite{Wiesmayr.2023,pfefferkorn,Testing_Midhun}, but a solution for porting test cases including their execution framework to different platforms is not yet available. We have presented our initial concept in \cite{biancaMidhunETFAwip}. 

\subsection{Validating Control Software}
Cengic and Akesson~\cite{cengic_executionsemantics} followed the approach of creating a formal semantics specification for each runtime environment. Subsequently, formal verification methods can be applied to identify errors in the code \cite{xavier2024framework}, which can enhance a system's reliability by checking various properties. Formal verification does not require access to any RTE, but a model of the execution behaviour is required. Sinha et al.~\cite{Sinha.2019} provide an overview of formal methods for IEC~61499. Formal verification may uncover errors that do not occur during simulations, thus, identifying certain undesirable situations \cite{lilli2023formal}. 
Control software has to be continuously adapted to changing requirements. Research has therefore provided guidance on the correct evolution of control software developed in IEC~61499 \cite{faqrizal_guidedevolution}. 
Verification and testing can complement each other \cite{Hussain.2006}. During development, tests provide early feedback, even if the model is still incomplete. Furthermore, errors that are introduced during the deployment may lead to issues encountered during runtime, but might not be revealed by formal methods \cite{ovsiannikova2023formal} \cite{xavier2023formal}. Hence, testing deployed software directly in the run-time environment remains valuable.

Developers can apply various testing strategies, which we differentiate based on the involved software activities (e.g., unit tests or integration tests), the maturity of the software, and the degree of automation~\cite{softwareTesting}. 
Unit testing is a fundamental approach to software testing, which evaluates the implementation of a piece of software \cite{softwareTesting} to ensure software reliability.
In IEC~61499, the relevant units are individual Function Blocks (FBs) \cite{hametner2014}. Executing a test requires providing event and data signals. For control engineers who develop FBs, it can be challenging to manually create a test FB and the required test application, which derives and collects the test results. Model-based testing can reduce manual effort and also supports a ``test first and fail'' methodology, known as Test-Driven Development (TDD) \cite{hametner2014}, which is used in agile software engineering.  
After developing the control program for the entire system, functional tests can be conducted. This involves evaluating the control system by providing input data and verifying the output against expected results.

Tools should support engineers in specifying test cases to reduce the required software engineering knowledge and increase efficiency \cite{hametner2014}. Model-based testing involves automating at least part of the testing activities. For IEC~61499 FBs, service sequences are suitable for specifying tests \cite{hametner2014}. 
A test runner can execute these tests in an RTE and automatically evaluate the results \cite{hametner2014}. Additionally, executing models directly can allow feedback without involving any RTE and is also feasible for service sequences \cite{wiesmayr2021}. The former approach requires specific tool support for a certain RTE, the latter cannot provide feedback regarding issues introduced in the deployment to an RTE. Our approach builds upon these works. 
As an alternative to service sequences, UML models have been used as test specifications \cite{Hussain.2006}. From a state-based model, test cases can be derived using coverage-driven algorithms \cite{Hussain.2006}. Using an evolutionary algorithm, test cases with a high coverage were generated directly from the FB model in \cite{Buzhinsky.2015}. Test case generation can augment our approach, which focusses on executing tests of any source on multiple platforms. Additional tool support would be required to use other test specifications than service sequences.

Two major problems are still associated with developing distributed control software that spans multiple platforms: 
\begin{itemize}
    \item The \emph{lack of automated tool support for RTE comparison} makes comparing the behaviour and performance of FBs across different RTEs a challenging task. Currently, manual comparison is time-consuming and error-prone. Dedicated tools should analyse and evaluate the behaviour of FBs in different RTEs to ensure an accurate comparison.
    \item \emph{Software development for different RTEs} is challenging because the compatibility and portability of an FB across different RTEs cannot be assumed.
\end{itemize}
For example, if an FB is initially developed and tested on one RTE, such as SE EcoRT, there might be a need to reuse that FB in another project in a different RTE, such as 4diac FORTE. Differences in RTE behaviour, programming languages, and underlying architectures can cause compatibility issues and hinder the seamless transfer of FBs between different RTEs. 

\subsection{Cross-Platform Tests of Control Software}
Two main strategies are relevant for IEC~61499-based software. Firstly, \textbf{manually created test FBs} \cite{Testing_Midhun} can reveal the behaviour implemented in an RTE. Each test FB encompasses multiple test scenarios and embeds control logic. To indicate whether a test was successful, the expected result is compared with the result obtained from executing the control logic. The tests are implemented as a Basic FB with event and data pins. Each input event represents a test scenario linked to specific data inputs, while output events indicate the expected result and corresponding data outputs. When a test scenario is triggered, the state diagram (i.e., Execution Control Chart, ECC) executes an algorithm that assigns input values, generates outputs based on those values, and triggers the output event. The main purpose of these FBs was to identify differences between the execution behaviour in RTEs, not to test FB libraries. Similarly, small networks of FBs can further expand these test suites \cite{pfefferkorn,Wiesmayr.2023}. 
As a second approach, \textbf{generating test code from a high-level test specification} has the potential to enable testing FB implementations \cite{biancaMidhunETFAwip}.

\section{Running Example: Simple Calculation FB}
\label{sec::running_example}
As a running example, we use an FB that performs a simple calculation (Figure~\ref{fig:running_example}) based on the inputs according to the formula \texttt{DO1:=DI1+2*DI2}. It contains the following language elements:

\begin{figure}[!htbp]
    \centering
    \includegraphics[width=0.75\columnwidth]{MX_Papers/Paper10/Figures/RunningExample.png}
    \caption{Running Example: FB performing simple calculation. FB interface defining the component, implementation as state diagram, and two usage scenarios modelled as service sequences.}
    \label{fig:running_example}
\end{figure}

\begin{figure}[!htbp]
	\centering
	\includegraphics[width=0.85\textwidth]{MX_Papers/Paper10/Figures/methodology_complete.png}
	\caption{Overview of process for testing FBs across various software platforms. The seven-step process is partly automated with tools (blue boxes), partly it requires tool-assisted manual development. The test application can be ported to tools of various vendors and validate the expected behaviour of an FB directly in the target RTE, thus, possibly identifying differences in the execution semantics that affect the FB under test.}
	\label{fig::methodology}
\end{figure}

\begin{itemize}
   \item Input event \texttt{REQ} triggers the calculation.
   \item Data inputs \texttt{DI1}, \texttt{DI2} receive values from other FBs and are used in the algorithm.
   \item The data output \texttt{DO1} stores the result of the algorithm.
   \item The output event \texttt{CNF} is issued to indicate that the algorithm was completed and that the output result is ready to be used by other FBs.
   \item The algorithm \texttt{REQ} within the FB is executed whenever the input event \texttt{REQ} occurs. The algorithm assigns the result of the formula to the output variable DO1.
\end{itemize}

In our example, the FB performs the calculation if the values of \texttt{DI1} and \texttt{DI2} are between 1 and 1000 (cf. implementation of the state diagram in Figure~\ref{fig:running_example}). When triggering the \texttt{REQ} event with appropriate input values, the FB executes the algorithm and produces the respective output. Both cases are modelled as service sequences and serve as test scenarios for the running example.

\section{Methodology for Testing FBs}
\label{sec::methodology}
Our proposed approach for cross-platform FB testing is suitable for a test-driven development process, as well as for testing existing implementations. It involves specifying tests, executing tests within an IDE, generating the portable test application, as well as executing these tests on all relevant RTEs. A test application that is compliant with IEC~61499 can be ported to RTEs of various vendors. The approach is visualised in Figure~\ref{fig::methodology} as a step-by-step approach using the generated test cases of our running example (Section~\ref{sec::running_example}). In the following, we describe each step of the process in detail. We describe the envisioned development process based on our running example.

\subsection{Test-driven Development of FBs}
The following steps describe a test-driven development process. For testing existing FB libraries, the process starts directly at step 4, as the implementation is already complete. In this case, test cases will be generated semi-automatically from the implementation to reduce the effort.

\subsubsection{Creating a new Function Block type (FBT)}
Reusable functionality should be encapsulated in an FB. This involves specifying the input/output events and data inputs/outputs. For the running example, this involves defining the interface of the calculation FB.

\subsubsection{Specify test cases as service sequence models}
Specification models are defined. In the running example, a service model with two sequences is provided to define the test scenarios for the FB (cf. Figure~\ref{fig::methodology}, step 2). The service model specifies the expected event occurrences, as well as the input values (\texttt{DI1} and \texttt{DI2}) and the expected output value (\texttt{DO1}) for each test case.  
The scenario of \texttt{test1} is triggered upon arrival of an event at the input \texttt{REQ}. The purpose of \texttt{test1} is to describe the FB behaviour by checking whether it correctly returns 19 when given input values of 5 and 7. 
Additionally, \texttt{test2} aims to evaluate the FB behaviour for an edge case, as one value will be out of range (i.e., \texttt{DI2:=INT\#1001}). We expect that no addition is performed, and no output events are sent (cf. second sequence in step 2). 
Where output data values are expected, they are specified together with the output event.

\subsubsection{Implement desired functionality of the FB type}
When following a TDD process, the functionality of the FB is implemented at this stage. The specified tests can be used for iteratively evaluating the correctness. For instance, the implemented behavior of an FB can be analysed using model interpreters to receive rapid feedback on any changes. 
For the running example, we assume that the limit check of DI2 was not yet implemented. This scenario is illustrated in step 3 of Figure~\ref{fig::methodology}. When the test cases are executed using a model interpreter \cite{wiesmayr2021}, the feedback shows that an unexpected event occurrence (i.e., CNF) has been output by the FB under test. 
By comparing the actual output with the expected output for each test case, the implemented FB behaviour is evaluated automatically.
As a result, the transition condition of the state diagram can be updated to include the missing check for \texttt{DI2<=1000}. Afterwards, evaluating the service sequence is successful.

\subsection{Testing Implemented FBs}
After an FB has been developed, it needs to be evaluated in an RTE. While manual testing is feasible on all existing RTEs, automating the process allows to reduce the development time and effort. Hence, the following steps describe the process of model-based testing by generating an IEC~61499-based test application from the test specification.

\subsubsection{Defining (additional) test cases for the implemented FB}
Especially when validating the behavior of an FB in different platforms, creating test cases for additional corner cases may be useful. Even when a test-driven development approach is followed, a comprehensive test suite may not be available. Using a model interpreter and its accompanying execution framework~\cite{wiesmayr2021} allows recording additional test cases based on specified events and/or data inputs. This also allows adding test suites to existing implementations.  Step 4 in Figure~\ref{fig::methodology} shows the dialogue for recording a service sequence where DI1 is out of range. The resulting graphical diagram is added to the FB type specification and is shown on the right. Developers have to manually check whether the result matches the expected behaviour of the developed FB. Recorded tests can serve as regression tests, as they capture the actual behaviour of an executed FB. This can ensure that an FB is evaluated comprehensively during model evolution.

\subsubsection{Generate test application for specified test cases}
Once the implementation has been completed, the correct real-time behavior of an FB has to be evaluated in a run-time environment. The test application ensures that developers do not need to manually interact with an FB and observe the outputs. Hence, various components are required for issuing test signals, which can be automatically generated. They are visualized in step 5 of Figure~\ref{fig::methodology} and described as follows:
\begin{itemize}
    \item \textbf{(A) Test signal generator:} This FB generates the input signals based on the service model and supplies the required events and data to the FB under test (REQ, DI1, DI2). It also notifies the test application of the expected output events (CNF or none). Additionally, it provides the expected output values (DO1), which are set in algorithms. 
    \item \textbf{(B) Matcher:} This FB compares the execution results of the FB under test with the expected results that are provided by the test signal generator. 
    \item \textbf{(C) Multiplexer: } The next FB forwards the result (ERROR, SUCCESS) of each test case (i.e., service sequence) to the output, together with the name of the executed test case. This helps developers to identify the faulty test case if there are any problems. 
    \item \textbf{(D) Test application composite: } A composite FB encapsulate the test application so that it can be easily deployed to an RTE. All components are interconnected to provide a simple interface. Developers can run each test case by triggering the respective events. An additional event pin ``run\_all'' is provided to execute all test cases sequentially based on a single event trigger. This functionality is handled by an additional FB which initiates these further test cases. 
\end{itemize}
Some of the components (e.g., the matcher) also require a timer to wait for the results of the FB under test, thus, ensuring that no unexpected event outputs are detected. As a result, timer FBs are included in the test application.
Note that the test signal generator, the FB under test, and the matcher are instantiated once per test case. This ensures that the internal state of these components does not affect further test cases. The FBs are guaranteed to initiate the execution from the START-state. As all FBs for the test application require an internal state, they are realized as Basic FBs. Although Figure~\ref{fig::methodology} visualizes the test environment for a Basic FB (i.e., the running example), any kind of FB can serve as the FB under test. Only the interface definition and the service model are required. This also means that application parts can be tested as long as they are integrated into an FB.

Based on the results provided in \cite{Testing_Midhun,biancaMidhunETFAwip}, we derived transformation rules for creating test code from service models. We implemented these rules in Eclipse 4diac~\cite{eclipse4diac}, which provides an open source IDE for IEC 61499-based software.

\begin{figure}[!htbp]
    \centering
    \includegraphics[width=\linewidth]{MX_Papers/Paper10/Figures/generation_rules.png}
    \caption{Test application generated for two service sequences of the running example. Relevant regions are highlighted including their relation to the service model.}
    \label{fig::testapp}
\end{figure}

\begin{itemize}
    \item  A \textbf{service model} serves as the test suite and includes one or more service sequences. A single test application is generated for the whole service model. 
    
    \item Each \textbf{service sequence} serves as one test case. We need one event input per test case in the test application FB, which will trigger the execution of this test. We include its name in the event pin to relate the parts of the test application with the respective service sequences. Once the service sequence was completed, an ERROR or SUCCESS event is issued, together with the name of the service sequence. A service sequence can consist of several service transactions, which define the flow of events along the sequence. 
     
     \item Each \textbf{service transaction} is comprised of an input primitive (ingoing arrow) and any number of output primitives (outgoing arrows). The service transactions are processed one after another. An event issued by the matcher (nextCase) indicates that a transaction has been completed.
    
    \item The \textbf{input primitive} describes the event that initiates a transaction. 
    The test signal generator FB has to issue the input event specified in a transaction. They are supplied to the FB under test via the respective connections together with the associated data. As a result, the test signal generator FB requires one output pin (events and data) for each input pin of the FB under test. 
    
    \item The \textbf{output primitives} describe the event(s) that is/are caused by the input event.
    The test signal generator FB issues an event indicating all output events that are part of a test sequence. They are supplied to the FB under test via the respective connections together with the associated data. As a result, the test signal generator FB requires one output pin (events and data) for each input pin of the FB under test. The matcher FB receives information about the expected events and data values from the signal generator FB. It compares them with the events and data received from the FB under test.
\end{itemize}

\subsubsection{Port test application to other platforms}
The presented development approach is fully supported in 4diac~IDE~\cite{eclipse4diac}. Although service sequences are defined in the standard, they are not fully supported in other tools. Also the model execution framework for evaluating FBs is provided in 4diac~IDE. Hence, the test application is generated in 4diac~IDE following the notation of the IEC~61499 standard, and is then ported to other platforms. Manual effort may be required for importing FBs developed in one IDE to other vendors \cite{cheng_pang_portability}.

\subsubsection{Execute Tests in All Relevant RTEs}
The generated test application (i.e., the composite FB), is deployed to and executed on different RTEs. The behaviour and output results of the FB are evaluated manually in each RTE.

\section{Implementation}
\label{sec::implementation}
Sophisticated tool support can automate a large part of the process, especially for the steps 4 to 6. The systematic approach ensures that FBs can be thoroughly tested for functionality and compatibility across various RTEs.
Tool support for specifying service sequences and simulating their results is available in 4diac IDE from previous work \cite{wiesmayr2021}. We have extended the tool with a test FB generator, which uses the information provided in service models to create test code. Unlike the prototype presented in \cite{biancaMidhunETFAwip}, the current implementation fully automates the process of creating control code and supports all kinds of test cases that can be modeled in service sequences. The code is available open source on Github.\footnote{\url{https://github.com/eclipse-4diac/4diac-ide/tree/release/plugins/org.eclipse.fordiac.ide.fb.interpreter}}

\subsection{Case Study: Processing Station}
\label{sec::casestudy}
\label{sec:drilling}
 
The processing station (Figure~\ref{fig:ps}) is composed of several mechatronic components, including the table, tester,  and drill component. They are considered smart, i.e., each is equipped with its own control devices, implementing the provided operations. The table component undergoes rotation from one fixed position to another. A complete cycle is achieved when the table rotates six times. Whenever a material is positioned in the loading area, the table rotates to align it underneath the tester component. The tester then checks whether the material has been drilled. If necessary, the drill component is triggered to process the material as soon as its sensor detects it. 

\begin{figure}[!htbp]
	\centering
		\includegraphics[width=0.5\linewidth,clip]{MX_Papers/Paper10/Figures/processingStation.jpg}
		\caption{Processing station}
		\label{fig:ps}	
 \end{figure}

The processing station system implemented in IEC~61499 is designed to control various mechatronic components through an integrated application. Each of these components is equipped with its own control device, which execute distinct control programs, i.e., the TableControl, TesterControl, and DrillControl programs. The application is implemented following the "Chain of Actions" design pattern \cite{Patil.2018}, which draws inspiration from the "Chain of Responsibility" design pattern. Each of the components is implemented as a FB network that is shown in Figure~\ref{fig:Application}. For the evaluation, three Basic FBs controlling the respective substations will be presented in detail.

\subsubsection{Table Control: Rotation}
The TableControl FB network (cf. Figure~\ref{fig:Application} \textit{top}) is responsible for rotating the table to position the material appropriately under the tester and drill components. The \texttt{WPdeliveryService} FB, a composite FB encapsulating another FB network, manages this rotation through the \texttt{TableDriver} FB and a pair of \texttt{E\_DELAY} FBs. The latter are part of the core FB library defined in the standard. Multiple instances of \texttt{WPdeliveryService} are included in the control application. 

The \texttt{TableRotate} FB encapsulates the core control logic within the network, making it the primary focus for testing the component's features. Its interface and state diagram are shown in Figure~\ref{fig:TableRotate}. The FB controls the rotational movement of a table machine which can rotate to different positions as required by the system's operation through state transitions. The primary functions include starting the rotation, monitoring the rotation process, checking if the table is in the correct position, and handling timeouts. This FB ensures that the table's motion is accurately controlled and stops when the desired position is reached or if a timeout occurs. As shown in the FB interface, the main events are \texttt{ROTATE} and \texttt{TIMEOUT\_EXCEED}. The \texttt{inPosition} input variable signals that the table has reached its target position, while output events such as \texttt{DRIVE\_ON}, \texttt{DRIVE\_OFF}, and \texttt{DONE} manage the rotation process. The state diagram (i.e., ECC) initially has an active \texttt{START} state, moves to \texttt{ROTATE\_START} when the rotation begins, and transitions to \texttt{ROTATE\_CONTINUE} if a timeout occurs. Once the table is in position, the FB transitions to \texttt{DONE}, stops the rotation, and then returns to \texttt{START}, ensuring precise control of the table's movement. Two test scenarios for the timeout are included as service sequences in Figure~\ref{fig:TableRotate}. They show that the drive is switched off if the position has been reached. The scenarios ensure that the rotation is stopped even if the signal \texttt{inPosition} is not received correctly.

\begin{figure}[!htbp]
	\centering
		\includegraphics[width=0.99\linewidth,clip]{MX_Papers/Paper10/Figures/TableRotate.png}
            \includegraphics[width=0.99\linewidth]{MX_Papers/Paper10/Figures/tests_casestudy/Service-TableRotate_selected.png}
        \caption{TableRotate: Interface, state diagram, and selected test cases.}
		\label{fig:TableRotate}	
 \end{figure}

\begin{figure}[!htbp]
	\centering
		\includegraphics[width=0.99\linewidth,clip]{MX_Papers/Paper10/Figures/FB_App.png}
		\caption{Control applications for the three substations including input FBs for processing signals from the physical system, control blocks orchestrating the station, and output FBs for writing information to actuators.}
		\label{fig:Application}	
 \end{figure}

\subsubsection{Tester Control: Inspection}
The FB network for the Tester component, TesterControl, detects holes in the workpiece to prevent that a workpiece is drilled more than once and to verify that the workpiece has undergone drilling. The main component is the Composite FB \texttt{Test} which is comprised of a standard library FB (\texttt{E\_DELAY}) and the \texttt{TestCtrl} Basic FB. 

Upon activation, the \texttt{TestCtrl} FB uses sensor data to check whether a hole is present in the workpiece. In this case, it confirms that drilling was complete. The FB manages the sequence of extending, checking, and retracting the probe, and handles timeouts in case the process takes too long. The interface and ECC of the FB are shown in Figure~\ref{fig:TestCtrl}. 
The primary input event, \texttt{TRIGGER}, initiates the detection sequence, with \texttt{QI} qualifying this event and \texttt{QO} reflecting the detection result. The FB has four output events: \texttt{EXTEND} to extend the sensor, \texttt{RETRACT} to retract it, \texttt{DONE} to signal that the process was completed, and \texttt{TIMEOUT} to initiate a timer. The ECC transitions from \texttt{START} to \texttt{EXTEND} upon receiving \texttt{TRIGGER}, and if a timeout occurs, moves to \texttt{CHECK} to determine whether a hole was detected. It then transitions to \texttt{RETRACT} and finally to \texttt{DONE}. A test case that shows the behaviour in the case of a timeout is shown in Figure~\ref{fig:TestCtrl}.

\begin{figure}[!htbp]
	\centering
		\includegraphics[width=0.99\linewidth,clip]{MX_Papers/Paper10/Figures/TestCtrl.png}
            \includegraphics[width=0.99\linewidth]{MX_Papers/Paper10/Figures/tests_casestudy/Service-TestCtrl_timeout.png}
		\caption{TestCtrl: Interface, state diagram and a selected test case.}
		\label{fig:TestCtrl}	
 \end{figure}
 
\subsubsection{Drill Control: Processing}
The FB network for the drill component, DrillControl, orchestrates the drilling process. The main FBs are \texttt{DrillingSequence} and \texttt{DrillDriver}. The \texttt{DrillingSequence} is a composite FB that uses two FBs, \texttt{DoubleActingCylinder} and \texttt{SingleActingAc\-tu\-a\-tor}. The former is responsible for controlling the vertical movement of the drill, while the latter controls the rotation of the drill motor. As the \texttt{DoubleActingCylinder} FB contains the core logic of the drilling process, it is shown in Figure~\ref{fig:DoubleActingCylinder}. It controls the bidirectional motion of a drill machine, specifically managing its upward and downward movements. This control is essential for operating machinery that uses linear actuators, such as hydraulic or pneumatic cylinders, to extend and retract, corresponding to the downstroke and upstroke of the drill. The FB handles initialization, execution requests, extension (downward movement), and retraction (upward movement), with input conditions determining the cylinder's state transitions and output commands controlling the actuators.
Specifically, the cylinder movement is controlled via the events \texttt{EXTEND} and \texttt{RETRACT}. The input variables \texttt{atHome} and \texttt{atEnd} indicate the cylinder's position, while the output variables \texttt{extend} and \texttt{retract} command its motion. The ECC ensures that the cylinder moves correctly based on events and position feedback. The algorithms activate or stop the cylinder's movement, ensuring precise control for upward and downward motion.

\begin{figure}[!htbp]
	\centering
		\includegraphics[width=0.99\linewidth,clip]{MX_Papers/Paper10/Figures/DoubleActingCylinderV2.png}
            \includegraphics[width=0.99\linewidth]{MX_Papers/Paper10/Figures/Service-DoubleActingCylinder_selected.png}
		\caption{Double Acting Cylinder: Interface, state diagram, and selected test cases}
		\label{fig:DoubleActingCylinder}	
 \end{figure}

\section{Evaluation}
We tested the main control FBs from the processing station, but also validated the implementation itself. Following our methodology, we imported any FBs from other tools into 4diac IDE. In 4diac IDE, we then generated the test application, which was executed in 4diac~FORTE and in EcoRT.

\subsection{Evaluating the Generation Framework in Eclipse 4diac}
Selected test cases defined as service sequences as shown in Figures~\ref{fig:TableRotate}, \ref{fig:TestCtrl} and \ref{fig:DoubleActingCylinder}. In total, 12 test cases were created which described both the expected interactions with an FB and deviations (such as events from the environment that occur in a reversed order). To additionally validate that the framework can successfully detect failed test cases, we intentionally defined five service sequences with unexpected behaviour. An example for the TableRotate FB is shown in Figure~\ref{fig:table_rotate_fail}. These experiments validate that the generated test application can indeed detect deviations between the implementation and the specification.

\begin{figure}[!htbp]
    \centering
    \includegraphics[width=0.7\linewidth]{MX_Papers/Paper10/Figures/tests_casestudy/Service-TableRotate_negative.png}
    \caption{Service sequences that result in failed test cases for a correct implementation.}
    \label{fig:table_rotate_fail}
\end{figure}

\subsection{Semantic Variants in Ported Software}
All involved FBs were manually imported in a second tool environment to evaluate the behavior of a ported test application. Unfortunately, the testing of FBs on the EcoStruxure Automation Expert (EAE) revealed that certain test cases were failing when executed on the Tester Control. Modifications to the framework were required to compensate the variations in the implemented execution semantics.

Upon thorough analysis, it was determined that these failures were attributed to a semantic execution issue inherent in the EAE. Unlike specified the IEC 61499 standard, EAE adopts a specific semantic execution model. This divergence in execution semantics implies that when a real-world application is ported from Eclipse 4diac to EAE, it may not function correctly, and in some cases, it may even result in system failures.

EAE's semantic execution model operates such that when two or more identical events are used as a sequence of events to transition through multiple states, a single event trigger results in a direct transition to the final state. Specifically, when the event triggers once, the system bypasses intermediate states and directly reaches the final state. 
To illustrate this issue, consider the ECC shown in Figure~\ref{fig:SemanticExexcissue} where the event \texttt{TIMEOUT\_EXCEEDED} is expected to facilitate a sequence of state transitions. According to the desired behavior, upon the first occurrence of the \texttt{TIMEOUT\_EXCEEDED} event, the system should transition to a state labeled \texttt{CHECK}. It should remain in this \texttt{CHECK} state, awaiting a subsequent \texttt{TIMEOUT\_EXCEEDED} event to be triggered before progressing to the final state labeled \texttt{DONE}. However, due to the semantic execution model employed by EAE, when the \texttt{TIMEOUT\_EXCEEDED} event is triggered only once, the system bypasses the \texttt{CHECK} state entirely and directly transitions to the \texttt{DONE} state. This behavior deviates from the intended design, leading to incorrect execution flow.

To address this issue, the test case generation framework was modified to compensate the semantic execution differences. This algorithm modifies the \texttt{TIMEOUT\_EXCEEDED} event by setting its value to \texttt{FALSE}, ensuring that the system remains in the \texttt{CHECK} state after the initial trigger. Consequently, the system only moves to the next state when a subsequent \texttt{TIMEOUT\_EXCEEDED} event is triggered. This modification aligns the execution flow with the expected sequence, preventing premature transitions to the final state.

\begin{figure}[!htbp]
	\centering
		\includegraphics[width=0.99\linewidth,clip]{MX_Papers/Paper10/Figures/SemanticExexcissue.png}
		\caption{Semantic Execution issue}
		\label{fig:SemanticExexcissue}	
 \end{figure}

\section{Results and Discussion}
\label{sec::results}
The semantic differences outlined in the previous section demonstrate the need for a cross-platform testing framework. While the semantic differences could be bridged within our implemented generators, transferring real-world FBs between execution environments can lead to undetected failures. In any portability scenario, this would mean that the distribution of software parts across various vendor's IDEs introduces bugs that are difficult to detect.

It was observed that the control application for our demonstrator contained certain bugs after porting to Eclipse 4diac, which were uncovered during the execution of the test cases. These issues highlight the importance of recognizing the semantic execution differences between EAE and the IEC 61499 standard when porting applications. Addressing such discrepancies is crucial to ensure that applications function as intended within the EAE environment.

The migration of Function Blocks (FBs) from Eclipse 4diac to EcoStruxure Automation Expert (EAE) presents several challenges. Initially, the FBs were imported into 4diac IDE, where tests were defined and test applications were generated. 
These test applications were executed on the accompanying runtime environment, 4diac FORTE. However, when these test applications were ported to EcoStruxure and executed on the EcoRT runtime environment, portability issues arose. The following outlines the key challenges encountered during this migration process.

One significant challenge involves the direct addition of a composite FB containing the \texttt{E\_DELAY} FB. The \texttt{E\_DELAY} FB is a standard function block that is pre-compiled into the vendor's runtime environment. Due to this pre-compiled nature, adding any standard FB directly is not possible. Standard FBs are already available within the vendor's IDE, which necessitates a replacement approach to ensure proper functionality. Consequently, when a composite FB that contains a standard FB is added, it becomes necessary to manually replace the standard FB with an equivalent vendor-provided standard FB within the vendor's IDE.

Another issue encountered pertains to adapters and namespaces. In EAE, an adapter cannot be located unless the correct namespace is specified as \texttt{Main}. The definition of the appropriate namespace is essential; otherwise, the adapter will not be displayed. Correcting the namespace can resolve this particular issue. However, even after setting the correct namespace, the use of adapters in composite FBs poses additional problems. These problems require the removal and subsequent redrawing of connections, indicating that the mere correction of namespaces is insufficient when dealing with composite FBs involving adapters.

Further challenges were identified while porting the test FBs in EAE, particularly in relation to the algorithm section within the \texttt{.fbt} file. It was observed that the \texttt{start\_algorithm} and \texttt{end\_algorithm} statements were repeated twice. This redundancy results in errors that must be rectified manually by removing the duplicate statements. Additionally, case sensitivity issues arise within the algorithms. For instance, Boolean variable values are written as \texttt{"false"} and \texttt{"true"} instead of the expected \texttt{"FALSE"} and \texttt{"TRUE"}. This discrepancy requires manual correction to conform to the syntax standards of the target environment.

Another issue relates to the naming conventions used in algorithms. Algorithm names containing double underscores (\texttt{\_\_}) are not supported. As a result, any algorithm names with double underscores must be modified to eliminate this character sequence. This change is necessary to ensure that the algorithms are compatible with the EAE environment. These migration challenges highlight the need for some manual adjustments to ensure the successful transfer and execution of FBs within different automation environments.

In summary, the proposed framework can reduce the effort of porting libraries because the test execution is automated rather than requiring manual labour.

\section{Limitations}
\label{sec::limitations}
We have proposed and implemented a framework for generating cross-platform tests for IEC~61499-based software. We used two tool environments to evaluate the portability and built on previous results of known execution issues. Currently, no industrial-scale FB libraries are publicly available for testing our approach in practice. As described in the paper, service sequences can serve as test cases for IEC~61499 FBs. However, there is no possibility to specify timing information \cite{Wiesmayr_ifac}. Our approach mainly targets the event-based behaviour of IEC~61499 FBs, while other approaches can cover timing verification (c.f. \cite{Bruns.2023}).

\section{Conclusion and Future Work}
\label{sec::conclusions}
In conclusion, the proposed testing methodology for IEC 61499 FBs offers systematic and reliable means to verify the correct behaviour of FBs across diverse RTEs. The generation of test FBs from a model-based specification, represented as a service sequence, was accomplished through semi-automated means. Engineers can manually create the test specification (e.g., for test-driven development), or derive it from an existing implementation via the IDE. Our created test FBs are portable across platforms to allow platform-independent testing. This paper presented the overall approach and a first proof-of-concept implementation. We provide an initial set of transformation rules and the corresponding tool support for automating part of the process.

In future work, we aim to support all kinds of FB implementations, provide also the test applications automatically, and evaluate our approach based on a realistic use case to evaluate the scalability and feasibility of the approach in practice. Additionally, developing a runtime comparison tool to analyse and compare the behaviour and performance of FBs across different platforms would enable control engineers to identify and address discrepancies. Finally, integrating the testing approach with further model-based development techniques, such as formal methods or simulation, would provide a holistic approach to system verification and enhance the overall reliability of developed systems.

\clearpage
\putbib
\end{bibunit} 




\printglossary[type=\acronymtype]

\end{document}
